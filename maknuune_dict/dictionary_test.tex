\documentclass[11pt]{book}
\usepackage{preamble}

\newcommand\Cstar{CALIMA$_{Star}$}

\newcommand{\caphi}[1]{{/{{\it #1}}/}}
\usepackage{xspace}
\newcommand{\curras}{{Curras}\xspace}
\newcommand{\maknuune}{{Maknuune}\xspace}
\newcommand{\hide}[1]{}

\newcommand{\AMADDA}{{\={A}}}
\newcommand{\AHAMZAUP}{{\^{A}}}
\newcommand{\WHAMZA}{{\^{w}}}
\newcommand{\AHAMZADN}{{\v{A}}}
\newcommand{\YHAMZA}{{\^{y}}}
\newcommand{\TAMARBUTA}{{$\hbar$}}
\newcommand{\TAMAR}{{$\hbar$}}
\newcommand{\THA}{{$\theta$}}
\newcommand{\DHA}{{\dh}}
\newcommand{\SHIN}{{\v{s}}}
\newcommand{\DAD}{{\v{D}}} %Z
\newcommand{\ZA}{{\v{D}}} %Z
\newcommand{\AYN}{{$\varsigma$}}
\newcommand{\GAYN}{{$\gamma$}}
\newcommand{\AMAQSURA}{{\'{y}}}
\newcommand{\AMAQ}{{\'{y}}}
\newcommand{\FATHATAN}{{\~{a}}}
\newcommand{\KASRATAN}{{\~{\i}}}
\newcommand{\DAMMATAN}{{\~{u}}}
\newcommand{\SHADDA}{{$\sim$}}
\newcommand{\DAGGER}{{\'{a}}}

\newcommand{\LONGA}{{\={a}}}
\newcommand{\LONGU}{{\={u}}}
\newcommand{\LONGI}{{\={\i}}}
\newcommand{\LONGE}{{\={e}}}
\newcommand{\LONGO}{{\={o}}}

\newcommand{\LEVJ}{{\v{z}}}
\newcommand{\TSHA}{{\v{c}}}

\newcommand{\tab}{\hspace*{1em} }
 
 
 
\newcommand{\adam}{\sc{ADAM}}
\newcommand{\magead}{\sc{MAGEAD}}
\newcommand{\mada}{\sc{MADA}}
\newcommand{\tokan}{\sc{TOKAN}}
\newcommand{\amira}{\sc{AMIRA}}
%\newcommand{\amiratok}{\sc{Amira-Tok}}
%\newcommand{\amirapos}{\sc{Amira-Pos}}
%\newcommand{\amirabpc}{\sc{Amira-Bpc}}
 
\newcommand{\REGEX}{\sc{RegEx}}
%%
\newcommand{\vs}{\sc V-Sent}
\newcommand{\ns}{\sc N-Sent}
\newcommand{\np}{\sc N-Phrase}
\newcommand{\pp}{\sc P-Phrase}
\newcommand{\masc}{\sc Masc}
\newcommand{\fem}{\sc Fem}

\newcommand{\sing}{\sc Sg}
\newcommand{\dual}{\sc Du}
\newcommand{\plur}{\sc Pl}
\newcommand{\nom}{\sc Nom}
\newcommand{\acc}{\sc Acc}
\newcommand{\gen}{\sc Gen}
\newcommand{\scon}{\sc Con}
\newcommand{\sdef}{\sc Def}
\newcommand{\sindef}{\sc InDef}

\newcommand{\third}{\sc 3rd}
\newcommand{\second}{\sc 2nd}
\newcommand{\first}{\sc 1st}

\newcommand{\CATiB}{Columbia Arabic Treebank}
\newcommand{\CTB} {CATiB}

\newcommand{\parent} {\bf{ATT}}
\newcommand{\rel} {\bf{LAB}}
\newcommand{\parrel} {\bf{LABATT}}

\newcommand{\msa}{{\sc MSA}}
\newcommand{\lev}{{\sc Lev}}
\newcommand{\irq}{{\sc Irq}}
\newcommand{\glf}{{\sc Glf}}
\newcommand{\egy}{{\sc EGY}}
 

\newcommand{\calima}{{\sc CALIMA}}
\newcommand{\bama}{{\sc BAMA}}
\newcommand{\sama}{{\sc SAMA}}
\newcommand{\ecal}{{\sc ECAL}}
\newcommand{\Sama}{{\sc SAMA3.1}}
\newcommand{\da}{{\sc DA}}
\newcommand{\coda}{{\sc CODA}}
%
% \newcommand{\num}{{\sc Num}}
%\newcommand{\dt}{{\sc Det}}
%\newcommand{\enword}{{\sc EngWord}}
%\newcommand{\pos}{{\sc Pos}}

%\newcommand{\test}{{\sc MT06}}
%\newcommand{\dev}{{\sc MT05}}


\newcommand{\train}{{\sc Train}}

\bibliography{camel-bib-v2,extra,maknuune-cite}

\makeatletter
\renewcommand{\@chapapp}{}% Not necessary...
\newenvironment{chapquote}[2][2em]
  {\setlength{\@tempdima}{#1}%
   \def\chapquote@author{#2}%
   \parshape 1 \@tempdima \dimexpr\textwidth-2\@tempdima\relax%
   \itshape}
  {\par\normalfont\hfill--\ \chapquote@author\hspace*{\@tempdima}\par\bigskip}
\makeatother

\fancypagestyle{logo}{\fancyhf{}\renewcommand{\headrulewidth}{0pt}\fancyfoot[C]{\includegraphics[width=0.4\textwidth]{camel-lab-logo.png}}
\fancyfoot[R]{{\normalfont\fontsize{15}{15}\selectfont \textbf{v1.0.2}}}}

\begin{document}
\setsansfont{Arial}

\frontmatter %Use lowercase Roman numerals for page numbers

%-----------------------------------------------------------
% COVER PAGE
%-----------------------------------------------------------

\title{Maknuune}

\begingroup
\thispagestyle{logo}
%\AddToShipoutPicture*{\put(-175,0){\transparent{0.5}\includegraphics[scale=1]{unnamed-4.jpg}}} % Image background
\AddToShipoutPicture*{\put(-175,0){\includegraphics[scale=1]{maknuune-egg.png}}} % Image background
\centering
\vspace*{3cm}
{\par\fontsize{35}{35}\textbf{MAKNUUNE}\\}
\vspace{20pt}
{\fontsize{45}{45}\foreignlanguage{arabic}{مكنونة} \\}
\vspace*{1cm}
{\par\normalfont\fontsize{30}{30}\selectfont A Large Open \\ Palestinian Lexicon} \\ % Book title
\vspace{1mm}
{\par\normalfont\fontsize{32}{32}\selectfont \foreignlanguage{arabic}{قاموس} \\ \foreignlanguage{arabic}{اللهجة العربية الفلسطينية} \\ \foreignlanguage{arabic}{المفتوح المصدر}} \\ % Book title
\vspace*{2.5cm}
{\normalfont\fontsize{18}{18}\selectfont \textbf{SHAHD DIBAS}}\par % Author name
\vspace{1mm}
{\normalfont\fontsize{20}{20}\selectfont \textbf{\foreignlanguage{arabic}{شهد دعباس}}}\par % Author name
\endgroup

%-----------------------------------------------------------
% INSIDE COVER
%-----------------------------------------------------------
\newpage
\thispagestyle{logo}

{
\centering
\vspace*{2cm}
{\par\fontsize{35}{35}\textbf{MAKNUUNE}\\}
\vspace{20pt}
{\fontsize{45}{45}\foreignlanguage{arabic}{مكنونة} \\}
\vspace*{1cm}
{\par\normalfont\fontsize{30}{30}\selectfont A Large Open \\ Palestinian Lexicon} \\ % Book title
\vspace{1mm}
{\par\normalfont\fontsize{32}{32}\selectfont \foreignlanguage{arabic}{قاموس} \\ \foreignlanguage{arabic}{اللهجة العربية الفلسطينية} \\ \foreignlanguage{arabic}{المفتوح المصدر}} \\ % Book title
\vspace*{2cm}
{\normalfont\fontsize{18}{18}\selectfont \textbf{SHAHD DIBAS}}\par % Author name
\vspace{1mm}
{\normalfont\fontsize{20}{20}\selectfont \textbf{\foreignlanguage{arabic}{شهد دعباس}}}\par % Author name

\vspace{1cm}
{\normalfont\fontsize{15}{15}\selectfont in collaboration with}\par
\vspace{1mm}
{\normalfont\fontsize{15}{15}\selectfont \foreignlanguage{arabic}{بالتعاون مع}}\par
\vspace{1cm}
}
{
{\normalfont\fontsize{12}{12}\selectfont \textbf{CHRISTIAN KHAIRALLAH}}
\hspace{1.5cm} and \hspace{2cm}
{\normalfont\fontsize{12}{12}\selectfont \textbf{NIZAR HABASH}}
\par
\hspace{2cm}
{\normalfont\fontsize{12}{12}\selectfont \textbf{\foreignlanguage{arabic}{كريستيان خيرالله}}}
\hspace{3.5cm}
{\normalfont\fontsize{12}{12}\selectfont \textbf{\foreignlanguage{arabic}{و}}}
\hspace{3.5cm}
{\normalfont\fontsize{12}{12}\selectfont \textbf{\foreignlanguage{arabic}{نزار حبش}}}

}

%-----------------------------------------------------------
% COPYRIGHT PAGE
%-----------------------------------------------------------

\newpage
~\vfill
\thispagestyle{empty}

%\noindent Copyright \copyright\ 2014 Andrea Hidalgo\\ % Copyright notice

\noindent \textbf{New York University Abu Dhabi}\\
\noindent Abu Dhabi, United Arab Emirates\\

\copyright\ \textsc{Shahd Dibas and CAMeL Lab at New York University Abu Dhabi, 2022}\\
All rights reserved. The content in this work is licensed under a \href{https://creativecommons.org/licenses/by-sa/4.0/}{Creative Commons Attribution-ShareAlike 4.0 International} license. \\

\textbf{Website}\\
\noindent \url{www.palestine-lexicon.org}\\ % URL

\textbf{Citation}\\
\href{https://arxiv.org/abs/2210.12985}{Shahd, Dibas, Christian Khairallah, Nizar Habash, Omar Fayez Sadi, Tariq Sairafy, Karmel Sarabta, and Abrar Ardah. 2022. "Maknuune: A Large Open Palestinian Arabic Lexicon." In: \textit{Proceedings of the Workshop for Arabic Natural Language Processing (WANLP)}. Abu Dhabi, United Arab Emirates.} \\

\noindent This work was done in collaboration with the \href{https://nyuad.nyu.edu/en/research/faculty-labs-and-projects/computational-approaches-to-modeling-language-lab.html}{Computational Approaches to Modeling Language Lab (CAMeL Lab)} at New York University Abu Dhabi.\\ % License information

\noindent \textit{First release, December 2022} % Printing/edition date

%-----------------------------------------------------------
% ENDORSEMENTS
%-----------------------------------------------------------

\newpage
\thispagestyle{empty}
{\par\normalfont\fontsize{30}{30}\selectfont \textbf{Endorsements}} \\
\vspace{2cm}

\begin{chapquote}{\textbf{Prof. Noam Chomsky}, \textit{Massachusetts Institute of Technology (MIT)}}
``The new Palestinian Arabic Lexicon [Maknuune] is a valuable contribution to the linguistic scholarship generally, and is especially welcome for the insights it provides into the rich cultural and social life of Palestine.''
\end{chapquote}

\vspace{5mm}

\begin{chapquote}{\textbf{Prof. Hamid Dabashi}, \textit{Columbia University}}
``The language a people speak is like the air they breathe, the food they eat, the life they live, the children they raise, the hope they invest in the future of their humanity. The timely and ambitious Maknuune Corpus is the sign of a trust rooted in a prolonged past and a heroically defended future. Collected mainly from the elderly Palestinians living in refugee camps, villages and towns, this dictionary is the living memory of a life happily and defiantly lived, and it is to preserve the Palestinian linguistic and cultural identity for the posterity. The Palestinian Arabic that this project seeks to record and preserve is as precious for our future as the very landscape they have spent generations defending against memoricide.''
\end{chapquote}

\vspace{5mm}

\begin{chapquote}{\textbf{Prof.  Abdelkader Fassi Fehri}, \textit{Mohammed V University
}}
``Produced by a group of outstanding NLP Palestinian scholars from Oxford University, NYU Abu Dhabi, and UCES at UNRWA, Maknuune is a large open lexicon of significant varieties of the Palestinian Arabic dialect (PAL). Originally designed and motivated by the need to document and preserve the cultural heritage and unique identities of the various PAL sub-varieties (or typical elderly idiolects), it is open to expand and cover PAL’s evolution, not only as a living Arabic language variety within the context of Arabic pluriglossia, but also and chiefly as the tongue and voices of the Palestinian people, and their heroic combat for life and freedom. Moreover, it aims not only to provide a resource to support research and development in the specifics of this Arabic dialect in natural language processing (NLP), or more broadly comparative Arabic linguistics, but it also serves as an important pedagogical tool for learning PAL, help teaching PAL to Palestinian children in the diaspora, and to non-Arabic speakers.
The Maknuune lexicon specifies a significant number of lexical entries in terms of roots, lemmas, and forms. Entries include Arabic diacritized orthography, phonological transcription, English and Standard Arabic glosses, associated phrases and collocations, broken plural and templatic feminine inflectional morphology, examples or notes on grammar, usage, or location of the collected entries, syntax and collocations, with expected expansions to cover more PAL sub-varieties, additional entries, richer annotations, and sufficiently specified phonological transcriptions to help in speech recognition, or morpho-lexical information for developing morphological analyzers or taggers.  
I expect that Maknuune — as one of the largest open machine-readable dictionaries for PAL — will enjoy success, support, and interest from a large public of culture, linguistics, sociolinguistics, computation and NLP, comparative lexicology, pedagogy, and planning.''
\end{chapquote}

\vspace{5mm}
\thispagestyle{empty}

\begin{chapquote}{\textbf{Prof. Clive Holes}, \textit{University of Oxford
}}
``The language a people speak is like the air they breathe, the food they eat, the life they live, the children they raise, the hope they invest in the future of their humanity. The timely and ambitious Maknuune Corpus is the sign of a trust rooted in a prolonged past and a heroically defended future. Collected mainly from the elderly Palestinians living in refugee camps, villages and towns, this dictionary is the living memory of a life happily and defiantly lived, and it is to preserve the Palestinian linguistic and cultural identity for the posterity. The Palestinian Arabic that this project seeks to record and preserve is as precious for our future as the very landscape they have spent generations defending against memoricide.''
\end{chapquote}

\vspace{5mm}

\begin{chapquote}{\textbf{Prof. Ilan Pappe}, \textit{University of Exeter}}
``Maknuune — the egg left for future hens to lay their eggs in the future — is a fitting name for this incredible project that will make sure that Palestinian culture in its widest possible definition, will light our way in the future. The ethnic cleansing, occupation and oppression perpetrated against the Palestinians, targeted also the Palestinian culture. Resistance and resilience defeated the attempt to culturicide Palestine and this lexicon is testimony to this incredible Palestinian endeavor and survival. Its width and depth will amaze any reader and it will become one of the most useful accessories for anyone's interest in Palestinian culture in the future. This is a project that commemorates forever the language of the people themselves and is an organic part of their identity, their past and future aspirations.''
\end{chapquote}

\vspace{5mm}

\begin{chapquote}{\textbf{Dr. Walid Saif}, \textit{Palestinian poet, short-story writer, playwright and critic}}
``Language is the most important vehicle of cultural heritage and identity, hence the significance of this linguistic project recording Palestinian Arabic. [This] lexicon is not just an inventory of independent entries, but rather a complex and dynamic network of concepts, ideas and values that reflects and helps construct [a] worldview as it feeds into social interactions. Thus, as a Palestinian linguist and writer, I enthusiastically endorse this project, and invite all concerned to contribute thereto.''

\end{chapquote}

%-----------------------------------------------------------
% MAKNUUNE TEAM
%-----------------------------------------------------------

\newpage
\thispagestyle{empty}
{\par\normalfont\fontsize{30}{30}\selectfont \textbf{The Maknuune Project Team}} \\
\vspace{3cm}

\begin{minipage}{0.2\textwidth}
\includegraphics[width=\textwidth]{shahd-profile.png}
\end{minipage}
\begin{minipage}{0.8\textwidth}
\textbf{Shahd Dibas} is a doctoral student in linguistics (Syntax-semantics Interface) at the University of Oxford. She was the leading force behind the Maknuune project. She collected and annotated most of the data present in this work, and did the field data collection in various villages, towns and refugee camps. Shahd has always wanted to work on a lexicon for Palestinian Arabic. She has always wanted to preserve the linguistic heritage and cultural identity of Palestinians. Her goal is to expand Maknuune in the future and write more research on Palestinian Arabic using the Maknuune lexicon. 
\end{minipage} \\

\vspace{1cm}

\begin{minipage}{0.2\textwidth}
\includegraphics[width=\textwidth]{christian-profile.png}
\end{minipage}
\begin{minipage}{0.8\textwidth}
\textbf{Christian Khairallah} is a Research Assistant at the Computational Approaches to Modeling Language Lab (CAMeL Lab) in NYU Abu Dhabi. He was the primary software engineer and computational linguist on the project. He worked on maintaining the resources using careful linguistic analysis, and developed code for streamlining well-formedness checks and quality control, as well as provided automatic slot filling to speed up repetitive tasks. Finally, he produced the dictionary that you are currently reading in pdf format. It has always been a goal of his to collect a machine-readable lexicon of levantine Arabic words and expressions, and this work is a solid step in that direction.
\end{minipage} \\

\vspace{1cm}

\begin{minipage}{0.2\textwidth}
\includegraphics[width=\textwidth]{nizar-profile.png}
\end{minipage}
\begin{minipage}{0.8\textwidth}
\textbf{Nizar Habash} is the director of the Computational Approaches to Modeling Language Lab (CAMeL Lab) at NYU Abu Dhabi. He was the primary advisor of the project, and guided the definition of the guidelines for the various components, and the evaluation and quality check process of the lexicon. His area of research is artificial intelligence, specifically natural language processing and computational linguistics. He primarily works on Arabic and Arabic dialect language processing (in terms of orthography, morphology, syntax, semantics, lexicons and corpora), machine translation and dialogue systems.
\end{minipage} \\

%-----------------------------------------------------------
% TABLE OF CONTENTS
%-----------------------------------------------------------

\pagestyle{empty} % No headers

\tableofcontents % Print the table of contents itself

%-----------------------------------------------------------
% ACKNOWLEDGMENTS
%-----------------------------------------------------------

\chapter{Acknowledgments}
This work would have never been possible without the help of so many outstanding people who believed in Maknuune from the very beginning and helped it grow with authentic data that was collected from different settings and sources in order to reflect language diversity. Some of the language users agreed to be interviewed more than one time, and they kindly connected Shahd with their grandparents and other relatives in the different villages and refugee camps in Palestine. Unfortunately, some language users preferred to be anonymous and asked us not to write their names.

We are deeply grateful to all of the language users for their time and patience in answering Shahd's questions! All of them contributed largely to the growth of Maknuune. It is worth mentioning that some of the language users whom Shahd interviewed were the last living people in some villages that can now only be found in old British maps.

We also need to highlight that some language users passed away since the beginning of the project. Sadly, with their death, we lost part of the truth about how our beautiful Palestine was before the \textit{Nakba} in 1948. Therefore, this work is dedicated to them and to all Palestinians no matter where they live.

We hope that this collective effort would enable us to preserve the cultural identity and the linguistic heritage of Palestinians, and help the young generation of Palestinians who live in the diaspora learn more about Palestine from the language that their people use.

\vspace{5mm}
Special thanks go to Omar Sadi, Dr. Tariq Sairafy, Karmel Sarabta and Abrar Arda for their contribution to Maknuune. 

\vspace{5mm}
Finally, we also thank Ahmad Habash for generously providing the artwork used on the book cover.

%-----------------------------------------------------------
% INTRODUCTION
%-----------------------------------------------------------

\chapter{Introduction}

\textit{The content of this introduction is a modified republication of \citet{dibas2022maknuune}}.

\pagestyle{plain}

\section*{Background}
\addcontentsline{toc}{section}{\protect\numberline{}Background}%
Arabic is a collective of historically related variants that co-exist in a diglossic \citep{Ferguson:1959:diglossia} relationship between a Standard variant and geographically specific dialectal variants. Standard Arabic (SA, \foreignlanguage{arabic}{العربية الفصحى})
is typically used to refer to the older Classical Arabic (CA) used in Quranic texts and pre-islamic poetry, all the way to Modern SA (MSA), the official language of news and culture in the Arab World.  Dialectal Arabic (DA) is classified geographically into regions such as Egyptian, Levantine, Maghrebi, and Gulf.
%\todo{cite transliteration}
%\cite{}.\todo{add citations?}   
The dialects, which  differ among themselves and SA, are the primary mode of spoken communication, although increasingly they are dominating in written form on social media.  That said, DA has no official prescriptive grammars or orthographic standards, unlike the highly standardized and regulated MSA.  In the realm of natural language processing (NLP), MSA has relatively more annotated and parallel resources than DA; although there are many notable efforts to fill gaps in all Arabic variants \citep{alyafeai2022masader}.

In this work, we focus on Palestinian Arabic (PAL), which is part of the South Levantine Arabic dialect subgroup. PAL consists of several sub-dialects in the region of Historic Palestine that %generally 
vary in terms of their phonology and lexical choice \citep{Jarrar:2016:curras}. 
PAL, like all other DA,  has been historically influenced by many languages, specifically, in its case, Syriac, Turkish, Persian, English and most recently Modern Hebrew \citep{moin2019etymological}, as well as other Arabic dialects that came in interaction with PAL after the Nakba. %\todo{this is tough to write about unemotionally!}
%
While this research effort was originally motivated by the need to document and preserve the cultural heritage and unique identities 
of the various PAL sub-dialects, it has expanded to cover PAL's ever-evolving nature as a living language, and provides a resource to support research and development in Arabic dialect NLP.

Concretely, we present \textbf{Maknuune}~\foreignlanguage{arabic}{مكنونة},\footnote{\foreignlanguage{arabic}{مكنونة}~/maknūne/ is a PAL farming term that refers to an egg intentionally left behind in a specific location to encourage the chicken to lay more eggs in that location.
We hope that the lexicon will encourage other researchers and citizen linguists to contribute to it.}
%We name our open-source lexicon after it, hoping that more researchers and citizen linguists will contribute to it.}
%
a large open lexicon for PAL, with over 36K entries from 17K lemmas, and 3.7K roots.\footnote{In this initial phase of Maknuune, we focus on the PAL sub-dialects spoken in the West Bank, an area with dialectal diversity across many dimensions such as \textit{lifestyle} (urban, rural, bedouin), religion, gender, and social class.}
%
All entries include diacritized Arabic orthography and phonological transcription following \citep{Habash:2018:unified}, as well as English glosses. Important inflectional variants are included for some lemmas, such as broken plural and templatic feminine. %, as well as verbal aspect 
About 10\%  of the entries are phrases (multiword expressions) indexed by their primary lemmas. And about 67\%  
of the entries include MSA glosses,  examples, and/or notes on grammar, usage, or location of collected entry.
%
To our knowledge, Maknuune is the largest open machine-readable dictionary for PAL. Maknuune is publicly viewable and downloadable.\footnote{\url{www.palestine-lexicon.org}}
%
%We present our data collection process and annotation guidelines, which we hope can be of use for similar efforts on other languages and dialects.

We discuss some related work in Section~\ref{related}, and highlight some PAL linguistic facts %and challenges 
that motivated many of our 
%lexicon 
design choices in Section~\ref{lingfacts}.  Section~\ref{method} presents our data collection process and annotation guidelines. We present statistics for our lexicon and evaluate its coverage 
%compare it with the Curras corpus \citep{Jarrar:2016:curras} 
%and Madar Lexicon-Jerusalem?
in Section~\ref{eval}.

%%%%%%%%%%%%%%%%%%%%%%%%%%%%%%%%%%%%%%%%%%%%%%%%%%%%%%%%%%%%%%%%%%%%%%%%%%%%%%%%%%%%%%%%%%%%%%%%%%%%%%%%%%%%%%%%%%%%%%%%%%%%%%%%%%%%%%%%%%%%%%%%%%%%%%%%%%%%%%%%%%%%%%%%%%%%%%%%%%%%%%%%%%%%%%%%%%%%%%%%%%%%%%%%%%%%%%%%%%%%%%%%%%%%%%%%%%%%%%%%%%%%%%%%%%%%%%%%%%%%%%%%%%%%%%%%%%%%%%%%%%%%%%%%%%%%%%%%%%%%%%%%%%%%%%%%%%%%%%%%%%%%%%%%%%%%%%%%%%%%%%%%%%%%%%%%%%%%%%%%%%%%%%%%%%%%%%%%%%%%%%%%%%%%%%%%%%%%%%%%%%%%%%%%%%%%%%%%%%%%%%%%%%%%%%%%%%%%%%%%%%%%%%%%%%%%%%%%%%%%%%%%%%%%%%%%%%%%%%%%


\section*{Related Work}
\addcontentsline{toc}{section}{\protect\numberline{}Related Work}%
\label{related}
%Previous important NLP efforts on PAL include the annotated Curras corpus \citep{Jarrar:2016:curras,NewJarrar}, the Shami corpus \cite{}\todo{@Chris: add citation for Shami; and new Curras}. The MADAR corpus lexicon included 
%\todo{German linguistic atlas of Syria and Palestine; Syrian dictionary from Georgetown; Anis Friha for Lebanese; Olive tree compare}
%A Dictionary of Syrian Arabic : English-Arabic
%Paperback Georgetown Classics in Arabic Languages and Linguistics series Arabic
%  Karl Stowasser and  Moukhtar Ani


\paragraph{Linguistic Descriptions} There are several linguistic references describing various aspects of PAL \citep{Rice:1979:eastern, herzallah1990aspects, hopkins1995sarar, elihai2004olive, talmon200419th, bassal2012hebrew, cotter2015sociolinguistics}. These are mostly targeting academics and language learners. We consulted many of these resources as part of developing our annotation guidelines.

%Furthermore, an increasing amount of attention has been  allotted to the development of resources for DA, which in the past has tended to take the back seat to the benefit of MSA.

%DA Datasets can roughly be divided into two categories: lexicons and corpora. The former constitutes a listing of possibly inflected lemmas, and the latter being an annotated collection of sentences. A corpus can be turned into a lexicon by uniquifying its entries based on the lemma and form fields. Below, we provide an account of a few relevant examples.

\paragraph{Dialectal Corpora}
We can group DA corpora based on the degree of richness in their annotations.
%
%which are either completely free of annotations, or are annotated for some simple features such as dialect id. 
Some noteworthy examples of unannotated or lightly annotated corpora of relevance include the MADAR Corpus \citep{Bouamor:2018:madar}, comprising 2K parallel sentences spread across 25 dialects of Arabic, including PAL (Jerusalem variety) and the NADI corpus for nuanced dialect identification \citep{abdulmageed2021nadi}. The Shami Corpus \citep{abu-kwaik-etal-2018-shami} includes 21K PAL sentences, and the Parallel Arabic Dialect Corpus (PADIC) contains 6.4K PAL sentences \citep{Meftouh:2015:machine}. In the spirit of genre diversification and wider coverage across dialects, \citep{el-haj-2020-habibi}  introduced the Habibi Corpus for song lyrics, which comprises songs from many Arab countries including all Levantine Arab countries.

Public and freely available morphologically annotated corpora are scarce for DA and often do not agree on annotation guidelines. A notable annotated dataset for PAL is the Curras corpus \citep{Jarrar:2016:curras}, a 56K-token morphologically annotated corpus. 
%
Other annotated  Levantine dialect efforts include the Jordan Comprehensive Contemporary Arabic Corpus (JCCA)
\citep{Sawalha:2019:construction}, the Jordanian and Syrian corpora by \citep{alshargi:2019:morphologically}, and the
Baladi corpus of Lebanese Arabic \citep{alhaff-EtAl:2022:LREC}.

We consulted some of the public corpora as part of the development of Maknuune. However, most of the above datasets are based on web scrapes, which limits the amount of actual lemma coverage that they could attain.
%, which is why lexicons are also available. 


\paragraph{Dialectal Lexicons} 
Examples of machine-readable DA lexicons include the 36K-lemma lexicon used for the CALIMA EGY fully inflected morphological analyzer \citep{Habash:2012:morphological}, based on the CALLHOME Egypt lexicon \citep{Gadalla:1997:callhome}, and the 51K-lemma Egyptian Arabic Tharwa lexicon  \citep{Diab:2014:tharwa}, which provides some morphological annotations.

The \textit{Palestinian Colloquial Arabic Vocabulary} comprises 4.5K entries including expressions \citep{younis2021palestinian}, and  the MADAR Lexicon contains 2.7K entries dedicated to the Jerusalem variety of PAL, including lemmas, phonological transcriptions, and glosses in MSA, English and French \citep{Bouamor:2018:madar}.

In addition to the above there are a number of dictionaries for Levantine Arabic variants, e.g., for PAL, 
\citet{barghouti2001palestinian}, \citet{elihai2004olive} (9K entries and 17K phrases), \citet{moin2019etymological}, and \citet{seeger2022dictionary} (more than 30K entries and phrases); for Lebanese Arabic, \citet{freiha:1973:dictionary} (ca. 5K entries), and for Syrian Arabic \citet{stowasser2004dictionary} (15K entries).
These resources include  base lemma forms, occasional plural forms, verb aspect inflections, and expressions; however,
none of them are publicly available in a machine-readable (i.e., tabular or structured) format, to the best of our knowledge.

%The lexicon presented in this work strives to increase coverage of dialectal content, complementing the above resources with entries that may not be easily be found in web-scraped content as will be discussed in the evaluation section (Section~\ref{eval}). More importantly, our morphologically annotated lexicon is computer-readable.
%
The lexicon presented in this work strives to be a large-scale and open resource with rich entries covering  phonology, morphology, and lexical expressions, and with a wide-ranging coverage of PAL sub-dialects. The lexicon may never be complete, but by making it open to sharing and contribution, we hope it will become central and useful to NLP researchers and developers, as well as to linguists working on Arabic and its dialects.

%%%%%%%%%%%%%%%%%%%%%%%%%%%%%%%%%%%%%%%%%%%%%%%%%%%%%%%%%%%%%%%%%%%%%%%%%%%%%%%%%%%%%%%%%%%%%%%%%%%%%%%%%%%%%%%%%%%%%%%%%%%%%%%%%%%%%%%%%%%%%%%%%%%%%%%%%%%%%%%%%%%%%%%%%%%%%%%%%%%%%%%%%%%%%%%%%%%%%%%%%%%%%%%%%%%%%%%%%%%%%%%%%%%%%%%%%%%%%%%%%%%%%%%%%%%%%%%%%%%%%%%%%%%%%%%%%%%%%%%%%%%%%%%%%%%%%%%%%%%%%%%%%%%%%%%%%%%%%%%%%%%%%%%%%%%%%%%%%%%%%%%%%%%%%%%%%%%%%%%%%%%%%%%%%%%%%%%%%%%%%%%%%%%%%%%%%%%%%%%%%%%%%%%%%%%%%%%%%%%%%%%%%%%%%%%%%%%%%%%%%%%%%%%%%%%%%%%%%%%%%%%%%%%%%%%%%%%%%%%%
%\newpage 
\section*{Linguistic Facts}
\addcontentsline{toc}{section}{\protect\numberline{}Linguistic Facts}%
\label{lingfacts}
In this section we present some general linguistic facts about PAL and highlight specific challenging phenomena that motivated many of our annotation decisions.

 
\subsection*{Phonology and Orthography}
Like all other DA, and unlike MSA, PAL has no standard orthography rules \citep{Jarrar:2016:curras,Habash:2018:unified}.  In practice, PAL is primarily written in Arabic script, and to a lesser extent in Arabizi style romanization \citep{Darwish:2014:arabizi}. Some of the variations in the written form reflect the words' phonology, morphology, and/or etymological connections to MSA.  Orthogonal and detrimental to the orthography challenge, PAL has a high degree of variability within it sub-dialects in phonological terms. We highlight some below, noting that some also exist in other DA.

\paragraph{Consonantal Variables} 
A number of PAL consonants vary widely within sub-dialects. 
For example, the  voiceless velar stop \caphi{k} is affricated to the palatal  \caphi{tsh} in many PAL rural varieties \citep{herzallah1990aspects}, e.g., \foreignlanguage{arabic}{كَيف} 
{\it kayf} `how' appears as \caphi{k ee f} (urban) or \caphi{tsh ee f} (rural).\footnote{Arabic orthographic transliteration is presented in the HSB Scheme (italics) \citep{Habash:2007:arabic-transliteration}. Arabic script orthography is presented in the CODA* scheme, and Arabic phonology is presented in the CAPHI scheme (between /../) \citep{Habash:2018:unified}.}
%
Similarly, the MSA voiceless uvular stop \caphi{q} in the word \foreignlanguage{arabic}{قَلْب} 
{\it qal.b}
`heart'  is realized either as glottal stop \caphi{2 a l b} in urban dialects, as a voiceless velar stop \caphi{k a l b} in rural dialects, or a voiced velar stop \caphi{g a l b} in Bedouin dialects \citep{herzallah1990aspects}. 
%
It should be noted that there are some exceptions that do not conform to the above generalizations. For example, in Beit Fajjar,\footnote{A Palestinian town located 8 kilometers south of Bethlehem in the West Bank.} the word \foreignlanguage{arabic}{قَهْوَة} 
{\it qah.wa{\TAMARBUTA}}
`coffee' typically varying elsewhere  as \caphi{\{2,q,g,k\} a h w e} is realized as \caphi{tsh~h~ee~w~a}.
Moreover, some words do not have varying pronunciations such as \foreignlanguage{arabic}{عْقَال} 
{\it {\AYN}.qaAl}
\caphi{3~g~aa~l} `Egal headband'.


%Whereas some researchers claim the glottal stop /2/ developed directly from the voiceless uvular stop itself \citep{levin1994grammar, horesh2000toward} 
%others assumed that /q/ and /2/ and  other variants like /k/, /q./ and /g/ existed side by side, until  the glottal stop ultimately became the
%sole phonetic representation that reflects the phoneme /q/ \citep{cotter2015sociolinguistics}.

\paragraph{Monophthongization} 
Some PAL diphthongs shift to different monophthongs in different locations. 
For example the \caphi{a y} diphthong in \foreignlanguage{arabic}{شَيخ}
{\it {\SHIN}ayx} \caphi{sh~a~y~kh} `Sheikh' shifts often to \caphi{ee} (\caphi{sh ee kh}), but also to \caphi{ii} (\caphi{sh ii kh}).\footnote{In the Palestinian village of Ramadin, near Hebron in the West Bank.}
%
Following the CODA*  guidelines for diacritizing DA \citep{Habash:2018:unified}, we spell the \caphi{oo} and \caphi{ee} sounds using 
\foreignlanguage{arabic}{ىَو}~{\it aw}
and \foreignlanguage{arabic}{ىَي}~{\it ay} 
(without a \textit{sukun} on the \foreignlanguage{arabic}{و} \textit{w} or \foreignlanguage{arabic}{ي} 
\textit{y}), respectively, e.g.,
\foreignlanguage{arabic}{كَوم} \textit{kawm} \caphi{k oo m} `pile' and \foreignlanguage{arabic}{بَيت}
\textit{bayt} \caphi{b ee t} `house'.
%%%%


%Another phonological process that can be observed in the dialect of Ramadin is monophthongization. It is a type of vowel shift and it is defined as a sound change by which a diphthong becomes a monophthong \citep{dressler1984explaining}. The words \foreignlanguage{arabic}{صَيفْ} \caphi{s.~a~y~f}, \foreignlanguage{arabic}{ضَيْف} \caphi{d.~a~y~f} "guest",  and \foreignlanguage{arabic}{شيْخ} "\caphi{sh~a~y~kh} "Sheikh", all change into \caphi{dh. ii f}, \caphi{s. ii f} and \caphi{sh ii kh}  respectively. 

%NEED CITATION FOR One typical feature of Bedouin dialects is commonly known as "Gahawah/Ghawah syndrome". It simply means the insertion of /a/ in a cluster aGC... where (G = gutturals /x, \textsubdot{g}, \textipa{\.h, P, Q}, h/). This can be seen in examples like \foreignlanguage{arabic}{قهوة} \caphi{g~h~a~w~a} `coffee', \foreignlanguage{arabic}{بغلة}  \caphi{b~gh~a.~l.~a} `mule',  \foreignlanguage{arabic}{سخلة} \caphi{s~kh~a.~l.~a} `lamb',  and \foreignlanguage{arabic}{سعوة}  \caphi{s~3~a~w~a} `hen'.

\paragraph{Metathesis} 
In some rural dialects in villages near Tulkarem, Jenin and Ramallah, there are words with consonant pairs within a syllable that appear in a different order than is the norm in PAL, e.g., a word like \foreignlanguage{arabic}{كَهْرَبَا} 
{\it kah.rabaA} \caphi{k a h r a b a} `electricity' realizes as \caphi{k a r h a b a}.

\paragraph{Epenthesis}
PAL exhibits systematic epenthesis of the \caphi{i} or \caphi{u} sounds producing paired word alternations
such as \caphi{b a 3 d} and \caphi{b a 3 i d} for \foreignlanguage{arabic}{بعد} `still;after'
or
\caphi{kh u b z} and \caphi{kh u b u z} or \caphi{kh~u~b~i~z} (in different sub-dialects) for \foreignlanguage{arabic}{خبز} `bread'.
We opted to use the fully epenthesized forms in the lexicon, i.e., 
\foreignlanguage{arabic}{بَعِد}
\textit{ba{\AYN}id},
\foreignlanguage{arabic}{خُبُز}
\textit{xubuz},
and
\foreignlanguage{arabic}{خُبِز}
\textit{xubiz}, for the above mentioned examples.




\subsection*{Morphology}

Like other DA, PAL has a complex morphology employing templatic and concatenative morphemes, and including a  rich set of morphological features: gender, number, person, state, aspect, in addition to numerous clitics.  We highlight some specific morphological phenomena that we needed to handle.

\paragraph{Ta Marbuta}
The so-called feminine singular suffix morpheme, or Ta Marbuta (\foreignlanguage{arabic}{ة} \TAMARBUTA), is a morpheme that can be used to mark feminine singular nominals, but that also appears with masculine singular and plural nominals.
Morphophonemically, it has a number of forms in PAL that vary contextually. 
%
First, in some PAL sub-dialects, the Ta Marbuta is pronounced as \caphi{a} when preceded by an emphatic consonant,  velars, and pharyngeal fricatives, e.g., 
\foreignlanguage{arabic}{بَطَّة}
{\it baT{\SHADDA}a{\TAMARBUTA}}
\caphi{b a t. t. a}
`duck'; otherwise it realizes as \caphi{e}, e.g., \foreignlanguage{arabic}{بِسِّة}
{\it bis{\SHADDA}i{\TAMARBUTA}}
\caphi{b i s s e}. 
In some northern PAL dialects, the \caphi{e} variant appears as \caphi{i}; and in some southern PAL dialects, the distinction is gone and all Ta Marbutas are pronounced \caphi{a}.
%
Second, the Ta Marbuta turns into its allomorph \caphi{i t} in {\it Idafa} constructions, e.g., \caphi{b i s s i t} `the/a cat of'. 
Finally, for some active participle deverbal nouns, the Ta Marbuta realizes as \caphi{aa} or \caphi{ii t} when followed by a pronominal object clitic, e.g., \foreignlanguage{arabic}{كَاتْبَاه}
{\it kaAt.baAh} \caphi{k aa t b aa (h)} or \foreignlanguage{arabic}{كَاتْبِيْتُه}
{\it kaAt.biy.tuh} or \caphi{k~a~t~b~ii~t~u~(h)} `she wrote it'.



%One of the most prominent phenomena regarding allomorphy in Arabic (both MSA and DA) is the realization of the feminine singular suffix morpheme (Ta Marbuta), which is
%pronounced in MSA as /2atan/ except at utterance final positions (where it is pronounced as /a/ ). In
%most PAL dialects and sub dialects, the feminine suffix is pronounced as follows:

%[a] when the feminine singular suffix morpheme (Ta Marbuta) is preceded by an emphatic consonant, i.e. uvularized coronals \caphi{s., d., t., dh.}, velars \caphi{gh, kh,q} and pharyngeal Fricatives \caphi{7, 3}. For example, \foreignlanguage{arabic}{بطَّة} "duck" \caphi{b a t. t. a}, \foreignlanguage{arabic}{بلغة} "one item of slippers" \caphi{b a l gh a}, \foreignlanguage{arabic}{طلعة} "upill or going out for a picnic or shopping" \caphi{t. a l 3 a}\footnote{Some speakers in the North of Palestine pronounce the feminine singular suffix morpheme that is preceded by the Alveolar ejective fricative [s.] as [e]}.\\
%[e] elsewhere.

%On the other hand, those who live in the south of Palestine, in areas such as, Hebron and Bethlehem, %pronounce the feminine singular suffix morpheme as [a] e.g. \foreignlanguage{arabic}{بسَّة} "cat"  
%\caphi{b i s s a}, \foreignlanguage{arabic}{معلمة} "teacher"
%\caphi{m 3 a l m a}, and \foreignlanguage{arabic}{بتَّة} "single item"
%\caphi{b a t t a}.


\paragraph{Complex Plural Forms}
Besides the common use of broken plural (templatic plural) in DA, we encountered cases of {\it blocked} plurals where a typical sound plural or templatic plural is not generated because another word form is used in its place \citep{aronoff1976word}. One example from Ramadin, is the plural form  of 
 the word
\foreignlanguage{arabic}{عَيِّل}
{\it {\AYN}ay{\SHADDA}il} 
\caphi{3~a~y~y~i~l} `child [lit. dependent]', which is blocked by the word form \foreignlanguage{arabic}{ضْعُوف} 
{\it D.{\AYN}uwf} 
\caphi{dh.~3~uu~f} `children [lit. weaklings]'.
%imilar plural words that do not have a singular form were widely used among PAL speakers from the different areas of the West Bank. The table below clearly demonstrates some of the examples used in PAL. 



\subsection*{Syntax}

Previous research on Arabic dialects reveals that the syntactic differences between these dialects
are considered to be minor compared to the morphological ones \citep{Brustad:2000:syntax}. 
%
%In line with previous findings, single negation with the negative particle \foreignlanguage{arabic}{ما} "not" coupled with or without negation enclitic \foreignlanguage{arabic}{ش} can be found in PAL . For example, \foreignlanguage{arabic}{ما أكلت} and \foreignlanguage{arabic}{ما أكلتش} "I did not eat."
%
One particular challenging phenomenon we encountered is a class of nouns used in adjectival constructions, but violating noun-adjective agreement rules, which involve gender, number and rationality \citep{Alkuhlani:2011:corpus}.  For instance, the word \foreignlanguage{arabic}{خِيخَة}
{\it xiyxa{\TAMARBUTA}} \caphi{kh~ii~kh~a} `weak/lame' does not typically agree with the nouns it modifies unlike a normal adjective such \foreignlanguage{arabic}{كْبِير}
{\it k.biyr} \caphi{k b ii r} `old [human]/large [nonhuman]'.  
So, the words
\foreignlanguage{arabic}{سِيَّارَة}
{\it siy{\SHADDA}aAra{\TAMARBUTA}} `car [f.s.]', 
\foreignlanguage{arabic}{عُرُس} {\it {\AYN}urus} `wedding [m.s.]', 
and \foreignlanguage{arabic}{نَاس} {\it naAs} `people [m.p]' can all be modified by \foreignlanguage{arabic}{خِيخَة}
{\it xiyxa{\TAMARBUTA}}; however, they need three different forms of \foreignlanguage{arabic}{كْبِير}
{\it k.biyr}: 
\foreignlanguage{arabic}{كْبِيرِة}
{\it k.biyri{\TAMARBUTA}},
\foreignlanguage{arabic}{كْبِير}
{\it k.biyr}, and
\foreignlanguage{arabic}{كْبَار}
{\it k.baAr}, respectively.
%
We mark the POS of such nominals as ADJ/NOUN in our lexicon, as it is a class that deserves further study.


%\citet{harley2011compounding} notes that a compound is a word-sized unit that is composed of two or more Roots. The meaning of a compound is usually compositional, i.e., predictable and the parts contribute to the whole. For example, the compound “popcorn” is a kind of corn which pops \citep{fabb2017compounding}. On the other hand, they can be non-compositional. For example, the meaning of the compound “watershed” has noting to do with the meanings of “water” and “shed” in isolation. Maknuune lexicon is very rich with both compositional and non-compositional compounds (CC and NC respectively). Examples for CC's include  \foreignlanguage{arabic}{جواز سفر} "passport" \caphi{J a w aa z \# s a f a r}, \foreignlanguage{arabic}{فقر دم}  "anemia" \caphi{f a q i r \# d a m m}. As for NC's, the word \foreignlanguage{arabic}{بيت} combines with many words to create new meanings. The table below summarizes some of the compounds found in Maknuune lexicon.

%
%\citet{borer2013structuring} maintains that categorial exocentricity simply means that the compound is not %a sub-kind of its head and therefore its overall category may differ from those of its constituents. As it %can be shown in the Arabic examples below in A, the resulting two NC's whose both of their categories are %nouns were made of two imperative verbs that combined together \foreignlanguage{arabic}{عص مص} a 
%and  \foreignlanguage{arabic}{قرمز ونقِّي}.
%
%%\ag
%3 u s. s. \# 3 m u s. s.\\
%Squeeze (Imp.V.2MS)   lick(Imp. V.2MS)\\
%%\glt
%'a type of ice-cream'.\\
%%\bg
%g a r m i z \# w u n a g g i\\
%squat(Imp. V.2MS)    and.choose (Imp. V.2MS)\\
%%\glt
%'second-hand clothing market'. 
%
%It must be noted that exocentric compounds can never be  found in Modern Standard Arabic, and rarely found in Dialectal Arabic as in the examples above. One might notice that the examples tend to be fixed and they never undergo pluralization at all.

%\subsection{Semantics, Pragmatics, and Collocations}
\subsection*{Figures of Speech and Multiword Expressions}

PAL has a rich culture of figures of speech and multiword expressions (compounds, collocations, etc.) that has not been well documented. We highlight some phenomena that we  cover in Maknuune.

\paragraph{Collocations}
As part of working on Maknuune, we encountered numeorus collocations (words that tend to co-occur with certain words more often than they do  with others). For example, the verbs used for trimming off the tough ends of some vegetables vary based on the vegetable:
%\foreignlanguage{arabic}{يقمِّع} 
%\caphi{y~Q~a~m~m~i~3} `trim okra', 
%\foreignlanguage{arabic}{يقرِّم}
%\caphi{y~q~a~r~r~i~m} `trim green beans',
%\foreignlanguage{arabic}{يعكِّب}  
%\caphi{y~3~a~k~k~i~b} 'dethorn artichoke', 
%and \foreignlanguage{arabic}{يطَرْطِف} 
%\caphi{y~t.~a~r~t.~i~f} 'cut the blossom ends of the maize stalks'.
\foreignlanguage{arabic}{يْقَمِّع بَامْيِا} 
\caphi{y~Q~a~m~m~i~3 \# b~aa~m~y~e} `trim off the tough ends of okra', \foreignlanguage{arabic}{يْقَرِّم فَاصَولْيَا}
\caphi{y~q~a~r~r~i~m \# f~aa~s.~uu~l~y~a} `trim off the tough ends of green beans', \foreignlanguage{arabic}{يْعَكِّب عَكُّوب}
\caphi{y~3~a~k~k~i~b \# 3~a~k~k~uu~b} `remove the thorns from artichoke (Gundelia)', and  \foreignlanguage{arabic}{يْطَرْطِف ذُرَة} 
\caphi{y~t.~a~r~t.~i~f \# D~u~r~a} `cut the blossom ends of the maize stalks'.


\paragraph{Compounds}
We encountered many compositional and non-compositional compounds. Examples include  \foreignlanguage{arabic}{جَوَاز سَفَر} 
{\it jawaAz safar}
\caphi{J a w aa z \# s~a~f~a~r} `[lit. permission-of-travel, passport]', which is also used in MSA. Some words appear in many compounds with a wide range of meaning, e.g.,
%\foreignlanguage{arabic}{فقر دم}  "anemia" \caphi{f a q i r \# d a m m}. As for NC's,
the word \foreignlanguage{arabic}{بَيت} {\it bayt} `[lit. house]' appears in compounds referring to celebrations, funerals, bathrooms, and whether or not a family has children (see the examples in  Table~\ref{tab:phrases}).

\paragraph{Synecdoches}
It has  been widely observed that PAL speakers use synecdoches\footnote{A figure of speech in which a term for a part of something is used to refer to the whole, or vice versa.} in their dialects \citep{seto1999distinguishing}.
%Synecdoche, which is  \footnote{\citep{lakoff2008metaphors} also include synecdoche within the term metonymy.}. 
Examples include the use of \foreignlanguage{arabic}{كَوم لَحِم} \caphi{k oo m \# l a 7 i m} `[lit. a pile of meat]', and \foreignlanguage{arabic}{كَبَابِيش}  \caphi{k~a~b~aa~b~ii sh} `[lit. plural of hair]' to mean `children'.

%On the other hand, the terms \foreignlanguage{arabic}{ضلع إعوج}  "lit:crooked rib" \caphi{D. i l i 3 \# 2 i 3 w a J},  \foreignlanguage{arabic}{أربع وعشرين ضلع}  "24 ribs" \caphi{2 a r b a 3 a \# w u 3 i sh r ii n \# D. i l i 3}  and \foreignlanguage{arabic}{ضلع قاصر} "lit:a juvenile rib" \caphi{D. i l i 3 \# Q aa s. i r} all mean "woman".

\paragraph{Euphemisms}
PAL speakers use many euphemistic expressions. For example, in some villages 
in Nablus, the expression \foreignlanguage{arabic}{يَوم تْهَنَّى} 
\caphi{y~oo~m~\# t h a n n a} `[lit. the day he felt happy]' to mean `the day he passed away'. 
In other areas in the West Bank, 
the phrase \foreignlanguage{arabic}{عَينُه كَرِيمِة} 
\caphi{3~ee~n~o~\# k~a~r~ii~m~e}
`[lit. his eye is generous]'
to mean `one-eyed'; and the phrase
\foreignlanguage{arabic}{بَيت خَالْتِي}
\caphi{b~ee~t~\# kh~aa~l~t~i}
`[lit. my aunt's house]' means 'prison'.
%As for the people in Hebreon, some of them say \foreignlanguage{arabic}{يطيِّر مي} y t. a y y i r \# m a. y y "lit: spray water, fig: go to the bathroom".

%We specifically targeted collecting many of these kinds of constructions and included them in Maknuune.
%%%%%%%%%%%%%%%%%%%%%%%%%%%%%%%%%%%%%%%%%%%%%%%%%%%%%%%%%%%%%%%%%%%%%%%%%%%%%%%%%%%%%%%%%%%%%%%%%%%%%%%%%%%%%%%%%%%%%%%%%%%%%%%%%%%%%%%%%%%%%%%%%%%%%%%%%%%%%%%%%%%%%%%%%%%%%%%%%%%%%%%%%%%%%%%%%%%%%%%%%%%%%%%%%%%%%%%%%%%%%%%%%%%%%%%%%%%%%%%%%%%%%%%%%%%%%%%%%%%%%%%%%%%%%%%%%%%%%%%%%%%%%%%%%%%%%%%%%%%%%%%%%%%%%%%%%%%%%%%%%%%%%%%%%%%%%%%%%%%%%%%%%%%%%%%%%%%%%%%%%%%%%%%%%%%%%%%%%%%%%%%%%%%%%%%%%%%%%%%%%%%%%%%%%%%%%%%%%%%%%%%%%%%%%%%%%%%%%%%%%%%%%%%%%%%%%%%%%%%%%%%%%%%%%%%%%%%%%%%%

\section*{Methodology}
\addcontentsline{toc}{section}{\protect\numberline{}Methodology}%
\label{method}
In this section, we discuss the methodology  we adopted in data collection for Maknuune, as well as the guidelines we followed for creating the lexicon entries.

\subsection*{Data Sources}
The current work spans over five years of effort, and a large number of volunteering informants, linguistics students, and citizen linguists (over 130 people).
%The first author and last four authors...
The data was collected from many different sources.

First are \textbf{interviews} with (mostly but not entirely) elderly people  who live in rural areas such as villages and towns or in refugee camps in the West Bank.
%
The researchers went to the field and met with several people. 
They attended several social gatherings and participated in different events, e.g. weddings, funerals, field harvests, traditional cooking sessions, sewing, etc. They asked the language users several questions pertaining to the following themes: weddings, funerals, occupations, illnesses, cooking traditional dishes, plants, animals, myths, games, weather terms, tools and utensils, etc. They were particularly interested in documenting terms and expressions that are used mainly by the old generation. 

Secondly, to achieve the needed balance in the lexicon, the researchers consulted an in-house \textbf{balanced corpus}, that contains $\sim$40,000 words. The corpus comprises data that was transcribed from several recorded conversations that revolve around the same themes as above, written chats and texts, and some internet material (both written and spoken). Common words including verbs, adjectives, adverbs, and function words (e.g., prepositions, conjunctions, particles) were taken from the balanced corpus. At a later stage in the development of Maknuune, we consulted with the Curras Corpus \citep{Jarrar:2016:curras} to identify additional missing lemmas, with limited yield. We compare to Curras in terms of coverage in  Section~\ref{eval}. All of the above was also supplemented by methodical rounds of well-formedness checking to improve consistency across all fields, i.e., diacritization, transcription, root validity, etc.

Finally, in addition to the previous two methods, the researchers employed their \textbf{linguistic intuition} skills, knowledge of Palestinian Arabic (as native speakers) and the knowledge of the language users to provide additional word classes and multiword expressions that are associated with the existing lemmas.
%Note that only words that were considered by the researchers to be a representative sample of PAL (as a whole, i.e., all of the sub-dialects) were used in the lexicon, and this includes MSA lemmas (or pronunciations or meanings thereof) that would possibly not qualify as representative in other varieties of Arabic or even in some PAL sub-dialects.

It should be noted that whether an MSA lemma cognate of a PAL lemma (with similar or exact pronunciation, or meaning) exists was not considered a factor in including the PAL lemma in the lexicon.  We focused on creating a  representative sample of PAL including all its sub-dialects.


\begin{table*}[t!]
    \centering
\includegraphics[width=\linewidth]{examples.pdf}
    \caption{Eight entries from {\maknuune} that share the same root, and are paired with four distinct lemmas.}
    \label{tab:tf7}
\end{table*}

\subsection*{Lexical Entries}


Each entry in the Maknuune lexicon consists of six required and three optional fields.
The six required fields are the \textbf{Root}, \textbf{Lemma}, \textbf{Form}, \textbf{Transcription},  \textbf{POS \& Features}, and \textbf{English Gloss}. The optional fields are the \textbf{MSA Gloss}, \textbf{Example} and \textbf{Notes}.
Figure~\ref{tab:tf7} presents an example of a number of entries coming from the same root.

%\subsection{Manual Annotation}
%\subsubsection{Root}

%The root is an abstraction of all derivations. Arabic morphologists classified roots based on the number of their radicals into triliteral (three radicals), quadriliteral (four radicals) and quintiliteral (five radicals) roots. Templatic morphemes that are equally needed to create a word templatic stem come in three types: roots, patterns and vocalisms. In terms of the root morpheme, it is a sequence of three, four, or very rarely five consonants that come in a fixed order. \\

%1a2a3 + k.t.b = katab\\
%1aa2i3+k.t.b=kaatib\\
%1a22a3 + k.t.b = kattab = kat~ab\\
%ista12a3 +k.t.b = istaktab\\
%1u22aa3+k.t.b=kuttab\\
%ma12a3+k.t.b =maktab\\
%ma12a3a+k.t.b =maktaba\\

%The root signifies some abstract meaning or notion that is shared by all the derivations. For example, The root \foreignlanguage{arabic}{ك.ت.ب}has many words associated with it and that share similar meanings to the root
%\foreignlanguage{arabic}{كَتَب} "write.3rd.Masc.SG) k a t a b, \foreignlanguage{arabic}{كَتَّب} "make sb write.causative.3rd.Masc.SG) k a t t a b, \foreignlanguage{arabic}{كُتُب} "books" k u t u b, \foreignlanguage{arabic}{مكتب} "office" m a k t a b, \foreignlanguage{arabic}{مكتبة} "library" m a k t a b e, \foreignlanguage{arabic}{كُتَّاب} "an old school where kids in the past used to go to in order to learn reading, writing and reciting Qura'an", \foreignlanguage{arabic}{كاتب} "write to one another or carry on a correspondence"

\subsubsection*{Root, Lemma, and Form}
The \textbf{Root}, \textbf{Lemma} and \textbf{Form} represent three degrees of morphological abstraction.
The \textbf{root} in Arabic in general is a templatic morpheme that interdigitates with a pattern or template to form a word stem that can then be inflected further. Roots are very abstract representations that broadly define the morphological family a word belongs to at the derivational and inflectional level. 
%
\textbf{Lemmas} on the other hand are abstractions of the inflectional space that is limited by variations in the morphological features of person, gender, number, aspect, etc. Lemmas are the central entries of the lexicon. 
\textbf{Forms} are base words (i.e., without clitics) that are inflected in a specific way. 
We follow the same general guidelines of determining lemmas as used in large Arabic morphological analyzers \citep{Graff:2009:standard,Habash:2012:morphological,Khalifa:2017:morphological}. There are of course some constructions that have grammaticalized into new lemmas, e.g., 
\foreignlanguage{arabic}{عَشَان} 
{\it {\AYN}a{\SHIN}aAn} can be treated as the noun  
\foreignlanguage{arabic}{شَان}
{\it {\SHIN}aAn} `situation;status' with a proclitic, or the subordinating conjunction meaning `because'.

For nouns and adjectives, we provide the lemma in the masculine singular form, unless it is a feminine form that does not vary in gender, in which case it is provided in the feminine singular. Very infrequently, some nouns only appear in plural form, which become their lemma, e.g. \foreignlanguage{arabic}{أَوَاعِي} {\it {\AHAMZAUP}awaA{\AYN}iy} \caphi{2~a~w~aa~3~i} `clothes'.  We do not list the sound plural and sound feminine inflections of nouns and adjectives. However, broken plurals and templatic feminine forms are provided and linked through the same lemma as the singular form.

For verbs, we provide the lemmas in the third masculine singular perfective form as is normally done in Arabic lexicography. We provide three forms linked to the lemma: the third masculine singular perfective, the third masculine singular imperfective, and the second person masculine imperative (command) forms.  These are provided for completeness to identify the basic verbal inflectional paradigm (albeit, not completely).

These three representations are provided in Arabic script.
Since PAL does not have an official standard orthography, we intentionally decided to follow the Conventional Orthography for Dialectal Arabic (CODA*) \citep{Habash:2018:unified}. In addition to being used in developing Curras \citep{Jarrar:2016:curras}, CODA* has been adopted  by a website for teaching PAL to non-native speakers.\footnote{\url{https://www.palestinianarabic.com/}}
%\todo{if we have space... refer to Figure 1}

\begin{table}[t!]
    \centering
    \includegraphics[width=0.6\linewidth]{caphi_table-v2.pdf}
    \caption{The CAPHI++ symbols set and its expanded CAPHI symbols, with examples.}
    \label{tab:caphiplus}
\end{table}

\subsubsection*{Transcription with CAPHI++}
One of CODA*'s limitations is that it abstracts over some of the phonological variations. As such, we follow the suggestions by \citep{Habash:2018:unified} to use a phonological representation, CAPHI, to indicate the specific phonology of the entries.  CAPHI, which stands for Camel Phonetic Inventory is inspired by the  International Phonetic Alphabet (IPA)  and Arpabet \citep{Shoup:1980:phonological}, and is designed to only use characters directly accessible on the common keyboard to ease the job of annotators.

Owing to the phonological variations that are found in PAL, we extended CAPHI's symbol set with \textit{cover phonemes} that represent a number of possible interchangeable phones.  We call our extended set CAPHI++.  Table~\ref{tab:caphiplus} presents the new 9 symbols we introduced. All of these symbols are to be presented in upper case, while normal CAPHI symbols are in lower case. The new CAPHI++ symbols represent specific sets of mostly two variants in common use in different PAL sub-dialects.
For example, instead of including four entries for the word \foreignlanguage{arabic}{قَلَم} {\it qalam} 
(\caphi{q~a~l~a~m}, \caphi{k~a~l~a~m},  \caphi{2~a~l~a~m}, 
\caphi{g~a~l~a~m}),
we only provide one form (\caphi{Q~a~l~a~m}).
Exceptional usages that do not conform to the specific generalizations of the CAPHI++ cover symbols are listed independently, e.g., a second entry for the above example is provided for the Beit Fajjar pronunciation of \caphi{tsh~a~l~a~m}.

We acknowledge that the transcriptions provided may not represent the full breadth of PAL sub-dialects.  We make our resource open so that additional forms and variants can be added in the future, as needed.

\subsubsection*{Phonological Transcription in this Book}
While CAPHI++ is used in the introduction of this book and the development of the lexicon on the Google Sheets interface for a smoother annotator experience, we use IPA in this book to represent the phonological transcriptions. To accommodate the CAPHI++ extensions, we introduce parallel IPA++ additions (see Table \ref{tab:caphiplus}).

%However, it is the opinion of the lexicographers working on Maknuune that most of the time, the different pronunciations do not conflict with the CODA form (and to a lesser extent diacritization) which is rather robust to PAL sub-dialect phonological variation.

%The word \foreignlanguage{arabic}{قلم} "pen" Q a l a m can be pronounced as q a l a m, k a l a m, 2 a l a m, g a l a m, \footnote{the word \caphi{tsh a l a m}, which means pen, is used in Bayt Fajar} \caphi{tsh a l a m}. The table below clearly illustrates the symbols employed in CAPHI++.
%It should be noted that there are some exceptions that do not conform to the the generalizations captured in the new symbols syggested in CAPHI++. For example, in Beit Fajjar,  a Palestinian town located 8 kilometers south of Bethlehem in the West Bank, pronounce the word \foreignlanguage{arabic}{قهوة} "coffee" Q a h w e as tsh h ee w a
%Moreover, some words have only one or two pronunciations; such as, \foreignlanguage{arabic}{عقال} "Agal" 3 g aa l, \foreignlanguage{arabic}{نيقة} "fussy" n ii 2 a , and \foreignlanguage{arabic}{قندرة} "shoe" qIIk u n d a r a. It is worth mentioning that the symbol II was used to give two or three possible pronunciations of the same word as indicated in the example \foreignlanguage{arabic}{قندرة} above.
%Certain words that have the same meaning but were spelled differently were written in separate lexical entries with different roots; such as, \foreignlanguage{arabic}{أنطى} "give" 2 a n t. a and \foreignlanguage{arabic}{أعطى}  "give" 2 a 3 t. a, \foreignlanguage{arabic}{نيرة} "dinar" n ee r a and \foreignlanguage{arabic}{ليرة} "dinar" l ee r a, and \foreignlanguage{arabic}{فنجال} "cup" f i n J aa l and \foreignlanguage{arabic}{فنجان} "cup" f i n J aa n.

\subsubsection*{POS and Features}

The analysis cell in every entry indicates the POS and features of the word form. 
We use 35 POS tags based on a combination of previously used POS tagsets in Arabic NLP \citep{Graff:2009:standard,Pasha:2014:madamira,Khalifa:2018:morphologically}.  Our closest relative is the tagset used by \citep{Khalifa:2018:morphologically} for work on Emirtai Arabic annotation. See the full list of POS tags in Table~\ref{tab:pos} in Appendix~\ref{pos-mapping}. %\todo{@shahd table needs cleaning; check comparison with Khalifa's Camel POS}  
However, we  extend their POS list with three tags: ADJ/NOUN (for adjectives with exceptional agreement), NOUN\_ACT (active participle deverbal noun), and NOUN\_PASS (passive participle deverbal noun).

For features, we use MS (masculine singular), FS (feminine singular), and P (plural) for nominals, % \todo{or NOUN and ADJ only ? what about NOUN\_ACT/PASS others..}
and P (perfective), I (imperfective) and C (command) for third masculine singular verb forms only.


%The annotators provided all the possible word forms that are associated with the same root. The table below shows that the root \foreignlanguage{arabic}{ح}.\foreignlanguage{arabic}{ف}.\foreignlanguage{arabic}{ت}
 
%has  several  lexical entries ; such as, unit noun, collective noun, verb and phrase. 

%In Figure~\ref{fig:tf7}, we see...
%Figure~\ref{fig:tf7}(a) is an example of...

%It must be noted that the annotators provided the readers with the irregular feminine and broken plurals of the nouns and adjectives. The table below shows some examples.

%\begin{figure*}[t!]
%    \centering
%\includegraphics[width=0.99\linewidth]{Irregular Fem and %Plurals.jpg}
%    \caption{Irregular Femminine and Broken Plurals}
%    \label{fig:tf7}
%\end{figure*}

\subsubsection*{Phrases} 
In addition to basic word forms, we overload the use of the form cells to list phrases (multiword expressions, collocations, and figures of speech) that are paired with the lemma. In such cases, the POS:Features cell is given the POS of the lemma, with the extension \textbf{PHRASE}, e.g., line (d) in Table~\ref{tab:tf7}, and 
%. {\maknuune} contains a large number of phrasal entries. For some additional examples associated with a single lemma, see 
Table~\ref{tab:phrases}.

\begin{table*}[th!]
    \centering
\includegraphics[width=\linewidth]{phrases.pdf}
    \caption{Examples of NC compounds in Maknuune for the lemma \foreignlanguage{arabic}{بَيت} `house'.}
    \label{tab:phrases}
\end{table*}
 



\subsubsection*{Glosses, Examples and Notes}
We provided the English gloss equivalents of all the PAL words. The MSA gloss was provided for about a third of the entries at the time of writing. 
In cases where no single word in MSA or English can encode a culturally specific concept, the annotators translated the whole situation/concept. 
For example, in Ramadin, there are two words for `baby camel' depending on its age: \foreignlanguage{arabic}{ذَلُول} 
{\it {\DHA}aluwl} \caphi{dh~a~l~uu~l}, `barely a few days old'  and
\foreignlanguage{arabic}{حْوَيِّر}
{\it H.way{\SHADDA}ir} \caphi{7~w~a~y~y~i~r} `around 14-15 months old'. 
Another complex example is the word \foreignlanguage{arabic}{تَلْجِيم} {\it tal.jiym} \caphi{t a l J ii m} `[lit. harnessing or bridling]' which can refer also to `reciting some verses from the Quran (Surat Al-Takweer, Ayat Al-Kursi or Surat Al-Hashr) on a razor or a thread and closing the razor or tying the thread and leaving them aside until a lost or missing riding animal has returned home.' 
%the word \foreignlanguage{arabic}{يبنِّق} "loosen the garment by sewing extra fabric to its sides" \caphi{y b a n n i q}, and the word \foreignlanguage{arabic}{يتبعَّر} "pick olives after the main harvest"  \caphi{y i t b a 3 3 a r}. 

Finally, we provide usage examples for some entries, as well as grammatical or collection notes.  Notes vary in type from {\it Collective Noun} and {\it Collected near Nablus}, to {\it Vulgar}.

%The simple words had equivalents in both MSA and English as can be seen in the table below.

%\begin{figure*}[t!]
%    \centering
%\includegraphics[width=0.99\linewidth]{Glosses.jpg}
%    \caption{Glosses}
%    \label{fig:tf7}
%\end{figure*}




%%%%%%%%%%%%%%%%%%%%%%%%%%%%%%%%%%%%%%%%%%%%%%%%%%%%%%%%%%%%%%%%%%%%%%%%%%%%%%%%%%%%%%%%%%%%%%%%%%%%%%%%%%%%%%%%%%%%%%%%%%%%%%%%%%%%%%%%%%%%%%%%%%%%%%%%%%%%%%%%%%%%%%%%%%%%%%%%%%%%%%%%%%%%%%%%%%%%%%%%%%%%%%%%%%%%%%%%%%%%%%%%%%%%%%%%%%%%%%%%%%%%%%%%%%%%%%%%%%%%%%%%%%%%%%%%%%%%%%%%%%%%%%%%%%%%%%%%%%%%%%%%%%%%%%%%%%%%%%%%%%%%%%%%%%%%%%%%%%%%%%%%%%%%%%%%%%%%%%%%%%%%%%%%%%%%%%%%%%%%%%%%%%%%%%%%%%%%%%%%%%%%%%%%%%%%%%%%%%%%%%%%%%%%%%%%%%%%%%%%%%%%%%%%%%%%%%%%%%%%%%%%%%%%%%%%%%%%%%%%


\section*{Coverage Evaluation}
\addcontentsline{toc}{section}{\protect\numberline{}Coverage Evaluation}%
\label{eval}
We approximate the coverage of our lexicon by comparing it with the {\curras} corpus \citep{Jarrar:2016:curras}, the largest resource available for PAL.\footnote{\citet{alhaff-EtAl:2022:LREC} describe a revised version of that corpus, but it was not made available at the time of writing.} Since \curras is a corpus and our resource is a lexicon, the analysis is carried out in such a way to account for that difference. 
%
We  present next some high-level corpus statistics and then a detailed comparison between \maknuune and \curras.
%
Then, we  provide some comparison between \maknuune and the lexicons of two morphological analyzers for MSA and EGY.

\begin{table}[t]
    \centering
    \includegraphics[width=0.6\linewidth]{maknuune_stats.pdf}
    \caption{POS type and entry  statistics in \maknuune.}
    \label{fig:stats-maknuune}
\end{table}

\begin{table}[t]
    \centering
\includegraphics[width=0.6\linewidth]{maknuune_curras_comparison-v3.pdf}
    \caption{Side-by-side view of the statistics of both \maknuune and the lexicon extracted from \curras.}
    \label{fig:stats-comp}
\end{table}

\subsection*{Maknuune \& Curras Statistics}
\paragraph{Maknuune POS Types}
Table \ref{fig:stats-maknuune} shows some basic statistics about \maknuune, dividing entries across four basic POS types (see Table~\ref{tab:pos}).
 %
\maknuune has about three times more verb entries than verb lemmas, reflecting the fact that almost each verb appears in all three aspects (perfective, imperfective, and command) in third person masculine singular form. Similarly for nominals (nouns, adjectives, etc.), the ratio of 1.2 forms per lemma reflects the inclusion of plural entries for many nominals. % which can take a plural form.
Phrasal entries account for 10\% of all Maknuune entries, and close to three quarters of them are associated with nominals (63\% of all lemmas). 

\paragraph{The Curras Lexicon}

In order to compare \maknuune with \curras, we  extract a lexicon, henceforth Curras Lexicon, out of the Curras corpus by uniquing its entries based on lemma, inflected form, POS, and grammatical features (for \curras, aspect, person, gender, and number). 
%This way, we obtain the \curras lexicon, the numbers of which are contrasted against those of 
We compare the Curras Lexicon to \maknuune in Table~\ref{fig:stats-comp}.

Firstly, Curras does not include  roots; and although it is a corpus, it does not identify phrases in the way Maknuune does. As such, we do not compare them in those terms in Table~\ref{fig:stats-comp}.
%and technically, since \curras is a corpus, then quoting the number of phrases in it is not really useful, which explains why these numbers are missing from Table~\ref{fig:stats-comp}. 
%Each phrase is represented by one or more lemmas which are annotated in-context for POS, explaining the difference between the total number phrase entries and the number of unique phrase entries.

Secondly, by virtue of being a lexicon, \maknuune possesses more unique lemmas, weighing in at 17,369 lemmas taking POS into account (lemma:POS), while the total number of inflected forms is at 32,759, both of which are about 50\% more than in the Curras Lexicon. This clearly showcases \maknuune's richness in terms that go beyond the day-to-day language that one sees frequently in corpora like \curras. In contrast, \curras being a corpus, its extracted lexicon showcases a greater inflectional coverage with 224 unique word analyses as opposed to 76 for \maknuune. 

%Furthermore, the difference between the unique number of lemma:POS:features 3-tuples and unique number of inflected forms reflects the inflectional and derivational syncretism in PAL.
Finally, as inferable from the difference between the number of unique lemmas and lemma:POS, 548 lemmas are associated to more than one POS in \maknuune.

\subsection*{Corpus Coverage Analysis}
\label{corpus-coverage-analysis}
In the interest of estimating how well our lexicon would fare with real-world data, we perform an analysis between the \curras and \maknuune lemmas, to see how many of the \curras lemmas \maknuune actually covers. From an initial investigation, we note that there are numerous minor differences that need to be normalized to ensure a more meaningful evaluation.
As such, we first pre-process all lemmas (in both lexicons) by stripping the \foreignlanguage{arabic}{سكون} {\it sukun} diacritic, stripping all the \foreignlanguage{arabic}{فتحة} diacritics that appear before a \foreignlanguage{arabic}{ا}~\textit{A},
%all diacritics at the end of the lemmas,
converting the \foreignlanguage{arabic}{همزة وصل} \foreignlanguage{arabic}{ٱ}~\textit{Ä} to \foreignlanguage{arabic}{ا}~\textit{A}, and stripping the \foreignlanguage{arabic}{كسرة} (\textit{i}) and \foreignlanguage{arabic}{فتحة} (\textit{a}) diacritics if they appear before \foreignlanguage{arabic}{ة}~\textit{\TAMARBUTA}. We then compare all the annotated lemma:POSType
%\footnote{Occurence of a lemma which has a specific POS type (see mapping available in Appendix \ref{}.} 
in \curras (56,004 tokens and 8,315 normalized types) to the lemmas in Maknuune.

We exclude 12,673 (23\%) of the tokens pertaining to punctuation, digits and proper noun POS, none of which were especially targeted by \maknuune. Of the remaining 43,331 entries, 49\% have exact match in \maknuune. We sample 10\% of the unique entries with no exact match (433 types and 1,965 tokens), and manually annotate them for their mismatch class.  We found that 74\% of all the sampled types (80\% in tokens) are actually present in \maknuune, but with slight differences in orthography mainly in the presence or absence of diacritics but also some spelling conventions. For about 20\% of sampled types (17\% in tokens), the lemma type is not one that we targeted such as foreign words and proper nouns that are differently labeled in \curras, or MSA words. Finally, 6\% of sampled types (3\% in tokens) are entries that are admittedly missing in \maknuune and can be added.

This suggests that we have very good coverage although the annotation errors and differences make it less obvious to see. A simple projected estimate assuming that our 10\% sample is representative would suggest that \maknuune's coverage of \curras' lexical terms (other than proper nouns and punctuation) is close to 94\% (97\% in token space); however a full detailed classification would be needed to confirm this projection. 

\subsection*{Overlap with MSA and EGY}
In this section we conduct an evaluation similar to the one carried out in Section \ref{corpus-coverage-analysis} but with an MSA lexicon (Calima$_{MSA}$), and an Egyptian Arabic lexicon (Calima$_{EGY}$).\footnote{For MSA, we compared with the \texttt{calima-msa-s31\_0.4.2.utf8.db} version \citep{Taji:2018:arabic-morphological} based on SAMA \citep{Graff:2009:standard} and for EGY we only compared to the {\tt calima-egy-c044\_0.2.0.utf8.db} based on \citep{Habash:2012:morphological}. For EGY, only {\tt CALIMA} analyses entries are selected.}
%
The analysis reveals that 44\% of \maknuune overlaps with Calima$_{MSA}$ at the lemma:POSType level (63\% if all entries are dediacritized),\footnote{The \textit{shadda} ({\SHADDA}) is not included in dediacritization.}
and that 49\% of \maknuune overlaps similarly with Calima$_{EGY}$ (75\% dediacritized). 
%
Taking into account that {\maknuune} spelling follows the CODA* guidelines,
the analysis suggests that the 37\% of {\maknuune}  lemma:POSTypes, which do not exist in the MSA lexicon we used, are heavily dialectal. The overlap with EGY is predictably higher, and the 25\% of Maknuune lemma:POSTypes (dediacritized) not existing in EGY highlights the differences between the two dialects despite their many similarities.

\subsection*{Observations on Lexical Richness and Diversity}
The quantitative analyses we presented above allow us to see the big picture in terms of lexical richness and diversity in {\maknuune} and its complementarity to existing resources. However, we acknowledge that such an approach misses a lot of details that are collapsed or lost when ignoring subtle differences in semantics, phonology and morphology.

We first point at homonyms showing semantic changes and spread, such as  \foreignlanguage{arabic}{آوَى} 
/2 aa w a/ which is  `thread a needle' in PAL and ‘shelter sb’ in both MSA and PAL,
% \foreignlanguage{arabic}{جرجير} \caphi{J a r J ii r} which means ‘black olives that have been collected from the ground’ in some Palestinian villages and ‘arugula (rocket)’ in MSA, 
\foreignlanguage{arabic}{بَطّ} \caphi{b a t. t.} which means `very small olives that people find hard to pick' in some villages in Palestine and `ducks' in both MSA and PAL, and \foreignlanguage{arabic}{آخرة}
\caphi{2 aa kh r e} which means `desserts' in Nablus and `the Day of the Judgment' in both MSA and PAL, albeit with a different pronunciation. Clearly, additional entries are needed to mark these difference.

Furthermore, the majority of the entries in \maknuune are actually pronounced differently from MSA even if spelled the same without diacritics and thus warrant entries of their own, with clear phonological specifications.

Finally, if we consider morphology (which is not modeled here per se), many PAL lemmas that have MSA lemma cognates are actually inflected differently, e.g.,
\foreignlanguage{arabic}{مَدّ}
{\it mad{\SHADDA}} `extend;stretch'
(in PAL and MSA),
has different inflections for some parts of the paradigm: the 2nd person masculine plural is
\foreignlanguage{arabic}{مَدَّيتوا} {\it mad{\SHADDA}aytuwA} in PAL and
\foreignlanguage{arabic}{مَدَدْتُم} {\it madad.tum} in MSA.
Hence, each lemma in our lexicon heads a morphological paradigm which differs from its MSA counterpart.

\newpage

\section*{POS Type Mapping and Examples}
\addcontentsline{toc}{section}{\protect\numberline{}POS Type Mapping and Examples}%
\label{pos-mapping}

\begin{table}[h!]
    \centering
\includegraphics[width=0.5\linewidth]{pos_table.pdf}
    \caption{Mapping of part-of-speech (POS) types to POS tags used to annotate base words in Maknuune, and associated examples.}
    \label{tab:pos}
\end{table}



%-----------------------------------------------------------
% DICTIONARY
%-----------------------------------------------------------

\mainmatter
\chapter*{\normalfont\fontsize{40}{40}\selectfont \textbf{The Dictionary \\ {\normalfont\fontsize{30}{30}\selectfont By First Root Radical}}}
\addcontentsline{toc}{chapter}{\protect\numberline{}The Dictionary}%

\textit{Phonological transcription in the rest of the book is provided under IPA specifications.}
\thispagestyle{empty}

\newpage
\thispagestyle{empty}
\section*{Abbreviations}
\addcontentsline{toc}{section}{\protect\numberline{}Abbreviations}%

For the more information about the part-of-speech (POS) tags that usually appears after the IPA transcription, refer to the following webpage: \url{https://camel-guidelines.readthedocs.io/en/latest/morphology/}.

\vspace{5mm}
For abbreviations that come between square brackets after the POS tags, the following mapping applies:

\begin{table}[h]
  \begin{tabular}{ll}
  m.    & Masculine singular               \\
  f.    & Feminine singular                \\
  pl.   & Plural                           \\
  f.pl. & Feminine plural                  \\
  p.    & Perfective aspect                \\
  i.    & Imperfective aspect              \\
  c.    & Command aspect                   \\
  1p    & First person plural              \\
  1s    & First person singular            \\
  2fp   & Second person feminine plural    \\
  2fs   & Second person feminine singular  \\
  2ms   & Second person masculine singular \\
  2p    & Second person plural             \\
  3fp   & Third person feminine plural     \\
  3fs   & Third person feminine singular   \\
  3ms   & Third person masculine singular  \\
  3p    & Third person plural              \\
  d.    & Dual                             \\
  \end{tabular}
\end{table}

\vspace{5mm}
Finally, whenever \textsc{ph.} appears after a bullet ($\bullet$\ ), then the following entry is a phrase (collocation, saying, multiword expression, etc.)



\include{letter_sections/ء.tex}
% \include{letter_sections/ب.tex}
% \include{letter_sections/ت.tex}
% \include{letter_sections/ث.tex}
% \include{letter_sections/ج.tex}
% \include{letter_sections/ح.tex}
% \include{letter_sections/خ.tex}
% \include{letter_sections/د.tex}
% \include{letter_sections/ذ.tex}
% \include{letter_sections/ر.tex}
% \include{letter_sections/ز.tex}
% \include{letter_sections/س.tex}
% \include{letter_sections/ش.tex}
% \include{letter_sections/ص.tex}
% \include{letter_sections/ض.tex}
% \include{letter_sections/ط.tex}
% \include{letter_sections/ظ.tex}
% \include{letter_sections/ع.tex}
% \include{letter_sections/غ.tex}
% 
\documentclass[10pt,a4paper,twoside]{article} % 10pt font size, A4 paper and two-sided margins
\usepackage{preamble}
\usepackage{standalone}

\begin{document}

\begin{figure*}[t!]\centering\includegraphics[width=0.15\linewidth]{letter_images/ف.png}\end{figure*}
\color{white}

 \section*{\foreignlanguage{arabic}{ف}} 
 \begin{multicols}{2} 

\addcontentsline{toc}{section}{\protect\numberline{}\foreignlanguage{arabic}{ف}}%
\color{black}
\vspace{-3mm}
\markboth{\color{blue}\foreignlanguage{arabic}{ف}\color{blue}{ (ntws)}}{\color{blue}\foreignlanguage{arabic}{ف}\color{blue}{ (ntws)}}\subsection*{\color{blue}\foreignlanguage{arabic}{ف}\color{blue}{ (ntws)}\index{\color{blue}\foreignlanguage{arabic}{ف}\color{blue}{ (ntws)}}} 

{\setlength\topsep{0pt}\textbf{\foreignlanguage{arabic}{فَ}}\ {\color{gray}\texttt{/\sffamily {{\sffamily fa}}/}\color{black}}\ \textsc{conj}\ \color{gray}(msa. \foreignlanguage{arabic}{ف (حرف عطف)}~\foreignlanguage{arabic}{\textbf{١.}})\color{black}\ \textbf{1.}~then  \textbf{2.}~and then.  \textbf{3.}~so\  \begin{flushright}\color{gray}\foreignlanguage{arabic}{\textbf{\underline{\foreignlanguage{arabic}{أمثلة}}}: سمعت شو حكالك المدير؟ الأول فالأول}\end{flushright}\color{black}} \vspace{2mm}

\vspace{-3mm}
\markboth{\color{blue}\foreignlanguage{arabic}{ف..ن.د.ق}\color{blue}{ (ntws)}}{\color{blue}\foreignlanguage{arabic}{ف..ن.د.ق}\color{blue}{ (ntws)}}\subsection*{\color{blue}\foreignlanguage{arabic}{ف..ن.د.ق}\color{blue}{ (ntws)}\index{\color{blue}\foreignlanguage{arabic}{ف..ن.د.ق}\color{blue}{ (ntws)}}} 

{\setlength\topsep{0pt}\textbf{\foreignlanguage{arabic}{فُنْدُق}}\ {\color{gray}\texttt{/\sffamily {{\sffamily fundu(q)}}/}\color{black}}\ \textsc{noun}\ [m.]\ \color{gray}(msa. \foreignlanguage{arabic}{فُنْدُق}~\foreignlanguage{arabic}{\textbf{١.}})\color{black}\ \textbf{1.}~hotel\ \ $\bullet$\ \ \setlength\topsep{0pt}\textbf{\foreignlanguage{arabic}{فَنَادِق}}\ {\color{gray}\texttt{/\sffamily {{\sffamily fanaːdi(q)}}/}\color{black}}\ [pl.]\  \begin{flushright}\color{gray}\foreignlanguage{arabic}{\textbf{\underline{\foreignlanguage{arabic}{أمثلة}}}: بس بيجي بآخر الليل بينام وبس يصحى بياكل وبيطلع وبنشوفش خلقته لليوم الثاني. عدنه معتبر الدار فُنْدُق}\end{flushright}\color{black}} \vspace{2mm}

\vspace{-3mm}
\markboth{\color{blue}\foreignlanguage{arabic}{ف.ء.ر}\color{blue}{}}{\color{blue}\foreignlanguage{arabic}{ف.ء.ر}\color{blue}{}}\subsection*{\color{blue}\foreignlanguage{arabic}{ف.ء.ر}\color{blue}{}\index{\color{blue}\foreignlanguage{arabic}{ف.ء.ر}\color{blue}{}}} 

{\setlength\topsep{0pt}\textbf{\foreignlanguage{arabic}{فَار}}\ {\color{gray}\texttt{/\sffamily {{\sffamily faːr}}/}\color{black}}\ \textsc{noun}\ [m.]\ \color{gray}(msa. \foreignlanguage{arabic}{فأر}~\foreignlanguage{arabic}{\textbf{١.}})\color{black}\ \textbf{1.}~mouse\ \ $\bullet$\ \ \setlength\topsep{0pt}\textbf{\foreignlanguage{arabic}{فِيرَان}}\ {\color{gray}\texttt{/\sffamily {{\sffamily fiːraːn}}/}\color{black}}\ [pl.]\  \begin{flushright}\color{gray}\foreignlanguage{arabic}{\textbf{\underline{\foreignlanguage{arabic}{أمثلة}}}: الدار كلها فِيران وصراصير}\end{flushright}\color{black}} \vspace{2mm}

{\setlength\topsep{0pt}\textbf{\foreignlanguage{arabic}{فَارَة}}\ {\color{gray}\texttt{/\sffamily {{\sffamily faːra}}/}\color{black}}\ \textsc{noun}\ [f.]\ \color{gray}(msa. \foreignlanguage{arabic}{آداة تستعمل في تنعيم وصقل أسطح الأخشاب والمشغولات}~\foreignlanguage{arabic}{\textbf{١.}})\color{black}\ \textbf{1.}~A tool used for smoothing and polishing wood surfaces\  \begin{flushright}\color{gray}\foreignlanguage{arabic}{\textbf{\underline{\foreignlanguage{arabic}{أمثلة}}}: ناولني الفارة بدي أنعم الخشبة}\end{flushright}\color{black}} \vspace{2mm}

\vspace{-3mm}
\markboth{\color{blue}\foreignlanguage{arabic}{ف.ء.س}\color{blue}{}}{\color{blue}\foreignlanguage{arabic}{ف.ء.س}\color{blue}{}}\subsection*{\color{blue}\foreignlanguage{arabic}{ف.ء.س}\color{blue}{}\index{\color{blue}\foreignlanguage{arabic}{ف.ء.س}\color{blue}{}}} 

{\setlength\topsep{0pt}\textbf{\foreignlanguage{arabic}{فَأْس}}\ {\color{gray}\texttt{/\sffamily {{\sffamily faʔs}}/}\color{black}}\ \textsc{noun}\ [m.]\ \color{gray}(msa. \foreignlanguage{arabic}{فأس}~\foreignlanguage{arabic}{\textbf{١.}})\color{black}\ \textbf{1.}~axe\ \ $\bullet$\ \ \setlength\topsep{0pt}\textbf{\foreignlanguage{arabic}{فُؤُوس}}\ {\color{gray}\texttt{/\sffamily {{\sffamily fuʔuːs}}/}\color{black}}\ [pl.]\ \ $\bullet$\ \ \textsc{ph.} \color{gray} \foreignlanguage{arabic}{وقع الفَاس بَالرَاس}\color{black}\ {\color{gray}\texttt{/{\sffamily wi(q)iʕ ʔilfaːs birraːs}/}\color{black}}\ \color{gray} (msa. \foreignlanguage{arabic}{حصلت المصيبة ولا مفر منها}~\foreignlanguage{arabic}{\textbf{١.}})\color{black}\ \textbf{1.}~(It is an idiomatic expression that means that the die is cast)\  \begin{flushright}\color{gray}\foreignlanguage{arabic}{\textbf{\underline{\foreignlanguage{arabic}{أمثلة}}}: لما وِقِع الفاس بالرّاس تعال الحق يا صلاح}\end{flushright}\color{black}} \vspace{2mm}

{\setlength\topsep{0pt}\textbf{\foreignlanguage{arabic}{فَاس}}\ {\color{gray}\texttt{/\sffamily {{\sffamily faːs}}/}\color{black}}\ \textsc{noun}\ [m.]\ \color{gray}(msa. \foreignlanguage{arabic}{فأس}~\foreignlanguage{arabic}{\textbf{١.}})\color{black}\ \textbf{1.}~axe\  \begin{flushright}\color{gray}\foreignlanguage{arabic}{\textbf{\underline{\foreignlanguage{arabic}{أمثلة}}}: مسك الفاس وإِجى بده بقطع إِيد أخوه}\end{flushright}\color{black}} \vspace{2mm}

\vspace{-3mm}
\markboth{\color{blue}\foreignlanguage{arabic}{ف.ء.ل}\color{blue}{}}{\color{blue}\foreignlanguage{arabic}{ف.ء.ل}\color{blue}{}}\subsection*{\color{blue}\foreignlanguage{arabic}{ف.ء.ل}\color{blue}{}\index{\color{blue}\foreignlanguage{arabic}{ف.ء.ل}\color{blue}{}}} 

{\setlength\topsep{0pt}\textbf{\foreignlanguage{arabic}{تَفَاؤُل}}\ {\color{gray}\texttt{/\sffamily {{\sffamily tafaːʔul}}/}\color{black}}\ \textsc{noun}\ [m.]\ \color{gray}(msa. \foreignlanguage{arabic}{تَفاؤُل}~\foreignlanguage{arabic}{\textbf{١.}})\color{black}\ \textbf{1.}~optimism\ } \vspace{2mm}

{\setlength\topsep{0pt}\textbf{\foreignlanguage{arabic}{تْفَائَل}}\ {\color{gray}\texttt{/\sffamily {{\sffamily tfaːʔal}}/}\color{black}}\ \textsc{verb}\ [p.]\ \textbf{1.}~be optimistic\ \ $\bullet$\ \ \setlength\topsep{0pt}\textbf{\foreignlanguage{arabic}{اِتْفَائَل}}\ {\color{gray}\texttt{/\sffamily {{\sffamily ʔitfaːʔal}}/}\color{black}}\ [c.]\ \ $\bullet$\ \ \setlength\topsep{0pt}\textbf{\foreignlanguage{arabic}{يِتْفَائَل}}\ {\color{gray}\texttt{/\sffamily {{\sffamily jitfaːʔal}}/}\color{black}}\ [i.]\ \color{gray}(msa. \foreignlanguage{arabic}{يَتَفائَل}~\foreignlanguage{arabic}{\textbf{١.}})\color{black}\  \begin{flushright}\color{gray}\foreignlanguage{arabic}{\textbf{\underline{\foreignlanguage{arabic}{أمثلة}}}: أنا بتْفائَل فيك وجهك حلو علي}\end{flushright}\color{black}} \vspace{2mm}

{\setlength\topsep{0pt}\textbf{\foreignlanguage{arabic}{فَأَل}}\ {\color{gray}\texttt{/\sffamily {{\sffamily faʔal}}/}\color{black}}\ \textsc{verb}\ [p.]\ \textbf{1.}~make sb optimistic\ \ $\bullet$\ \ \setlength\topsep{0pt}\textbf{\foreignlanguage{arabic}{اِفْئِل}}\ {\color{gray}\texttt{/\sffamily {{\sffamily ʔifʔil}}/}\color{black}}\ [c.]\ \ $\bullet$\ \ \setlength\topsep{0pt}\textbf{\foreignlanguage{arabic}{يِفْئِل}}\ {\color{gray}\texttt{/\sffamily {{\sffamily jifʔil}}/}\color{black}}\ [i.]\ \color{gray}(msa. \foreignlanguage{arabic}{يَجعل شخص مُتَفائِل}~\foreignlanguage{arabic}{\textbf{١.}})\color{black}\  \begin{flushright}\color{gray}\foreignlanguage{arabic}{\textbf{\underline{\foreignlanguage{arabic}{أمثلة}}}: أنت دايما بتِفْئِلني هيك الله يسعدك ويباركلك}\end{flushright}\color{black}} \vspace{2mm}

{\setlength\topsep{0pt}\textbf{\foreignlanguage{arabic}{فَاءَل}}\ {\color{gray}\texttt{/\sffamily {{\sffamily faːʔal}}/}\color{black}}\ \textsc{verb}\ [p.]\ \textbf{1.}~make sb optimistic\ \ $\bullet$\ \ \setlength\topsep{0pt}\textbf{\foreignlanguage{arabic}{فَائِل}}\ {\color{gray}\texttt{/\sffamily {{\sffamily faːʔil}}/}\color{black}}\ [c.]\ \ $\bullet$\ \ \setlength\topsep{0pt}\textbf{\foreignlanguage{arabic}{يْفَائِل}}\ {\color{gray}\texttt{/\sffamily {{\sffamily jfaːʔil}}/}\color{black}}\ [i.]\ \color{gray}(msa. \foreignlanguage{arabic}{يَجعل شخص مُتَفائِل}~\foreignlanguage{arabic}{\textbf{١.}})\color{black}\  \begin{flushright}\color{gray}\foreignlanguage{arabic}{\textbf{\underline{\foreignlanguage{arabic}{أمثلة}}}: هو حرام حاول يفائِلني بس أنا مش قادرة أشوف غير السواد}\end{flushright}\color{black}} \vspace{2mm}

{\setlength\topsep{0pt}\textbf{\foreignlanguage{arabic}{فَال}}\ {\color{gray}\texttt{/\sffamily {{\sffamily faːl}}/}\color{black}}\ \textsc{noun}\ [m.]\ \textbf{1.}~good omen\ \ $\bullet$\ \ \textsc{ph.} \color{gray} \foreignlanguage{arabic}{فَال الله ولَا فَالك}\color{black}\ {\color{gray}\texttt{/{\sffamily faːl ʔalˤlˤa wala faːlak}/}\color{black}}\ \textbf{1.}~it is an expression that the speaker says to a pessimistic person wo expects the worst\ \ $\bullet$\ \ \textsc{ph.} \color{gray} \foreignlanguage{arabic}{الفَال الك ان شَاء الله}\color{black}\ {\color{gray}\texttt{/{\sffamily ʔilfaːl ʔilak ʔinʃaːlˤlˤa}/}\color{black}}\ \textbf{1.}~it is an expression that means the same to you in reply to sb who has just congratulated someone on sth\  \begin{flushright}\color{gray}\foreignlanguage{arabic}{\textbf{\underline{\foreignlanguage{arabic}{أمثلة}}}: الحمامة البيضا فال منيح الك}\end{flushright}\color{black}} \vspace{2mm}

{\setlength\topsep{0pt}\textbf{\foreignlanguage{arabic}{فَاوَل}}\ {\color{gray}\texttt{/\sffamily {{\sffamily faːwal}}/}\color{black}}\ \textsc{verb}\ [p.]\ \textbf{1.}~foretell misfortune\ \ $\bullet$\ \ \setlength\topsep{0pt}\textbf{\foreignlanguage{arabic}{فَاوِل}}\ {\color{gray}\texttt{/\sffamily {{\sffamily faːwil}}/}\color{black}}\ [c.]\ \ $\bullet$\ \ \setlength\topsep{0pt}\textbf{\foreignlanguage{arabic}{يفَاوِل}}\ {\color{gray}\texttt{/\sffamily {{\sffamily jfaːwil}}/}\color{black}}\ [i.]\  \begin{flushright}\color{gray}\foreignlanguage{arabic}{\textbf{\underline{\foreignlanguage{arabic}{أمثلة}}}: تضلكاش تْفاوِل عليه.}\end{flushright}\color{black}} \vspace{2mm}

{\setlength\topsep{0pt}\textbf{\foreignlanguage{arabic}{مُتَفَائِل}}\ {\color{gray}\texttt{/\sffamily {{\sffamily mutafaːʔil}}/}\color{black}}\ \textsc{adj}\ [m.]\ \color{gray}(msa. \foreignlanguage{arabic}{مُتَفائِل}~\foreignlanguage{arabic}{\textbf{١.}})\color{black}\ \textbf{1.}~optimistic\  \begin{flushright}\color{gray}\foreignlanguage{arabic}{\textbf{\underline{\foreignlanguage{arabic}{أمثلة}}}: صاحب ناس مُتَفائِلة مش بُوَم!}\end{flushright}\color{black}} \vspace{2mm}

{\setlength\topsep{0pt}\textbf{\foreignlanguage{arabic}{مُتَفَائِل}}\ {\color{gray}\texttt{/\sffamily {{\sffamily mutafaːʔil}}/}\color{black}}\ \textsc{noun\textunderscore act}\ [m.]\ \textbf{1.}~being optimistic about sth\  \begin{flushright}\color{gray}\foreignlanguage{arabic}{\textbf{\underline{\foreignlanguage{arabic}{أمثلة}}}: أنا للأمانة مُتَفائِل بالأخبار اللي سمعتها}\end{flushright}\color{black}} \vspace{2mm}

\vspace{-3mm}
\markboth{\color{blue}\foreignlanguage{arabic}{ف.ب.ر.ك}\color{blue}{ (ntws)}}{\color{blue}\foreignlanguage{arabic}{ف.ب.ر.ك}\color{blue}{ (ntws)}}\subsection*{\color{blue}\foreignlanguage{arabic}{ف.ب.ر.ك}\color{blue}{ (ntws)}\index{\color{blue}\foreignlanguage{arabic}{ف.ب.ر.ك}\color{blue}{ (ntws)}}} 

{\setlength\topsep{0pt}\textbf{\foreignlanguage{arabic}{تْفَبْرَك}}\ {\color{gray}\texttt{/\sffamily {{\sffamily tfabrak}}/}\color{black}}\ \textsc{verb}\ [p.]\ \textbf{1.}~be fabricated\ \ $\bullet$\ \ \setlength\topsep{0pt}\textbf{\foreignlanguage{arabic}{اِتْفَبْرَك}}\ {\color{gray}\texttt{/\sffamily {{\sffamily ʔitfabrak}}/}\color{black}}\ [c.]\ \ $\bullet$\ \ \setlength\topsep{0pt}\textbf{\foreignlanguage{arabic}{يِتْفَبْرَك}}\ {\color{gray}\texttt{/\sffamily {{\sffamily jitfabrak}}/}\color{black}}\ [i.]\  \begin{flushright}\color{gray}\foreignlanguage{arabic}{\textbf{\underline{\foreignlanguage{arabic}{أمثلة}}}: هذا أخوها بلال بالزمانات تْفَبْرَكتله قصة هيك. الله يستر علينا وعليه!}\end{flushright}\color{black}} \vspace{2mm}

{\setlength\topsep{0pt}\textbf{\foreignlanguage{arabic}{فَبْرَك}}\ {\color{gray}\texttt{/\sffamily {{\sffamily fabrak}}/}\color{black}}\ \textsc{verb}\ [p.]\ \textbf{1.}~fabricate\ \ $\bullet$\ \ \setlength\topsep{0pt}\textbf{\foreignlanguage{arabic}{فَبْرِك}}\ {\color{gray}\texttt{/\sffamily {{\sffamily fabrik}}/}\color{black}}\ [c.]\ \ $\bullet$\ \ \setlength\topsep{0pt}\textbf{\foreignlanguage{arabic}{يفَبْرِك}}\footnote{English loanword}\ \ {\color{gray}\texttt{/\sffamily {{\sffamily jfabrik}}/}\color{black}}\ [i.]\ \color{gray}(msa. \foreignlanguage{arabic}{يفَبْرِك}~\foreignlanguage{arabic}{\textbf{١.}})\color{black}\  \begin{flushright}\color{gray}\foreignlanguage{arabic}{\textbf{\underline{\foreignlanguage{arabic}{أمثلة}}}: ولاد الحرام فَبْرَكوله فيديو}\end{flushright}\color{black}} \vspace{2mm}

{\setlength\topsep{0pt}\textbf{\foreignlanguage{arabic}{فَبْرَكِة}}\footnote{English loanword}\ \ {\color{gray}\texttt{/\sffamily {{\sffamily fabrake}}/}\color{black}}\ \textsc{noun}\ [f.]\ \color{gray}(msa. \foreignlanguage{arabic}{فَبْرَكَة}~\foreignlanguage{arabic}{\textbf{١.}})\color{black}\ \textbf{1.}~fabrication\  \begin{flushright}\color{gray}\foreignlanguage{arabic}{\textbf{\underline{\foreignlanguage{arabic}{أمثلة}}}: في فَبْرَكِة بالقضية شكلهم والله العليم لفقوله فصة}\end{flushright}\color{black}} \vspace{2mm}

{\setlength\topsep{0pt}\textbf{\foreignlanguage{arabic}{فَبْرِيكِة}}\ {\color{gray}\texttt{/\sffamily {{\sffamily fabriːke}}/}\color{black}}\ \textsc{noun}\ [f.]\ \color{gray}(msa. \foreignlanguage{arabic}{مصنع}~\foreignlanguage{arabic}{\textbf{١.}})\color{black}\ \textbf{1.}~factory\ \ $\bullet$\ \ \setlength\topsep{0pt}\textbf{\foreignlanguage{arabic}{فَبَارِيك}}\ {\color{gray}\texttt{/\sffamily {{\sffamily fabaːriːk}}/}\color{black}}\ [pl.]\  \begin{flushright}\color{gray}\foreignlanguage{arabic}{\textbf{\underline{\foreignlanguage{arabic}{أمثلة}}}: هاي الفَبْريكِة اللي أنت شايفها ورثها عن أبوه}\end{flushright}\color{black}} \vspace{2mm}

{\setlength\topsep{0pt}\textbf{\foreignlanguage{arabic}{مْفَبْرَك}}\footnote{English loanword}\ \ {\color{gray}\texttt{/\sffamily {{\sffamily mfabrak}}/}\color{black}}\ \textsc{adj}\ [m.]\ \color{gray}(msa. \foreignlanguage{arabic}{مُفَبْرَك}~\foreignlanguage{arabic}{\textbf{١.}})\color{black}\ \textbf{1.}~fabricated\  \begin{flushright}\color{gray}\foreignlanguage{arabic}{\textbf{\underline{\foreignlanguage{arabic}{أمثلة}}}: هاي قِصَّة مْفَبْرَكة عشان يبروا ذمتهم}\end{flushright}\color{black}} \vspace{2mm}

\vspace{-3mm}
\markboth{\color{blue}\foreignlanguage{arabic}{ف.ت.ت}\color{blue}{}}{\color{blue}\foreignlanguage{arabic}{ف.ت.ت}\color{blue}{}}\subsection*{\color{blue}\foreignlanguage{arabic}{ف.ت.ت}\color{blue}{}\index{\color{blue}\foreignlanguage{arabic}{ف.ت.ت}\color{blue}{}}} 

{\setlength\topsep{0pt}\textbf{\foreignlanguage{arabic}{اِنْفَتّ}}\ {\color{gray}\texttt{/\sffamily {{\sffamily ʔinfatt}}/}\color{black}}\ \textsc{verb}\ [p.]\ \textbf{1.}~be paid for sth or sb (a lot of money).  \textbf{2.}~be spent (a lot of money)\ \ $\bullet$\ \ \setlength\topsep{0pt}\textbf{\foreignlanguage{arabic}{اِنْفَتّ}}\ {\color{gray}\texttt{/\sffamily {{\sffamily ʔinfatt}}/}\color{black}}\ [c.]\ \ $\bullet$\ \ \setlength\topsep{0pt}\textbf{\foreignlanguage{arabic}{يِنْفَتّ}}\ {\color{gray}\texttt{/\sffamily {{\sffamily jinfatt}}/}\color{black}}\ [i.]\  \begin{flushright}\color{gray}\foreignlanguage{arabic}{\textbf{\underline{\foreignlanguage{arabic}{أمثلة}}}: ابنها الكبير اِنْفَتّ عليه بلاوي!}\end{flushright}\color{black}} \vspace{2mm}

{\setlength\topsep{0pt}\textbf{\foreignlanguage{arabic}{تْفَتَّت}}\ {\color{gray}\texttt{/\sffamily {{\sffamily tfattat}}/}\color{black}}\ \textsc{verb}\ [p.]\ \textbf{1.}~break into small pieces.  \textbf{2.}~shred sth into small pieces\ \ $\bullet$\ \ \setlength\topsep{0pt}\textbf{\foreignlanguage{arabic}{اِتْفَتَّت}}\ {\color{gray}\texttt{/\sffamily {{\sffamily ʔitfattat}}/}\color{black}}\ [c.]\ \ $\bullet$\ \ \setlength\topsep{0pt}\textbf{\foreignlanguage{arabic}{يِتْفَتَّت}}\ {\color{gray}\texttt{/\sffamily {{\sffamily jitfattat}}/}\color{black}}\ [i.]\  \begin{flushright}\color{gray}\foreignlanguage{arabic}{\textbf{\underline{\foreignlanguage{arabic}{أمثلة}}}: حطي المعمولات بعلبة ولا بيتْفَتَّتوا بعدين}\end{flushright}\color{black}} \vspace{2mm}

{\setlength\topsep{0pt}\textbf{\foreignlanguage{arabic}{فَتّ}}\ {\color{gray}\texttt{/\sffamily {{\sffamily fatt}}/}\color{black}}\ \textsc{noun}\ [m.]\ \textbf{1.}~paying for sth a lot\ \ $\bullet$\ \ \textsc{ph.} \color{gray} \foreignlanguage{arabic}{فَتّ خُبِز}\color{black}\ {\color{gray}\texttt{/{\sffamily fatt xubiz}/}\color{black}}\ \textbf{1.}~sb needs alot of life experiences in order to toughen him/her up\  \begin{flushright}\color{gray}\foreignlanguage{arabic}{\textbf{\underline{\foreignlanguage{arabic}{أمثلة}}}: ابنك بده فَت خُبِز كثير عشان يصير زلمة}\end{flushright}\color{black}} \vspace{2mm}

{\setlength\topsep{0pt}\textbf{\foreignlanguage{arabic}{فَتّ}}\ {\color{gray}\texttt{/\sffamily {{\sffamily fatt}}/}\color{black}}\ \textsc{verb}\ [p.]\ \textbf{1.}~pay for sth or sb a lot\ \ $\bullet$\ \ \setlength\topsep{0pt}\textbf{\foreignlanguage{arabic}{فِتّ}}\ {\color{gray}\texttt{/\sffamily {{\sffamily fitt}}/}\color{black}}\ [c.]\ \ $\bullet$\ \ \setlength\topsep{0pt}\textbf{\foreignlanguage{arabic}{يفِتّ}}\ {\color{gray}\texttt{/\sffamily {{\sffamily jfitt}}/}\color{black}}\ [i.]\  \begin{flushright}\color{gray}\foreignlanguage{arabic}{\textbf{\underline{\foreignlanguage{arabic}{أمثلة}}}: أبوي فَتّ عليه بلاوي بالجامعة}\end{flushright}\color{black}} \vspace{2mm}

{\setlength\topsep{0pt}\textbf{\foreignlanguage{arabic}{فَتَّت}}\ {\color{gray}\texttt{/\sffamily {{\sffamily fattat}}/}\color{black}}\ \textsc{verb}\ [p.]\ \textbf{1.}~break into small pieces.  \textbf{2.}~shred sth into small pieces\ \ $\bullet$\ \ \setlength\topsep{0pt}\textbf{\foreignlanguage{arabic}{فَتِّت}}\ {\color{gray}\texttt{/\sffamily {{\sffamily fattit}}/}\color{black}}\ [c.]\ \ $\bullet$\ \ \setlength\topsep{0pt}\textbf{\foreignlanguage{arabic}{يفَتِّت}}\ {\color{gray}\texttt{/\sffamily {{\sffamily jfattit}}/}\color{black}}\ [i.]\  \begin{flushright}\color{gray}\foreignlanguage{arabic}{\textbf{\underline{\foreignlanguage{arabic}{أمثلة}}}: بدي أفَتِّت حصى كلى بموت من الوجع بالشتا}\end{flushright}\color{black}} \vspace{2mm}

{\setlength\topsep{0pt}\textbf{\foreignlanguage{arabic}{فَتِّة}}\ {\color{gray}\texttt{/\sffamily {{\sffamily fatte}}/}\color{black}}\ \textsc{noun}\ [m.]\ \textbf{1.}~porridge  \textbf{2.}~Fatteh is a dish consisting of pieces of fresh, toasted, grilled, or stale flatbread covered with other ingredients that vary according to region.\ } \vspace{2mm}

{\setlength\topsep{0pt}\textbf{\foreignlanguage{arabic}{فُتَات}}\ {\color{gray}\texttt{/\sffamily {{\sffamily futaːt}}/}\color{black}}\ \textsc{noun}\ [pl.]\ \textbf{1.}~bits\  \begin{flushright}\color{gray}\foreignlanguage{arabic}{\textbf{\underline{\foreignlanguage{arabic}{أمثلة}}}: هبش أكبر قسم وتركلنا الفُتات نتقاسمه}\end{flushright}\color{black}} \vspace{2mm}

{\setlength\topsep{0pt}\textbf{\foreignlanguage{arabic}{مْفَتَّت}}\ {\color{gray}\texttt{/\sffamily {{\sffamily mfattat}}/}\color{black}}\ \textsc{noun\textunderscore pass}\ \textbf{1.}~broken into small pieces\  \begin{flushright}\color{gray}\foreignlanguage{arabic}{\textbf{\underline{\foreignlanguage{arabic}{أمثلة}}}: بعثت صحن معمول لعندها عالخليل وصل مْفَتَّت}\end{flushright}\color{black}} \vspace{2mm}

\vspace{-3mm}
\markboth{\color{blue}\foreignlanguage{arabic}{ف.ت.ح}\color{blue}{}}{\color{blue}\foreignlanguage{arabic}{ف.ت.ح}\color{blue}{}}\subsection*{\color{blue}\foreignlanguage{arabic}{ف.ت.ح}\color{blue}{}\index{\color{blue}\foreignlanguage{arabic}{ف.ت.ح}\color{blue}{}}} 

{\setlength\topsep{0pt}\textbf{\foreignlanguage{arabic}{اِسْتَفْتَح}}\ {\color{gray}\texttt{/\sffamily {{\sffamily ʔistaftaħ}}/}\color{black}}\ \textsc{verb}\ [p.]\ \textbf{1.}~open a shop.  \textbf{2.}~open one's day's or life's work.  \textbf{3.}~sell the first item in the shop on this day\ \ $\bullet$\ \ \setlength\topsep{0pt}\textbf{\foreignlanguage{arabic}{اِسْتَفْتِح}}\ {\color{gray}\texttt{/\sffamily {{\sffamily ʔistaftiħ}}/}\color{black}}\ [c.]\ \ $\bullet$\ \ \setlength\topsep{0pt}\textbf{\foreignlanguage{arabic}{يِسْتَفْتَح}}\ {\color{gray}\texttt{/\sffamily {{\sffamily jistaftiħ}}/}\color{black}}\ [i.]\  \begin{flushright}\color{gray}\foreignlanguage{arabic}{\textbf{\underline{\foreignlanguage{arabic}{أمثلة}}}: بدنا نِسْتَفْتَح ونبيعلك القميصين ب50 شو رأيك؟}\end{flushright}\color{black}} \vspace{2mm}

{\setlength\topsep{0pt}\textbf{\foreignlanguage{arabic}{اِسْتِفْتَاحِيِّة}}\ {\color{gray}\texttt{/\sffamily {{\sffamily ʔistiftaːħijje}}/}\color{black}}\ \textsc{noun}\ [f.]\ \textbf{1.}~opening a shop.  \textbf{2.}~opening one's day's or life's work.  \textbf{3.}~selling the first item in the shop on this day\  \begin{flushright}\color{gray}\foreignlanguage{arabic}{\textbf{\underline{\foreignlanguage{arabic}{أمثلة}}}: اِسْتِفْتاحِيِّة مباركة ان شاء الله}\end{flushright}\color{black}} \vspace{2mm}

{\setlength\topsep{0pt}\textbf{\foreignlanguage{arabic}{اِفْتَتَح}}\ {\color{gray}\texttt{/\sffamily {{\sffamily ʔiftataħ}}/}\color{black}}\ \textsc{verb}\ [p.]\ \textbf{1.}~inaugurate\ \ $\bullet$\ \ \setlength\topsep{0pt}\textbf{\foreignlanguage{arabic}{اِفْتَتِح}}\ {\color{gray}\texttt{/\sffamily {{\sffamily ʔiftatiħ}}/}\color{black}}\ [c.]\ \ $\bullet$\ \ \setlength\topsep{0pt}\textbf{\foreignlanguage{arabic}{يِفْتَتِح}}\ {\color{gray}\texttt{/\sffamily {{\sffamily jiftatiħ}}/}\color{black}}\ [i.]\ \color{gray}(msa. \foreignlanguage{arabic}{يَفْتَتِح}~\foreignlanguage{arabic}{\textbf{١.}})\color{black}\  \begin{flushright}\color{gray}\foreignlanguage{arabic}{\textbf{\underline{\foreignlanguage{arabic}{أمثلة}}}: اِفْتَتَحوا فرع لجامعة الخضوري برام الله بشارع المعاهد جنب وزارة التربية والتعليم}\end{flushright}\color{black}} \vspace{2mm}

{\setlength\topsep{0pt}\textbf{\foreignlanguage{arabic}{اِفْتِتَاحِيِّة}}\ {\color{gray}\texttt{/\sffamily {{\sffamily ʔiftitaːħijje}}/}\color{black}}\ \textsc{noun}\ [f.]\ \color{gray}(msa. \foreignlanguage{arabic}{اِفْتِتاحِيِّة}~\foreignlanguage{arabic}{\textbf{١.}})\color{black}\ \textbf{1.}~inauguration\  \begin{flushright}\color{gray}\foreignlanguage{arabic}{\textbf{\underline{\foreignlanguage{arabic}{أمثلة}}}: عملوا اِفْتِتاحِيِّة المحل بنص الشهر}\end{flushright}\color{black}} \vspace{2mm}

{\setlength\topsep{0pt}\textbf{\foreignlanguage{arabic}{اِنْفَتَح}}\footnote{Taboo}\ \ {\color{gray}\texttt{/\sffamily {{\sffamily ʔinfataħ}}/}\color{black}}\ \textsc{verb}\ [p.]\ \textbf{1.}~be opened.  \textbf{2.}~be deflowered.  \textbf{3.}~start yelling and scolding.  \textbf{4.}~cry loudly\ \ $\bullet$\ \ \setlength\topsep{0pt}\textbf{\foreignlanguage{arabic}{اِنْفِتِح}}\ {\color{gray}\texttt{/\sffamily {{\sffamily ʔinfitiħ}}/}\color{black}}\ [c.]\ \ $\bullet$\ \ \setlength\topsep{0pt}\textbf{\foreignlanguage{arabic}{اِنِفْتِح}}\ {\color{gray}\texttt{/\sffamily {{\sffamily ʔiniftiħ}}/}\color{black}}\ [c.]\ \ $\bullet$\ \ \setlength\topsep{0pt}\textbf{\foreignlanguage{arabic}{يِنْفِتِح}}\ {\color{gray}\texttt{/\sffamily {{\sffamily jinfitiħ}}/}\color{black}}\ [i.]\ \color{gray}(msa. \foreignlanguage{arabic}{يبكي بصوت مرتفع}~\foreignlanguage{arabic}{\textbf{٤.}}  \foreignlanguage{arabic}{يصرخ}~\foreignlanguage{arabic}{\textbf{٣.}}  .\foreignlanguage{arabic}{تفقِد عذريتها}~\foreignlanguage{arabic}{\textbf{٢.}}  \foreignlanguage{arabic}{يُفْتَح}~\foreignlanguage{arabic}{\textbf{١.}})\color{black}\ \ $\bullet$\ \ \setlength\topsep{0pt}\textbf{\foreignlanguage{arabic}{يِنِفْتِح}}\ {\color{gray}\texttt{/\sffamily {{\sffamily jiniftiħ}}/}\color{black}}\ [i.]\ \color{gray}(msa. \foreignlanguage{arabic}{يبكي بصوت مرتفع}~\foreignlanguage{arabic}{\textbf{٤.}}  \foreignlanguage{arabic}{يصرخ}~\foreignlanguage{arabic}{\textbf{٣.}}  .\foreignlanguage{arabic}{تفقِد عذريتها}~\foreignlanguage{arabic}{\textbf{٢.}}  \foreignlanguage{arabic}{يُفْتَح}~\foreignlanguage{arabic}{\textbf{١.}})\color{black}\ \ $\bullet$\ \ \textsc{ph.} \color{gray} \foreignlanguage{arabic}{اِنفتحت منَافسي}\color{black}\ {\color{gray}\texttt{/{\sffamily ʔinfatħat manaːfsi}/}\color{black}}\ \color{gray} (msa. \foreignlanguage{arabic}{يشتهي شيء}~\foreignlanguage{arabic}{\textbf{١.}})\color{black}\ \textbf{1.}~crave for sth\ \ $\bullet$\ \ \textsc{ph.} \color{gray} \foreignlanguage{arabic}{اِنْفَتَح زي الشِّشمة}\color{black}\ {\color{gray}\texttt{/{\sffamily ʔinfataħ zajj ʔiʃʃiʃme}/}\color{black}}\ \color{gray} (msa. \foreignlanguage{arabic}{يبكي بصوت مرتفع وبشكل هستيري}~\foreignlanguage{arabic}{\textbf{١.}})\color{black}\ \textbf{1.}~cry loudly and uncontrollably\  \begin{flushright}\color{gray}\foreignlanguage{arabic}{\textbf{\underline{\foreignlanguage{arabic}{أمثلة}}}: بعد شوفة ابنهم والله انفَتْحَت مَنافْسِي عالخلفة\ $\bullet$\ \  ليش أخذت منه اللهاية هسه بينِفْتِح وهات يسكت\ $\bullet$\ \  بخاف بس يِنْفِتِح الباب عشان بكون الصف مكركب\ $\bullet$\ \  اِنْفِتِح عليه وتسمحلوش يتطاول عليك أبدا\ $\bullet$\ \  البنت اِنْفَتَحت من وهي صغيرة يعن قبل لا تعرفك}\end{flushright}\color{black}} \vspace{2mm}

{\setlength\topsep{0pt}\textbf{\foreignlanguage{arabic}{تْفَتَّح}}\ {\color{gray}\texttt{/\sffamily {{\sffamily tfattaħ}}/}\color{black}}\ \textsc{verb}\ [p.]\ \textbf{1.}~open  \textbf{2.}~become worldly-wise and hard-bitten.  \textbf{3.}~look fresh\ \ $\bullet$\ \ \setlength\topsep{0pt}\textbf{\foreignlanguage{arabic}{اِتْفَتَّح}}\ {\color{gray}\texttt{/\sffamily {{\sffamily ʔitfattaħ}}/}\color{black}}\ [c.]\ \ $\bullet$\ \ \setlength\topsep{0pt}\textbf{\foreignlanguage{arabic}{يِتْفَتَّح}}\ {\color{gray}\texttt{/\sffamily {{\sffamily jitfattaħ}}/}\color{black}}\ [i.]\  \begin{flushright}\color{gray}\foreignlanguage{arabic}{\textbf{\underline{\foreignlanguage{arabic}{أمثلة}}}: بحب أول ما يِتْفَتَّح الورد\ $\bullet$\ \  يازلمة اِتْفَتَّح وشوف الدنيا والله الدنيا مالهاش أمان\ $\bullet$\ \  هيك اسم الله تْفَتَّحتي واحلويتي بعد الجيزة. ليش عملتي كل هالنكد؟}\end{flushright}\color{black}} \vspace{2mm}

{\setlength\topsep{0pt}\textbf{\foreignlanguage{arabic}{فَاتَح}}\ {\color{gray}\texttt{/\sffamily {{\sffamily faːtaħ}}/}\color{black}}\ \textsc{verb}\ [p.]\ \textbf{1.}~approach sb.  \textbf{2.}~raise an issue.  \textbf{3.}~discuss sth for the first time\ \ $\bullet$\ \ \setlength\topsep{0pt}\textbf{\foreignlanguage{arabic}{فَاتِح}}\ {\color{gray}\texttt{/\sffamily {{\sffamily faːtiħ}}/}\color{black}}\ [c.]\ \ $\bullet$\ \ \setlength\topsep{0pt}\textbf{\foreignlanguage{arabic}{يفَاتِح}}\ {\color{gray}\texttt{/\sffamily {{\sffamily jfaːtiħ}}/}\color{black}}\ [i.]\  \begin{flushright}\color{gray}\foreignlanguage{arabic}{\textbf{\underline{\foreignlanguage{arabic}{أمثلة}}}: وينتا أجي عليكم عالدار؟ بدي أفاتِح أبوك بموضوع الخطبة}\end{flushright}\color{black}} \vspace{2mm}

{\setlength\topsep{0pt}\textbf{\foreignlanguage{arabic}{فَاتِح}}\ {\color{gray}\texttt{/\sffamily {{\sffamily faːtiħ}}/}\color{black}}\ \textsc{adj}\ [m.]\ \textbf{1.}~bright  \textbf{2.}~light\ \ $\bullet$\ \ \setlength\topsep{0pt}\textbf{\foreignlanguage{arabic}{فَوَاتِح}}\ {\color{gray}\texttt{/\sffamily {{\sffamily fawaːtiħ}}/}\color{black}}\ [pl.]\  \begin{flushright}\color{gray}\foreignlanguage{arabic}{\textbf{\underline{\foreignlanguage{arabic}{أمثلة}}}: يختي البسي فَواتِح بدل هالنكد والأسود اللي أنت بتضلك تلبسيه أكنه عزا!}\end{flushright}\color{black}} \vspace{2mm}

{\setlength\topsep{0pt}\textbf{\foreignlanguage{arabic}{فَاتِح}}\ {\color{gray}\texttt{/\sffamily {{\sffamily faːtiħ}}/}\color{black}}\ \textsc{noun\textunderscore act}\ [m.]\ \textbf{1.}~opening\ \ $\bullet$\ \ \textsc{ph.} \color{gray} \foreignlanguage{arabic}{فَاتِحهَا عليه}\color{black}\ {\color{gray}\texttt{/{\sffamily faːtiħha ʕaleːh}/}\color{black}}\ \color{gray} (msa. \foreignlanguage{arabic}{ثري}~\foreignlanguage{arabic}{\textbf{١.}})\color{black}\ \textbf{1.}~It is an idiomatic expression that means tha sb has a lot of money (very rich)\ \ $\bullet$\ \ \textsc{ph.} \color{gray} \foreignlanguage{arabic}{فَاتِح بطنه}\color{black}\ {\color{gray}\texttt{/{\sffamily faːtiħ batˤno}/}\color{black}}\ \color{gray} (msa. \foreignlanguage{arabic}{يطمع بشيء}~\foreignlanguage{arabic}{\textbf{١.}})\color{black}\ \textbf{1.}~to covet sth\ \ $\bullet$\ \ \textsc{ph.} \color{gray} \foreignlanguage{arabic}{الله فَاتِح عليه}\color{black}\ {\color{gray}\texttt{/{\sffamily ʔalˤlˤa faːtiħ ʕaleːh}/}\color{black}}\ \color{gray} (msa. \foreignlanguage{arabic}{ثري}~\foreignlanguage{arabic}{\textbf{١.}})\color{black}\ \textbf{1.}~wealthy\ \ $\bullet$\ \ \textsc{ph.} \color{gray} \foreignlanguage{arabic}{فَاتِح ثمُّه ورَاخي بيضه}\color{black}\ {\color{gray}\texttt{/{\sffamily faːtiħ (t)immo wuraːxi beː(dˤ)o}/}\color{black}}\ \color{gray} (msa. \foreignlanguage{arabic}{شارد الذِّهن}~\foreignlanguage{arabic}{\textbf{١.}})\color{black}\ \textbf{1.}~absent-minded\  \begin{flushright}\color{gray}\foreignlanguage{arabic}{\textbf{\underline{\foreignlanguage{arabic}{أمثلة}}}: يامن تراه فاتِح ثمُّه وراخي بيضه\ $\bullet$\ \  ليش رفضتي العريس؟ اللَّه فاتِح عليه وعنده محلين وشقة وسيارة\ $\bullet$\ \  جوزي فاتح بطنه بده ياخذ ورثتي كلها يحطها بالبنا\ $\bullet$\ \  اللي الله فاتحها عليه بقآ يشتري لوكس}\end{flushright}\color{black}} \vspace{2mm}

{\setlength\topsep{0pt}\textbf{\foreignlanguage{arabic}{فَاتْحَة}}\ {\color{gray}\texttt{/\sffamily {{\sffamily faːtħa}}/}\color{black}}\ \textsc{noun}\ [f.]\ \color{gray}(msa. \foreignlanguage{arabic}{بِدايَة}~\foreignlanguage{arabic}{\textbf{١.}})\color{black}\ \textbf{1.}~beginning\ \ $\smblkdiamond$\ \ \setlength\topsep{0pt}\textbf{\foreignlanguage{arabic}{فَاتْحَة}}\ \color{gray}(msa. \foreignlanguage{arabic}{فاتِحَة}~\foreignlanguage{arabic}{\textbf{١.}})\color{black}\ \textbf{1.}~Soorat Al-Fatiha\ \ $\bullet$\ \ \textsc{ph.} \color{gray} \foreignlanguage{arabic}{اِقروَا عروحه الفَاتْحَة}\color{black}\ {\color{gray}\texttt{/{\sffamily ʔiqru ʕaroːħo ʔilfaːtħa}/}\color{black}}\ \textbf{1.}~read Suurat AL-Fatiha on a dead person or his grave\ \ $\bullet$\ \ \textsc{ph.} \color{gray} \foreignlanguage{arabic}{مقريِّة فَاتِحتي}\color{black}\ {\color{gray}\texttt{/{\sffamily ma(q)rijje faːtiħti}/}\color{black}}\ \textbf{1.}~sb is engaged but need to make the marriage official in a court\  \begin{flushright}\color{gray}\foreignlanguage{arabic}{\textbf{\underline{\foreignlanguage{arabic}{أمثلة}}}: أنا مقريِّة فاتِحتي الأسبوع الماضي وان شاء الله الأسبوع الجاي كتب الكتاب وحفلة الخطبة\ $\bullet$\ \  الله يجعلها فاتْحَة خير عليكم}\end{flushright}\color{black}} \vspace{2mm}

{\setlength\topsep{0pt}\textbf{\foreignlanguage{arabic}{فَتَح}}\ {\color{gray}\texttt{/\sffamily {{\sffamily fataħ}}/}\color{black}}\ \textsc{verb}\ [p.]\ \textbf{1.}~open  \textbf{2.}~start crying.  \textbf{3.}~start cursing at sb.  \textbf{4.}~starl letting out a stream of invectives\ \ $\bullet$\ \ \setlength\topsep{0pt}\textbf{\foreignlanguage{arabic}{اِفْتَح}}\ {\color{gray}\texttt{/\sffamily {{\sffamily ʔiftaħ}}/}\color{black}}\ [c.]\ \ $\bullet$\ \ \setlength\topsep{0pt}\textbf{\foreignlanguage{arabic}{يِفْتَح}}\ {\color{gray}\texttt{/\sffamily {{\sffamily jiftaħ}}/}\color{black}}\ [i.]\ \color{gray}(msa. \foreignlanguage{arabic}{يبدأ بالشتائِم}~\foreignlanguage{arabic}{\textbf{٣.}}  .\foreignlanguage{arabic}{يبدأ بالبكاء}~\foreignlanguage{arabic}{\textbf{٢.}}  \foreignlanguage{arabic}{يَفْتَح}~\foreignlanguage{arabic}{\textbf{١.}})\color{black}\ \ $\bullet$\ \ \textsc{ph.} \color{gray} \foreignlanguage{arabic}{مجَاري وفتحت}\color{black}\ {\color{gray}\texttt{/{\sffamily ma(dʒ)ari wufatħat}/}\color{black}}\ \color{gray} (msa. \foreignlanguage{arabic}{تعبير اصطلاحي يقصد به الشخص الذي يتكلم كلام بذيء دون توقف}~\foreignlanguage{arabic}{\textbf{١.}})\color{black}\ \textbf{1.}~an ideomatic expression that means  a person that talks trash non-stop\ \ $\bullet$\ \ \textsc{ph.} \color{gray} \foreignlanguage{arabic}{فتحت جروحي}\color{black}\ {\color{gray}\texttt{/{\sffamily fattaħit (dʒ)ruːħi}/}\color{black}}\ \color{gray} (msa. \foreignlanguage{arabic}{يذكر شخص بذكريات مؤلمة}~\foreignlanguage{arabic}{\textbf{١.}})\color{black}\ \textbf{1.}~remind sb with painful memories\ \ $\bullet$\ \ \textsc{ph.} \color{gray} \foreignlanguage{arabic}{تفتحش علي بَاب}\color{black}\ {\color{gray}\texttt{/{\sffamily tiftaħiʃ ʕalaj baːb}/}\color{black}}\ \textbf{1.}~problems come up/arise\ \ $\bullet$\ \ \textsc{ph.} \color{gray} \foreignlanguage{arabic}{فتح عينك}\color{black}\ {\color{gray}\texttt{/{\sffamily fattiħ ʕeːnak}/}\color{black}}\ \color{gray} (msa. \foreignlanguage{arabic}{يحذَر أو انتبِه}~\foreignlanguage{arabic}{\textbf{١.}})\color{black}\ \textbf{1.}~be aware.  \textbf{2.}~watch out!\  \begin{flushright}\color{gray}\foreignlanguage{arabic}{\textbf{\underline{\foreignlanguage{arabic}{أمثلة}}}: فَتِّح عينَك منيح الطريق ملان يهود\ $\bullet$\ \  ما صدَّقِت وهو ناسي الموضوع تفْتَحِش علي باب أبوس إِيدك\ $\bullet$\ \  والله يا أخي أبو محمد وحياة هالنعمة إِنَّك فَتَّْحت جْروحي\ $\bullet$\ \  ما كنا هاديين بأمان الله. ليس لتِفْتحلنا اياه هيه من عياطه سيدي صحي\ $\bullet$\ \  افْتَح الباب شوي شوي مش زي البقر\ $\bullet$\ \  بس حكيتله انه يستنى علي شوي. هاتلك! فَتَح وصار يحكي مسبات من الزنار ونازِل.}\end{flushright}\color{black}} \vspace{2mm}

{\setlength\topsep{0pt}\textbf{\foreignlanguage{arabic}{فَتَّاح}}\ {\color{gray}\texttt{/\sffamily {{\sffamily fatˤtˤaːħ}}/}\color{black}}\ \textsc{noun\textunderscore prop}\ \textbf{1.}~Al-Fattah (Allah's name which means the Opener, the One who opens for His slaves the closed worldly and religious matters)\ \ $\bullet$\ \ \textsc{ph.} \color{gray} \foreignlanguage{arabic}{يَا فَتَّاح يَا عَلِيم}\color{black}\ {\color{gray}\texttt{/{\sffamily jaː fatˤtˤaːħ jaː ʕaliːm}/}\color{black}}\ \textbf{1.}~it is an expression that means that sb saw sth inappropriate in the morning\  \begin{flushright}\color{gray}\foreignlanguage{arabic}{\textbf{\underline{\foreignlanguage{arabic}{أمثلة}}}: يا فتّاح يا عليم! شو هالمناظر عساعة هالصبح!}\end{flushright}\color{black}} \vspace{2mm}

{\setlength\topsep{0pt}\textbf{\foreignlanguage{arabic}{فَتَّاحَة}}\ {\color{gray}\texttt{/\sffamily {{\sffamily fattaːħa}}/}\color{black}}\ \textsc{noun}\ [f.]\ \color{gray}(msa. \foreignlanguage{arabic}{ساحرة}~\foreignlanguage{arabic}{\textbf{١.}})\color{black}\ \textbf{1.}~witch\ } \vspace{2mm}

{\setlength\topsep{0pt}\textbf{\foreignlanguage{arabic}{فَتِّح}}\ {\color{gray}\texttt{/\sffamily {{\sffamily fattaħ}}/}\color{black}}\ \textsc{verb}\ [p.]\ \textbf{1.}~open  \textbf{2.}~lighten\ \ $\bullet$\ \ \setlength\topsep{0pt}\textbf{\foreignlanguage{arabic}{فَتِّح}}\ {\color{gray}\texttt{/\sffamily {{\sffamily fattiħ}}/}\color{black}}\ [c.]\ \ $\bullet$\ \ \setlength\topsep{0pt}\textbf{\foreignlanguage{arabic}{يفَتِّح}}\ {\color{gray}\texttt{/\sffamily {{\sffamily jfattiħ}}/}\color{black}}\ [i.]\ \color{gray}(msa. \foreignlanguage{arabic}{يُفَتِّح اللون}~\foreignlanguage{arabic}{\textbf{٢.}}  \foreignlanguage{arabic}{يَفْتَح}~\foreignlanguage{arabic}{\textbf{١.}})\color{black}\ \ $\bullet$\ \ \textsc{ph.} \color{gray} \foreignlanguage{arabic}{فتح ذنيك معي}\color{black}\ {\color{gray}\texttt{/{\sffamily fattiħ (d)ineːk maʕi}/}\color{black}}\ \color{gray} (msa. \foreignlanguage{arabic}{يُركِّز}~\foreignlanguage{arabic}{\textbf{١.}})\color{black}\ \textbf{1.}~concentrate\  \begin{flushright}\color{gray}\foreignlanguage{arabic}{\textbf{\underline{\foreignlanguage{arabic}{أمثلة}}}: فتح ذنيك معي! بديش ولاد هسَّه!\ $\bullet$\ \  رحت عالكوفيرة عشان تفَتِِّحلي لون شعري}\end{flushright}\color{black}} \vspace{2mm}

{\setlength\topsep{0pt}\textbf{\foreignlanguage{arabic}{فَتْحَة}}\ {\color{gray}\texttt{/\sffamily {{\sffamily fatħa}}/}\color{black}}\ \textsc{noun}\ [f.]\ \color{gray}(msa. \foreignlanguage{arabic}{مَعْبَر}~\foreignlanguage{arabic}{\textbf{٢.}}  \foreignlanguage{arabic}{حُفْرَة}~\foreignlanguage{arabic}{\textbf{١.}})\color{black}\ \textbf{1.}~hole  \textbf{2.}~crossing border\ \ $\smblkdiamond$\ \ \setlength\topsep{0pt}\textbf{\foreignlanguage{arabic}{فَتْحَة}}\ \textbf{1.}~diacritic fatha\ \ $\bullet$\ \ \textsc{ph.} \color{gray} \foreignlanguage{arabic}{تجوَّزهَا بَالفَتْحَة وَالشيخ رُسْلَان}\color{black}\ {\color{gray}\texttt{/{\sffamily ʔit(dʒ)awwazha bilfatħa wiʃʃeːx rislaːn}/}\color{black}}\ \textbf{1.}~get married without paying any dowry to the bride\  \begin{flushright}\color{gray}\foreignlanguage{arabic}{\textbf{\underline{\foreignlanguage{arabic}{أمثلة}}}: والله فش أحلى من هالجيزة، تجوَّزها بالفَتْحَة والشيخ رُسْلان\ $\bullet$\ \  حط فَتْحَة عالحرف الأول\ $\bullet$\ \  بنفوت عالقدس من فَتْحَة فرعون بطولكرم}\end{flushright}\color{black}} \vspace{2mm}

{\setlength\topsep{0pt}\textbf{\foreignlanguage{arabic}{فِتِح}}\ {\color{gray}\texttt{/\sffamily {{\sffamily fitih}}/}\color{black}}\ \textsc{adj}\ [m.]\ \color{gray}(msa. \foreignlanguage{arabic}{له خبرة بالحياة}~\foreignlanguage{arabic}{\textbf{١.}})\color{black}\ \textbf{1.}~worldly-wise/hard-bitten\  \begin{flushright}\color{gray}\foreignlanguage{arabic}{\textbf{\underline{\foreignlanguage{arabic}{أمثلة}}}: الأب فِتِح ما حدا بضحك عليه بسهولة}\end{flushright}\color{black}} \vspace{2mm}

{\setlength\topsep{0pt}\textbf{\foreignlanguage{arabic}{مَفْتُوح}}\ {\color{gray}\texttt{/\sffamily {{\sffamily maftuːħ}}/}\color{black}}\ \textsc{adj}\ [m.]\ \color{gray}(msa. \foreignlanguage{arabic}{مَفْتوح}~\foreignlanguage{arabic}{\textbf{١.}})\color{black}\ \textbf{1.}~open\ \ $\bullet$\ \ \setlength\topsep{0pt}\textbf{\foreignlanguage{arabic}{مَفْتُوحَة}}\footnote{Taboo}\ \ {\color{gray}\texttt{/\sffamily {{\sffamily maftuːħa}}/}\color{black}}\ [m.]\ \textbf{1.}~deflowered  \textbf{2.}~not virgin\  \begin{flushright}\color{gray}\foreignlanguage{arabic}{\textbf{\underline{\foreignlanguage{arabic}{أمثلة}}}: أنا شو الله جابرني أدفع دم قلبي بجوازرة وبالأخير تطلع المرة مَفْتوحَة أصلا\ $\bullet$\ \  الباب مَفْتوح رده شوي شوي}\end{flushright}\color{black}} \vspace{2mm}

{\setlength\topsep{0pt}\textbf{\foreignlanguage{arabic}{مُنْفَتِح}}\ {\color{gray}\texttt{/\sffamily {{\sffamily munfatiħ}}/}\color{black}}\ \textsc{adj}\ [m.]\ \color{gray}(msa. \foreignlanguage{arabic}{مُنْفَتِح}~\foreignlanguage{arabic}{\textbf{١.}})\color{black}\ \textbf{1.}~open-minded\  \begin{flushright}\color{gray}\foreignlanguage{arabic}{\textbf{\underline{\foreignlanguage{arabic}{أمثلة}}}: أبوي زلمة مُنْفَتِح عشان قضى عمره أغلبه برة}\end{flushright}\color{black}} \vspace{2mm}

{\setlength\topsep{0pt}\textbf{\foreignlanguage{arabic}{مِتْفَتَّح}}\ {\color{gray}\texttt{/\sffamily {{\sffamily mitfattiħ}}/}\color{black}}\ \textsc{adj}\ [m.]\ \textbf{1.}~worldly-wise and hard-bitten.  \textbf{2.}~bright  \textbf{3.}~shining\  \begin{flushright}\color{gray}\foreignlanguage{arabic}{\textbf{\underline{\foreignlanguage{arabic}{أمثلة}}}: صاير مِتْفَتَّح كثير عن أول}\end{flushright}\color{black}} \vspace{2mm}

{\setlength\topsep{0pt}\textbf{\foreignlanguage{arabic}{مِفْتَاح}}\ {\color{gray}\texttt{/\sffamily {{\sffamily miftaːħ}}/}\color{black}}\ \textsc{noun}\ [m.]\ \color{gray}(msa. \foreignlanguage{arabic}{مِفْتاح}~\foreignlanguage{arabic}{\textbf{١.}})\color{black}\ \textbf{1.}~key\ \ $\bullet$\ \ \setlength\topsep{0pt}\textbf{\foreignlanguage{arabic}{مَفَاتِيح}}\ {\color{gray}\texttt{/\sffamily {{\sffamily mafatiːħ}}/}\color{black}}\ [pl.]\ \ $\bullet$\ \ \textsc{ph.} \color{gray} \foreignlanguage{arabic}{مِفْتَاح العودَة}\color{black}\ {\color{gray}\texttt{/{\sffamily miftaːħ ʔilʕawda}/}\color{black}}\ \textbf{1.}~the Palestinian right of return key\  \begin{flushright}\color{gray}\foreignlanguage{arabic}{\textbf{\underline{\foreignlanguage{arabic}{أمثلة}}}: ليش لسة عندك مِفْتاح العودَة يا ستي؟\ $\bullet$\ \  بدي أعمل نسخة من المَفاتِيح}\end{flushright}\color{black}} \vspace{2mm}

{\setlength\topsep{0pt}\textbf{\foreignlanguage{arabic}{مْفَتَّح}}\ {\color{gray}\texttt{/\sffamily {{\sffamily mfattaħ}}/}\color{black}}\ \textsc{noun\textunderscore pass}\ \color{gray}(msa. \foreignlanguage{arabic}{مَفْتُوح من عدة جوانب}~\foreignlanguage{arabic}{\textbf{١.}})\color{black}\ \textbf{1.}~has many holes\  \begin{flushright}\color{gray}\foreignlanguage{arabic}{\textbf{\underline{\foreignlanguage{arabic}{أمثلة}}}: خير يختي ما تتستري القميص مْفَتَّح من كل الجهات كانك رايحة كازينو استغفر الله}\end{flushright}\color{black}} \vspace{2mm}

{\setlength\topsep{0pt}\textbf{\foreignlanguage{arabic}{مْفَتِّح}}\ {\color{gray}\texttt{/\sffamily {{\sffamily mfattiħ}}/}\color{black}}\ \textsc{adj}\ [m.]\ \color{gray}(msa. \foreignlanguage{arabic}{يُصْبِح اللون فاتحا}~\foreignlanguage{arabic}{\textbf{١.}})\color{black}\ \textbf{1.}~the colour lightened\ \ $\smblkdiamond$\ \ \setlength\topsep{0pt}\textbf{\foreignlanguage{arabic}{مْفَتِّح}}\ \textbf{1.}~sensible  \textbf{2.}~rational  \textbf{3.}~aware\ \ $\bullet$\ \ \textsc{ph.} \color{gray} \foreignlanguage{arabic}{مفتح ببلَاد عميَان}\color{black}\ {\color{gray}\texttt{/{\sffamily mfattiħ biblaːd ʕimjaːn}/}\color{black}}\ \color{gray}(src. \foreignlanguage{arabic}{رام الله > قرى})\color{black}\ \color{gray} (msa. \foreignlanguage{arabic}{شخص حكيم بمكان مليئ بالجاهلين والحمقى}~\foreignlanguage{arabic}{\textbf{١.}})\color{black}\ \textbf{1.}~It is an idiomatic expression that means that a wise person in a place that is full of idiots and ignorant people\  \begin{flushright}\color{gray}\foreignlanguage{arabic}{\textbf{\underline{\foreignlanguage{arabic}{أمثلة}}}: يا عمي أنت مْفَتِّح ببْلاد عِمْيان شو الك بكل هالقصى؟\ $\bullet$\ \  يختي خالد مْفَتِّح مش مثل الطَّبل اللي عندك\ $\bullet$\ \  شايفة بس بطلتي تطلعي بالشمس وبس صرتي توكلي منيح. هياته لونك مْفَتِّح}\end{flushright}\color{black}} \vspace{2mm}

\vspace{-3mm}
\markboth{\color{blue}\foreignlanguage{arabic}{ف.ت.ر}\color{blue}{}}{\color{blue}\foreignlanguage{arabic}{ف.ت.ر}\color{blue}{}}\subsection*{\color{blue}\foreignlanguage{arabic}{ف.ت.ر}\color{blue}{}\index{\color{blue}\foreignlanguage{arabic}{ف.ت.ر}\color{blue}{}}} 

{\setlength\topsep{0pt}\textbf{\foreignlanguage{arabic}{فَاتُورَة}}\ {\color{gray}\texttt{/\sffamily {{\sffamily faːtuːra}}/}\color{black}}\ \textsc{noun}\ [f.]\ \textbf{1.}~invoice  \textbf{2.}~bill\ \ $\bullet$\ \ \setlength\topsep{0pt}\textbf{\foreignlanguage{arabic}{فَوَاتِير}}\ {\color{gray}\texttt{/\sffamily {{\sffamily fawaːtiːr}}/}\color{black}}\ [pl.]\  \begin{flushright}\color{gray}\foreignlanguage{arabic}{\textbf{\underline{\foreignlanguage{arabic}{أمثلة}}}: بدي أدفع كل الفَواتِير اللي علي}\end{flushright}\color{black}} \vspace{2mm}

{\setlength\topsep{0pt}\textbf{\foreignlanguage{arabic}{فَاتِر}}\ {\color{gray}\texttt{/\sffamily {{\sffamily faːtr}}/}\color{black}}\ \textsc{adj}\ [m.]\ \color{gray}(msa. \foreignlanguage{arabic}{دافِئ}~\foreignlanguage{arabic}{\textbf{٢.}}  \foreignlanguage{arabic}{فاتِرْ}~\foreignlanguage{arabic}{\textbf{١.}})\color{black}\ \textbf{1.}~lukewarm  \textbf{2.}~warm\ } \vspace{2mm}

{\setlength\topsep{0pt}\textbf{\foreignlanguage{arabic}{فَتْرَة}}\ {\color{gray}\texttt{/\sffamily {{\sffamily fatra}}/}\color{black}}\ \textsc{noun}\ [f.]\ \color{gray}(msa. \foreignlanguage{arabic}{فترة زمنية}~\foreignlanguage{arabic}{\textbf{١.}})\color{black}\ \textbf{1.}~period\  \begin{flushright}\color{gray}\foreignlanguage{arabic}{\textbf{\underline{\foreignlanguage{arabic}{أمثلة}}}: الكنب بيضاين فترة طويلة وما بصيرله اشي}\end{flushright}\color{black}} \vspace{2mm}

{\setlength\topsep{0pt}\textbf{\foreignlanguage{arabic}{فَتْرِي}}\ {\color{gray}\texttt{/\sffamily {{\sffamily fatri}}/}\color{black}}\ \textsc{adj}\ [m.]\ \color{gray}(msa. \foreignlanguage{arabic}{فَتْرَِي}~\foreignlanguage{arabic}{\textbf{١.}})\color{black}\ \textbf{1.}~periodic\  \begin{flushright}\color{gray}\foreignlanguage{arabic}{\textbf{\underline{\foreignlanguage{arabic}{أمثلة}}}: هاد الاشي فَتْرَِي مش طول العُمُر}\end{flushright}\color{black}} \vspace{2mm}

{\setlength\topsep{0pt}\textbf{\foreignlanguage{arabic}{فِتِر}}\ {\color{gray}\texttt{/\sffamily {{\sffamily fitir}}/}\color{black}}\ \textsc{verb}\ [p.]\ \textbf{1.}~get tired\ \ $\bullet$\ \ \setlength\topsep{0pt}\textbf{\foreignlanguage{arabic}{اِفْتَر}}\ {\color{gray}\texttt{/\sffamily {{\sffamily ʔiftar}}/}\color{black}}\ [c.]\ \ $\bullet$\ \ \setlength\topsep{0pt}\textbf{\foreignlanguage{arabic}{يِفْتَر}}\ {\color{gray}\texttt{/\sffamily {{\sffamily jiftar}}/}\color{black}}\ [i.]\ \color{gray}(msa. \foreignlanguage{arabic}{يَتْعَب}~\foreignlanguage{arabic}{\textbf{١.}})\color{black}\  \begin{flushright}\color{gray}\foreignlanguage{arabic}{\textbf{\underline{\foreignlanguage{arabic}{أمثلة}}}: أنا فْتِرِت من هيك عيشة}\end{flushright}\color{black}} \vspace{2mm}

\vspace{-3mm}
\markboth{\color{blue}\foreignlanguage{arabic}{ف.ت.ر.ن}\color{blue}{ (ntws)}}{\color{blue}\foreignlanguage{arabic}{ف.ت.ر.ن}\color{blue}{ (ntws)}}\subsection*{\color{blue}\foreignlanguage{arabic}{ف.ت.ر.ن}\color{blue}{ (ntws)}\index{\color{blue}\foreignlanguage{arabic}{ف.ت.ر.ن}\color{blue}{ (ntws)}}} 

{\setlength\topsep{0pt}\textbf{\foreignlanguage{arabic}{فَتْرِينِة}}\ {\color{gray}\texttt{/\sffamily {{\sffamily vatriːne}}/}\color{black}}\ \textsc{noun}\ [f.]\ \color{gray}(msa. \foreignlanguage{arabic}{صندوق زجاج}~\foreignlanguage{arabic}{\textbf{١.}})\color{black}\ \textbf{1.}~a glass box.  \textbf{2.}~display window.  \textbf{3.}~shop window\ \ $\bullet$\ \ \setlength\topsep{0pt}\textbf{\foreignlanguage{arabic}{فَتَارِين}}\ {\color{gray}\texttt{/\sffamily {{\sffamily vataːriːn}}/}\color{black}}\ [pl.]\ } \vspace{2mm}

\vspace{-3mm}
\markboth{\color{blue}\foreignlanguage{arabic}{ف.ت.ش}\color{blue}{}}{\color{blue}\foreignlanguage{arabic}{ف.ت.ش}\color{blue}{}}\subsection*{\color{blue}\foreignlanguage{arabic}{ف.ت.ش}\color{blue}{}\index{\color{blue}\foreignlanguage{arabic}{ف.ت.ش}\color{blue}{}}} 

{\setlength\topsep{0pt}\textbf{\foreignlanguage{arabic}{تَفْتِيش}}\ {\color{gray}\texttt{/\sffamily {{\sffamily taftiːʃ}}/}\color{black}}\ \textsc{noun}\ [m.]\ \color{gray}(msa. \foreignlanguage{arabic}{تَفْتِيش}~\foreignlanguage{arabic}{\textbf{١.}})\color{black}\ \textbf{1.}~investigation  \textbf{2.}~examination\  \begin{flushright}\color{gray}\foreignlanguage{arabic}{\textbf{\underline{\foreignlanguage{arabic}{أمثلة}}}: ليش الشنطة لهلا بالتَّفْتِيش}\end{flushright}\color{black}} \vspace{2mm}

{\setlength\topsep{0pt}\textbf{\foreignlanguage{arabic}{فَاتَش}}\ {\color{gray}\texttt{/\sffamily {{\sffamily faːtaʃ}}/}\color{black}}\ \textsc{verb}\ [p.]\ \textbf{1.}~cross-question sb\ \ $\bullet$\ \ \setlength\topsep{0pt}\textbf{\foreignlanguage{arabic}{فَاتِش}}\ {\color{gray}\texttt{/\sffamily {{\sffamily faːtiʃ}}/}\color{black}}\ [c.]\ \ $\bullet$\ \ \setlength\topsep{0pt}\textbf{\foreignlanguage{arabic}{يفَاتِش}}\ {\color{gray}\texttt{/\sffamily {{\sffamily jfaːtiʃ}}/}\color{black}}\ [i.]\  \begin{flushright}\color{gray}\foreignlanguage{arabic}{\textbf{\underline{\foreignlanguage{arabic}{أمثلة}}}: ولا شي! صار يفاتِش فيه عشان ال100 شيكل اللي ضاعت}\end{flushright}\color{black}} \vspace{2mm}

{\setlength\topsep{0pt}\textbf{\foreignlanguage{arabic}{فَتَّش}}\ {\color{gray}\texttt{/\sffamily {{\sffamily fattaʃ}}/}\color{black}}\ \textsc{verb}\ [p.]\ \textbf{1.}~investigate  \textbf{2.}~examine\ \ $\bullet$\ \ \setlength\topsep{0pt}\textbf{\foreignlanguage{arabic}{فَتِّش}}\ {\color{gray}\texttt{/\sffamily {{\sffamily fattiʃ}}/}\color{black}}\ [c.]\ \ $\bullet$\ \ \setlength\topsep{0pt}\textbf{\foreignlanguage{arabic}{يفَتِّش}}\ {\color{gray}\texttt{/\sffamily {{\sffamily jfattiʃ}}/}\color{black}}\ [i.]\ \color{gray}(msa. \foreignlanguage{arabic}{يَتَفَحَّص}~\foreignlanguage{arabic}{\textbf{٢.}}  \foreignlanguage{arabic}{يُحَقِّق}~\foreignlanguage{arabic}{\textbf{١.}})\color{black}\  \begin{flushright}\color{gray}\foreignlanguage{arabic}{\textbf{\underline{\foreignlanguage{arabic}{أمثلة}}}: فَتِّش منيح يجيوبك بتلاقيها}\end{flushright}\color{black}} \vspace{2mm}

{\setlength\topsep{0pt}\textbf{\foreignlanguage{arabic}{فَتُّوش}}\ {\color{gray}\texttt{/\sffamily {{\sffamily fattuːʃ}}/}\color{black}}\ \textsc{noun\textunderscore prop}\ \textbf{1.}~Fattoush is a Levantine salad made of mixed  vegetables, such as lettuce, cucumber and tomatoes and seasoned with olive oil, lemon juice and salt. On the top of it, people put toasted or fried pieces of bread.\  \begin{flushright}\color{gray}\foreignlanguage{arabic}{\textbf{\underline{\foreignlanguage{arabic}{أمثلة}}}: التبولة أزكى وأبه لعزومة من الفتُّوش}\end{flushright}\color{black}} \vspace{2mm}

{\setlength\topsep{0pt}\textbf{\foreignlanguage{arabic}{فُتَّيش}}\ {\color{gray}\texttt{/\sffamily {{\sffamily futteːʃ}}/}\color{black}}\ \textsc{noun}\ [m.]\ \color{gray}(msa. \foreignlanguage{arabic}{ألعاب نارية}~\foreignlanguage{arabic}{\textbf{١.}})\color{black}\ \textbf{1.}~firework (s)\  \begin{flushright}\color{gray}\foreignlanguage{arabic}{\textbf{\underline{\foreignlanguage{arabic}{أمثلة}}}: بدي أولِّع فُتّيش أول أيام العيد}\end{flushright}\color{black}} \vspace{2mm}

{\setlength\topsep{0pt}\textbf{\foreignlanguage{arabic}{فُتَّيشِة}}\ {\color{gray}\texttt{/\sffamily {{\sffamily futteːʃe}}/}\color{black}}\ \textsc{noun}\ [f.]\ \color{gray}(msa. \foreignlanguage{arabic}{ألعاب نارية}~\foreignlanguage{arabic}{\textbf{٢.}}  .\foreignlanguage{arabic}{مشكلة عنيفة وخلاف حاد}~\foreignlanguage{arabic}{\textbf{١.}})\color{black}\ \textbf{1.}~violent argument / bust-up.  \textbf{2.}~firework (s)\  \begin{flushright}\color{gray}\foreignlanguage{arabic}{\textbf{\underline{\foreignlanguage{arabic}{أمثلة}}}: رمَى فُتّيشِة وبعدها قامت القيامة وهي حردت عند أهلها ووصلت الأمور بينهم للطلاق}\end{flushright}\color{black}} \vspace{2mm}

{\setlength\topsep{0pt}\textbf{\foreignlanguage{arabic}{مُفَتِّش}}\ {\color{gray}\texttt{/\sffamily {{\sffamily mufattiʃ}}/}\color{black}}\ \textsc{noun}\ [m.]\ \color{gray}(msa. \foreignlanguage{arabic}{مُفَتِّش}~\foreignlanguage{arabic}{\textbf{١.}})\color{black}\ \textbf{1.}~inspector  \textbf{2.}~investigator\  \begin{flushright}\color{gray}\foreignlanguage{arabic}{\textbf{\underline{\foreignlanguage{arabic}{أمثلة}}}: أخده المُفَتِّش عغرفة تانية وهو ايديه مربطات}\end{flushright}\color{black}} \vspace{2mm}

\vspace{-3mm}
\markboth{\color{blue}\foreignlanguage{arabic}{ف.ت.ف.ت}\color{blue}{}}{\color{blue}\foreignlanguage{arabic}{ف.ت.ف.ت}\color{blue}{}}\subsection*{\color{blue}\foreignlanguage{arabic}{ف.ت.ف.ت}\color{blue}{}\index{\color{blue}\foreignlanguage{arabic}{ف.ت.ف.ت}\color{blue}{}}} 

{\setlength\topsep{0pt}\textbf{\foreignlanguage{arabic}{تْفَتْفَت}}\ {\color{gray}\texttt{/\sffamily {{\sffamily tfatfat}}/}\color{black}}\ \textsc{verb}\ [p.]\ \textbf{1.}~be smashed.  \textbf{2.}~be shreded.  \textbf{3.}~be incandescent with rage\ \ $\bullet$\ \ \setlength\topsep{0pt}\textbf{\foreignlanguage{arabic}{اِتْفَتْفَت}}\ {\color{gray}\texttt{/\sffamily {{\sffamily ʔitfatfat}}/}\color{black}}\ [c.]\ \ $\bullet$\ \ \setlength\topsep{0pt}\textbf{\foreignlanguage{arabic}{يِتْفَتْفَت}}\ {\color{gray}\texttt{/\sffamily {{\sffamily jitfatfat}}/}\color{black}}\ [i.]\ \color{gray}(msa. \foreignlanguage{arabic}{يستشيط غضباً}~\foreignlanguage{arabic}{\textbf{٣.}}  .\foreignlanguage{arabic}{يُقَطَّع إِلى قِطع صغيرة}~\foreignlanguage{arabic}{\textbf{٢.}}  \foreignlanguage{arabic}{يُسْحَق}~\foreignlanguage{arabic}{\textbf{١.}})\color{black}\  \begin{flushright}\color{gray}\foreignlanguage{arabic}{\textbf{\underline{\foreignlanguage{arabic}{أمثلة}}}: هو بيحكي وبيضحك عادي وأنا بتفَتْفَت من جوا\ $\bullet$\ \  شو أعمل؟ الخبز تفَتْفَت لحاله؟}\end{flushright}\color{black}} \vspace{2mm}

{\setlength\topsep{0pt}\textbf{\foreignlanguage{arabic}{فَتْفَت}}\ {\color{gray}\texttt{/\sffamily {{\sffamily fatfat}}/}\color{black}}\ \textsc{verb}\ [p.]\ \textbf{1.}~smash  \textbf{2.}~shred  \textbf{3.}~threaten to beat sb\ \ $\bullet$\ \ \setlength\topsep{0pt}\textbf{\foreignlanguage{arabic}{فَتْفِت}}\ {\color{gray}\texttt{/\sffamily {{\sffamily fatfit}}/}\color{black}}\ [c.]\ \ $\bullet$\ \ \setlength\topsep{0pt}\textbf{\foreignlanguage{arabic}{يفَتْفِت}}\ {\color{gray}\texttt{/\sffamily {{\sffamily jfatfit}}/}\color{black}}\ [i.]\ \color{gray}(msa. \foreignlanguage{arabic}{يُهدد بضرب شخص}~\foreignlanguage{arabic}{\textbf{٣.}}  .\foreignlanguage{arabic}{يُقَطِّع قِطع صغيرة}~\foreignlanguage{arabic}{\textbf{٢.}}  \foreignlanguage{arabic}{يِسْحَق}~\foreignlanguage{arabic}{\textbf{١.}})\color{black}\  \begin{flushright}\color{gray}\foreignlanguage{arabic}{\textbf{\underline{\foreignlanguage{arabic}{أمثلة}}}: نفسي أفتْفِتك وحياة الله. كيف بتعمل هيك بدون إِذني؟\ $\bullet$\ \  خليه يفَتْفِت الكعك مليح قبل ما يطعميه للجاجات}\end{flushright}\color{black}} \vspace{2mm}

{\setlength\topsep{0pt}\textbf{\foreignlanguage{arabic}{فَتْفُوتِة}}\ {\color{gray}\texttt{/\sffamily {{\sffamily fatfuːte}}/}\color{black}}\ \textsc{noun}\ [f.]\ \color{gray}(msa. \foreignlanguage{arabic}{قطعة صغيرة}~\foreignlanguage{arabic}{\textbf{١.}})\color{black}\ \textbf{1.}~a bit\ \ $\bullet$\ \ \setlength\topsep{0pt}\textbf{\foreignlanguage{arabic}{فَتَافِيت}}\ {\color{gray}\texttt{/\sffamily {{\sffamily fataːfiːt}}/}\color{black}}\ [pl.]\  \begin{flushright}\color{gray}\foreignlanguage{arabic}{\textbf{\underline{\foreignlanguage{arabic}{أمثلة}}}: حُطِّيلي فَتْفُوتِة رُز أَطْعَم هالحمامات}\end{flushright}\color{black}} \vspace{2mm}

\vspace{-3mm}
\markboth{\color{blue}\foreignlanguage{arabic}{ف.ت.ل}\color{blue}{}}{\color{blue}\foreignlanguage{arabic}{ف.ت.ل}\color{blue}{}}\subsection*{\color{blue}\foreignlanguage{arabic}{ف.ت.ل}\color{blue}{}\index{\color{blue}\foreignlanguage{arabic}{ف.ت.ل}\color{blue}{}}} 

{\setlength\topsep{0pt}\textbf{\foreignlanguage{arabic}{اِنْفَتَل}}\ {\color{gray}\texttt{/\sffamily {{\sffamily ʔinfatal}}/}\color{black}}\ \textsc{verb}\ [p.]\ \textbf{1.}~be twisted.  \textbf{2.}~be rotated.  \textbf{3.}~the Maftul is prepare out of groats\ \ $\bullet$\ \ \setlength\topsep{0pt}\textbf{\foreignlanguage{arabic}{اِنْفِتِل}}\ {\color{gray}\texttt{/\sffamily {{\sffamily ʔinfitil}}/}\color{black}}\ [c.]\ \ $\bullet$\ \ \setlength\topsep{0pt}\textbf{\foreignlanguage{arabic}{يِنْفِتِل}}\ {\color{gray}\texttt{/\sffamily {{\sffamily jinfitil}}/}\color{black}}\ [i.]\ \ $\bullet$\ \ \textsc{ph.} \color{gray} \foreignlanguage{arabic}{اِنْفَتَل رَاسِي}\color{black}\ {\color{gray}\texttt{/{\sffamily ʔinfatal raːsi}/}\color{black}}\ \textbf{1.}~have a headache.  \textbf{2.}~be very tired of sth\  \begin{flushright}\color{gray}\foreignlanguage{arabic}{\textbf{\underline{\foreignlanguage{arabic}{أمثلة}}}: اِنْفَتَل راسِي من ورا فقرات الإذاعة المدرسية تبعتهم\ $\bullet$\ \  البرغي اِنْفَتَل أكثر من اللازم عشان هيك صدَّى}\end{flushright}\color{black}} \vspace{2mm}

{\setlength\topsep{0pt}\textbf{\foreignlanguage{arabic}{تْفَتَّل}}\ {\color{gray}\texttt{/\sffamily {{\sffamily tfattal}}/}\color{black}}\ \textsc{verb}\ [p.]\ \textbf{1.}~go around.  \textbf{2.}~promenade\ \ $\bullet$\ \ \setlength\topsep{0pt}\textbf{\foreignlanguage{arabic}{تْفَتَّل}}\ {\color{gray}\texttt{/\sffamily {{\sffamily tfattal}}/}\color{black}}\ [c.]\ \ $\bullet$\ \ \setlength\topsep{0pt}\textbf{\foreignlanguage{arabic}{يتْفَتَّل}}\ {\color{gray}\texttt{/\sffamily {{\sffamily jitfattal}}/}\color{black}}\ [i.]\ \color{gray}(msa. \foreignlanguage{arabic}{يلف ويدور ويتمشى}~\foreignlanguage{arabic}{\textbf{١.}})\color{black}\  \begin{flushright}\color{gray}\foreignlanguage{arabic}{\textbf{\underline{\foreignlanguage{arabic}{أمثلة}}}: تْفَتَّلْت شوي بالمطبخ والصالة قبل ما يجوا الجماعة}\end{flushright}\color{black}} \vspace{2mm}

{\setlength\topsep{0pt}\textbf{\foreignlanguage{arabic}{تْفَتْوَل}}\ {\color{gray}\texttt{/\sffamily {{\sffamily tfatwal}}/}\color{black}}\ \textsc{verb}\ [p.]\ \textbf{1.}~go around.  \textbf{2.}~promenade\ \ $\bullet$\ \ \setlength\topsep{0pt}\textbf{\foreignlanguage{arabic}{اِتْفَتْوَل}}\ {\color{gray}\texttt{/\sffamily {{\sffamily ʔitfatwal}}/}\color{black}}\ [c.]\ \ $\bullet$\ \ \setlength\topsep{0pt}\textbf{\foreignlanguage{arabic}{يِتْفَتْوَل}}\ {\color{gray}\texttt{/\sffamily {{\sffamily jitfatwal}}/}\color{black}}\ [i.]\  \begin{flushright}\color{gray}\foreignlanguage{arabic}{\textbf{\underline{\foreignlanguage{arabic}{أمثلة}}}: اِتْفَتْوَلولكم شوي بالسوق عبين ما ألاقيلي صفة للسيارة}\end{flushright}\color{black}} \vspace{2mm}

{\setlength\topsep{0pt}\textbf{\foreignlanguage{arabic}{فَتَل}}\ {\color{gray}\texttt{/\sffamily {{\sffamily fatal}}/}\color{black}}\ \textsc{verb}\ [p.]\ \textbf{1.}~twist  \textbf{2.}~rotate  \textbf{3.}~prepare Maftul out of groats\ \ $\bullet$\ \ \setlength\topsep{0pt}\textbf{\foreignlanguage{arabic}{اِفْتِل}}\ {\color{gray}\texttt{/\sffamily {{\sffamily ʔiftil}}/}\color{black}}\ [c.]\ \ $\bullet$\ \ \setlength\topsep{0pt}\textbf{\foreignlanguage{arabic}{يِفْتِل}}\ {\color{gray}\texttt{/\sffamily {{\sffamily jiftil}}/}\color{black}}\ [i.]\ \color{gray}(msa. \foreignlanguage{arabic}{يَصْنَع مفتول}~\foreignlanguage{arabic}{\textbf{٣.}}  \foreignlanguage{arabic}{يدوِّر}~\foreignlanguage{arabic}{\textbf{٢.}}  \foreignlanguage{arabic}{يَلوِي}~\foreignlanguage{arabic}{\textbf{١.}})\color{black}\  \begin{flushright}\color{gray}\foreignlanguage{arabic}{\textbf{\underline{\foreignlanguage{arabic}{أمثلة}}}: ستي فَتَلَت مفتول للحارة كلها\ $\bullet$\ \  افتل الراس تبعها}\end{flushright}\color{black}} \vspace{2mm}

{\setlength\topsep{0pt}\textbf{\foreignlanguage{arabic}{فَتْلِة}}\ {\color{gray}\texttt{/\sffamily {{\sffamily fatle}}/}\color{black}}\ \textsc{noun}\ [f.]\ \color{gray}(msa. \foreignlanguage{arabic}{مِشْوار}~\foreignlanguage{arabic}{\textbf{٢.}}  \foreignlanguage{arabic}{لَوِي}~\foreignlanguage{arabic}{\textbf{١.}})\color{black}\ \textbf{1.}~twisting  \textbf{2.}~promenade\ } \vspace{2mm}

{\setlength\topsep{0pt}\textbf{\foreignlanguage{arabic}{فَتْوَلِة}}\ {\color{gray}\texttt{/\sffamily {{\sffamily fatwale}}/}\color{black}}\ \textsc{noun}\ [f.]\ \textbf{1.}~going around.  \textbf{2.}~promenade\ } \vspace{2mm}

{\setlength\topsep{0pt}\textbf{\foreignlanguage{arabic}{فْتِيل}}\ {\color{gray}\texttt{/\sffamily {{\sffamily ftiːl}}/}\color{black}}\ \textsc{noun}\ [m.]\ \color{gray}(msa. \foreignlanguage{arabic}{فَتِيل}~\foreignlanguage{arabic}{\textbf{١.}})\color{black}\ \textbf{1.}~fuse\  \begin{flushright}\color{gray}\foreignlanguage{arabic}{\textbf{\underline{\foreignlanguage{arabic}{أمثلة}}}: أتوقع إِنُّه بده تغيير فْتيل}\end{flushright}\color{black}} \vspace{2mm}

{\setlength\topsep{0pt}\textbf{\foreignlanguage{arabic}{فْتِيلِة}}\ {\color{gray}\texttt{/\sffamily {{\sffamily ftiːle}}/}\color{black}}\ \textsc{noun}\ [f.]\ \color{gray}(msa. \foreignlanguage{arabic}{قطعة قماش مبلَّلة بزيت أو كاز}~\foreignlanguage{arabic}{\textbf{١.}})\color{black}\ \textbf{1.}~fabric soaked with oil or kerosene\ \ $\bullet$\ \ \setlength\topsep{0pt}\textbf{\foreignlanguage{arabic}{فَتَايِل}}\ {\color{gray}\texttt{/\sffamily {{\sffamily fataːjil}}/}\color{black}}\ [pl.]\  \begin{flushright}\color{gray}\foreignlanguage{arabic}{\textbf{\underline{\foreignlanguage{arabic}{أمثلة}}}: امسح الدهان اللي اجآ عليها بقطعة فتيلة}\end{flushright}\color{black}} \vspace{2mm}

{\setlength\topsep{0pt}\textbf{\foreignlanguage{arabic}{مَفْتُول}}\ {\color{gray}\texttt{/\sffamily {{\sffamily maftuːl}}/}\color{black}}\ \textsc{noun}\ [f.]\ \color{gray}(msa. \foreignlanguage{arabic}{طعام تقليدي شعبي شتوي يتكون من السميد المضاف إِليه طحين القمح على شكل كرات صغيرة ويطهى بطناجر خاصة، على البخار المتصاعد من مرق اللحم وخليط الخضراوات. تتكون طنجرة المفتول من قطعتين، تعلو إِحداهما الأخرى، وتكون القطعة العليا عبارة عن مصفاة مخرمة يوضع بها المفتول، وتركب على السفلى بشكل لا يسمح للبخار بالصعود إِلا من خلال حبيبات المفتول.}~\foreignlanguage{arabic}{\textbf{١.}})\color{black}\ \textbf{1.}~A popular traditional wintery food consisting of semolina and wheat flour in small balls, cooked with special pots and steamed from the meat broth and vegetable mixture. The maftoul cooker consists of two pieces, one of which is on top of the other, and the upper piece is an openwork strainer in which the maftuul is placed, and it is installed on the bottom in a manner that does not allow steam to rise except through the granules of the maftoul.\ \ $\bullet$\ \ \textsc{ph.} \color{gray} \foreignlanguage{arabic}{قور المَفْتُول}\color{black}\ {\color{gray}\texttt{/{\sffamily quːr ʔilmaftuːl}/}\color{black}}\ \textbf{1.}~the special cooking pot of Maftul\  \begin{flushright}\color{gray}\foreignlanguage{arabic}{\textbf{\underline{\foreignlanguage{arabic}{أمثلة}}}: عملت طبق مفتول كتير زاكي}\end{flushright}\color{black}} \vspace{2mm}

\vspace{-3mm}
\markboth{\color{blue}\foreignlanguage{arabic}{ف.ت.ن}\color{blue}{}}{\color{blue}\foreignlanguage{arabic}{ف.ت.ن}\color{blue}{}}\subsection*{\color{blue}\foreignlanguage{arabic}{ف.ت.ن}\color{blue}{}\index{\color{blue}\foreignlanguage{arabic}{ف.ت.ن}\color{blue}{}}} 

{\setlength\topsep{0pt}\textbf{\foreignlanguage{arabic}{فَتَّان}}\ {\color{gray}\texttt{/\sffamily {{\sffamily fattaːn}}/}\color{black}}\ \textsc{adj}\ [m.]\ \color{gray}(msa. \foreignlanguage{arabic}{واش}~\foreignlanguage{arabic}{\textbf{١.}})\color{black}\ \textbf{1.}~snitch\  \begin{flushright}\color{gray}\foreignlanguage{arabic}{\textbf{\underline{\foreignlanguage{arabic}{أمثلة}}}: انت واحد فتّان وما بتتأمَّن عأسرار}\end{flushright}\color{black}} \vspace{2mm}

{\setlength\topsep{0pt}\textbf{\foreignlanguage{arabic}{فَتَّن}}\ {\color{gray}\texttt{/\sffamily {{\sffamily fattan}}/}\color{black}}\ \textsc{verb}\ [p.]\ \textbf{1.}~divulge/reveal a secret\ \ $\bullet$\ \ \setlength\topsep{0pt}\textbf{\foreignlanguage{arabic}{فَتِّن}}\ {\color{gray}\texttt{/\sffamily {{\sffamily fattin}}/}\color{black}}\ [c.]\ \ $\bullet$\ \ \setlength\topsep{0pt}\textbf{\foreignlanguage{arabic}{يفَتِّن}}\ {\color{gray}\texttt{/\sffamily {{\sffamily jfattin}}/}\color{black}}\ [i.]\ \color{gray}(msa. \foreignlanguage{arabic}{يفشي سراً}~\foreignlanguage{arabic}{\textbf{١.}})\color{black}\  \begin{flushright}\color{gray}\foreignlanguage{arabic}{\textbf{\underline{\foreignlanguage{arabic}{أمثلة}}}: اللي فَتَّن على أخوك هو مصطَفَى}\end{flushright}\color{black}} \vspace{2mm}

{\setlength\topsep{0pt}\textbf{\foreignlanguage{arabic}{فِتْنِة}}\ {\color{gray}\texttt{/\sffamily {{\sffamily fitne}}/}\color{black}}\ \textsc{noun}\ [f.]\ \color{gray}(msa. \foreignlanguage{arabic}{فِتْنَة}~\foreignlanguage{arabic}{\textbf{١.}})\color{black}\ \textbf{1.}~sedition\ \ $\bullet$\ \ \setlength\topsep{0pt}\textbf{\foreignlanguage{arabic}{فِتَن}}\ {\color{gray}\texttt{/\sffamily {{\sffamily fitan}}/}\color{black}}\ [pl.]\  \begin{flushright}\color{gray}\foreignlanguage{arabic}{\textbf{\underline{\foreignlanguage{arabic}{أمثلة}}}: حشبي الله بكل واحد بيعمل فِتَن بين ولاد الشعب الفلسطيني}\end{flushright}\color{black}} \vspace{2mm}

\vspace{-3mm}
\markboth{\color{blue}\foreignlanguage{arabic}{ف.ت.ي}\color{blue}{}}{\color{blue}\foreignlanguage{arabic}{ف.ت.ي}\color{blue}{}}\subsection*{\color{blue}\foreignlanguage{arabic}{ف.ت.ي}\color{blue}{}\index{\color{blue}\foreignlanguage{arabic}{ف.ت.ي}\color{blue}{}}} 

{\setlength\topsep{0pt}\textbf{\foreignlanguage{arabic}{أَفْتَى}}\ {\color{gray}\texttt{/\sffamily {{\sffamily ʔafta}}/}\color{black}}\ \textsc{verb}\ [p.]\ \textbf{1.}~issue Fatwa (is a formal ruling or interpretation on a point of Islamic law given by a qualified legal scholar (known as a mufti)\ \ $\bullet$\ \ \setlength\topsep{0pt}\textbf{\foreignlanguage{arabic}{اِفْتِي}}\ {\color{gray}\texttt{/\sffamily {{\sffamily ʔifti}}/}\color{black}}\ [c.]\ \ $\bullet$\ \ \setlength\topsep{0pt}\textbf{\foreignlanguage{arabic}{يِفْتِي}}\ {\color{gray}\texttt{/\sffamily {{\sffamily jifti}}/}\color{black}}\ [i.]\ \color{gray}(msa. \foreignlanguage{arabic}{يُفْتِي}~\foreignlanguage{arabic}{\textbf{١.}})\color{black}\  \begin{flushright}\color{gray}\foreignlanguage{arabic}{\textbf{\underline{\foreignlanguage{arabic}{أمثلة}}}: هاي الشغلة تحديدا أفْتالي فيها الشيخ انها حلال}\end{flushright}\color{black}} \vspace{2mm}

{\setlength\topsep{0pt}\textbf{\foreignlanguage{arabic}{إِفْتَاء}}\ {\color{gray}\texttt{/\sffamily {{\sffamily ʔiftaːʔ}}/}\color{black}}\ \textsc{noun}\ [m.]\ \textbf{1.}~Ifta is defined as the Efforts to provide an explanation of the law of sharia by experts to people who do not know it.\  \begin{flushright}\color{gray}\foreignlanguage{arabic}{\textbf{\underline{\foreignlanguage{arabic}{أمثلة}}}: اتصل عالإِفْتاء بلكي بيردوا عليك وبيجكولك حكمها}\end{flushright}\color{black}} \vspace{2mm}

{\setlength\topsep{0pt}\textbf{\foreignlanguage{arabic}{اِسْتِفْتَاء}}\ {\color{gray}\texttt{/\sffamily {{\sffamily ʔistiftaːʔ}}/}\color{black}}\ \textsc{noun}\ [m.]\ \textbf{1.}~questionnaires  \textbf{2.}~polls  \textbf{3.}~referendums\  \begin{flushright}\color{gray}\foreignlanguage{arabic}{\textbf{\underline{\foreignlanguage{arabic}{أمثلة}}}: عملوا اِسْتِفْتاء أحسن صف بالمدرسة واحنا اللي فزنا}\end{flushright}\color{black}} \vspace{2mm}

{\setlength\topsep{0pt}\textbf{\foreignlanguage{arabic}{فَتَى}}\ {\color{gray}\texttt{/\sffamily {{\sffamily fata}}/}\color{black}}\ \textsc{verb}\ [p.]\ \textbf{1.}~pontificate over sth\ \ $\bullet$\ \ \setlength\topsep{0pt}\textbf{\foreignlanguage{arabic}{اِفْتِي}}\ {\color{gray}\texttt{/\sffamily {{\sffamily ʔifti}}/}\color{black}}\ [c.]\ \ $\bullet$\ \ \setlength\topsep{0pt}\textbf{\foreignlanguage{arabic}{يِفْتِي}}\ {\color{gray}\texttt{/\sffamily {{\sffamily jifti}}/}\color{black}}\ [i.]\  \begin{flushright}\color{gray}\foreignlanguage{arabic}{\textbf{\underline{\foreignlanguage{arabic}{أمثلة}}}: كل شي بيِفْتِي فيه حتى الحمل والولادة يا زلمة اِفْتِي بشي بتفهمه!}\end{flushright}\color{black}} \vspace{2mm}

{\setlength\topsep{0pt}\textbf{\foreignlanguage{arabic}{فَتْوِة}}\ {\color{gray}\texttt{/\sffamily {{\sffamily fatwe}}/}\color{black}}\ \textsc{noun}\ [f.]\ \color{gray}(msa. \foreignlanguage{arabic}{فَتْوَة}~\foreignlanguage{arabic}{\textbf{١.}})\color{black}\ \textbf{1.}~Fatwa, in Islam, is a formal ruling or interpretation on a point of Islamic law given by a qualified legal scholar (known as a mufti)\ \ $\bullet$\ \ \setlength\topsep{0pt}\textbf{\foreignlanguage{arabic}{فَتَاوَى}}\ {\color{gray}\texttt{/\sffamily {{\sffamily fataːwa}}/}\color{black}}\ [pl.]\  \begin{flushright}\color{gray}\foreignlanguage{arabic}{\textbf{\underline{\foreignlanguage{arabic}{أمثلة}}}: في فَتاوَى كثيرة بخصوص الطلاق بقدرش أعطيك من عندي لازم تشوف مُفْتِي}\end{flushright}\color{black}} \vspace{2mm}

{\setlength\topsep{0pt}\textbf{\foreignlanguage{arabic}{مُفْتِي}}\ {\color{gray}\texttt{/\sffamily {{\sffamily mufti}}/}\color{black}}\ \textsc{noun}\ [m.]\ \color{gray}(msa. \foreignlanguage{arabic}{مُفْتِي}~\foreignlanguage{arabic}{\textbf{١.}})\color{black}\ \textbf{1.}~Mufti (a qualified legal scholar who is responsible for issuing formal ruling or interpretation on a point of Islamic law)\ } \vspace{2mm}

\vspace{-3mm}
\markboth{\color{blue}\foreignlanguage{arabic}{ف.ج.ء}\color{blue}{}}{\color{blue}\foreignlanguage{arabic}{ف.ج.ء}\color{blue}{}}\subsection*{\color{blue}\foreignlanguage{arabic}{ف.ج.ء}\color{blue}{}\index{\color{blue}\foreignlanguage{arabic}{ف.ج.ء}\color{blue}{}}} 

{\setlength\topsep{0pt}\textbf{\foreignlanguage{arabic}{تْفَاجَأ}}\ {\color{gray}\texttt{/\sffamily {{\sffamily tfaː(dʒ)aʔ}}/}\color{black}}\ \textsc{verb}\ [p.]\ \textbf{1.}~be surprised\ \ $\bullet$\ \ \setlength\topsep{0pt}\textbf{\foreignlanguage{arabic}{اِتْفَاجَأ}}\ {\color{gray}\texttt{/\sffamily {{\sffamily ʔitfaː(dʒ)aʔ}}/}\color{black}}\ [c.]\ \color{gray}(msa. \foreignlanguage{arabic}{يَِتَفاجأ}~\foreignlanguage{arabic}{\textbf{١.}})\color{black}\ \ $\bullet$\ \ \setlength\topsep{0pt}\textbf{\foreignlanguage{arabic}{يِتْفَاجَأ}}\ {\color{gray}\texttt{/\sffamily {{\sffamily jitfaː(dʒ)aʔ}}/}\color{black}}\ [i.]\  \begin{flushright}\color{gray}\foreignlanguage{arabic}{\textbf{\underline{\foreignlanguage{arabic}{أمثلة}}}: هو تْفاجأ بس شافني بدون الأولاد}\end{flushright}\color{black}} \vspace{2mm}

{\setlength\topsep{0pt}\textbf{\foreignlanguage{arabic}{فَاجَأ}}\ {\color{gray}\texttt{/\sffamily {{\sffamily faː(dʒ)aʔ}}/}\color{black}}\ \textsc{verb}\ [p.]\ \textbf{1.}~surprise sb\ \ $\bullet$\ \ \setlength\topsep{0pt}\textbf{\foreignlanguage{arabic}{فَاجِئ}}\ {\color{gray}\texttt{/\sffamily {{\sffamily faː(dʒ)iʔ}}/}\color{black}}\ [c.]\ \ $\bullet$\ \ \setlength\topsep{0pt}\textbf{\foreignlanguage{arabic}{يْفَاجِئ}}\ {\color{gray}\texttt{/\sffamily {{\sffamily jfaː(dʒ)iʔ}}/}\color{black}}\ [i.]\ \color{gray}(msa. \foreignlanguage{arabic}{يُفاجِئ}~\foreignlanguage{arabic}{\textbf{١.}})\color{black}\  \begin{flushright}\color{gray}\foreignlanguage{arabic}{\textbf{\underline{\foreignlanguage{arabic}{أمثلة}}}: حاول فاجِئها بطلعة أو عزومة عمطعم بلكي بترضى عليك}\end{flushright}\color{black}} \vspace{2mm}

{\setlength\topsep{0pt}\textbf{\foreignlanguage{arabic}{فَجْأَة}}\ {\color{gray}\texttt{/\sffamily {{\sffamily fa(dʒ)ʔa}}/}\color{black}}\ \textsc{noun}\ [f.]\ \textbf{1.}~suddenly\  \begin{flushright}\color{gray}\foreignlanguage{arabic}{\textbf{\underline{\foreignlanguage{arabic}{أمثلة}}}: كنا زي السمنة عالعسل وفَجْأة كل شي تغير}\end{flushright}\color{black}} \vspace{2mm}

{\setlength\topsep{0pt}\textbf{\foreignlanguage{arabic}{فُجَائِي}}\ {\color{gray}\texttt{/\sffamily {{\sffamily fu(dʒ)aːʔi}}/}\color{black}}\ \textsc{adj}\ [m.]\ \textbf{1.}~surprising\  \begin{flushright}\color{gray}\foreignlanguage{arabic}{\textbf{\underline{\foreignlanguage{arabic}{أمثلة}}}: بحبش الزيارات الفُجائِية لو سمحت حماتي. الله يرضى عليك ابعثيلي خبر بس بدك تيجي لعنا.}\end{flushright}\color{black}} \vspace{2mm}

{\setlength\topsep{0pt}\textbf{\foreignlanguage{arabic}{مُفَاجَأَة}}\ {\color{gray}\texttt{/\sffamily {{\sffamily mufaː(dʒ)aʔa}}/}\color{black}}\ \textsc{noun}\ [f.]\ \color{gray}(msa. \foreignlanguage{arabic}{مُفاجَأة}~\foreignlanguage{arabic}{\textbf{١.}})\color{black}\ \textbf{1.}~surprise\  \begin{flushright}\color{gray}\foreignlanguage{arabic}{\textbf{\underline{\foreignlanguage{arabic}{أمثلة}}}: عاملتلك مُفاجَأة! بحب المُفاجَآت أنا!}\end{flushright}\color{black}} \vspace{2mm}

\vspace{-3mm}
\markboth{\color{blue}\foreignlanguage{arabic}{ف.ج.ر}\color{blue}{}}{\color{blue}\foreignlanguage{arabic}{ف.ج.ر}\color{blue}{}}\subsection*{\color{blue}\foreignlanguage{arabic}{ف.ج.ر}\color{blue}{}\index{\color{blue}\foreignlanguage{arabic}{ف.ج.ر}\color{blue}{}}} 

{\setlength\topsep{0pt}\textbf{\foreignlanguage{arabic}{اِنْفَجَر}}\ {\color{gray}\texttt{/\sffamily {{\sffamily ʔinfa(dʒ)ar}}/}\color{black}}\ \textsc{verb}\ [p.]\ \textbf{1.}~explode  \textbf{2.}~explode with anger.  \textbf{3.}~burts into tears\ \ $\bullet$\ \ \setlength\topsep{0pt}\textbf{\foreignlanguage{arabic}{اِنْفِجِر}}\ {\color{gray}\texttt{/\sffamily {{\sffamily ʔinfi(dʒ)ir}}/}\color{black}}\ [c.]\ \ $\bullet$\ \ \setlength\topsep{0pt}\textbf{\foreignlanguage{arabic}{اِنِفْجِر}}\ {\color{gray}\texttt{/\sffamily {{\sffamily ʔinif(dʒ)ir}}/}\color{black}}\ [c.]\ \ $\bullet$\ \ \setlength\topsep{0pt}\textbf{\foreignlanguage{arabic}{يِنْفِجِر}}\ {\color{gray}\texttt{/\sffamily {{\sffamily jinfi(dʒ)ir}}/}\color{black}}\ [i.]\ \ $\bullet$\ \ \setlength\topsep{0pt}\textbf{\foreignlanguage{arabic}{يِنِفْجِر}}\ {\color{gray}\texttt{/\sffamily {{\sffamily jinif(dʒ)ir}}/}\color{black}}\ [i.]\  \begin{flushright}\color{gray}\foreignlanguage{arabic}{\textbf{\underline{\foreignlanguage{arabic}{أمثلة}}}: والله الواحد بده يِنِفْجِر بس شو بده يعمل\ $\bullet$\ \  كان بيشتغل عمواد كيميائية وبعديها مش عارف شو صار واِنْفَجَر المختبر}\end{flushright}\color{black}} \vspace{2mm}

{\setlength\topsep{0pt}\textbf{\foreignlanguage{arabic}{اِنْفِجَار}}\ {\color{gray}\texttt{/\sffamily {{\sffamily ʔinfi(dʒ)aːr}}/}\color{black}}\ \textsc{noun}\ [m.]\ \color{gray}(msa. \foreignlanguage{arabic}{اِنْفِجار}~\foreignlanguage{arabic}{\textbf{١.}})\color{black}\ \textbf{1.}~explosion\  \begin{flushright}\color{gray}\foreignlanguage{arabic}{\textbf{\underline{\foreignlanguage{arabic}{أمثلة}}}: دار أبو أحمد بقى عندهم مصنع كبير للإِسمنت. صار في اِنْفِجار ضخم وواحد من العُمّال انحرق بالكامل وتوفى الله يرحمه.}\end{flushright}\color{black}} \vspace{2mm}

{\setlength\topsep{0pt}\textbf{\foreignlanguage{arabic}{تَفْجِير}}\ {\color{gray}\texttt{/\sffamily {{\sffamily taf(dʒ)iːr}}/}\color{black}}\ \textsc{noun}\ [m.]\ \color{gray}(msa. \foreignlanguage{arabic}{تَفْجِير}~\foreignlanguage{arabic}{\textbf{١.}})\color{black}\ \textbf{1.}~explosion\  \begin{flushright}\color{gray}\foreignlanguage{arabic}{\textbf{\underline{\foreignlanguage{arabic}{أمثلة}}}: مش هاد المبنى اللي صار فيه تَفْجِير العام؟}\end{flushright}\color{black}} \vspace{2mm}

{\setlength\topsep{0pt}\textbf{\foreignlanguage{arabic}{تْفَجَّر}}\ {\color{gray}\texttt{/\sffamily {{\sffamily tfa(dʒ)(dʒ)ar}}/}\color{black}}\ \textsc{verb}\ [p.]\ \textbf{1.}~explode  \textbf{2.}~burst\ \ $\bullet$\ \ \setlength\topsep{0pt}\textbf{\foreignlanguage{arabic}{اِتْفَجَّر}}\ {\color{gray}\texttt{/\sffamily {{\sffamily ʔitfa(dʒ)(dʒ)ar}}/}\color{black}}\ [c.]\ \ $\bullet$\ \ \setlength\topsep{0pt}\textbf{\foreignlanguage{arabic}{يِتْفَجَّر}}\ {\color{gray}\texttt{/\sffamily {{\sffamily jitfa(dʒ)(dʒ)ar}}/}\color{black}}\ [i.]\  \begin{flushright}\color{gray}\foreignlanguage{arabic}{\textbf{\underline{\foreignlanguage{arabic}{أمثلة}}}: راسي رح يِتْفَجَّر وحياة الله\ $\bullet$\ \  محسسني إنه أخرى شوي رح تِتْفَجَّر أنهار وينابيع}\end{flushright}\color{black}} \vspace{2mm}

{\setlength\topsep{0pt}\textbf{\foreignlanguage{arabic}{فَجَرَة}}\ {\color{gray}\texttt{/\sffamily {{\sffamily fa(dʒ)ara}}/}\color{black}}\ \textsc{adj}\ [pl.]\ \textbf{1.}~licentious\ } \vspace{2mm}

{\setlength\topsep{0pt}\textbf{\foreignlanguage{arabic}{فَاجِر}}\ {\color{gray}\texttt{/\sffamily {{\sffamily faː(dʒ)ir}}/}\color{black}}\ \textsc{noun}\ [m.]\ \color{gray}(msa. \foreignlanguage{arabic}{فاجِر}~\foreignlanguage{arabic}{\textbf{١.}})\color{black}\ \textbf{1.}~licentious\ } \vspace{2mm}

{\setlength\topsep{0pt}\textbf{\foreignlanguage{arabic}{فَجَر}}\ {\color{gray}\texttt{/\sffamily {{\sffamily fa(dʒ)ar}}/}\color{black}}\ \textsc{verb}\ [p.]\ \textbf{1.}~act licentiously\ \ $\bullet$\ \ \setlength\topsep{0pt}\textbf{\foreignlanguage{arabic}{اِفْجُر}}\ {\color{gray}\texttt{/\sffamily {{\sffamily ʔuf(dʒ)ur}}/}\color{black}}\ [c.]\ \ $\bullet$\ \ \setlength\topsep{0pt}\textbf{\foreignlanguage{arabic}{يُفْجُر}}\ {\color{gray}\texttt{/\sffamily {{\sffamily juf(dʒ)ur}}/}\color{black}}\ [i.]\ \color{gray}(msa. \foreignlanguage{arabic}{يَفْجُر}~\foreignlanguage{arabic}{\textbf{١.}})\color{black}\  \begin{flushright}\color{gray}\foreignlanguage{arabic}{\textbf{\underline{\foreignlanguage{arabic}{أمثلة}}}: ابنها بعد ماراح عالغربة فَجَر}\end{flushright}\color{black}} \vspace{2mm}

{\setlength\topsep{0pt}\textbf{\foreignlanguage{arabic}{فَجِر}}\ {\color{gray}\texttt{/\sffamily {{\sffamily fa(dʒ)ir}}/}\color{black}}\ \textsc{noun}\ [m.]\ \color{gray}(msa. \foreignlanguage{arabic}{الصباح الباكر}~\foreignlanguage{arabic}{\textbf{٢.}}  \foreignlanguage{arabic}{الفَجْر}~\foreignlanguage{arabic}{\textbf{١.}})\color{black}\ \textbf{1.}~Fajr  \textbf{2.}~dawn\ \ $\bullet$\ \ \textsc{ph.} \color{gray} \foreignlanguage{arabic}{من طيز الفجر}\color{black}\ \footnote{Taboo}\ {\color{gray}\texttt{/{\sffamily min tˤiːzˤ ʔilfa(dʒ)ir}/}\color{black}}\ \color{gray} (msa. \foreignlanguage{arabic}{الصباح الباكر}~\foreignlanguage{arabic}{\textbf{١.}})\color{black}\ \textbf{1.}~It is an idiomatic expression that means very early in the morning\  \begin{flushright}\color{gray}\foreignlanguage{arabic}{\textbf{\underline{\foreignlanguage{arabic}{أمثلة}}}: يعني أنا صاحي من من طِيز الفَجِر أحرث بهالرأرض عشان بالأخير همي يجوا يقشوا كل هالخير عالبارد المستريح}\end{flushright}\color{black}} \vspace{2mm}

{\setlength\topsep{0pt}\textbf{\foreignlanguage{arabic}{فَجَّر}}\ {\color{gray}\texttt{/\sffamily {{\sffamily fa(dʒ)(dʒ)ar}}/}\color{black}}\ \textsc{verb}\ [p.]\ \textbf{1.}~explode  \textbf{2.}~denotate\ \ $\bullet$\ \ \setlength\topsep{0pt}\textbf{\foreignlanguage{arabic}{فَجِّر}}\ {\color{gray}\texttt{/\sffamily {{\sffamily fa(dʒ)(dʒ)ir}}/}\color{black}}\ [c.]\ \ $\bullet$\ \ \setlength\topsep{0pt}\textbf{\foreignlanguage{arabic}{يفَجِّر}}\ {\color{gray}\texttt{/\sffamily {{\sffamily jfa(dʒ)(dʒ)ir}}/}\color{black}}\ [i.]\ \color{gray}(msa. \foreignlanguage{arabic}{يُفَجِّر}~\foreignlanguage{arabic}{\textbf{١.}})\color{black}\ } \vspace{2mm}

{\setlength\topsep{0pt}\textbf{\foreignlanguage{arabic}{فَجْرِيِّة}}\ {\color{gray}\texttt{/\sffamily {{\sffamily fa(dʒ)rijje}}/}\color{black}}\ \textsc{noun}\ [f.]\ \color{gray}(msa. \foreignlanguage{arabic}{فَجْر}~\foreignlanguage{arabic}{\textbf{١.}})\color{black}\ \textbf{1.}~(time of) dawn\  \begin{flushright}\color{gray}\foreignlanguage{arabic}{\textbf{\underline{\foreignlanguage{arabic}{أمثلة}}}: اجا عنا من الفَجْرِيِّة}\end{flushright}\color{black}} \vspace{2mm}

{\setlength\topsep{0pt}\textbf{\foreignlanguage{arabic}{فُجُور}}\ {\color{gray}\texttt{/\sffamily {{\sffamily fu(dʒ)uːr}}/}\color{black}}\ \textsc{noun}\ [m.]\ \color{gray}(msa. \foreignlanguage{arabic}{فُجُور}~\foreignlanguage{arabic}{\textbf{١.}})\color{black}\ \textbf{1.}~licentiousness\  \begin{flushright}\color{gray}\foreignlanguage{arabic}{\textbf{\underline{\foreignlanguage{arabic}{أمثلة}}}: أنا بحياتي ماشفت فُجُور عند البهود زي اللي شفته بهالحفلة}\end{flushright}\color{black}} \vspace{2mm}

\vspace{-3mm}
\markboth{\color{blue}\foreignlanguage{arabic}{ف.ج.ع}\color{blue}{}}{\color{blue}\foreignlanguage{arabic}{ف.ج.ع}\color{blue}{}}\subsection*{\color{blue}\foreignlanguage{arabic}{ف.ج.ع}\color{blue}{}\index{\color{blue}\foreignlanguage{arabic}{ف.ج.ع}\color{blue}{}}} 

{\setlength\topsep{0pt}\textbf{\foreignlanguage{arabic}{اِنْفَجَع}}\ {\color{gray}\texttt{/\sffamily {{\sffamily ʔinfa(dʒ)aʕ}}/}\color{black}}\ \textsc{verb}\ [p.]\ \textbf{1.}~be afflicted.  \textbf{2.}~be bereaved\ \ $\bullet$\ \ \setlength\topsep{0pt}\textbf{\foreignlanguage{arabic}{اِنْفِجِع}}\ {\color{gray}\texttt{/\sffamily {{\sffamily ʔinfi(dʒ)iʕ}}/}\color{black}}\ [c.]\ \ $\bullet$\ \ \setlength\topsep{0pt}\textbf{\foreignlanguage{arabic}{اِنِفْجِع}}\ {\color{gray}\texttt{/\sffamily {{\sffamily ʔinif(dʒ)iʕ}}/}\color{black}}\ [c.]\ \ $\bullet$\ \ \setlength\topsep{0pt}\textbf{\foreignlanguage{arabic}{يِنْفِجِع}}\ {\color{gray}\texttt{/\sffamily {{\sffamily jinfi(dʒ)iʕ}}/}\color{black}}\ [i.]\ \ $\bullet$\ \ \setlength\topsep{0pt}\textbf{\foreignlanguage{arabic}{يِنِفْجِع}}\ {\color{gray}\texttt{/\sffamily {{\sffamily jinif(dʒ)iʕ}}/}\color{black}}\ [i.]\  \begin{flushright}\color{gray}\foreignlanguage{arabic}{\textbf{\underline{\foreignlanguage{arabic}{أمثلة}}}: اِنْفَجَعنا بخبر وفاته الله يرحمه\ $\bullet$\ \  مش عارف ايش مالي اِنْفَجَعت مرة وحدة عالأكل}\end{flushright}\color{black}} \vspace{2mm}

{\setlength\topsep{0pt}\textbf{\foreignlanguage{arabic}{تْفَجْعَن}}\ {\color{gray}\texttt{/\sffamily {{\sffamily tfa(dʒ)ʕan}}/}\color{black}}\ \textsc{verb}\ [p.]\ \textbf{1.}~eat with gluttony.  \textbf{2.}~be gluttonous\ \ $\bullet$\ \ \setlength\topsep{0pt}\textbf{\foreignlanguage{arabic}{اِتْفَجْعَن}}\ {\color{gray}\texttt{/\sffamily {{\sffamily ʔitfa(dʒ)ʕan}}/}\color{black}}\ [c.]\ \ $\bullet$\ \ \setlength\topsep{0pt}\textbf{\foreignlanguage{arabic}{يِتْفَجْعَن}}\ {\color{gray}\texttt{/\sffamily {{\sffamily jitfa(dʒ)ʕan}}/}\color{black}}\ [i.]\ \color{gray}(msa. \foreignlanguage{arabic}{يأكل بشراهَة}~\foreignlanguage{arabic}{\textbf{١.}})\color{black}\  \begin{flushright}\color{gray}\foreignlanguage{arabic}{\textbf{\underline{\foreignlanguage{arabic}{أمثلة}}}: أخذناهم عالعرس خزونا صاروا يِتْفَجْعَنوا}\end{flushright}\color{black}} \vspace{2mm}

{\setlength\topsep{0pt}\textbf{\foreignlanguage{arabic}{فَجَع}}\ {\color{gray}\texttt{/\sffamily {{\sffamily fa(dʒ)aʕ}}/}\color{black}}\ \textsc{noun}\ [m.]\ \color{gray}(msa. \foreignlanguage{arabic}{شراهَة}~\foreignlanguage{arabic}{\textbf{٢.}}  .\foreignlanguage{arabic}{فَجَع الألم}~\foreignlanguage{arabic}{\textbf{١.}})\color{black}\ \textbf{1.}~bereavement  \textbf{2.}~gluttony\ } \vspace{2mm}

{\setlength\topsep{0pt}\textbf{\foreignlanguage{arabic}{فَجَع}}\ {\color{gray}\texttt{/\sffamily {{\sffamily fa(dʒ)aʕ}}/}\color{black}}\ \textsc{verb}\ [p.]\ \textbf{1.}~afflict  \textbf{2.}~cause bereavement to sb\ \ $\bullet$\ \ \setlength\topsep{0pt}\textbf{\foreignlanguage{arabic}{اِفْجَع}}\ {\color{gray}\texttt{/\sffamily {{\sffamily ʔif(dʒ)aʕ}}/}\color{black}}\ [c.]\ \ $\bullet$\ \ \setlength\topsep{0pt}\textbf{\foreignlanguage{arabic}{يِفْجَع}}\ {\color{gray}\texttt{/\sffamily {{\sffamily jif(dʒ)aʕ}}/}\color{black}}\ [i.]\ \color{gray}(msa. \foreignlanguage{arabic}{يَفْجَع}~\foreignlanguage{arabic}{\textbf{١.}})\color{black}\  \begin{flushright}\color{gray}\foreignlanguage{arabic}{\textbf{\underline{\foreignlanguage{arabic}{أمثلة}}}: الله لا يِفْجَع أم بضناها}\end{flushright}\color{black}} \vspace{2mm}

{\setlength\topsep{0pt}\textbf{\foreignlanguage{arabic}{فَجْعَان}}\ {\color{gray}\texttt{/\sffamily {{\sffamily fa(dʒ)ʕaːn}}/}\color{black}}\ \textsc{adj}\ [m.]\ \color{gray}(msa. \foreignlanguage{arabic}{غير قنوع}~\foreignlanguage{arabic}{\textbf{٢.}}  \foreignlanguage{arabic}{شَرِِه}~\foreignlanguage{arabic}{\textbf{١.}})\color{black}\ \textbf{1.}~gluttonous  \textbf{2.}~dicontent with what sb has\  \begin{flushright}\color{gray}\foreignlanguage{arabic}{\textbf{\underline{\foreignlanguage{arabic}{أمثلة}}}: مرتك فَجْعانة ولو شو ماتعمل مشش رح يملا عينها}\end{flushright}\color{black}} \vspace{2mm}

{\setlength\topsep{0pt}\textbf{\foreignlanguage{arabic}{فَجْعَنِة}}\ {\color{gray}\texttt{/\sffamily {{\sffamily fa(dʒ)ʕane}}/}\color{black}}\ \textsc{noun}\ [f.]\ \color{gray}(msa. \foreignlanguage{arabic}{شراهَة}~\foreignlanguage{arabic}{\textbf{١.}})\color{black}\ \textbf{1.}~gluttony\  \begin{flushright}\color{gray}\foreignlanguage{arabic}{\textbf{\underline{\foreignlanguage{arabic}{أمثلة}}}: بكفي فَجْعَنِة عيب}\end{flushright}\color{black}} \vspace{2mm}

{\setlength\topsep{0pt}\textbf{\foreignlanguage{arabic}{مَفْجُوع}}\ {\color{gray}\texttt{/\sffamily {{\sffamily maf(dʒ)uːʕ}}/}\color{black}}\ \textsc{adj}\ [m.]\ \color{gray}(msa. \foreignlanguage{arabic}{مَفجوع من شدة الألم والحزن}~\foreignlanguage{arabic}{\textbf{٢.}}  \foreignlanguage{arabic}{شَرِه}~\foreignlanguage{arabic}{\textbf{١.}})\color{black}\ \textbf{1.}~gluttonous  \textbf{2.}~bereaved\ \ $\bullet$\ \ \setlength\topsep{0pt}\textbf{\foreignlanguage{arabic}{مَفَاجِيع}}\ {\color{gray}\texttt{/\sffamily {{\sffamily mafaː(dʒ)iːʕ}}/}\color{black}}\ [pl.]\  \begin{flushright}\color{gray}\foreignlanguage{arabic}{\textbf{\underline{\foreignlanguage{arabic}{أمثلة}}}: شو الله بلاني أقعد مش شلة مَفاجِيع\ $\bullet$\ \  المرة مسكينة مَفْجُوعَة بوفاة زوجها}\end{flushright}\color{black}} \vspace{2mm}

\vspace{-3mm}
\markboth{\color{blue}\foreignlanguage{arabic}{ف.ج.ف.ج}\color{blue}{}}{\color{blue}\foreignlanguage{arabic}{ف.ج.ف.ج}\color{blue}{}}\subsection*{\color{blue}\foreignlanguage{arabic}{ف.ج.ف.ج}\color{blue}{}\index{\color{blue}\foreignlanguage{arabic}{ف.ج.ف.ج}\color{blue}{}}} 

{\setlength\topsep{0pt}\textbf{\foreignlanguage{arabic}{فَجْفَجّ}}\ {\color{gray}\texttt{/\sffamily {{\sffamily fadʒfadʒdʒ}}/}\color{black}}\ \textsc{verb}\ [p.]\ \textbf{1.}~open sth widely\ \ $\bullet$\ \ \setlength\topsep{0pt}\textbf{\foreignlanguage{arabic}{فَجْفِجّ}}\ {\color{gray}\texttt{/\sffamily {{\sffamily fadʒfidʒdʒ}}/}\color{black}}\ [c.]\ \ $\bullet$\ \ \setlength\topsep{0pt}\textbf{\foreignlanguage{arabic}{يفَجْفِجّ}}\ {\color{gray}\texttt{/\sffamily {{\sffamily jfadʒfidʒdʒ}}/}\color{black}}\ [i.]\  \begin{flushright}\color{gray}\foreignlanguage{arabic}{\textbf{\underline{\foreignlanguage{arabic}{أمثلة}}}: لما سمعت إِنه أبوي رجع فَجْفَجت عيونها هيك وصارت تحكي ليش ماخبَّرني؟}\end{flushright}\color{black}} \vspace{2mm}

{\setlength\topsep{0pt}\textbf{\foreignlanguage{arabic}{مْفَجْفِجّ}}\ {\color{gray}\texttt{/\sffamily {{\sffamily mfadʒfidʒdʒ}}/}\color{black}}\ \textsc{adj}\ [m.]\ \textbf{1.}~be opened widely.  \textbf{2.}~swollen\  \begin{flushright}\color{gray}\foreignlanguage{arabic}{\textbf{\underline{\foreignlanguage{arabic}{أمثلة}}}: عيونها مْفَجْفِجات من العياط}\end{flushright}\color{black}} \vspace{2mm}

\vspace{-3mm}
\markboth{\color{blue}\foreignlanguage{arabic}{ف.ج.م}\color{blue}{ (ntws)}}{\color{blue}\foreignlanguage{arabic}{ف.ج.م}\color{blue}{ (ntws)}}\subsection*{\color{blue}\foreignlanguage{arabic}{ف.ج.م}\color{blue}{ (ntws)}\index{\color{blue}\foreignlanguage{arabic}{ف.ج.م}\color{blue}{ (ntws)}}} 

{\setlength\topsep{0pt}\textbf{\foreignlanguage{arabic}{فَيجَم}}\ {\color{gray}\texttt{/\sffamily {{\sffamily feː(dʒ)am}}/}\color{black}}\ \textsc{noun}\ [m.]\ \color{gray}(msa. \foreignlanguage{arabic}{سذاب أذفر (يُغْلى ويُشْرَب لعلاج الجلطات)}~\foreignlanguage{arabic}{\textbf{١.}})\color{black}\ \textbf{1.}~Common rue (people boil it and drink it as a remedy for strokes)\  \begin{flushright}\color{gray}\foreignlanguage{arabic}{\textbf{\underline{\foreignlanguage{arabic}{أمثلة}}}: ضلك اغليلها فيجَم واعمليلها تبخيرة وان شاء الله بتتحسن}\end{flushright}\color{black}} \vspace{2mm}

\vspace{-3mm}
\markboth{\color{blue}\foreignlanguage{arabic}{ف.ج.ن}\color{blue}{ (ntws)}}{\color{blue}\foreignlanguage{arabic}{ف.ج.ن}\color{blue}{ (ntws)}}\subsection*{\color{blue}\foreignlanguage{arabic}{ف.ج.ن}\color{blue}{ (ntws)}\index{\color{blue}\foreignlanguage{arabic}{ف.ج.ن}\color{blue}{ (ntws)}}} 

{\setlength\topsep{0pt}\textbf{\foreignlanguage{arabic}{فَيجَن}}\ {\color{gray}\texttt{/\sffamily {{\sffamily feː(dʒ)an}}/}\color{black}}\ \textsc{noun}\ [m.]\ \color{gray}(msa. \foreignlanguage{arabic}{سذاب أذفر (يُغْلى ويُشْرَب لعلاج الجلطات)}~\foreignlanguage{arabic}{\textbf{١.}})\color{black}\ \textbf{1.}~Common rue (people boil it and drink it as a remedy for strokes)\ } \vspace{2mm}

\vspace{-3mm}
\markboth{\color{blue}\foreignlanguage{arabic}{ف.ج.و}\color{blue}{}}{\color{blue}\foreignlanguage{arabic}{ف.ج.و}\color{blue}{}}\subsection*{\color{blue}\foreignlanguage{arabic}{ف.ج.و}\color{blue}{}\index{\color{blue}\foreignlanguage{arabic}{ف.ج.و}\color{blue}{}}} 

{\setlength\topsep{0pt}\textbf{\foreignlanguage{arabic}{فَجْوِة}}\ {\color{gray}\texttt{/\sffamily {{\sffamily fa(dʒ)we}}/}\color{black}}\ \textsc{noun}\ [f.]\ \color{gray}(msa. \foreignlanguage{arabic}{فَجْوَة}~\foreignlanguage{arabic}{\textbf{١.}})\color{black}\ \textbf{1.}~gap\  \begin{flushright}\color{gray}\foreignlanguage{arabic}{\textbf{\underline{\foreignlanguage{arabic}{أمثلة}}}: في مية فَجْوِة بيني وبينك طلقني خلاص}\end{flushright}\color{black}} \vspace{2mm}

\vspace{-3mm}
\markboth{\color{blue}\foreignlanguage{arabic}{ف.ج.و.ل}\color{blue}{}}{\color{blue}\foreignlanguage{arabic}{ف.ج.و.ل}\color{blue}{}}\subsection*{\color{blue}\foreignlanguage{arabic}{ف.ج.و.ل}\color{blue}{}\index{\color{blue}\foreignlanguage{arabic}{ف.ج.و.ل}\color{blue}{}}} 

{\setlength\topsep{0pt}\textbf{\foreignlanguage{arabic}{فَجْوَل}}\ {\color{gray}\texttt{/\sffamily {{\sffamily fa(dʒ)wal}}/}\color{black}}\ \textsc{verb}\ [p.]\ (src. \color{gray}\foreignlanguage{arabic}{الخليل}\color{black})\ \color{gray}(msa. \foreignlanguage{arabic}{انحسرت الغيوم}~\foreignlanguage{arabic}{\textbf{١.}})\color{black}\ \textbf{1.}~the clouds receded\ \ $\bullet$\ \ \setlength\topsep{0pt}\textbf{\foreignlanguage{arabic}{فَجْوِل}}\ {\color{gray}\texttt{/\sffamily {{\sffamily fa(dʒ)wil}}/}\color{black}}\ [c.]\ \ $\bullet$\ \ \setlength\topsep{0pt}\textbf{\foreignlanguage{arabic}{يفَجْوِل}}\ {\color{gray}\texttt{/\sffamily {{\sffamily jfa(dʒ)wil}}/}\color{black}}\ [i.]\  \begin{flushright}\color{gray}\foreignlanguage{arabic}{\textbf{\underline{\foreignlanguage{arabic}{أمثلة}}}: أبوي طلع بس فَجْوَلَت الدنيا}\end{flushright}\color{black}} \vspace{2mm}

{\setlength\topsep{0pt}\textbf{\foreignlanguage{arabic}{فَجْوَلِة}}\ {\color{gray}\texttt{/\sffamily {{\sffamily fa(dʒ)wale}}/}\color{black}}\ \textsc{noun}\ [f.]\ (src. \color{gray}\foreignlanguage{arabic}{الخليل}\color{black})\ \color{gray}(msa. \foreignlanguage{arabic}{عندما تنحسر الغيوم}~\foreignlanguage{arabic}{\textbf{١.}})\color{black}\ \textbf{1.}~when clouds recede\  \begin{flushright}\color{gray}\foreignlanguage{arabic}{\textbf{\underline{\foreignlanguage{arabic}{أمثلة}}}: أحسن وقت الها وقت الفَجْوَلِة أو قبلها بشوي}\end{flushright}\color{black}} \vspace{2mm}

\vspace{-3mm}
\markboth{\color{blue}\foreignlanguage{arabic}{ف.ح.ت}\color{blue}{}}{\color{blue}\foreignlanguage{arabic}{ف.ح.ت}\color{blue}{}}\subsection*{\color{blue}\foreignlanguage{arabic}{ف.ح.ت}\color{blue}{}\index{\color{blue}\foreignlanguage{arabic}{ف.ح.ت}\color{blue}{}}} 

{\setlength\topsep{0pt}\textbf{\foreignlanguage{arabic}{فَحَت}}\ {\color{gray}\texttt{/\sffamily {{\sffamily faħat}}/}\color{black}}\ \textsc{verb}\ [p.]\ \textbf{1.}~dig  \textbf{2.}~dig up.  \textbf{3.}~excavate\ \ $\bullet$\ \ \setlength\topsep{0pt}\textbf{\foreignlanguage{arabic}{اِفْحَت}}\ {\color{gray}\texttt{/\sffamily {{\sffamily ʔifħat}}/}\color{black}}\ [c.]\ \ $\bullet$\ \ \setlength\topsep{0pt}\textbf{\foreignlanguage{arabic}{يِفْحَت}}\ {\color{gray}\texttt{/\sffamily {{\sffamily jifħat}}/}\color{black}}\ [i.]\ \color{gray}(msa. \foreignlanguage{arabic}{يَحْفِر}~\foreignlanguage{arabic}{\textbf{١.}})\color{black}\  \begin{flushright}\color{gray}\foreignlanguage{arabic}{\textbf{\underline{\foreignlanguage{arabic}{أمثلة}}}: اِفْحَت هون بلكي بنلاقيلنا كنز وبنصير أغنياء هههه}\end{flushright}\color{black}} \vspace{2mm}

{\setlength\topsep{0pt}\textbf{\foreignlanguage{arabic}{فَحَّت}}\ {\color{gray}\texttt{/\sffamily {{\sffamily faħħat}}/}\color{black}}\ \textsc{verb}\ [p.]\ \textbf{1.}~dig  \textbf{2.}~dig up.  \textbf{3.}~excavate (with force and repeatedly)\ \ $\bullet$\ \ \setlength\topsep{0pt}\textbf{\foreignlanguage{arabic}{فَحِّت}}\ {\color{gray}\texttt{/\sffamily {{\sffamily faħħit}}/}\color{black}}\ [c.]\ \ $\bullet$\ \ \setlength\topsep{0pt}\textbf{\foreignlanguage{arabic}{يفَحِّت}}\ {\color{gray}\texttt{/\sffamily {{\sffamily jfaħħit}}/}\color{black}}\ [i.]\ \color{gray}(msa. \foreignlanguage{arabic}{يَحْفِر بقوة وبشكل متكرر}~\foreignlanguage{arabic}{\textbf{١.}})\color{black}\  \begin{flushright}\color{gray}\foreignlanguage{arabic}{\textbf{\underline{\foreignlanguage{arabic}{أمثلة}}}: ضلينا نفَحِّت من الصبح للمغربيات آخر شي فزرنا ماسورة تحت الأرض}\end{flushright}\color{black}} \vspace{2mm}

\vspace{-3mm}
\markboth{\color{blue}\foreignlanguage{arabic}{ف.ح.ج}\color{blue}{}}{\color{blue}\foreignlanguage{arabic}{ف.ح.ج}\color{blue}{}}\subsection*{\color{blue}\foreignlanguage{arabic}{ف.ح.ج}\color{blue}{}\index{\color{blue}\foreignlanguage{arabic}{ف.ح.ج}\color{blue}{}}} 

{\setlength\topsep{0pt}\textbf{\foreignlanguage{arabic}{فَاحَج}}\ {\color{gray}\texttt{/\sffamily {{\sffamily faːħa(dʒ)}}/}\color{black}}\ \textsc{verb}\ [p.]\ \textbf{1.}~walk with long steps.  \textbf{2.}~stride\ \ $\bullet$\ \ \setlength\topsep{0pt}\textbf{\foreignlanguage{arabic}{فَاحِج}}\ {\color{gray}\texttt{/\sffamily {{\sffamily faːħi(dʒ)}}/}\color{black}}\ [c.]\ \ $\bullet$\ \ \setlength\topsep{0pt}\textbf{\foreignlanguage{arabic}{يفَاحِج}}\ {\color{gray}\texttt{/\sffamily {{\sffamily jfaːħi(dʒ)}}/}\color{black}}\ [i.]\  \begin{flushright}\color{gray}\foreignlanguage{arabic}{\textbf{\underline{\foreignlanguage{arabic}{أمثلة}}}: مالك بِتفاحِج مْفاحَجِة مثل المرة الوالد جديد}\end{flushright}\color{black}} \vspace{2mm}

{\setlength\topsep{0pt}\textbf{\foreignlanguage{arabic}{فَحَج}}\ {\color{gray}\texttt{/\sffamily {{\sffamily faħa(dʒ)}}/}\color{black}}\ \textsc{verb}\ [p.]\ \textbf{1.}~open one's legs widely.  \textbf{2.}~avoid stepping on sth\ \ $\bullet$\ \ \setlength\topsep{0pt}\textbf{\foreignlanguage{arabic}{اِفْحَج}}\ {\color{gray}\texttt{/\sffamily {{\sffamily ʔifħa(dʒ)}}/}\color{black}}\ [c.]\ \ $\bullet$\ \ \setlength\topsep{0pt}\textbf{\foreignlanguage{arabic}{يِفْحَج}}\ {\color{gray}\texttt{/\sffamily {{\sffamily jifħa(dʒ)}}/}\color{black}}\ [i.]\  \begin{flushright}\color{gray}\foreignlanguage{arabic}{\textbf{\underline{\foreignlanguage{arabic}{أمثلة}}}: اِفْحَج عن الخبزة بلاش تدعس عليعا حرام}\end{flushright}\color{black}} \vspace{2mm}

{\setlength\topsep{0pt}\textbf{\foreignlanguage{arabic}{فَحَّج}}\ {\color{gray}\texttt{/\sffamily {{\sffamily faħħa(dʒ)}}/}\color{black}}\ \textsc{verb}\ [p.]\ \textbf{1.}~spread legs.  \textbf{2.}~sit in a W-sitting position\ \ $\bullet$\ \ \setlength\topsep{0pt}\textbf{\foreignlanguage{arabic}{فَحِّج}}\ {\color{gray}\texttt{/\sffamily {{\sffamily faħħi(dʒ)}}/}\color{black}}\ [c.]\ \ $\bullet$\ \ \setlength\topsep{0pt}\textbf{\foreignlanguage{arabic}{يفَحِّج}}\ {\color{gray}\texttt{/\sffamily {{\sffamily jfaħħi(dʒ)}}/}\color{black}}\ [i.]\  \begin{flushright}\color{gray}\foreignlanguage{arabic}{\textbf{\underline{\foreignlanguage{arabic}{أمثلة}}}: تفحِّجِش اجريك اقعد منيح زي الناس}\end{flushright}\color{black}} \vspace{2mm}

{\setlength\topsep{0pt}\textbf{\foreignlanguage{arabic}{فَحْجِة}}\ {\color{gray}\texttt{/\sffamily {{\sffamily faħ(dʒ)e}}/}\color{black}}\ \textsc{noun}\ [f.]\ \textbf{1.}~the state of opening one's legs widely\  \begin{flushright}\color{gray}\foreignlanguage{arabic}{\textbf{\underline{\foreignlanguage{arabic}{أمثلة}}}: كل واحد فيكم يِفْحَج فَحْجِة قد البلد وبعديها بنحط حجارة مكان إِجرك}\end{flushright}\color{black}} \vspace{2mm}

{\setlength\topsep{0pt}\textbf{\foreignlanguage{arabic}{مْفَاحَجِة}}\ {\color{gray}\texttt{/\sffamily {{\sffamily mfaːħa(dʒ)e}}/}\color{black}}\ \textsc{noun}\ [f.]\ \textbf{1.}~walking with long steps.  \textbf{2.}~striding\  \begin{flushright}\color{gray}\foreignlanguage{arabic}{\textbf{\underline{\foreignlanguage{arabic}{أمثلة}}}: أخوها بيمشيش عادي زينا. بيقعد يفاحِج مْفاحَجِة}\end{flushright}\color{black}} \vspace{2mm}

{\setlength\topsep{0pt}\textbf{\foreignlanguage{arabic}{مْفَحِّج}}\ {\color{gray}\texttt{/\sffamily {{\sffamily mfaħħi(dʒ)}}/}\color{black}}\ \textsc{noun\textunderscore act}\ [m.]\ \color{gray}(msa. \foreignlanguage{arabic}{جلس على شكل حرف} W~\foreignlanguage{arabic}{\textbf{١.}})\color{black}\ \textbf{1.}~spreading legs.  \textbf{2.}~sitting in a W-sitting position\ \ $\bullet$\ \ \textsc{ph.} \color{gray} \foreignlanguage{arabic}{مْفَحِّج عَمِية خَازُوق}\color{black}\ {\color{gray}\texttt{/{\sffamily mfaħħi(dʒ) ʕamiːt xazuː(q)}/}\color{black}}\ \color{gray} (msa. \foreignlanguage{arabic}{يقوم بأكثر من شيئ في نفس الوقت}~\foreignlanguage{arabic}{\textbf{١.}})\color{black}\ \textbf{1.}~It is an idiomatic expression that means that sb is engaged in more than one task at a time, i.e. Jack of all trades, master of none\  \begin{flushright}\color{gray}\foreignlanguage{arabic}{\textbf{\underline{\foreignlanguage{arabic}{أمثلة}}}: جوزك مْفحِّج عمِية خازُوق\ $\bullet$\ \  دعست عاجره بالغلط وهو قاعد و مِفحِّج اجريه}\end{flushright}\color{black}} \vspace{2mm}

\vspace{-3mm}
\markboth{\color{blue}\foreignlanguage{arabic}{ف.ح.ر}\color{blue}{}}{\color{blue}\foreignlanguage{arabic}{ف.ح.ر}\color{blue}{}}\subsection*{\color{blue}\foreignlanguage{arabic}{ف.ح.ر}\color{blue}{}\index{\color{blue}\foreignlanguage{arabic}{ف.ح.ر}\color{blue}{}}} 

{\setlength\topsep{0pt}\textbf{\foreignlanguage{arabic}{فَحَر}}\ {\color{gray}\texttt{/\sffamily {{\sffamily faħar}}/}\color{black}}\ \textsc{verb}\ [p.]\ \textbf{1.}~hollow out (zucchine or eggplant)\ \ $\bullet$\ \ \setlength\topsep{0pt}\textbf{\foreignlanguage{arabic}{اِفْحَر}}\ {\color{gray}\texttt{/\sffamily {{\sffamily ʔifħar}}/}\color{black}}\ [c.]\ \ $\bullet$\ \ \setlength\topsep{0pt}\textbf{\foreignlanguage{arabic}{يِفْحَر}}\ {\color{gray}\texttt{/\sffamily {{\sffamily jifħar}}/}\color{black}}\ [i.]\  \begin{flushright}\color{gray}\foreignlanguage{arabic}{\textbf{\underline{\foreignlanguage{arabic}{أمثلة}}}: إِمي بتفحَر كوسا، أناديلك اياها}\end{flushright}\color{black}} \vspace{2mm}

{\setlength\topsep{0pt}\textbf{\foreignlanguage{arabic}{مَفْحُور}}\ {\color{gray}\texttt{/\sffamily {{\sffamily mafħuːr}}/}\color{black}}\ \textsc{noun\textunderscore pass}\ \textbf{1.}~hollowed out (zucchine or eggplant)\  \begin{flushright}\color{gray}\foreignlanguage{arabic}{\textbf{\underline{\foreignlanguage{arabic}{أمثلة}}}: اشتريلنا كوسا مَفْحُور وجاهِز بدل ما احنا نقعد نفحر فيه}\end{flushright}\color{black}} \vspace{2mm}

\vspace{-3mm}
\markboth{\color{blue}\foreignlanguage{arabic}{ف.ح.ص}\color{blue}{}}{\color{blue}\foreignlanguage{arabic}{ف.ح.ص}\color{blue}{}}\subsection*{\color{blue}\foreignlanguage{arabic}{ف.ح.ص}\color{blue}{}\index{\color{blue}\foreignlanguage{arabic}{ف.ح.ص}\color{blue}{}}} 

{\setlength\topsep{0pt}\textbf{\foreignlanguage{arabic}{تْفَحَّص}}\ {\color{gray}\texttt{/\sffamily {{\sffamily tfaħħasˤ}}/}\color{black}}\ \textsc{verb}\ [p.]\ \textbf{1.}~scrutinize  \textbf{2.}~examine sth closely\ \ $\bullet$\ \ \setlength\topsep{0pt}\textbf{\foreignlanguage{arabic}{اِتْفَحَّص}}\ {\color{gray}\texttt{/\sffamily {{\sffamily ʔitfaħħasˤ}}/}\color{black}}\ [c.]\ \ $\bullet$\ \ \setlength\topsep{0pt}\textbf{\foreignlanguage{arabic}{يِتْفَحَّص}}\ {\color{gray}\texttt{/\sffamily {{\sffamily jitfaħħasˤ}}/}\color{black}}\ [i.]\ \color{gray}(msa. \foreignlanguage{arabic}{يَتَفَحَّص}~\foreignlanguage{arabic}{\textbf{١.}})\color{black}\  \begin{flushright}\color{gray}\foreignlanguage{arabic}{\textbf{\underline{\foreignlanguage{arabic}{أمثلة}}}: خلي الموظف يِتْفَحَّص الأوراق ويخبرك بالنتيجة}\end{flushright}\color{black}} \vspace{2mm}

{\setlength\topsep{0pt}\textbf{\foreignlanguage{arabic}{فَحَص}}\ {\color{gray}\texttt{/\sffamily {{\sffamily faħasˤ}}/}\color{black}}\ \textsc{verb}\ [p.]\ \textbf{1.}~examine  \textbf{2.}~test  \textbf{3.}~see a doctor\ \ $\bullet$\ \ \setlength\topsep{0pt}\textbf{\foreignlanguage{arabic}{اِفْحَص}}\ {\color{gray}\texttt{/\sffamily {{\sffamily ʔifħasˤ}}/}\color{black}}\ [c.]\ \ $\bullet$\ \ \setlength\topsep{0pt}\textbf{\foreignlanguage{arabic}{يِفْحَص}}\ {\color{gray}\texttt{/\sffamily {{\sffamily jifħasˤ}}/}\color{black}}\ [i.]\ \color{gray}(msa. \foreignlanguage{arabic}{يَفْحَص}~\foreignlanguage{arabic}{\textbf{١.}})\color{black}\  \begin{flushright}\color{gray}\foreignlanguage{arabic}{\textbf{\underline{\foreignlanguage{arabic}{أمثلة}}}: الميكانيكي بيِفْحَص السيارة بساعة\ $\bullet$\ \  روح اِفْحَص عند الدكتور أحسن}\end{flushright}\color{black}} \vspace{2mm}

{\setlength\topsep{0pt}\textbf{\foreignlanguage{arabic}{فَحِص}}\ {\color{gray}\texttt{/\sffamily {{\sffamily faħisˤ}}/}\color{black}}\ \textsc{noun}\ [m.]\ \color{gray}(msa. \foreignlanguage{arabic}{فَحْص}~\foreignlanguage{arabic}{\textbf{١.}})\color{black}\ \textbf{1.}~scrutiny  \textbf{2.}~test  \textbf{3.}~medical test\ } \vspace{2mm}

\vspace{-3mm}
\markboth{\color{blue}\foreignlanguage{arabic}{ف.ح.ك.ش}\color{blue}{}}{\color{blue}\foreignlanguage{arabic}{ف.ح.ك.ش}\color{blue}{}}\subsection*{\color{blue}\foreignlanguage{arabic}{ف.ح.ك.ش}\color{blue}{}\index{\color{blue}\foreignlanguage{arabic}{ف.ح.ك.ش}\color{blue}{}}} 

{\setlength\topsep{0pt}\textbf{\foreignlanguage{arabic}{فَحْكَش}}\ {\color{gray}\texttt{/\sffamily {{\sffamily faħkaʃ}}/}\color{black}}\ \textsc{verb}\ [p.]\ \textbf{1.}~make sth disorganized.  \textbf{2.}~mess sth up\ \ $\bullet$\ \ \setlength\topsep{0pt}\textbf{\foreignlanguage{arabic}{فَحْكِش}}\ {\color{gray}\texttt{/\sffamily {{\sffamily faħkiʃ}}/}\color{black}}\ [c.]\ \ $\bullet$\ \ \setlength\topsep{0pt}\textbf{\foreignlanguage{arabic}{يفَحْكِش}}\ {\color{gray}\texttt{/\sffamily {{\sffamily jfaħkiʃ}}/}\color{black}}\ [i.]\  \begin{flushright}\color{gray}\foreignlanguage{arabic}{\textbf{\underline{\foreignlanguage{arabic}{أمثلة}}}: اجوا ساعتين فَحْكَشوا الدار عن شهر وتيسروا الله ييسر امورهم}\end{flushright}\color{black}} \vspace{2mm}

{\setlength\topsep{0pt}\textbf{\foreignlanguage{arabic}{مْفَحْكَش}}\ {\color{gray}\texttt{/\sffamily {{\sffamily mfaħkaʃ}}/}\color{black}}\ \textsc{adj}\ [m.]\ \color{gray}(msa. \foreignlanguage{arabic}{غير منظَّم}~\foreignlanguage{arabic}{\textbf{١.}})\color{black}\ \textbf{1.}~disorganized  \textbf{2.}~messed up\  \begin{flushright}\color{gray}\foreignlanguage{arabic}{\textbf{\underline{\foreignlanguage{arabic}{أمثلة}}}: الغرفة مْفَحْكَشِة وحالتها حالة والله بدنا يوم للتعزيل}\end{flushright}\color{black}} \vspace{2mm}

\vspace{-3mm}
\markboth{\color{blue}\foreignlanguage{arabic}{ف.ح.ل}\color{blue}{}}{\color{blue}\foreignlanguage{arabic}{ف.ح.ل}\color{blue}{}}\subsection*{\color{blue}\foreignlanguage{arabic}{ف.ح.ل}\color{blue}{}\index{\color{blue}\foreignlanguage{arabic}{ف.ح.ل}\color{blue}{}}} 

{\setlength\topsep{0pt}\textbf{\foreignlanguage{arabic}{اِسْتَفْحَل}}\ {\color{gray}\texttt{/\sffamily {{\sffamily ʔistafħal}}/}\color{black}}\ \textsc{verb}\ [p.]\ \textbf{1.}~aggravate  \textbf{2.}~exacerbate  \textbf{3.}~\ \ $\bullet$\ \ \setlength\topsep{0pt}\textbf{\foreignlanguage{arabic}{اِسْتَفْحِل}}\ {\color{gray}\texttt{/\sffamily {{\sffamily ʔistafħil}}/}\color{black}}\ [c.]\ \ $\bullet$\ \ \setlength\topsep{0pt}\textbf{\foreignlanguage{arabic}{يِسْتَفْحِل}}\ {\color{gray}\texttt{/\sffamily {{\sffamily jistafħil}}/}\color{black}}\ [i.]\  \begin{flushright}\color{gray}\foreignlanguage{arabic}{\textbf{\underline{\foreignlanguage{arabic}{أمثلة}}}: اِسْتَفْحَلت المشكلة أكثر شي لما رفضت توخذ ولادها}\end{flushright}\color{black}} \vspace{2mm}

{\setlength\topsep{0pt}\textbf{\foreignlanguage{arabic}{فَحِل}}\ {\color{gray}\texttt{/\sffamily {{\sffamily faħil}}/}\color{black}}\ \textsc{noun}\ [m.]\ \color{gray}(msa. \foreignlanguage{arabic}{فَحْل}~\foreignlanguage{arabic}{\textbf{١.}})\color{black}\ \textbf{1.}~stallion\ \ $\bullet$\ \ \setlength\topsep{0pt}\textbf{\foreignlanguage{arabic}{فْحُول}}\ {\color{gray}\texttt{/\sffamily {{\sffamily fħuːl}}/}\color{black}}\ [pl.]\ } \vspace{2mm}

{\setlength\topsep{0pt}\textbf{\foreignlanguage{arabic}{مُسْتَفْحِل}}\ {\color{gray}\texttt{/\sffamily {{\sffamily mistafħil}}/}\color{black}}\ \textsc{adj}\ [m.]\ \textbf{1.}~growing  \textbf{2.}~massive\  \begin{flushright}\color{gray}\foreignlanguage{arabic}{\textbf{\underline{\foreignlanguage{arabic}{أمثلة}}}: هاي العيلة عندها غباء مُسْتَفْحِل}\end{flushright}\color{black}} \vspace{2mm}

{\setlength\topsep{0pt}\textbf{\foreignlanguage{arabic}{مْفَحْلَل}}\ {\color{gray}\texttt{/\sffamily {{\sffamily mfaħlal}}/}\color{black}}\ \textsc{adj/noun}\ \textbf{1.}~very vicious and rude\  \begin{flushright}\color{gray}\foreignlanguage{arabic}{\textbf{\underline{\foreignlanguage{arabic}{أمثلة}}}: مرتك مْفَحْلَل. فش حد بيقدر عليها غير ربنا.}\end{flushright}\color{black}} \vspace{2mm}

\vspace{-3mm}
\markboth{\color{blue}\foreignlanguage{arabic}{ف.ح.م}\color{blue}{}}{\color{blue}\foreignlanguage{arabic}{ف.ح.م}\color{blue}{}}\subsection*{\color{blue}\foreignlanguage{arabic}{ف.ح.م}\color{blue}{}\index{\color{blue}\foreignlanguage{arabic}{ف.ح.م}\color{blue}{}}} 

{\setlength\topsep{0pt}\textbf{\foreignlanguage{arabic}{أَفْحَم}}\ {\color{gray}\texttt{/\sffamily {{\sffamily ʔafħam}}/}\color{black}}\ \textsc{verb}\ [p.]\ \textbf{1.}~say sth cogent, reasonable and correct. Theredore, the speaker demolishes the argument of the hearer\ \ $\bullet$\ \ \setlength\topsep{0pt}\textbf{\foreignlanguage{arabic}{اِفْحِم}}\ {\color{gray}\texttt{/\sffamily {{\sffamily ʔifħim}}/}\color{black}}\ [c.]\ \ $\bullet$\ \ \setlength\topsep{0pt}\textbf{\foreignlanguage{arabic}{يِفْحِم}}\ {\color{gray}\texttt{/\sffamily {{\sffamily jifħim}}/}\color{black}}\ [i.]\  \begin{flushright}\color{gray}\foreignlanguage{arabic}{\textbf{\underline{\foreignlanguage{arabic}{أمثلة}}}: أفْحَمتني بصراحة!}\end{flushright}\color{black}} \vspace{2mm}

{\setlength\topsep{0pt}\textbf{\foreignlanguage{arabic}{فَحِم}}\footnote{Collective noun}\ \ {\color{gray}\texttt{/\sffamily {{\sffamily faħim}}/}\color{black}}\ \textsc{noun}\ [m.]\ \color{gray}(msa. \foreignlanguage{arabic}{فَحْم}~\foreignlanguage{arabic}{\textbf{١.}})\color{black}\ \textbf{1.}~charcoal\  \begin{flushright}\color{gray}\foreignlanguage{arabic}{\textbf{\underline{\foreignlanguage{arabic}{أمثلة}}}: اكتشفت إِنه الفَحِم اللي عنا بيكفِّيش للشوي}\end{flushright}\color{black}} \vspace{2mm}

{\setlength\topsep{0pt}\textbf{\foreignlanguage{arabic}{فَحَّم}}\ {\color{gray}\texttt{/\sffamily {{\sffamily faħħam}}/}\color{black}}\ \textsc{verb}\ [p.]\ \textbf{1.}~be charred.  \textbf{2.}~cry hesterically and uncontrollably\ \ $\bullet$\ \ \setlength\topsep{0pt}\textbf{\foreignlanguage{arabic}{فَحِّم}}\ {\color{gray}\texttt{/\sffamily {{\sffamily faħħim}}/}\color{black}}\ [c.]\ \ $\bullet$\ \ \setlength\topsep{0pt}\textbf{\foreignlanguage{arabic}{يفَحِّم}}\ {\color{gray}\texttt{/\sffamily {{\sffamily jfaħħim}}/}\color{black}}\ [i.]\ \color{gray}(msa. \foreignlanguage{arabic}{يبكي بشكل هستيري}~\foreignlanguage{arabic}{\textbf{٢.}}  \foreignlanguage{arabic}{يَتَفَحَّم}~\foreignlanguage{arabic}{\textbf{١.}})\color{black}\  \begin{flushright}\color{gray}\foreignlanguage{arabic}{\textbf{\underline{\foreignlanguage{arabic}{أمثلة}}}: بدي اياك تفَحِّملي الخبزة عشان عندي تجربة بالعلوم}\end{flushright}\color{black}} \vspace{2mm}

{\setlength\topsep{0pt}\textbf{\foreignlanguage{arabic}{فَحْمِة}}\footnote{Unit noun}\ \ {\color{gray}\texttt{/\sffamily {{\sffamily faħme}}/}\color{black}}\ \textsc{noun}\ [f.]\ \color{gray}(msa. \foreignlanguage{arabic}{فَحْمَة}~\foreignlanguage{arabic}{\textbf{١.}})\color{black}\ \textbf{1.}~one piece of charcoal\ \ $\bullet$\ \ \textsc{ph.} \color{gray} \foreignlanguage{arabic}{الجوز بَالبيت رحمة حتى لو كَان فَحْمِة}\color{black}\ {\color{gray}\texttt{/{\sffamily ʔil(dʒ)oːz bilbeːt raħme ħatta law kaːn faħme}/}\color{black}}\ \color{gray} (msa. \foreignlanguage{arabic}{أهمية وجود الرجل بحياة المرأة}~\foreignlanguage{arabic}{\textbf{١.}})\color{black}\ \textbf{1.}~It is an idiomatic expression that means that the man's role in the house is very important. Therefore, they use this expression to encourage women to get married and not to set up high expectations for their future husbands.\ } \vspace{2mm}

{\setlength\topsep{0pt}\textbf{\foreignlanguage{arabic}{مُفْحِم}}\ {\color{gray}\texttt{/\sffamily {{\sffamily mufħim}}/}\color{black}}\ \textsc{adj}\ [m.]\ \color{gray}(msa. \foreignlanguage{arabic}{مُفْحِم}~\foreignlanguage{arabic}{\textbf{١.}})\color{black}\ \textbf{1.}~cogent\  \begin{flushright}\color{gray}\foreignlanguage{arabic}{\textbf{\underline{\foreignlanguage{arabic}{أمثلة}}}: ردِّيت عليه رد مُفْحِم}\end{flushright}\color{black}} \vspace{2mm}

{\setlength\topsep{0pt}\textbf{\foreignlanguage{arabic}{مِفْحَمِة}}\ {\color{gray}\texttt{/\sffamily {{\sffamily mifħame}}/}\color{black}}\ \textsc{noun}\ [f.]\ \color{gray}(msa. \foreignlanguage{arabic}{منجم الفحم}~\foreignlanguage{arabic}{\textbf{١.}})\color{black}\ \textbf{1.}~coal mine\ \ $\bullet$\ \ \setlength\topsep{0pt}\textbf{\foreignlanguage{arabic}{مَفَاحِم}}\ {\color{gray}\texttt{/\sffamily {{\sffamily mafaːħim}}/}\color{black}}\ [pl.]\  \begin{flushright}\color{gray}\foreignlanguage{arabic}{\textbf{\underline{\foreignlanguage{arabic}{أمثلة}}}: أهله كثير أغنيا عندهم مِفْحَمِة و مِحْجَر}\end{flushright}\color{black}} \vspace{2mm}

{\setlength\topsep{0pt}\textbf{\foreignlanguage{arabic}{مْفَحِّم}}\ {\color{gray}\texttt{/\sffamily {{\sffamily mfaħħim}}/}\color{black}}\ \textsc{noun\textunderscore pass}\ \textbf{1.}~being charred.  \textbf{2.}~crying hesterically and uncontrollably\ \ $\bullet$\ \ \textsc{ph.} \color{gray} \foreignlanguage{arabic}{مفحم عيَاط}\color{black}\ {\color{gray}\texttt{/{\sffamily mfaħħim ʕjaːtˤ}/}\color{black}}\ \textbf{1.}~cry uncontrollably\  \begin{flushright}\color{gray}\foreignlanguage{arabic}{\textbf{\underline{\foreignlanguage{arabic}{أمثلة}}}: لما لقوه أهله بقى مْفَحِّم عْياط شربوه بطاسة الرجفة عشان يهدا وما ينقطع خلفه عكبر}\end{flushright}\color{black}} \vspace{2mm}

\vspace{-3mm}
\markboth{\color{blue}\foreignlanguage{arabic}{ف.خ.خ}\color{blue}{}}{\color{blue}\foreignlanguage{arabic}{ف.خ.خ}\color{blue}{}}\subsection*{\color{blue}\foreignlanguage{arabic}{ف.خ.خ}\color{blue}{}\index{\color{blue}\foreignlanguage{arabic}{ف.خ.خ}\color{blue}{}}} 

{\setlength\topsep{0pt}\textbf{\foreignlanguage{arabic}{تْفَخَّخ}}\ {\color{gray}\texttt{/\sffamily {{\sffamily tfaxxax}}/}\color{black}}\ \textsc{verb}\ [p.]\ \textbf{1.}~be trapped\ \ $\bullet$\ \ \setlength\topsep{0pt}\textbf{\foreignlanguage{arabic}{اِتْفَخَّخ}}\ {\color{gray}\texttt{/\sffamily {{\sffamily ʔitfaxxax}}/}\color{black}}\ [c.]\ \ $\bullet$\ \ \setlength\topsep{0pt}\textbf{\foreignlanguage{arabic}{يِتْفَخَّخ}}\ {\color{gray}\texttt{/\sffamily {{\sffamily jitfaxxax}}/}\color{black}}\ [i.]\ } \vspace{2mm}

{\setlength\topsep{0pt}\textbf{\foreignlanguage{arabic}{فَخّ}}\ {\color{gray}\texttt{/\sffamily {{\sffamily faxx}}/}\color{black}}\ \textsc{noun}\ [m.]\ \color{gray}(msa. \foreignlanguage{arabic}{فَخ}~\foreignlanguage{arabic}{\textbf{١.}})\color{black}\ \textbf{1.}~trap\ \ $\bullet$\ \ \setlength\topsep{0pt}\textbf{\foreignlanguage{arabic}{فْخُوخ}}\ {\color{gray}\texttt{/\sffamily {{\sffamily fxuːx}}/}\color{black}}\ [pl.]\ \ $\bullet$\ \ \setlength\topsep{0pt}\textbf{\foreignlanguage{arabic}{أَفْخَاخ}}\ {\color{gray}\texttt{/\sffamily {{\sffamily ʔafxaːx}}/}\color{black}}\ [pl.]\  \begin{flushright}\color{gray}\foreignlanguage{arabic}{\textbf{\underline{\foreignlanguage{arabic}{أمثلة}}}: المنطقة هاي كلها أَفْخاخ وألغام\ $\bullet$\ \  علمني أبوي كيف أعمل فخوخ للقنافِذ\ $\bullet$\ \  دير بالك ما توثع بالفَخ أنت كمان}\end{flushright}\color{black}} \vspace{2mm}

{\setlength\topsep{0pt}\textbf{\foreignlanguage{arabic}{فَخَّة}}\ {\color{gray}\texttt{/\sffamily {{\sffamily faxxa}}/}\color{black}}\ \textsc{noun}\ [f.]\ \color{gray}(msa. \foreignlanguage{arabic}{مصيدة}~\foreignlanguage{arabic}{\textbf{١.}})\color{black}\ \textbf{1.}~hunting trap\  \begin{flushright}\color{gray}\foreignlanguage{arabic}{\textbf{\underline{\foreignlanguage{arabic}{أمثلة}}}: بقينا نعمل للقنفذ فخَّة نحطِّله زواكي}\end{flushright}\color{black}} \vspace{2mm}

{\setlength\topsep{0pt}\textbf{\foreignlanguage{arabic}{فَخَّخ}}\ {\color{gray}\texttt{/\sffamily {{\sffamily faxxax}}/}\color{black}}\ \textsc{verb}\ [p.]\ \textbf{1.}~trap  \textbf{2.}~make sth explosives-laden\ \ $\bullet$\ \ \setlength\topsep{0pt}\textbf{\foreignlanguage{arabic}{فَخِّخ}}\ {\color{gray}\texttt{/\sffamily {{\sffamily faxxix}}/}\color{black}}\ [c.]\ \ $\bullet$\ \ \setlength\topsep{0pt}\textbf{\foreignlanguage{arabic}{يفَخِّخ}}\ {\color{gray}\texttt{/\sffamily {{\sffamily jfaxxix}}/}\color{black}}\ [i.]\ \color{gray}(msa. \foreignlanguage{arabic}{يُفَخِّخ}~\foreignlanguage{arabic}{\textbf{١.}})\color{black}\ } \vspace{2mm}

{\setlength\topsep{0pt}\textbf{\foreignlanguage{arabic}{مُفَخَّخ}}\ {\color{gray}\texttt{/\sffamily {{\sffamily mufaxxax}}/}\color{black}}\ \textsc{adj}\ [m.]\ \textbf{1.}~explosives-laden  \textbf{2.}~trapped\  \begin{flushright}\color{gray}\foreignlanguage{arabic}{\textbf{\underline{\foreignlanguage{arabic}{أمثلة}}}: ياحرام بالعراق انفجرت سيارة مُفَخَّخة وتوفوا أربعة}\end{flushright}\color{black}} \vspace{2mm}

\vspace{-3mm}
\markboth{\color{blue}\foreignlanguage{arabic}{ف.خ.ذ}\color{blue}{}}{\color{blue}\foreignlanguage{arabic}{ف.خ.ذ}\color{blue}{}}\subsection*{\color{blue}\foreignlanguage{arabic}{ف.خ.ذ}\color{blue}{}\index{\color{blue}\foreignlanguage{arabic}{ف.خ.ذ}\color{blue}{}}} 

{\setlength\topsep{0pt}\textbf{\foreignlanguage{arabic}{فَخِذ}}\ {\color{gray}\texttt{/\sffamily {{\sffamily faxi(d)}}/}\color{black}}\ \textsc{noun}\ [m.]\ \color{gray}(msa. \foreignlanguage{arabic}{فَخْذ}~\foreignlanguage{arabic}{\textbf{١.}})\color{black}\ \textbf{1.}~thigh  \textbf{2.}~leg\ \ $\bullet$\ \ \setlength\topsep{0pt}\textbf{\foreignlanguage{arabic}{فْخَاذ}}\ {\color{gray}\texttt{/\sffamily {{\sffamily fxaː(d)}}/}\color{black}}\ [pl.]\  \begin{flushright}\color{gray}\foreignlanguage{arabic}{\textbf{\underline{\foreignlanguage{arabic}{أمثلة}}}: بحب أطبخ اللبن عفْخاذ الخاروف}\end{flushright}\color{black}} \vspace{2mm}

\vspace{-3mm}
\markboth{\color{blue}\foreignlanguage{arabic}{ف.خ.ر}\color{blue}{}}{\color{blue}\foreignlanguage{arabic}{ف.خ.ر}\color{blue}{}}\subsection*{\color{blue}\foreignlanguage{arabic}{ف.خ.ر}\color{blue}{}\index{\color{blue}\foreignlanguage{arabic}{ف.خ.ر}\color{blue}{}}} 

{\setlength\topsep{0pt}\textbf{\foreignlanguage{arabic}{اِفْتَخَر}}\ {\color{gray}\texttt{/\sffamily {{\sffamily ʔiftaxar}}/}\color{black}}\ \textsc{verb}\ [p.]\ \textbf{1.}~be proud\ \ $\bullet$\ \ \setlength\topsep{0pt}\textbf{\foreignlanguage{arabic}{اِفْتِخِر}}\ {\color{gray}\texttt{/\sffamily {{\sffamily ʔiftixir}}/}\color{black}}\ [c.]\ \ $\bullet$\ \ \setlength\topsep{0pt}\textbf{\foreignlanguage{arabic}{يِفْتِخِر}}\ {\color{gray}\texttt{/\sffamily {{\sffamily jiftixir}}/}\color{black}}\ [i.]\ \color{gray}(msa. \foreignlanguage{arabic}{يَفْتَخِر}~\foreignlanguage{arabic}{\textbf{١.}})\color{black}\  \begin{flushright}\color{gray}\foreignlanguage{arabic}{\textbf{\underline{\foreignlanguage{arabic}{أمثلة}}}: كلنا بنفتخِر فيك وبإِنجازاتك يا رقية}\end{flushright}\color{black}} \vspace{2mm}

{\setlength\topsep{0pt}\textbf{\foreignlanguage{arabic}{فَاخُورَة}}\ {\color{gray}\texttt{/\sffamily {{\sffamily faːxuːra}}/}\color{black}}\ \textsc{noun}\ [f.]\ \color{gray}(msa. \foreignlanguage{arabic}{مَعْمَل فَخّار}~\foreignlanguage{arabic}{\textbf{١.}})\color{black}\ \textbf{1.}~pottery factory\ \ $\bullet$\ \ \setlength\topsep{0pt}\textbf{\foreignlanguage{arabic}{فَوَاخِير}}\ {\color{gray}\texttt{/\sffamily {{\sffamily fawaːxiːr}}/}\color{black}}\ [pl.]\  \begin{flushright}\color{gray}\foreignlanguage{arabic}{\textbf{\underline{\foreignlanguage{arabic}{أمثلة}}}: بقيت أشتغل بالفاخورَة اللي تلا دار حماي}\end{flushright}\color{black}} \vspace{2mm}

{\setlength\topsep{0pt}\textbf{\foreignlanguage{arabic}{فَاخِر}}\ {\color{gray}\texttt{/\sffamily {{\sffamily faːxir}}/}\color{black}}\ \textsc{adj}\ [m.]\ \textbf{1.}~fine  \textbf{2.}~selected  \textbf{3.}~de luxe.  \textbf{4.}~magnificent\ \ $\bullet$\ \ \textsc{ph.} \color{gray} \foreignlanguage{arabic}{فَاخِر من الآخر}\color{black}\ {\color{gray}\texttt{/{\sffamily faːxir min ʔilʔaːxir}/}\color{black}}\ \textbf{1.}~fine  \textbf{2.}~selected  \textbf{3.}~de luxe.  \textbf{4.}~magnificent\  \begin{flushright}\color{gray}\foreignlanguage{arabic}{\textbf{\underline{\foreignlanguage{arabic}{أمثلة}}}: الكنافة اللي جبناها للجاهة اشي فاخِر من الآخر!}\end{flushright}\color{black}} \vspace{2mm}

{\setlength\topsep{0pt}\textbf{\foreignlanguage{arabic}{فَخُور}}\ {\color{gray}\texttt{/\sffamily {{\sffamily faxuːr}}/}\color{black}}\ \textsc{adj}\ [m.]\ \color{gray}(msa. \foreignlanguage{arabic}{فَخور}~\foreignlanguage{arabic}{\textbf{١.}})\color{black}\ \textbf{1.}~proud\ } \vspace{2mm}

{\setlength\topsep{0pt}\textbf{\foreignlanguage{arabic}{فَخِر}}\ {\color{gray}\texttt{/\sffamily {{\sffamily faxir}}/}\color{black}}\ \textsc{noun}\ [m.]\ \textbf{1.}~boasting  \textbf{2.}~bragging  \textbf{3.}~pride\  \begin{flushright}\color{gray}\foreignlanguage{arabic}{\textbf{\underline{\foreignlanguage{arabic}{أمثلة}}}: الواحد بيحس بالفَخِر انه شبابنا ملاح وبيرفعوا الراس زي هيك}\end{flushright}\color{black}} \vspace{2mm}

{\setlength\topsep{0pt}\textbf{\foreignlanguage{arabic}{فَخَّار}}\ {\color{gray}\texttt{/\sffamily {{\sffamily faxxaːr}}/}\color{black}}\ \textsc{noun}\ [m.]\ \textbf{1.}~pottery\ \ $\bullet$\ \ \textsc{ph.} \color{gray} \foreignlanguage{arabic}{حَلُوبِة فَخَّار}\color{black}\ {\color{gray}\texttt{/{\sffamily ħaluːbit faxxaːr}/}\color{black}}\ \color{gray} (msa. \foreignlanguage{arabic}{وعاء فخاري يشبه الطبق العميق بعض الشئ وكان يستعمل لترويب الحليب ليصبح لبن رايب قبل بيعه}~\foreignlanguage{arabic}{\textbf{١.}})\color{black}\ \textbf{1.}~A clay pot that is somewhat similar to the deep dish. It was used to curdle the milk into yoghurt before it was sold.\ \ $\bullet$\ \ \textsc{ph.} \color{gray} \foreignlanguage{arabic}{فخَار يكسر بعضه}\color{black}\ {\color{gray}\texttt{/{\sffamily faxxaːr jkassir baʕ(dˤ)o}/}\color{black}}\ \color{gray} (msa. \foreignlanguage{arabic}{هذا الأمر لا يعنينا}~\foreignlanguage{arabic}{\textbf{١.}})\color{black}\ \textbf{1.}~to hell with sb\  \begin{flushright}\color{gray}\foreignlanguage{arabic}{\textbf{\underline{\foreignlanguage{arabic}{أمثلة}}}: فَخّار يكسِّر بَعْضُه ان شاء الله يشَلْخُوا بعض تَشْلِيخ واحنا مالنا؟}\end{flushright}\color{black}} \vspace{2mm}

{\setlength\topsep{0pt}\textbf{\foreignlanguage{arabic}{فَخَّارَة}}\ {\color{gray}\texttt{/\sffamily {{\sffamily faxxaːra}}/}\color{black}}\ \textsc{noun}\ [f.]\ \textbf{1.}~clay cooking pot.  \textbf{2.}~cooking pot made of pottery\  \begin{flushright}\color{gray}\foreignlanguage{arabic}{\textbf{\underline{\foreignlanguage{arabic}{أمثلة}}}: طبخنا القدرة عالفَخّارَة}\end{flushright}\color{black}} \vspace{2mm}

{\setlength\topsep{0pt}\textbf{\foreignlanguage{arabic}{فَوخَر}}\ {\color{gray}\texttt{/\sffamily {{\sffamily foːxar}}/}\color{black}}\ \textsc{verb}\ [p.]\ \textbf{1.}~blush\ \ $\bullet$\ \ \setlength\topsep{0pt}\textbf{\foreignlanguage{arabic}{فَوخِر}}\ {\color{gray}\texttt{/\sffamily {{\sffamily foːxir}}/}\color{black}}\ [c.]\ \ $\bullet$\ \ \setlength\topsep{0pt}\textbf{\foreignlanguage{arabic}{يفَوخِر}}\ {\color{gray}\texttt{/\sffamily {{\sffamily jfoːxir}}/}\color{black}}\ [i.]\ \color{gray}(msa. \foreignlanguage{arabic}{يَحْمَر خجلاً}~\foreignlanguage{arabic}{\textbf{١.}})\color{black}\  \begin{flushright}\color{gray}\foreignlanguage{arabic}{\textbf{\underline{\foreignlanguage{arabic}{أمثلة}}}: فُوخَر ابن أبو الرجا لما جبناله سيرة الجازة}\end{flushright}\color{black}} \vspace{2mm}

{\setlength\topsep{0pt}\textbf{\foreignlanguage{arabic}{فِخِر}}\ {\color{gray}\texttt{/\sffamily {{\sffamily fixir}}/}\color{black}}\ \textsc{verb}\ [p.]\ \textbf{1.}~be proud\ \ $\bullet$\ \ \setlength\topsep{0pt}\textbf{\foreignlanguage{arabic}{اِفْخَر}}\ {\color{gray}\texttt{/\sffamily {{\sffamily ʔifxar}}/}\color{black}}\ [c.]\ \ $\bullet$\ \ \setlength\topsep{0pt}\textbf{\foreignlanguage{arabic}{يِفْخَر}}\ {\color{gray}\texttt{/\sffamily {{\sffamily jifxar}}/}\color{black}}\ [i.]\ \color{gray}(msa. \foreignlanguage{arabic}{يَفْتَخِر}~\foreignlanguage{arabic}{\textbf{١.}})\color{black}\  \begin{flushright}\color{gray}\foreignlanguage{arabic}{\textbf{\underline{\foreignlanguage{arabic}{أمثلة}}}: لازم الواحد يِفْخَر بأصله}\end{flushright}\color{black}} \vspace{2mm}

{\setlength\topsep{0pt}\textbf{\foreignlanguage{arabic}{مْفَوخِر}}\ {\color{gray}\texttt{/\sffamily {{\sffamily mfoːxir}}/}\color{black}}\ \textsc{adj}\ [m.]\ (src. \color{gray}\foreignlanguage{arabic}{نابلس > قرى}\color{black})\ \color{gray}(msa. \foreignlanguage{arabic}{يَشْعُد بالدِّفء}~\foreignlanguage{arabic}{\textbf{١.}})\color{black}\ \textbf{1.}~feeling warm\ \ $\smblkdiamond$\ \ \setlength\topsep{0pt}\textbf{\foreignlanguage{arabic}{مْفَوخِر}}\ \color{gray}(msa. \foreignlanguage{arabic}{محمر الخدين}~\foreignlanguage{arabic}{\textbf{١.}})\color{black}\ \textbf{1.}~blushful\  \begin{flushright}\color{gray}\foreignlanguage{arabic}{\textbf{\underline{\foreignlanguage{arabic}{أمثلة}}}: مفوخرة من الخجل\ $\bullet$\ \  حاطط 3 حرامات ولسه مش مفوخر}\end{flushright}\color{black}} \vspace{2mm}

\vspace{-3mm}
\markboth{\color{blue}\foreignlanguage{arabic}{ف.خ.ف.خ}\color{blue}{}}{\color{blue}\foreignlanguage{arabic}{ف.خ.ف.خ}\color{blue}{}}\subsection*{\color{blue}\foreignlanguage{arabic}{ف.خ.ف.خ}\color{blue}{}\index{\color{blue}\foreignlanguage{arabic}{ف.خ.ف.خ}\color{blue}{}}} 

{\setlength\topsep{0pt}\textbf{\foreignlanguage{arabic}{تْفَخْفَخ}}\ {\color{gray}\texttt{/\sffamily {{\sffamily tfaxfax}}/}\color{black}}\ \textsc{verb}\ [p.]\ \textbf{1.}~lead a luxurious life\ \ $\bullet$\ \ \setlength\topsep{0pt}\textbf{\foreignlanguage{arabic}{اِتْفَخْفَخ}}\ {\color{gray}\texttt{/\sffamily {{\sffamily ʔitfaxfax}}/}\color{black}}\ [c.]\ \ $\bullet$\ \ \setlength\topsep{0pt}\textbf{\foreignlanguage{arabic}{يِتْفَخْفَخ}}\ {\color{gray}\texttt{/\sffamily {{\sffamily jitfaxfax}}/}\color{black}}\ [i.]\ \color{gray}(msa. \foreignlanguage{arabic}{يعيش حياة الرفاهية والترف}~\foreignlanguage{arabic}{\textbf{١.}})\color{black}\  \begin{flushright}\color{gray}\foreignlanguage{arabic}{\textbf{\underline{\foreignlanguage{arabic}{أمثلة}}}: لازم يِتْفَخْفَخ يعني ويجيب سيارة من شو بتشكي المواصلات}\end{flushright}\color{black}} \vspace{2mm}

{\setlength\topsep{0pt}\textbf{\foreignlanguage{arabic}{فَخْفَخَة}}\ {\color{gray}\texttt{/\sffamily {{\sffamily faxfaxa}}/}\color{black}}\ \textsc{noun}\ [f.]\ \color{gray}(msa. \foreignlanguage{arabic}{الترف}~\foreignlanguage{arabic}{\textbf{٢.}}  \foreignlanguage{arabic}{الرفاهية}~\foreignlanguage{arabic}{\textbf{١.}})\color{black}\ \textbf{1.}~the luxurey\  \begin{flushright}\color{gray}\foreignlanguage{arabic}{\textbf{\underline{\foreignlanguage{arabic}{أمثلة}}}: عاجبتها حياة الفَخْفَخَة هلّا بعد ما كانت هي وأهلها ساكنة بخُم}\end{flushright}\color{black}} \vspace{2mm}

\vspace{-3mm}
\markboth{\color{blue}\foreignlanguage{arabic}{ف.خ.م}\color{blue}{}}{\color{blue}\foreignlanguage{arabic}{ف.خ.م}\color{blue}{}}\subsection*{\color{blue}\foreignlanguage{arabic}{ف.خ.م}\color{blue}{}\index{\color{blue}\foreignlanguage{arabic}{ف.خ.م}\color{blue}{}}} 

{\setlength\topsep{0pt}\textbf{\foreignlanguage{arabic}{فَخَامِة}}\ {\color{gray}\texttt{/\sffamily {{\sffamily faxaːme}}/}\color{black}}\ \textsc{noun}\ [f.]\ \textbf{1.}~luxury  \textbf{2.}~the state of being fancy, grand and/or impressive\ } \vspace{2mm}

{\setlength\topsep{0pt}\textbf{\foreignlanguage{arabic}{فَخِم}}\ {\color{gray}\texttt{/\sffamily {{\sffamily faxim}}/}\color{black}}\ \textsc{adj}\ [m.]\ \textbf{1.}~luxurious  \textbf{2.}~fancy  \textbf{3.}~grand  \textbf{4.}~mpressive\ } \vspace{2mm}

{\setlength\topsep{0pt}\textbf{\foreignlanguage{arabic}{فَخَّم}}\ {\color{gray}\texttt{/\sffamily {{\sffamily faxxam}}/}\color{black}}\ \textsc{verb}\ [p.]\ \textbf{1.}~make sth luxurious.  \textbf{2.}~make sth grand.  \textbf{3.}~augment  \textbf{4.}~stress (a syllable)\ \ $\bullet$\ \ \setlength\topsep{0pt}\textbf{\foreignlanguage{arabic}{فَخِّم}}\ {\color{gray}\texttt{/\sffamily {{\sffamily faxxim}}/}\color{black}}\ [c.]\ \ $\bullet$\ \ \setlength\topsep{0pt}\textbf{\foreignlanguage{arabic}{يفَخِّم}}\ {\color{gray}\texttt{/\sffamily {{\sffamily jfaxxim}}/}\color{black}}\ [i.]\  \begin{flushright}\color{gray}\foreignlanguage{arabic}{\textbf{\underline{\foreignlanguage{arabic}{أمثلة}}}: وأنت بتقرأ حاول فَخَّم حرف الراء عشان مايفكوكاس خيخة}\end{flushright}\color{black}} \vspace{2mm}

{\setlength\topsep{0pt}\textbf{\foreignlanguage{arabic}{فِخِم}}\ {\color{gray}\texttt{/\sffamily {{\sffamily fixim}}/}\color{black}}\ \textsc{adj}\ [m.]\ \textbf{1.}~luxurious  \textbf{2.}~fancy  \textbf{3.}~grand  \textbf{4.}~mpressive\  \begin{flushright}\color{gray}\foreignlanguage{arabic}{\textbf{\underline{\foreignlanguage{arabic}{أمثلة}}}: البدل اللي جابتها من نابلس فِخمات}\end{flushright}\color{black}} \vspace{2mm}

\vspace{-3mm}
\markboth{\color{blue}\foreignlanguage{arabic}{ف.د.ي}\color{blue}{}}{\color{blue}\foreignlanguage{arabic}{ف.د.ي}\color{blue}{}}\subsection*{\color{blue}\foreignlanguage{arabic}{ف.د.ي}\color{blue}{}\index{\color{blue}\foreignlanguage{arabic}{ف.د.ي}\color{blue}{}}} 

{\setlength\topsep{0pt}\textbf{\foreignlanguage{arabic}{تْفَادَى}}\ {\color{gray}\texttt{/\sffamily {{\sffamily tfaːda}}/}\color{black}}\ \textsc{verb}\ [p.]\ \textbf{1.}~avoid\ \ $\bullet$\ \ \setlength\topsep{0pt}\textbf{\foreignlanguage{arabic}{اِتْفَادَى}}\ {\color{gray}\texttt{/\sffamily {{\sffamily ʔitfaːda}}/}\color{black}}\ [c.]\ \ $\bullet$\ \ \setlength\topsep{0pt}\textbf{\foreignlanguage{arabic}{يِتْفَادَى}}\ {\color{gray}\texttt{/\sffamily {{\sffamily jitfaːda}}/}\color{black}}\ [i.]\ \color{gray}(msa. \foreignlanguage{arabic}{يَتَجَنَّب}~\foreignlanguage{arabic}{\textbf{١.}})\color{black}\  \begin{flushright}\color{gray}\foreignlanguage{arabic}{\textbf{\underline{\foreignlanguage{arabic}{أمثلة}}}: أنا بتْفادَى أحكي معه هالفترة عشان مايجيبلي سيرة المصاري والميراث}\end{flushright}\color{black}} \vspace{2mm}

{\setlength\topsep{0pt}\textbf{\foreignlanguage{arabic}{فَدَى}}\ {\color{gray}\texttt{/\sffamily {{\sffamily fada}}/}\color{black}}\ \textsc{verb}\ [p.]\ \textbf{1.}~self-sacrifice  \textbf{2.}~ransom  \textbf{3.}~redeem\ \ $\bullet$\ \ \setlength\topsep{0pt}\textbf{\foreignlanguage{arabic}{اِفْدِي}}\ {\color{gray}\texttt{/\sffamily {{\sffamily ʔifdi}}/}\color{black}}\ [c.]\ \ $\bullet$\ \ \setlength\topsep{0pt}\textbf{\foreignlanguage{arabic}{يِفْدِي}}\ {\color{gray}\texttt{/\sffamily {{\sffamily jifdi}}/}\color{black}}\ [i.]\  \begin{flushright}\color{gray}\foreignlanguage{arabic}{\textbf{\underline{\foreignlanguage{arabic}{أمثلة}}}: والله بفْدِيك بروحي يابا}\end{flushright}\color{black}} \vspace{2mm}

{\setlength\topsep{0pt}\textbf{\foreignlanguage{arabic}{فِدَا}}\ {\color{gray}\texttt{/\sffamily {{\sffamily fida}}/}\color{black}}\ \textsc{noun}\ [m.]\ \color{gray}(msa. \foreignlanguage{arabic}{من أجل}~\foreignlanguage{arabic}{\textbf{٢.}}  \foreignlanguage{arabic}{فِداء}~\foreignlanguage{arabic}{\textbf{١.}})\color{black}\ \textbf{1.}~self-sacrifice  \textbf{2.}~for the sake of\  \begin{flushright}\color{gray}\foreignlanguage{arabic}{\textbf{\underline{\foreignlanguage{arabic}{أمثلة}}}: كلُّه فِدا تكون مبسوط!}\end{flushright}\color{black}} \vspace{2mm}

{\setlength\topsep{0pt}\textbf{\foreignlanguage{arabic}{فِدَائِي}}\ {\color{gray}\texttt{/\sffamily {{\sffamily fidaːʔi}}/}\color{black}}\ \textsc{adj}\ [m.]\ \textbf{1.}~guerrilla\  \begin{flushright}\color{gray}\foreignlanguage{arabic}{\textbf{\underline{\foreignlanguage{arabic}{أمثلة}}}: في ناس بتحكيلها أيلول الأسود وفي ناس بتقولها حرب الفدائيين}\end{flushright}\color{black}} \vspace{2mm}

{\setlength\topsep{0pt}\textbf{\foreignlanguage{arabic}{فِدَائِي}}\ {\color{gray}\texttt{/\sffamily {{\sffamily fidaːʔi}}/}\color{black}}\ \textsc{noun}\ [m.]\ \textbf{1.}~self-sacrificing\ } \vspace{2mm}

{\setlength\topsep{0pt}\textbf{\foreignlanguage{arabic}{فِدْيِة}}\ {\color{gray}\texttt{/\sffamily {{\sffamily fidje}}/}\color{black}}\ \textsc{noun}\ [f.]\ \color{gray}(msa. \foreignlanguage{arabic}{فِدْيَة}~\foreignlanguage{arabic}{\textbf{١.}})\color{black}\ \textbf{1.}~ransom\  \begin{flushright}\color{gray}\foreignlanguage{arabic}{\textbf{\underline{\foreignlanguage{arabic}{أمثلة}}}: خطفوه عصابة وطلبوا فِدْيِة 100 ألف شيكل}\end{flushright}\color{black}} \vspace{2mm}

\vspace{-3mm}
\markboth{\color{blue}\foreignlanguage{arabic}{ف.ر.ت.ك}\color{blue}{}}{\color{blue}\foreignlanguage{arabic}{ف.ر.ت.ك}\color{blue}{}}\subsection*{\color{blue}\foreignlanguage{arabic}{ف.ر.ت.ك}\color{blue}{}\index{\color{blue}\foreignlanguage{arabic}{ف.ر.ت.ك}\color{blue}{}}} 

{\setlength\topsep{0pt}\textbf{\foreignlanguage{arabic}{فَرْتَك}}\ {\color{gray}\texttt{/\sffamily {{\sffamily fartak}}/}\color{black}}\ \textsc{verb}\ [p.]\ \textbf{1.}~break  \textbf{2.}~smash  \textbf{3.}~rip sth off.  \textbf{4.}~beat sb severely\ \ $\bullet$\ \ \setlength\topsep{0pt}\textbf{\foreignlanguage{arabic}{فَرْتِك}}\ {\color{gray}\texttt{/\sffamily {{\sffamily fartik}}/}\color{black}}\ [c.]\ \ $\bullet$\ \ \setlength\topsep{0pt}\textbf{\foreignlanguage{arabic}{يفَرْتِك}}\ {\color{gray}\texttt{/\sffamily {{\sffamily jfartik}}/}\color{black}}\ [i.]\  \begin{flushright}\color{gray}\foreignlanguage{arabic}{\textbf{\underline{\foreignlanguage{arabic}{أمثلة}}}: مشكه أخوه وتوعد يفَرتِكه عاللي سواه قدام الجماعة}\end{flushright}\color{black}} \vspace{2mm}

{\setlength\topsep{0pt}\textbf{\foreignlanguage{arabic}{مْفَرْتَك}}\ {\color{gray}\texttt{/\sffamily {{\sffamily mfartak}}/}\color{black}}\ \textsc{adj}\ [m.]\ \color{gray}(msa. \foreignlanguage{arabic}{متكسر}~\foreignlanguage{arabic}{\textbf{١.}})\color{black}\ \textbf{1.}~chipped  \textbf{2.}~broken\  \begin{flushright}\color{gray}\foreignlanguage{arabic}{\textbf{\underline{\foreignlanguage{arabic}{أمثلة}}}: دخلت عالمطبخ لقيت الخزانة مفرتكة}\end{flushright}\color{black}} \vspace{2mm}

\vspace{-3mm}
\markboth{\color{blue}\foreignlanguage{arabic}{ف.ر.ج}\color{blue}{}}{\color{blue}\foreignlanguage{arabic}{ف.ر.ج}\color{blue}{}}\subsection*{\color{blue}\foreignlanguage{arabic}{ف.ر.ج}\color{blue}{}\index{\color{blue}\foreignlanguage{arabic}{ف.ر.ج}\color{blue}{}}} 

{\setlength\topsep{0pt}\textbf{\foreignlanguage{arabic}{أَفْرَج}}\ {\color{gray}\texttt{/\sffamily {{\sffamily ʔafra(dʒ)}}/}\color{black}}\ \textsc{verb}\ [p.]\ \textbf{1.}~release  \textbf{2.}~set sb free.  \textbf{3.}~alleviate\ \ $\bullet$\ \ \setlength\topsep{0pt}\textbf{\foreignlanguage{arabic}{اِفْرِج}}\ {\color{gray}\texttt{/\sffamily {{\sffamily ʔifri(dʒ)}}/}\color{black}}\ [c.]\ \ $\bullet$\ \ \setlength\topsep{0pt}\textbf{\foreignlanguage{arabic}{يِفْرِج}}\ {\color{gray}\texttt{/\sffamily {{\sffamily jifri(dʒ)}}/}\color{black}}\ [i.]\  \begin{flushright}\color{gray}\foreignlanguage{arabic}{\textbf{\underline{\foreignlanguage{arabic}{أمثلة}}}: مارضيوش اليهود يِفْرِجوا عنه لحديت ما أهله دفعوا\ $\bullet$\ \  ياربي اِفْرِجها من واسع فضلك ورحمك}\end{flushright}\color{black}} \vspace{2mm}

{\setlength\topsep{0pt}\textbf{\foreignlanguage{arabic}{اِفْرَاج}}\ {\color{gray}\texttt{/\sffamily {{\sffamily ʔifraː(dʒ)}}/}\color{black}}\ \textsc{noun}\ [m.]\ \color{gray}(msa. \foreignlanguage{arabic}{اِفْراج}~\foreignlanguage{arabic}{\textbf{١.}})\color{black}\ \textbf{1.}~release\  \begin{flushright}\color{gray}\foreignlanguage{arabic}{\textbf{\underline{\foreignlanguage{arabic}{أمثلة}}}: أخيراً أخذ اِفْراج الحمدلله}\end{flushright}\color{black}} \vspace{2mm}

{\setlength\topsep{0pt}\textbf{\foreignlanguage{arabic}{اِنْفَرَج}}\ {\color{gray}\texttt{/\sffamily {{\sffamily ʔinfara(dʒ)}}/}\color{black}}\ \textsc{verb}\ [p.]\ \textbf{1.}~be alleviated.  \textbf{2.}~be dispelled.  \textbf{3.}~be released\ \ $\bullet$\ \ \setlength\topsep{0pt}\textbf{\foreignlanguage{arabic}{اِنْفِرِج}}\ {\color{gray}\texttt{/\sffamily {{\sffamily ʔinfiri(dʒ)}}/}\color{black}}\ [c.]\ \ $\bullet$\ \ \setlength\topsep{0pt}\textbf{\foreignlanguage{arabic}{يِنْفِرِج}}\ {\color{gray}\texttt{/\sffamily {{\sffamily jinfiri(dʒ)}}/}\color{black}}\ [i.]\  \begin{flushright}\color{gray}\foreignlanguage{arabic}{\textbf{\underline{\foreignlanguage{arabic}{أمثلة}}}: ان شاء الله بتنفِرِج قريباً يا حبيبتي}\end{flushright}\color{black}} \vspace{2mm}

{\setlength\topsep{0pt}\textbf{\foreignlanguage{arabic}{اِنْفِرَاج}}\ {\color{gray}\texttt{/\sffamily {{\sffamily ʔinfiraː(dʒ)}}/}\color{black}}\ \textsc{noun}\ [m.]\ \textbf{1.}~release  \textbf{2.}~alleviation\ } \vspace{2mm}

{\setlength\topsep{0pt}\textbf{\foreignlanguage{arabic}{تَفْرِيج}}\ {\color{gray}\texttt{/\sffamily {{\sffamily tafriː(dʒ)}}/}\color{black}}\ \textsc{noun}\ [m.]\ \textbf{1.}~alleviation\  \begin{flushright}\color{gray}\foreignlanguage{arabic}{\textbf{\underline{\foreignlanguage{arabic}{أمثلة}}}: فش عمل أعظم من التَّفْريج عن المسلمين}\end{flushright}\color{black}} \vspace{2mm}

{\setlength\topsep{0pt}\textbf{\foreignlanguage{arabic}{تْفَرَّج}}\ {\color{gray}\texttt{/\sffamily {{\sffamily tfarra(dʒ)}}/}\color{black}}\ \textsc{verb}\ [p.]\ \textbf{1.}~see\ \ $\bullet$\ \ \setlength\topsep{0pt}\textbf{\foreignlanguage{arabic}{اِتْفَرَّج}}\ {\color{gray}\texttt{/\sffamily {{\sffamily ʔitfarra(dʒ)}}/}\color{black}}\ [c.]\ \ $\bullet$\ \ \setlength\topsep{0pt}\textbf{\foreignlanguage{arabic}{يِتْفَرَّج}}\ {\color{gray}\texttt{/\sffamily {{\sffamily jitfarra(dʒ)}}/}\color{black}}\ [i.]\ \color{gray}(msa. \foreignlanguage{arabic}{يَرَى}~\foreignlanguage{arabic}{\textbf{١.}})\color{black}\  \begin{flushright}\color{gray}\foreignlanguage{arabic}{\textbf{\underline{\foreignlanguage{arabic}{أمثلة}}}: جعبالي أتْفَرَّج عليكم بس!}\end{flushright}\color{black}} \vspace{2mm}

{\setlength\topsep{0pt}\textbf{\foreignlanguage{arabic}{فَرَج}}\ {\color{gray}\texttt{/\sffamily {{\sffamily fara(dʒ)}}/}\color{black}}\ \textsc{noun}\ [m.]\ \textbf{1.}~alleviation\  \begin{flushright}\color{gray}\foreignlanguage{arabic}{\textbf{\underline{\foreignlanguage{arabic}{أمثلة}}}: بستنى بفَرَج ربنا}\end{flushright}\color{black}} \vspace{2mm}

{\setlength\topsep{0pt}\textbf{\foreignlanguage{arabic}{فَرَّاجِة}}\ {\color{gray}\texttt{/\sffamily {{\sffamily farraːdʒe}}/}\color{black}}\ \textsc{noun}\ [m.]\ \color{gray}(msa. \foreignlanguage{arabic}{تلفاز}~\foreignlanguage{arabic}{\textbf{١.}})\color{black}\ \textbf{1.}~TV\  \begin{flushright}\color{gray}\foreignlanguage{arabic}{\textbf{\underline{\foreignlanguage{arabic}{أمثلة}}}: افتح الفراجة نشوف آخر الأخبار}\end{flushright}\color{black}} \vspace{2mm}

{\setlength\topsep{0pt}\textbf{\foreignlanguage{arabic}{فَرَّج}}\ {\color{gray}\texttt{/\sffamily {{\sffamily farra(dʒ)}}/}\color{black}}\ \textsc{verb}\ [p.]\ \textbf{1.}~alleviate  \textbf{2.}~dispel  \textbf{3.}~release  \textbf{4.}~show  \textbf{5.}~make sb see\ \ $\bullet$\ \ \setlength\topsep{0pt}\textbf{\foreignlanguage{arabic}{فَرِّج}}\ {\color{gray}\texttt{/\sffamily {{\sffamily farri(dʒ)}}/}\color{black}}\ [c.]\ \ $\bullet$\ \ \setlength\topsep{0pt}\textbf{\foreignlanguage{arabic}{يفَرِّج}}\ {\color{gray}\texttt{/\sffamily {{\sffamily jfarri(dʒ)}}/}\color{black}}\ [i.]\ \color{gray}(msa. \foreignlanguage{arabic}{يُري}~\foreignlanguage{arabic}{\textbf{٢.}}  \foreignlanguage{arabic}{يُخَفِّف}~\foreignlanguage{arabic}{\textbf{١.}})\color{black}\  \begin{flushright}\color{gray}\foreignlanguage{arabic}{\textbf{\underline{\foreignlanguage{arabic}{أمثلة}}}: الله يفَرِّجها عليك يارب\ $\bullet$\ \  يخرب بيتك فَرَّجت الناس علينا}\end{flushright}\color{black}} \vspace{2mm}

{\setlength\topsep{0pt}\textbf{\foreignlanguage{arabic}{فَرْجَى}}\ {\color{gray}\texttt{/\sffamily {{\sffamily far(dʒ)a}}/}\color{black}}\ \textsc{verb}\ [p.]\ \textbf{1.}~show  \textbf{2.}~make sb see sth\ \ $\bullet$\ \ \setlength\topsep{0pt}\textbf{\foreignlanguage{arabic}{فَرْجَي}}\ {\color{gray}\texttt{/\sffamily {{\sffamily far(dʒ)i}}/}\color{black}}\ [c.]\ \ $\bullet$\ \ \setlength\topsep{0pt}\textbf{\foreignlanguage{arabic}{يفَرْجَي}}\ {\color{gray}\texttt{/\sffamily {{\sffamily jfar(dʒ)i}}/}\color{black}}\ [i.]\ \color{gray}(msa. \foreignlanguage{arabic}{يُري}~\foreignlanguage{arabic}{\textbf{١.}})\color{black}\ \ $\bullet$\ \ \textsc{ph.} \color{gray} \foreignlanguage{arabic}{فَرْجَاه العين الحمرَا}\color{black}\ {\color{gray}\texttt{/{\sffamily far(dʒ)aː ʔilʕeːn ʔilħamra}/}\color{black}}\ \color{gray} (msa. \foreignlanguage{arabic}{يقسو على شخص}~\foreignlanguage{arabic}{\textbf{٢.}}  \foreignlanguage{arabic}{يُهَدِّد}~\foreignlanguage{arabic}{\textbf{١.}})\color{black}\ \textbf{1.}~threaten sb.  \textbf{2.}~be very tough to sb.  \textbf{3.}~be harsh on sb\  \begin{flushright}\color{gray}\foreignlanguage{arabic}{\textbf{\underline{\foreignlanguage{arabic}{أمثلة}}}: لما أبوه فَرْجاه العين الحمرا وطبله قدام الناس صار مؤدب\ $\bullet$\ \  فَرْجَيني الفواتير اللي دفعتها الشهر الماضي الا مايكونوا حاطين الزيادة}\end{flushright}\color{black}} \vspace{2mm}

{\setlength\topsep{0pt}\textbf{\foreignlanguage{arabic}{فُرْجِة}}\ {\color{gray}\texttt{/\sffamily {{\sffamily fur(dʒ)e}}/}\color{black}}\ \textsc{noun}\ [f.]\ \textbf{1.}~display case\  \begin{flushright}\color{gray}\foreignlanguage{arabic}{\textbf{\underline{\foreignlanguage{arabic}{أمثلة}}}: شو شايفيتني فُرْجِة؟ أنت خليتنا فُرْجِة للي بيسوى واللي مابيسوى}\end{flushright}\color{black}} \vspace{2mm}

{\setlength\topsep{0pt}\textbf{\foreignlanguage{arabic}{فِرْجِة}}\ {\color{gray}\texttt{/\sffamily {{\sffamily fir(dʒ)e}}/}\color{black}}\ \textsc{noun}\ [f.]\ \textbf{1.}~display case\ \ $\bullet$\ \ \textsc{ph.} \color{gray} \foreignlanguage{arabic}{صندوق الفرجِة}\color{black}\ {\color{gray}\texttt{/{\sffamily sˤunduːq ʔilfirdʒe}/}\color{black}}\ \textbf{1.}~the trade show booth for children in the past\ } \vspace{2mm}

{\setlength\topsep{0pt}\textbf{\foreignlanguage{arabic}{مُتَفَرِّج}}\ {\color{gray}\texttt{/\sffamily {{\sffamily mutafarri(dʒ)}}/}\color{black}}\ \textsc{noun}\ [m.]\ \color{gray}(msa. \foreignlanguage{arabic}{متفرِّج}~\foreignlanguage{arabic}{\textbf{١.}})\color{black}\ \textbf{1.}~spectator\  \begin{flushright}\color{gray}\foreignlanguage{arabic}{\textbf{\underline{\foreignlanguage{arabic}{أمثلة}}}: هي تشيل وتحط وتفاصِل وأنا صرت متفرِّج بس.}\end{flushright}\color{black}} \vspace{2mm}

\vspace{-3mm}
\markboth{\color{blue}\foreignlanguage{arabic}{ف.ر.ج.ن}\color{blue}{ (ntws)}}{\color{blue}\foreignlanguage{arabic}{ف.ر.ج.ن}\color{blue}{ (ntws)}}\subsection*{\color{blue}\foreignlanguage{arabic}{ف.ر.ج.ن}\color{blue}{ (ntws)}\index{\color{blue}\foreignlanguage{arabic}{ف.ر.ج.ن}\color{blue}{ (ntws)}}} 

{\setlength\topsep{0pt}\textbf{\foreignlanguage{arabic}{فَرْجَون}}\ {\color{gray}\texttt{/\sffamily {{\sffamily farɡoːn}}/}\color{black}}\ \textsc{noun}\ [m.]\ \color{gray}(msa. \foreignlanguage{arabic}{مقطورات}~\foreignlanguage{arabic}{\textbf{٢.}}  \foreignlanguage{arabic}{قاطرات}~\foreignlanguage{arabic}{\textbf{١.}})\color{black}\ \textbf{1.}~compartments\  \begin{flushright}\color{gray}\foreignlanguage{arabic}{\textbf{\underline{\foreignlanguage{arabic}{أمثلة}}}: هظول فَرْْجُونات من أيام الألمان}\end{flushright}\color{black}} \vspace{2mm}

\vspace{-3mm}
\markboth{\color{blue}\foreignlanguage{arabic}{ف.ر.ح}\color{blue}{}}{\color{blue}\foreignlanguage{arabic}{ف.ر.ح}\color{blue}{}}\subsection*{\color{blue}\foreignlanguage{arabic}{ف.ر.ح}\color{blue}{}\index{\color{blue}\foreignlanguage{arabic}{ف.ر.ح}\color{blue}{}}} 

{\setlength\topsep{0pt}\textbf{\foreignlanguage{arabic}{تَفْرِيحِة}}\ {\color{gray}\texttt{/\sffamily {{\sffamily tafriːħe}}/}\color{black}}\ \textsc{noun}\ [f.]\ \color{gray}(msa. \foreignlanguage{arabic}{ي نوع من الطعام (عادة حلويات) يقدمه الأب لأولاده بعد العمل}~\foreignlanguage{arabic}{\textbf{١.}})\color{black}\ \textbf{1.}~Any type of food (usually sweets) that the father brings to his children after work\ } \vspace{2mm}

{\setlength\topsep{0pt}\textbf{\foreignlanguage{arabic}{فَرَايْحِي}}\ {\color{gray}\texttt{/\sffamily {{\sffamily faraːjħi}}/}\color{black}}\ \textsc{adj}\ [m.]\ \color{gray}(msa. \foreignlanguage{arabic}{مُبْهِج}~\foreignlanguage{arabic}{\textbf{١.}})\color{black}\ \textbf{1.}~cheerful\  \begin{flushright}\color{gray}\foreignlanguage{arabic}{\textbf{\underline{\foreignlanguage{arabic}{أمثلة}}}: يختي البسي لمّاع أو ألوان فَرايحية ليش هالألوان اللي بتغِم}\end{flushright}\color{black}} \vspace{2mm}

{\setlength\topsep{0pt}\textbf{\foreignlanguage{arabic}{فَرَح}}\ {\color{gray}\texttt{/\sffamily {{\sffamily faraħ}}/}\color{black}}\ \textsc{noun}\ [m.]\ \textbf{1.}~happiness\ } \vspace{2mm}

{\setlength\topsep{0pt}\textbf{\foreignlanguage{arabic}{فَرَّح}}\ {\color{gray}\texttt{/\sffamily {{\sffamily farraħ}}/}\color{black}}\ \textsc{verb}\ [p.]\ \textbf{1.}~make sb happy.  \textbf{2.}~gladden\ \ $\bullet$\ \ \setlength\topsep{0pt}\textbf{\foreignlanguage{arabic}{فَرِّح}}\ {\color{gray}\texttt{/\sffamily {{\sffamily farriħ}}/}\color{black}}\ [c.]\ \ $\bullet$\ \ \setlength\topsep{0pt}\textbf{\foreignlanguage{arabic}{يفَرِّح}}\ {\color{gray}\texttt{/\sffamily {{\sffamily jfarriħ}}/}\color{black}}\ [i.]\ \color{gray}(msa. \foreignlanguage{arabic}{يُسْعِد}~\foreignlanguage{arabic}{\textbf{١.}})\color{black}\  \begin{flushright}\color{gray}\foreignlanguage{arabic}{\textbf{\underline{\foreignlanguage{arabic}{أمثلة}}}: الله يفَرِّح قلبك زي ما فَرَّحتني بزيارتك}\end{flushright}\color{black}} \vspace{2mm}

{\setlength\topsep{0pt}\textbf{\foreignlanguage{arabic}{فَرْحَان}}\ {\color{gray}\texttt{/\sffamily {{\sffamily farħaːn}}/}\color{black}}\ \textsc{adj}\ [m.]\ \color{gray}(msa. \foreignlanguage{arabic}{فَرْحان}~\foreignlanguage{arabic}{\textbf{١.}})\color{black}\ \textbf{1.}~happy  \textbf{2.}~glad\  \begin{flushright}\color{gray}\foreignlanguage{arabic}{\textbf{\underline{\foreignlanguage{arabic}{أمثلة}}}: فَرْحانِة بزيارتكم والله مش معطية فرحتي لحدا}\end{flushright}\color{black}} \vspace{2mm}

{\setlength\topsep{0pt}\textbf{\foreignlanguage{arabic}{فَرْحَة}}\ {\color{gray}\texttt{/\sffamily {{\sffamily farħa}}/}\color{black}}\ \textsc{noun}\ [f.]\ \textbf{1.}~happy event.  \textbf{2.}~happy moment\ \ $\bullet$\ \ \textsc{ph.} \color{gray} \foreignlanguage{arabic}{مُش معطي فَرْحِتي لحدَا}\color{black}\ {\color{gray}\texttt{/{\sffamily muʃ maʕtˤi farħiti laħada}/}\color{black}}\ \textbf{1.}~very happy\  \begin{flushright}\color{gray}\foreignlanguage{arabic}{\textbf{\underline{\foreignlanguage{arabic}{أمثلة}}}: مُش معطي فَرْحِتي لحدا الحمدلله\ $\bullet$\ \  فَرْحَة التخرج فش بعدها فَرْحَة!}\end{flushright}\color{black}} \vspace{2mm}

{\setlength\topsep{0pt}\textbf{\foreignlanguage{arabic}{فِرِح}}\ {\color{gray}\texttt{/\sffamily {{\sffamily firiħ}}/}\color{black}}\ \textsc{verb}\ [p.]\ \textbf{1.}~be happy.  \textbf{2.}~be glad\ \ $\bullet$\ \ \setlength\topsep{0pt}\textbf{\foreignlanguage{arabic}{اِْفْرَح}}\ {\color{gray}\texttt{/\sffamily {{\sffamily ʔifraħ}}/}\color{black}}\ [c.]\ \ $\bullet$\ \ \setlength\topsep{0pt}\textbf{\foreignlanguage{arabic}{يِفْرَح}}\ {\color{gray}\texttt{/\sffamily {{\sffamily jifraħ}}/}\color{black}}\ [i.]\ \color{gray}(msa. \foreignlanguage{arabic}{يشعر بالسعادة}~\foreignlanguage{arabic}{\textbf{١.}})\color{black}\  \begin{flushright}\color{gray}\foreignlanguage{arabic}{\textbf{\underline{\foreignlanguage{arabic}{أمثلة}}}: فرِحِت كثير عشانك والله}\end{flushright}\color{black}} \vspace{2mm}

{\setlength\topsep{0pt}\textbf{\foreignlanguage{arabic}{فْرَاحَة}}\ {\color{gray}\texttt{/\sffamily {{\sffamily fraːħa}}/}\color{black}}\ \textsc{noun}\ [f.]\ \color{gray}(msa. \foreignlanguage{arabic}{ي نوع من الطعام (عادة حلويات) يقدمه الأب لأولاده بعد العمل}~\foreignlanguage{arabic}{\textbf{١.}})\color{black}\ \textbf{1.}~Any type of food (usually sweets) that the father brings to his children after work\ } \vspace{2mm}

\vspace{-3mm}
\markboth{\color{blue}\foreignlanguage{arabic}{ف.ر.خ}\color{blue}{}}{\color{blue}\foreignlanguage{arabic}{ف.ر.خ}\color{blue}{}}\subsection*{\color{blue}\foreignlanguage{arabic}{ف.ر.خ}\color{blue}{}\index{\color{blue}\foreignlanguage{arabic}{ف.ر.خ}\color{blue}{}}} 

{\setlength\topsep{0pt}\textbf{\foreignlanguage{arabic}{فَرَّخ}}\ {\color{gray}\texttt{/\sffamily {{\sffamily farrax}}/}\color{black}}\ \textsc{verb}\ [p.]\ \textbf{1.}~give birth to many babies.  \textbf{2.}~have many kids\ \ $\bullet$\ \ \setlength\topsep{0pt}\textbf{\foreignlanguage{arabic}{فَرِّخ}}\ {\color{gray}\texttt{/\sffamily {{\sffamily farrix}}/}\color{black}}\ [c.]\ \ $\bullet$\ \ \setlength\topsep{0pt}\textbf{\foreignlanguage{arabic}{يفَرِّخ}}\ {\color{gray}\texttt{/\sffamily {{\sffamily jfarrix}}/}\color{black}}\ [i.]\  \begin{flushright}\color{gray}\foreignlanguage{arabic}{\textbf{\underline{\foreignlanguage{arabic}{أمثلة}}}: يا الله بهالقرية شو بيفَرخوا. كل عيلة عندهم أقل شي خمس أو ست صغار وكلهم ورا بعض.}\end{flushright}\color{black}} \vspace{2mm}

{\setlength\topsep{0pt}\textbf{\foreignlanguage{arabic}{فَرْخ}}\ {\color{gray}\texttt{/\sffamily {{\sffamily farx}}/}\color{black}}\ \textsc{noun}\ [m.]\ \textbf{1.}~chick  \textbf{2.}~small bird\ \ $\bullet$\ \ \textsc{ph.} \color{gray} \foreignlanguage{arabic}{فَرْخ البط عَوَّام}\color{black}\ {\color{gray}\texttt{/{\sffamily farx ʔilbatˤtˤ ʕawwaːm}/}\color{black}}\ \textbf{1.}~like father like son\  \begin{flushright}\color{gray}\foreignlanguage{arabic}{\textbf{\underline{\foreignlanguage{arabic}{أمثلة}}}: طالع لأبوه بحب ينقِّف العصافير فَرْخ البط عَوّام}\end{flushright}\color{black}} \vspace{2mm}

\vspace{-3mm}
\markboth{\color{blue}\foreignlanguage{arabic}{ف.ر.د}\color{blue}{}}{\color{blue}\foreignlanguage{arabic}{ف.ر.د}\color{blue}{}}\subsection*{\color{blue}\foreignlanguage{arabic}{ف.ر.د}\color{blue}{}\index{\color{blue}\foreignlanguage{arabic}{ف.ر.د}\color{blue}{}}} 

{\setlength\topsep{0pt}\textbf{\foreignlanguage{arabic}{اِسْتَفْرَد}}\ {\color{gray}\texttt{/\sffamily {{\sffamily ʔistafrad}}/}\color{black}}\ \textsc{verb}\ [p.]\ \textbf{1.}~take advantage of the situation where sb is being alone\ \ $\bullet$\ \ \setlength\topsep{0pt}\textbf{\foreignlanguage{arabic}{اِسْتَفْرِد}}\ {\color{gray}\texttt{/\sffamily {{\sffamily ʔistafrid}}/}\color{black}}\ [c.]\ \ $\bullet$\ \ \setlength\topsep{0pt}\textbf{\foreignlanguage{arabic}{يِسْتَفْرِد}}\ {\color{gray}\texttt{/\sffamily {{\sffamily jistafrid}}/}\color{black}}\ [i.]\  \begin{flushright}\color{gray}\foreignlanguage{arabic}{\textbf{\underline{\foreignlanguage{arabic}{أمثلة}}}: أوعك يِسْتَفْرِد فيك لحالكم واحنا برة الدار والله بشيله بزرة عينيه}\end{flushright}\color{black}} \vspace{2mm}

{\setlength\topsep{0pt}\textbf{\foreignlanguage{arabic}{اِنْفَرَد}}\ {\color{gray}\texttt{/\sffamily {{\sffamily ʔinfarad}}/}\color{black}}\ \textsc{verb}\ [p.]\ \textbf{1.}~take advantage of the situation where sb is being alone.  \textbf{2.}~be unique in sth\ \ $\bullet$\ \ \setlength\topsep{0pt}\textbf{\foreignlanguage{arabic}{اِنْفِرِد}}\ {\color{gray}\texttt{/\sffamily {{\sffamily ʔinfirid}}/}\color{black}}\ [c.]\ \ $\bullet$\ \ \setlength\topsep{0pt}\textbf{\foreignlanguage{arabic}{يِنْفِرِد}}\ {\color{gray}\texttt{/\sffamily {{\sffamily jinfirid}}/}\color{black}}\ [i.]\  \begin{flushright}\color{gray}\foreignlanguage{arabic}{\textbf{\underline{\foreignlanguage{arabic}{أمثلة}}}: احنا حابين نِنْفِرِد بصناعة هذا النوع من الحرامات بالبلد\ $\bullet$\ \  أول ما اِنْفَرَد فيها لحالهم صار يفهلق مثل الأهتر}\end{flushright}\color{black}} \vspace{2mm}

{\setlength\topsep{0pt}\textbf{\foreignlanguage{arabic}{اِنْفِرَاد}}\ {\color{gray}\texttt{/\sffamily {{\sffamily ʔinfiraːd}}/}\color{black}}\ \textsc{noun}\ [m.]\ \textbf{1.}~isolation\ \ $\bullet$\ \ \textsc{ph.} \color{gray} \foreignlanguage{arabic}{على اِنْفِرَاد}\color{black}\ {\color{gray}\texttt{/{\sffamily ʕala ʔinfiraːd}/}\color{black}}\ \textbf{1.}~alone  \textbf{2.}~in private\  \begin{flushright}\color{gray}\foreignlanguage{arabic}{\textbf{\underline{\foreignlanguage{arabic}{أمثلة}}}: بدي أحكي معك جوز كلام على  على اِنْفِراد}\end{flushright}\color{black}} \vspace{2mm}

{\setlength\topsep{0pt}\textbf{\foreignlanguage{arabic}{اِنْفِرَادي}}\ {\color{gray}\texttt{/\sffamily {{\sffamily ʔinfiraːdi}}/}\color{black}}\ \textsc{adj}\ [m.]\ \textbf{1.}~incommunicado\  \begin{flushright}\color{gray}\foreignlanguage{arabic}{\textbf{\underline{\foreignlanguage{arabic}{أمثلة}}}: ضله بالحبس الاِنْفِرادي لمدة أسلوع بدون أكل أو شرب}\end{flushright}\color{black}} \vspace{2mm}

{\setlength\topsep{0pt}\textbf{\foreignlanguage{arabic}{تْفَرَّد}}\ {\color{gray}\texttt{/\sffamily {{\sffamily tfarrad}}/}\color{black}}\ \textsc{verb}\ [p.]\ \textbf{1.}~be unique in sth\ \ $\bullet$\ \ \setlength\topsep{0pt}\textbf{\foreignlanguage{arabic}{اِتْفَرَّد}}\ {\color{gray}\texttt{/\sffamily {{\sffamily ʔitfarrad}}/}\color{black}}\ [c.]\ \ $\bullet$\ \ \setlength\topsep{0pt}\textbf{\foreignlanguage{arabic}{يِتْفَرَّد}}\ {\color{gray}\texttt{/\sffamily {{\sffamily jitfarrad}}/}\color{black}}\ [i.]\  \begin{flushright}\color{gray}\foreignlanguage{arabic}{\textbf{\underline{\foreignlanguage{arabic}{أمثلة}}}: أهل طولكرم تْفَرَّدوا بصنع الكعك بسميد ونابلس تْفَرَّدوا بالزلابية}\end{flushright}\color{black}} \vspace{2mm}

{\setlength\topsep{0pt}\textbf{\foreignlanguage{arabic}{فَارْدِة}}\ {\color{gray}\texttt{/\sffamily {{\sffamily faːrde}}/}\color{black}}\ \textsc{noun}\ [f.]\ \textbf{1.}~a ceremonial procession (in weddings or other celebrations )\  \begin{flushright}\color{gray}\foreignlanguage{arabic}{\textbf{\underline{\foreignlanguage{arabic}{أمثلة}}}: طلعنا فارْدِة عشان نجيبها}\end{flushright}\color{black}} \vspace{2mm}

{\setlength\topsep{0pt}\textbf{\foreignlanguage{arabic}{فَرَد}}\ {\color{gray}\texttt{/\sffamily {{\sffamily farad}}/}\color{black}}\ \textsc{verb}\ [p.]\ \textbf{1.}~spread  \textbf{2.}~smile\ \ $\bullet$\ \ \setlength\topsep{0pt}\textbf{\foreignlanguage{arabic}{اِفْرِد}}\ {\color{gray}\texttt{/\sffamily {{\sffamily ʔifrid}}/}\color{black}}\ [c.]\ \ $\bullet$\ \ \setlength\topsep{0pt}\textbf{\foreignlanguage{arabic}{يِفْرِد}}\ {\color{gray}\texttt{/\sffamily {{\sffamily jifrid}}/}\color{black}}\ [i.]\ \color{gray}(msa. \foreignlanguage{arabic}{يبتسم}~\foreignlanguage{arabic}{\textbf{٢.}}  \foreignlanguage{arabic}{يَفْرِد}~\foreignlanguage{arabic}{\textbf{١.}})\color{black}\ \ $\bullet$\ \ \textsc{ph.} \color{gray} \foreignlanguage{arabic}{إِفردهَا}\color{black}\ {\color{gray}\texttt{/{\sffamily ʔifridha}/}\color{black}}\ \color{gray} (msa. \foreignlanguage{arabic}{تَشَجَع}~\foreignlanguage{arabic}{\textbf{٢.}}  \foreignlanguage{arabic}{إِبتسم}~\foreignlanguage{arabic}{\textbf{١.}})\color{black}\ \textbf{1.}~smile  \textbf{2.}~cheer up\  \begin{flushright}\color{gray}\foreignlanguage{arabic}{\textbf{\underline{\foreignlanguage{arabic}{أمثلة}}}: خلص يا زلمة ما تضل زعلان افردها\ $\bullet$\ \  افْرِد الشرشف عالسرير}\end{flushright}\color{black}} \vspace{2mm}

{\setlength\topsep{0pt}\textbf{\foreignlanguage{arabic}{فَرِد}}\ {\color{gray}\texttt{/\sffamily {{\sffamily farid}}/}\color{black}}\ \textsc{noun}\ [m.]\ \textbf{1.}~stretching (dough)\  \begin{flushright}\color{gray}\foreignlanguage{arabic}{\textbf{\underline{\foreignlanguage{arabic}{أمثلة}}}: فَرِد العجين ولا فيه أسهل منه}\end{flushright}\color{black}} \vspace{2mm}

{\setlength\topsep{0pt}\textbf{\foreignlanguage{arabic}{فَرِيد}}\ {\color{gray}\texttt{/\sffamily {{\sffamily fariːd}}/}\color{black}}\ \textsc{adj}\ [m.]\ \color{gray}(msa. \foreignlanguage{arabic}{فَريد}~\foreignlanguage{arabic}{\textbf{١.}})\color{black}\ \textbf{1.}~unique\  \begin{flushright}\color{gray}\foreignlanguage{arabic}{\textbf{\underline{\foreignlanguage{arabic}{أمثلة}}}: كعك القدس فَريد من نوعه}\end{flushright}\color{black}} \vspace{2mm}

{\setlength\topsep{0pt}\textbf{\foreignlanguage{arabic}{فَرَّد}}\ {\color{gray}\texttt{/\sffamily {{\sffamily farrad}}/}\color{black}}\ \textsc{verb}\ [p.]\ \textbf{1.}~spread\ \ $\bullet$\ \ \setlength\topsep{0pt}\textbf{\foreignlanguage{arabic}{فَرِّد}}\ {\color{gray}\texttt{/\sffamily {{\sffamily farrid}}/}\color{black}}\ [c.]\ \ $\bullet$\ \ \setlength\topsep{0pt}\textbf{\foreignlanguage{arabic}{يفَرِّد}}\ {\color{gray}\texttt{/\sffamily {{\sffamily jfarrid}}/}\color{black}}\ [i.]\ \color{gray}(msa. \foreignlanguage{arabic}{يَفْرِد}~\foreignlanguage{arabic}{\textbf{١.}})\color{black}\  \begin{flushright}\color{gray}\foreignlanguage{arabic}{\textbf{\underline{\foreignlanguage{arabic}{أمثلة}}}: فَرَّد الخبزات عشان يبردن}\end{flushright}\color{black}} \vspace{2mm}

{\setlength\topsep{0pt}\textbf{\foreignlanguage{arabic}{فَرْد}}\ {\color{gray}\texttt{/\sffamily {{\sffamily fard}}/}\color{black}}\ \textsc{noun}\ [m.]\ \color{gray}(msa. \foreignlanguage{arabic}{مُسدَّس}~\foreignlanguage{arabic}{\textbf{١.}})\color{black}\ \textbf{1.}~gun\ \ $\smblkdiamond$\ \ \setlength\topsep{0pt}\textbf{\foreignlanguage{arabic}{فَرْد}}\ \color{gray}(msa. \foreignlanguage{arabic}{فَرْد}~\foreignlanguage{arabic}{\textbf{١.}})\color{black}\ \textbf{1.}~a single person.  \textbf{2.}~an individual.\ \ $\bullet$\ \ \setlength\topsep{0pt}\textbf{\foreignlanguage{arabic}{فْرُودِة}}\ {\color{gray}\texttt{/\sffamily {{\sffamily fruːde}}/}\color{black}}\ [pl.]\ \ $\bullet$\ \ \setlength\topsep{0pt}\textbf{\foreignlanguage{arabic}{أَفْرَاد}}\ {\color{gray}\texttt{/\sffamily {{\sffamily ʔafraːd}}/}\color{black}}\ [pl.]\ \textbf{1.}~a single person.  \textbf{2.}~an individual.\  \begin{flushright}\color{gray}\foreignlanguage{arabic}{\textbf{\underline{\foreignlanguage{arabic}{أمثلة}}}: اعتبرني فَرِد من أفْراد العيلة\ $\bullet$\ \  اعتبرني فَرِد من أفْراد العيلة\ $\bullet$\ \  كان ماسك الفَرْد بايده}\end{flushright}\color{black}} \vspace{2mm}

{\setlength\topsep{0pt}\textbf{\foreignlanguage{arabic}{فَرْدِة}}\ {\color{gray}\texttt{/\sffamily {{\sffamily farde}}/}\color{black}}\ \textsc{noun}\ [f.]\ \color{gray}(msa. \foreignlanguage{arabic}{كيس خيش}~\foreignlanguage{arabic}{\textbf{١.}})\color{black}\ \textbf{1.}~sackcloth bag.  \textbf{2.}~one half of a pair.  \textbf{3.}~complement\  \begin{flushright}\color{gray}\foreignlanguage{arabic}{\textbf{\underline{\foreignlanguage{arabic}{أمثلة}}}: فَرْدِة شبشبي الضايعة مش لاقيتها}\end{flushright}\color{black}} \vspace{2mm}

{\setlength\topsep{0pt}\textbf{\foreignlanguage{arabic}{مِفْرِد}}\ {\color{gray}\texttt{/\sffamily {{\sffamily mifrid}}/}\color{black}}\ \textsc{adj}\ [m.]\ \color{gray}(msa. \foreignlanguage{arabic}{فَرْدِي}~\foreignlanguage{arabic}{\textbf{١.}})\color{black}\ \textbf{1.}~single\  \begin{flushright}\color{gray}\foreignlanguage{arabic}{\textbf{\underline{\foreignlanguage{arabic}{أمثلة}}}: أعطيني جودل مِفْرِد}\end{flushright}\color{black}} \vspace{2mm}

\vspace{-3mm}
\markboth{\color{blue}\foreignlanguage{arabic}{ف.ر.ر}\color{blue}{}}{\color{blue}\foreignlanguage{arabic}{ف.ر.ر}\color{blue}{}}\subsection*{\color{blue}\foreignlanguage{arabic}{ف.ر.ر}\color{blue}{}\index{\color{blue}\foreignlanguage{arabic}{ف.ر.ر}\color{blue}{}}} 

{\setlength\topsep{0pt}\textbf{\foreignlanguage{arabic}{فَرّ}}\ {\color{gray}\texttt{/\sffamily {{\sffamily farr}}/}\color{black}}\ \textsc{verb}\ [p.]\ \textbf{1.}~escape\ \ $\bullet$\ \ \setlength\topsep{0pt}\textbf{\foreignlanguage{arabic}{فُرّ}}\ {\color{gray}\texttt{/\sffamily {{\sffamily furr}}/}\color{black}}\ [c.]\ \ $\bullet$\ \ \setlength\topsep{0pt}\textbf{\foreignlanguage{arabic}{يْفُرّ}}\ {\color{gray}\texttt{/\sffamily {{\sffamily jfurr}}/}\color{black}}\ [i.]\ \color{gray}(msa. \foreignlanguage{arabic}{يَهْرُب}~\foreignlanguage{arabic}{\textbf{١.}})\color{black}\ } \vspace{2mm}

{\setlength\topsep{0pt}\textbf{\foreignlanguage{arabic}{فُرَّيرَة}}\ {\color{gray}\texttt{/\sffamily {{\sffamily furreːra}}/}\color{black}}\ \textsc{noun}\ [f.]\ \color{gray}(msa. \foreignlanguage{arabic}{لعبة تقليدية تشبه البلبل}~\foreignlanguage{arabic}{\textbf{١.}})\color{black}\ \textbf{1.}~It is a traditional game that is similar to the spinning top.  \textbf{2.}~spinner  \textbf{3.}~spinning top\  \begin{flushright}\color{gray}\foreignlanguage{arabic}{\textbf{\underline{\foreignlanguage{arabic}{أمثلة}}}: متعتنا واحنا صغار لما بقينا نلعب بالفُرِّْيرَة والله بقينا نكيف عالآخر}\end{flushright}\color{black}} \vspace{2mm}

{\setlength\topsep{0pt}\textbf{\foreignlanguage{arabic}{فِرَار}}\ {\color{gray}\texttt{/\sffamily {{\sffamily firaːr}}/}\color{black}}\ \textsc{noun}\ [m.]\ \color{gray}(msa. \foreignlanguage{arabic}{هُرُوب}~\foreignlanguage{arabic}{\textbf{١.}})\color{black}\ \textbf{1.}~escape\  \begin{flushright}\color{gray}\foreignlanguage{arabic}{\textbf{\underline{\foreignlanguage{arabic}{أمثلة}}}: فِرار من المجهول}\end{flushright}\color{black}} \vspace{2mm}

{\setlength\topsep{0pt}\textbf{\foreignlanguage{arabic}{فْرَارِي}}\ {\color{gray}\texttt{/\sffamily {{\sffamily fraːri}}/}\color{black}}\ \textsc{noun}\ [m.]\ \color{gray}(msa. \foreignlanguage{arabic}{جُندي فار من غير إِذن}~\foreignlanguage{arabic}{\textbf{١.}})\color{black}\ \textbf{1.}~deserter\ \ $\bullet$\ \ \setlength\topsep{0pt}\textbf{\foreignlanguage{arabic}{فْرَارِيِّة}}\ {\color{gray}\texttt{/\sffamily {{\sffamily fraːrijje}}/}\color{black}}\ [pl.]\  \begin{flushright}\color{gray}\foreignlanguage{arabic}{\textbf{\underline{\foreignlanguage{arabic}{أمثلة}}}: مجموعة فْرارِيِّة بيتصرمحوا تلا دار الشهيد أبو ياسر}\end{flushright}\color{black}} \vspace{2mm}

\vspace{-3mm}
\markboth{\color{blue}\foreignlanguage{arabic}{ف.ر.ز}\color{blue}{}}{\color{blue}\foreignlanguage{arabic}{ف.ر.ز}\color{blue}{}}\subsection*{\color{blue}\foreignlanguage{arabic}{ف.ر.ز}\color{blue}{}\index{\color{blue}\foreignlanguage{arabic}{ف.ر.ز}\color{blue}{}}} 

{\setlength\topsep{0pt}\textbf{\foreignlanguage{arabic}{تَفْرِيز}}\ {\color{gray}\texttt{/\sffamily {{\sffamily tafriːz}}/}\color{black}}\ \textsc{noun}\ [m.]\ \color{gray}(msa. \foreignlanguage{arabic}{تَجْميد}~\foreignlanguage{arabic}{\textbf{١.}})\color{black}\ \textbf{1.}~freezing\  \begin{flushright}\color{gray}\foreignlanguage{arabic}{\textbf{\underline{\foreignlanguage{arabic}{أمثلة}}}: تَفْرِيز ورق العنب أسهل من ضبُّه بقناني}\end{flushright}\color{black}} \vspace{2mm}

{\setlength\topsep{0pt}\textbf{\foreignlanguage{arabic}{تْفَرَّز}}\ {\color{gray}\texttt{/\sffamily {{\sffamily tfarraz}}/}\color{black}}\ \textsc{verb}\ [p.]\ \textbf{1.}~be frozen\ \ $\bullet$\ \ \setlength\topsep{0pt}\textbf{\foreignlanguage{arabic}{اِتْفَرَّز}}\ {\color{gray}\texttt{/\sffamily {{\sffamily ʔitfarraz}}/}\color{black}}\ [c.]\ \ $\bullet$\ \ \setlength\topsep{0pt}\textbf{\foreignlanguage{arabic}{يِتْفَرَّز}}\ {\color{gray}\texttt{/\sffamily {{\sffamily jitfarraz}}/}\color{black}}\ [i.]\ \color{gray}(msa. \foreignlanguage{arabic}{يتَجَمَّد}~\foreignlanguage{arabic}{\textbf{١.}})\color{black}\  \begin{flushright}\color{gray}\foreignlanguage{arabic}{\textbf{\underline{\foreignlanguage{arabic}{أمثلة}}}: مالحقش الجاج يِتْفَرَّز. ارمح شيله من الفريزر}\end{flushright}\color{black}} \vspace{2mm}

{\setlength\topsep{0pt}\textbf{\foreignlanguage{arabic}{فَرَز}}\ {\color{gray}\texttt{/\sffamily {{\sffamily faraz}}/}\color{black}}\ \textsc{verb}\ [p.]\ \textbf{1.}~sort  \textbf{2.}~sort sth out\ \ $\bullet$\ \ \setlength\topsep{0pt}\textbf{\foreignlanguage{arabic}{اِفْرِز}}\ {\color{gray}\texttt{/\sffamily {{\sffamily ʔifriz}}/}\color{black}}\ [c.]\ \ $\bullet$\ \ \setlength\topsep{0pt}\textbf{\foreignlanguage{arabic}{يِفْرِز}}\ {\color{gray}\texttt{/\sffamily {{\sffamily jifriz}}/}\color{black}}\ [i.]\ \color{gray}(msa. \foreignlanguage{arabic}{يَفْرِز}~\foreignlanguage{arabic}{\textbf{١.}})\color{black}\  \begin{flushright}\color{gray}\foreignlanguage{arabic}{\textbf{\underline{\foreignlanguage{arabic}{أمثلة}}}: حاول افرِز الطلبات اللي عندك وأنا}\end{flushright}\color{black}} \vspace{2mm}

{\setlength\topsep{0pt}\textbf{\foreignlanguage{arabic}{فَرِز}}\ {\color{gray}\texttt{/\sffamily {{\sffamily fariz}}/}\color{black}}\ \textsc{noun}\ [m.]\ \textbf{1.}~sorting things out\  \begin{flushright}\color{gray}\foreignlanguage{arabic}{\textbf{\underline{\foreignlanguage{arabic}{أمثلة}}}: فَرز الأصوات بده يوم كامل بعديها بيعلنوا عالنتيجة الخايبة تبعتهم آخر النهار}\end{flushright}\color{black}} \vspace{2mm}

{\setlength\topsep{0pt}\textbf{\foreignlanguage{arabic}{فَرَّازِة}}\ {\color{gray}\texttt{/\sffamily {{\sffamily farr\#ze}}/}\color{black}}\ \textsc{noun}\ [f.]\ \textbf{1.}~a machine to sort thing out\ \ $\bullet$\ \ \textsc{ph.} \color{gray} \foreignlanguage{arabic}{عَالفَرَّازِة}\color{black}\ {\color{gray}\texttt{/{\sffamily ʕal farr\#ze}/}\color{black}}\ \color{gray} (msa. \foreignlanguage{arabic}{مُخْتار بعنايَ’}~\foreignlanguage{arabic}{\textbf{١.}})\color{black}\ \textbf{1.}~well-chosen\  \begin{flushright}\color{gray}\foreignlanguage{arabic}{\textbf{\underline{\foreignlanguage{arabic}{أمثلة}}}: الشركة تبعتهم بيختاروا البنات عاالفَرّازِة}\end{flushright}\color{black}} \vspace{2mm}

{\setlength\topsep{0pt}\textbf{\foreignlanguage{arabic}{فَرَّز}}\ {\color{gray}\texttt{/\sffamily {{\sffamily farraz}}/}\color{black}}\ \textsc{verb}\ [p.]\ \textbf{1.}~freeze sth.  \textbf{2.}~put sth in the freezer\ \ $\bullet$\ \ \setlength\topsep{0pt}\textbf{\foreignlanguage{arabic}{فَرِّز}}\ {\color{gray}\texttt{/\sffamily {{\sffamily farriz}}/}\color{black}}\ [c.]\ \ $\bullet$\ \ \setlength\topsep{0pt}\textbf{\foreignlanguage{arabic}{يفَرِّز}}\ {\color{gray}\texttt{/\sffamily {{\sffamily jfarriz}}/}\color{black}}\ [i.]\ \color{gray}(msa. \foreignlanguage{arabic}{يُجَمِّد}~\foreignlanguage{arabic}{\textbf{١.}})\color{black}\  \begin{flushright}\color{gray}\foreignlanguage{arabic}{\textbf{\underline{\foreignlanguage{arabic}{أمثلة}}}: بينفع أفَرِّز البامية وهي مطبوخة ولا بيروح طعمها؟}\end{flushright}\color{black}} \vspace{2mm}

{\setlength\topsep{0pt}\textbf{\foreignlanguage{arabic}{مْفَرَّز}}\ {\color{gray}\texttt{/\sffamily {{\sffamily mfarraz}}/}\color{black}}\ \textsc{noun\textunderscore pass}\ \color{gray}(msa. \foreignlanguage{arabic}{مُجَمَّد}~\foreignlanguage{arabic}{\textbf{١.}})\color{black}\ \textbf{1.}~frozen\  \begin{flushright}\color{gray}\foreignlanguage{arabic}{\textbf{\underline{\foreignlanguage{arabic}{أمثلة}}}: جبنا ملوخية مْفَرَّزة والله ما أزكاها بتفرقيهاش عن الطازة}\end{flushright}\color{black}} \vspace{2mm}

\vspace{-3mm}
\markboth{\color{blue}\foreignlanguage{arabic}{ف.ر.ز}\color{blue}{ (ntws)}}{\color{blue}\foreignlanguage{arabic}{ف.ر.ز}\color{blue}{ (ntws)}}\subsection*{\color{blue}\foreignlanguage{arabic}{ف.ر.ز}\color{blue}{ (ntws)}\index{\color{blue}\foreignlanguage{arabic}{ف.ر.ز}\color{blue}{ (ntws)}}} 

{\setlength\topsep{0pt}\textbf{\foreignlanguage{arabic}{فْرَيزَر}}\ {\color{gray}\texttt{/\sffamily {{\sffamily freːzar}}/}\color{black}}\ \textsc{noun}\ [m.]\ \textbf{1.}~freezer\ } \vspace{2mm}

{\setlength\topsep{0pt}\textbf{\foreignlanguage{arabic}{فْرِيزَر}}\ {\color{gray}\texttt{/\sffamily {{\sffamily friːzar}}/}\color{black}}\ \textsc{noun}\ [m.]\ \textbf{1.}~freezer\ } \vspace{2mm}

\vspace{-3mm}
\markboth{\color{blue}\foreignlanguage{arabic}{ف.ر.س}\color{blue}{}}{\color{blue}\foreignlanguage{arabic}{ف.ر.س}\color{blue}{}}\subsection*{\color{blue}\foreignlanguage{arabic}{ف.ر.س}\color{blue}{}\index{\color{blue}\foreignlanguage{arabic}{ف.ر.س}\color{blue}{}}} 

{\setlength\topsep{0pt}\textbf{\foreignlanguage{arabic}{اِفْتَرَس}}\ {\color{gray}\texttt{/\sffamily {{\sffamily ʔiftaras}}/}\color{black}}\ \textsc{verb}\ [p.]\ \textbf{1.}~prey on.  \textbf{2.}~devour  \textbf{3.}~scarf down\ \ $\bullet$\ \ \setlength\topsep{0pt}\textbf{\foreignlanguage{arabic}{اِفْتِرِس}}\ {\color{gray}\texttt{/\sffamily {{\sffamily ʔiftiris}}/}\color{black}}\ [c.]\ \ $\bullet$\ \ \setlength\topsep{0pt}\textbf{\foreignlanguage{arabic}{يِفْتِرِس}}\ {\color{gray}\texttt{/\sffamily {{\sffamily jiftiris}}/}\color{black}}\ [i.]\  \begin{flushright}\color{gray}\foreignlanguage{arabic}{\textbf{\underline{\foreignlanguage{arabic}{أمثلة}}}: اِفْتَرَسنا الأكل كله ماخليلانك ولا شي}\end{flushright}\color{black}} \vspace{2mm}

{\setlength\topsep{0pt}\textbf{\foreignlanguage{arabic}{فَارِس}}\ {\color{gray}\texttt{/\sffamily {{\sffamily faːris}}/}\color{black}}\ \textsc{noun}\ [m.]\ \textbf{1.}~knight  \textbf{2.}~cavalry\ } \vspace{2mm}

{\setlength\topsep{0pt}\textbf{\foreignlanguage{arabic}{فَرَس}}\ {\color{gray}\texttt{/\sffamily {{\sffamily faras}}/}\color{black}}\ \textsc{noun}\ [m.]\ \color{gray}(msa. \foreignlanguage{arabic}{حِصان}~\foreignlanguage{arabic}{\textbf{١.}})\color{black}\ \textbf{1.}~horse\ \ $\bullet$\ \ \textsc{ph.} \color{gray} \foreignlanguage{arabic}{مَالك بتركض وبَايدك مرس، قَال نسيب نسيبنَا شَاريله فرس}\color{black}\ {\color{gray}\texttt{/{\sffamily maːlak ʔibturku(dˤ) wubʔiːdak maras (q)aːl nsiːb nsiːbna ʃaːriːlo faras}/}\color{black}}\ \color{gray} (msa. \foreignlanguage{arabic}{هو تعبير مجازي يُقْصَد به أن الشخص يتدخَّل فيما لا يعنيه}~\foreignlanguage{arabic}{\textbf{١.}})\color{black}\ \textbf{1.}~It is an idiomatic expression that means that sb is very intrusive in an annoying way\ \ $\bullet$\ \ \textsc{ph.} \color{gray} \foreignlanguage{arabic}{اِبْن العَم هو اللي بينزِّل عن ظهر الفَرَس}\color{black}\ {\color{gray}\texttt{/{\sffamily ʔibnil ʕamm huwwe ʔilli binazzil ʕan (dˤ)ahr ʔilfaras}/}\color{black}}\ \textbf{1.}~it is an idiomatic expression that means that sb should get married to his paternal cousin\  \begin{flushright}\color{gray}\foreignlanguage{arabic}{\textbf{\underline{\foreignlanguage{arabic}{أمثلة}}}: والله ماحدا يوخذها غير ابن عمها. ابْن العَم هو اللي بينزِّل عن ظهر الفَرَس\ $\bullet$\ \  بديعة انطبشت وإِجاها الديسك من ورا وقعة الفَرَس}\end{flushright}\color{black}} \vspace{2mm}

{\setlength\topsep{0pt}\textbf{\foreignlanguage{arabic}{مُفْتَرِس}}\ {\color{gray}\texttt{/\sffamily {{\sffamily muftaris}}/}\color{black}}\ \textsc{adj}\ [m.]\ \color{gray}(msa. \foreignlanguage{arabic}{مُفْتَرِس}~\foreignlanguage{arabic}{\textbf{١.}})\color{black}\ \textbf{1.}~predatory\  \begin{flushright}\color{gray}\foreignlanguage{arabic}{\textbf{\underline{\foreignlanguage{arabic}{أمثلة}}}: طب علينا مثل الحيوان المُفْتَرِس}\end{flushright}\color{black}} \vspace{2mm}

\vspace{-3mm}
\markboth{\color{blue}\foreignlanguage{arabic}{ف.ر.س.م.ن}\color{blue}{ (ntws)}}{\color{blue}\foreignlanguage{arabic}{ف.ر.س.م.ن}\color{blue}{ (ntws)}}\subsection*{\color{blue}\foreignlanguage{arabic}{ف.ر.س.م.ن}\color{blue}{ (ntws)}\index{\color{blue}\foreignlanguage{arabic}{ف.ر.س.م.ن}\color{blue}{ (ntws)}}} 

{\setlength\topsep{0pt}\textbf{\foreignlanguage{arabic}{فَرْسَمَونَا}}\ {\color{gray}\texttt{/\sffamily {{\sffamily farsamuːna}}/}\color{black}}\ \textsc{noun}\ [m.]\ \color{gray}(msa. \foreignlanguage{arabic}{كاكي}~\foreignlanguage{arabic}{\textbf{١.}})\color{black}\ \textbf{1.}~Persimmon\ } \vspace{2mm}

\vspace{-3mm}
\markboth{\color{blue}\foreignlanguage{arabic}{ف.ر.س.و.ح}\color{blue}{ (ntws)}}{\color{blue}\foreignlanguage{arabic}{ف.ر.س.و.ح}\color{blue}{ (ntws)}}\subsection*{\color{blue}\foreignlanguage{arabic}{ف.ر.س.و.ح}\color{blue}{ (ntws)}\index{\color{blue}\foreignlanguage{arabic}{ف.ر.س.و.ح}\color{blue}{ (ntws)}}} 

{\setlength\topsep{0pt}\textbf{\foreignlanguage{arabic}{فَرْسُوحَة}}\ {\color{gray}\texttt{/\sffamily {{\sffamily farsuːħa}}/}\color{black}}\ \textsc{noun}\ [m.]\ \color{gray}(msa. \foreignlanguage{arabic}{شاورما}~\foreignlanguage{arabic}{\textbf{١.}})\color{black}\ \textbf{1.}~shawirma\  \begin{flushright}\color{gray}\foreignlanguage{arabic}{\textbf{\underline{\foreignlanguage{arabic}{أمثلة}}}: جاي عبالي فرسوحة بلحمة مع سلطة بتشهي}\end{flushright}\color{black}} \vspace{2mm}

\vspace{-3mm}
\markboth{\color{blue}\foreignlanguage{arabic}{ف.ر.ش}\color{blue}{}}{\color{blue}\foreignlanguage{arabic}{ف.ر.ش}\color{blue}{}}\subsection*{\color{blue}\foreignlanguage{arabic}{ف.ر.ش}\color{blue}{}\index{\color{blue}\foreignlanguage{arabic}{ف.ر.ش}\color{blue}{}}} 

{\setlength\topsep{0pt}\textbf{\foreignlanguage{arabic}{اِنْفَرَش}}\ {\color{gray}\texttt{/\sffamily {{\sffamily ʔinfaraʃ}}/}\color{black}}\ \textsc{verb}\ [p.]\ \textbf{1.}~be spread.  \textbf{2.}~be furnished\ \ $\bullet$\ \ \setlength\topsep{0pt}\textbf{\foreignlanguage{arabic}{اِنْفِرِش}}\ {\color{gray}\texttt{/\sffamily {{\sffamily ʔinfiriʃ}}/}\color{black}}\ [c.]\ \ $\bullet$\ \ \setlength\topsep{0pt}\textbf{\foreignlanguage{arabic}{يِنْفِرِش}}\ {\color{gray}\texttt{/\sffamily {{\sffamily jinfiriʃ}}/}\color{black}}\ [i.]\  \begin{flushright}\color{gray}\foreignlanguage{arabic}{\textbf{\underline{\foreignlanguage{arabic}{أمثلة}}}: بدي الدار تِنْفِرِش كلها قبل ما نعمل العرس}\end{flushright}\color{black}} \vspace{2mm}

{\setlength\topsep{0pt}\textbf{\foreignlanguage{arabic}{فَارِش}}\ {\color{gray}\texttt{/\sffamily {{\sffamily faːriʃ}}/}\color{black}}\ \textsc{noun}\ [m.]\ \textbf{1.}~spreading  \textbf{2.}~furnishing\  \begin{flushright}\color{gray}\foreignlanguage{arabic}{\textbf{\underline{\foreignlanguage{arabic}{أمثلة}}}: بقى الحزلوط فارِش الشقة بالكامل}\end{flushright}\color{black}} \vspace{2mm}

{\setlength\topsep{0pt}\textbf{\foreignlanguage{arabic}{فَرَاشِة}}\ {\color{gray}\texttt{/\sffamily {{\sffamily faraːʃa}}/}\color{black}}\ \textsc{noun}\ [f.]\ \color{gray}(msa. \foreignlanguage{arabic}{فَراشَة}~\foreignlanguage{arabic}{\textbf{١.}})\color{black}\ \textbf{1.}~butterfly\ } \vspace{2mm}

{\setlength\topsep{0pt}\textbf{\foreignlanguage{arabic}{فَرَش}}\ {\color{gray}\texttt{/\sffamily {{\sffamily faraʃ}}/}\color{black}}\ \textsc{verb}\ [p.]\ \textbf{1.}~spread  \textbf{2.}~lay sth out.  \textbf{3.}~furnish\ \ $\bullet$\ \ \setlength\topsep{0pt}\textbf{\foreignlanguage{arabic}{اِفْرِش}}\ {\color{gray}\texttt{/\sffamily {{\sffamily ʔifriʃ}}/}\color{black}}\ [c.]\ \ $\bullet$\ \ \setlength\topsep{0pt}\textbf{\foreignlanguage{arabic}{اُفْرُش}}\ {\color{gray}\texttt{/\sffamily {{\sffamily ʔufruʃ}}/}\color{black}}\ [c.]\ \ $\bullet$\ \ \setlength\topsep{0pt}\textbf{\foreignlanguage{arabic}{يِفْرِش}}\ {\color{gray}\texttt{/\sffamily {{\sffamily jifriʃ}}/}\color{black}}\ [i.]\ \color{gray}(msa. \foreignlanguage{arabic}{يُأثِّث}~\foreignlanguage{arabic}{\textbf{٢.}}  \foreignlanguage{arabic}{يَفْرُش}~\foreignlanguage{arabic}{\textbf{١.}})\color{black}\ \ $\bullet$\ \ \setlength\topsep{0pt}\textbf{\foreignlanguage{arabic}{يُفْرُش}}\ {\color{gray}\texttt{/\sffamily {{\sffamily jufruʃ}}/}\color{black}}\ [i.]\ \color{gray}(msa. \foreignlanguage{arabic}{يُأثِّث}~\foreignlanguage{arabic}{\textbf{٢.}}  \foreignlanguage{arabic}{يَفْرُش}~\foreignlanguage{arabic}{\textbf{١.}})\color{black}\ \ $\bullet$\ \ \textsc{ph.} \color{gray} \foreignlanguage{arabic}{فَرَشِلْهَا الأَرْض وَرد}\color{black}\ {\color{gray}\texttt{/{\sffamily faraʃilha ʔilʔari(dˤ) ward}/}\color{black}}\ \textbf{1.}~It is an idiomatic expression that means that sb made a tremendous effort to make sb happy\  \begin{flushright}\color{gray}\foreignlanguage{arabic}{\textbf{\underline{\foreignlanguage{arabic}{أمثلة}}}: من كثر ما كان يحبها كانوا يقولوا إِنه فَرَشلها الأرض ورد\ $\bullet$\ \  وعدني إِنه يُفْرُش الشقة كلها قبل العيد\ $\bullet$\ \  اُفْرُشي السجاد والله كرَّزنا من البرد}\end{flushright}\color{black}} \vspace{2mm}

{\setlength\topsep{0pt}\textbf{\foreignlanguage{arabic}{فَرَّش}}\ {\color{gray}\texttt{/\sffamily {{\sffamily farraʃ}}/}\color{black}}\ \textsc{verb}\ [p.]\ \textbf{1.}~brush\ \ $\bullet$\ \ \setlength\topsep{0pt}\textbf{\foreignlanguage{arabic}{فَرِّش}}\ {\color{gray}\texttt{/\sffamily {{\sffamily farriʃ}}/}\color{black}}\ [c.]\ \ $\bullet$\ \ \setlength\topsep{0pt}\textbf{\foreignlanguage{arabic}{يفَرِّش}}\ {\color{gray}\texttt{/\sffamily {{\sffamily jfarriʃ}}/}\color{black}}\ [i.]\ \color{gray}(msa. \foreignlanguage{arabic}{يُنَظِّف بالفُرشاة}~\foreignlanguage{arabic}{\textbf{١.}})\color{black}\  \begin{flushright}\color{gray}\foreignlanguage{arabic}{\textbf{\underline{\foreignlanguage{arabic}{أمثلة}}}: فَرِّش أسنانك قبل ما تنام}\end{flushright}\color{black}} \vspace{2mm}

{\setlength\topsep{0pt}\textbf{\foreignlanguage{arabic}{فَرْشَى}}\ {\color{gray}\texttt{/\sffamily {{\sffamily farʃa}}/}\color{black}}\ \textsc{verb}\ [p.]\ \textbf{1.}~brush\ \ $\bullet$\ \ \setlength\topsep{0pt}\textbf{\foreignlanguage{arabic}{فَرْشِي}}\ {\color{gray}\texttt{/\sffamily {{\sffamily farʃi}}/}\color{black}}\ [c.]\ \ $\bullet$\ \ \setlength\topsep{0pt}\textbf{\foreignlanguage{arabic}{يفَرْشِي}}\ {\color{gray}\texttt{/\sffamily {{\sffamily jfarʃi}}/}\color{black}}\ [i.]\ \color{gray}(msa. \foreignlanguage{arabic}{يُنَظِّف بالفُرشاة}~\foreignlanguage{arabic}{\textbf{١.}})\color{black}\  \begin{flushright}\color{gray}\foreignlanguage{arabic}{\textbf{\underline{\foreignlanguage{arabic}{أمثلة}}}: الله يقرفه مش عارف كيف بيستحمل ما يفَرْشِي سنانه طول هالفترة}\end{flushright}\color{black}} \vspace{2mm}

{\setlength\topsep{0pt}\textbf{\foreignlanguage{arabic}{فَرْشِة}}\ {\color{gray}\texttt{/\sffamily {{\sffamily farʃe}}/}\color{black}}\ \textsc{noun}\ [f.]\ \color{gray}(msa. \foreignlanguage{arabic}{فرْشَة}~\foreignlanguage{arabic}{\textbf{١.}})\color{black}\ \textbf{1.}~mattress\  \begin{flushright}\color{gray}\foreignlanguage{arabic}{\textbf{\underline{\foreignlanguage{arabic}{أمثلة}}}: جيبلي فرْشِة بدي أنام عالأرض وهند بتنام عالتخت}\end{flushright}\color{black}} \vspace{2mm}

{\setlength\topsep{0pt}\textbf{\foreignlanguage{arabic}{فُرْشَايِة}}\ {\color{gray}\texttt{/\sffamily {{\sffamily furʃaːje}}/}\color{black}}\ \textsc{noun}\ [f.]\ \color{gray}(msa. \foreignlanguage{arabic}{فُرْشايَة}~\foreignlanguage{arabic}{\textbf{١.}})\color{black}\ \textbf{1.}~brush\ \ $\bullet$\ \ \setlength\topsep{0pt}\textbf{\foreignlanguage{arabic}{فَرَاشِي}}\ {\color{gray}\texttt{/\sffamily {{\sffamily faraːʃi}}/}\color{black}}\ [pl.]\  \begin{flushright}\color{gray}\foreignlanguage{arabic}{\textbf{\underline{\foreignlanguage{arabic}{أمثلة}}}: عندك فراشِي طراشة ولا أشتري من عند عدنان الدعباس}\end{flushright}\color{black}} \vspace{2mm}

{\setlength\topsep{0pt}\textbf{\foreignlanguage{arabic}{مَفْرَش}}\ {\color{gray}\texttt{/\sffamily {{\sffamily mafraʃ}}/}\color{black}}\ \textsc{noun}\ [m.]\ \textbf{1.}~sheet cover\ \ $\bullet$\ \ \setlength\topsep{0pt}\textbf{\foreignlanguage{arabic}{مَفَارِش}}\ {\color{gray}\texttt{/\sffamily {{\sffamily mafaːriʃ}}/}\color{black}}\ [pl.]\ } \vspace{2mm}

{\setlength\topsep{0pt}\textbf{\foreignlanguage{arabic}{مَفْرُوش}}\ {\color{gray}\texttt{/\sffamily {{\sffamily mafruːʃ}}/}\color{black}}\ \textsc{noun\textunderscore pass}\ \textbf{1.}~furnished\  \begin{flushright}\color{gray}\foreignlanguage{arabic}{\textbf{\underline{\foreignlanguage{arabic}{أمثلة}}}: لما فسخوا البيت بقى مَفْرُوش بالكامل}\end{flushright}\color{black}} \vspace{2mm}

\vspace{-3mm}
\markboth{\color{blue}\foreignlanguage{arabic}{ف.ر.ش.و.ح}\color{blue}{ (ntws)}}{\color{blue}\foreignlanguage{arabic}{ف.ر.ش.و.ح}\color{blue}{ (ntws)}}\subsection*{\color{blue}\foreignlanguage{arabic}{ف.ر.ش.و.ح}\color{blue}{ (ntws)}\index{\color{blue}\foreignlanguage{arabic}{ف.ر.ش.و.ح}\color{blue}{ (ntws)}}} 

{\setlength\topsep{0pt}\textbf{\foreignlanguage{arabic}{فَرْشُوحَة}}\ {\color{gray}\texttt{/\sffamily {{\sffamily farʃuːħa}}/}\color{black}}\ \textsc{noun}\ [f.]\ \color{gray}(msa. \foreignlanguage{arabic}{شطيرة}~\foreignlanguage{arabic}{\textbf{١.}})\color{black}\ \textbf{1.}~Sandwitch\  \begin{flushright}\color{gray}\foreignlanguage{arabic}{\textbf{\underline{\foreignlanguage{arabic}{أمثلة}}}: بدي فَرْشوحَة شاورما بخبز صاج}\end{flushright}\color{black}} \vspace{2mm}

\vspace{-3mm}
\markboth{\color{blue}\foreignlanguage{arabic}{ف.ر.ص}\color{blue}{}}{\color{blue}\foreignlanguage{arabic}{ف.ر.ص}\color{blue}{}}\subsection*{\color{blue}\foreignlanguage{arabic}{ف.ر.ص}\color{blue}{}\index{\color{blue}\foreignlanguage{arabic}{ف.ر.ص}\color{blue}{}}} 

{\setlength\topsep{0pt}\textbf{\foreignlanguage{arabic}{فُرْصَة}}\ {\color{gray}\texttt{/\sffamily {{\sffamily fursˤa}}/}\color{black}}\ \textsc{noun}\ [f.]\ \color{gray}(msa. \foreignlanguage{arabic}{فُرْصَة}~\foreignlanguage{arabic}{\textbf{١.}})\color{black}\ \textbf{1.}~chance  \textbf{2.}~opportunity\ \ $\bullet$\ \ \setlength\topsep{0pt}\textbf{\foreignlanguage{arabic}{فُرَص}}\ {\color{gray}\texttt{/\sffamily {{\sffamily furasˤ}}/}\color{black}}\ [pl.]\ \ $\bullet$\ \ \textsc{ph.} \color{gray} \foreignlanguage{arabic}{فُرْصَة لَاتعوَّض}\color{black}\ {\color{gray}\texttt{/{\sffamily fursˤa laː tuʕawwa(dˤ)}/}\color{black}}\ \textbf{1.}~a golden opportunity\ \ $\bullet$\ \ \textsc{ph.} \color{gray} \foreignlanguage{arabic}{فُرْصَة ذهبية}\color{black}\ {\color{gray}\texttt{/{\sffamily fursˤa laː (d)ahabijje}/}\color{black}}\ \textbf{1.}~a golden opportunity\  \begin{flushright}\color{gray}\foreignlanguage{arabic}{\textbf{\underline{\foreignlanguage{arabic}{أمثلة}}}: هاد العريس يا خوتة فُرْصَة لاتعوَّض!\ $\bullet$\ \  إِجتك فُرَص كثير للتغير بس أنت دابِّة مابتفهم}\end{flushright}\color{black}} \vspace{2mm}

\vspace{-3mm}
\markboth{\color{blue}\foreignlanguage{arabic}{ف.ر.ض}\color{blue}{}}{\color{blue}\foreignlanguage{arabic}{ف.ر.ض}\color{blue}{}}\subsection*{\color{blue}\foreignlanguage{arabic}{ف.ر.ض}\color{blue}{}\index{\color{blue}\foreignlanguage{arabic}{ف.ر.ض}\color{blue}{}}} 

{\setlength\topsep{0pt}\textbf{\foreignlanguage{arabic}{اِفْتَرَض}}\ {\color{gray}\texttt{/\sffamily {{\sffamily ʔiftara(dˤ)}}/}\color{black}}\ \textsc{verb}\ [p.]\ \textbf{1.}~assume  \textbf{2.}~suppose\ \ $\bullet$\ \ \setlength\topsep{0pt}\textbf{\foreignlanguage{arabic}{اِفْتِرِض}}\ {\color{gray}\texttt{/\sffamily {{\sffamily ʔiftiri(dˤ)}}/}\color{black}}\ [c.]\ \ $\bullet$\ \ \setlength\topsep{0pt}\textbf{\foreignlanguage{arabic}{اِفْتَرِض}}\ {\color{gray}\texttt{/\sffamily {{\sffamily ʔiftari(dˤ)}}/}\color{black}}\ [c.]\ \ $\bullet$\ \ \setlength\topsep{0pt}\textbf{\foreignlanguage{arabic}{يِفْتِرِض}}\ {\color{gray}\texttt{/\sffamily {{\sffamily jiftiri(dˤ)}}/}\color{black}}\ [i.]\ \color{gray}(msa. \foreignlanguage{arabic}{يًفْتَرِض}~\foreignlanguage{arabic}{\textbf{١.}})\color{black}\ \ $\bullet$\ \ \setlength\topsep{0pt}\textbf{\foreignlanguage{arabic}{يِفْتَرِض}}\ {\color{gray}\texttt{/\sffamily {{\sffamily jiftari(dˤ)}}/}\color{black}}\ [i.]\ \color{gray}(msa. \foreignlanguage{arabic}{يًفْتَرِض}~\foreignlanguage{arabic}{\textbf{١.}})\color{black}\  \begin{flushright}\color{gray}\foreignlanguage{arabic}{\textbf{\underline{\foreignlanguage{arabic}{أمثلة}}}: مابيصير من راسك تِفْتِرضي إِنه معي مصاري ومابدي مساعدته\ $\bullet$\ \  اِفْتِرِض انه احنا نساوين لحالنا وفات علينا حرامي ابن حرام}\end{flushright}\color{black}} \vspace{2mm}

{\setlength\topsep{0pt}\textbf{\foreignlanguage{arabic}{اِفْتِرَاض}}\ {\color{gray}\texttt{/\sffamily {{\sffamily ʔiftiraː(dˤ)}}/}\color{black}}\ \textsc{noun}\ [m.]\ \color{gray}(msa. \foreignlanguage{arabic}{اِفْتِراض}~\foreignlanguage{arabic}{\textbf{١.}})\color{black}\ \textbf{1.}~assumption\  \begin{flushright}\color{gray}\foreignlanguage{arabic}{\textbf{\underline{\foreignlanguage{arabic}{أمثلة}}}: يعني حضرتك بتِفْتِرِض اِفْتِراضات مش صحيحة}\end{flushright}\color{black}} \vspace{2mm}

{\setlength\topsep{0pt}\textbf{\foreignlanguage{arabic}{اِنْفَرَض}}\ {\color{gray}\texttt{/\sffamily {{\sffamily ʔinfara(dˤ)}}/}\color{black}}\ \textsc{verb}\ [p.]\ \textbf{1.}~be imposed\ \ $\bullet$\ \ \setlength\topsep{0pt}\textbf{\foreignlanguage{arabic}{اِنْفِرِض}}\ {\color{gray}\texttt{/\sffamily {{\sffamily ʔinfiri(dˤ)}}/}\color{black}}\ [c.]\ \ $\bullet$\ \ \setlength\topsep{0pt}\textbf{\foreignlanguage{arabic}{يِنْفِرِض}}\ {\color{gray}\texttt{/\sffamily {{\sffamily jinfiri(dˤ)}}/}\color{black}}\ [i.]\  \begin{flushright}\color{gray}\foreignlanguage{arabic}{\textbf{\underline{\foreignlanguage{arabic}{أمثلة}}}: اِنْفَرَض علي هالزواج وما كان عندي خيار أرفض}\end{flushright}\color{black}} \vspace{2mm}

{\setlength\topsep{0pt}\textbf{\foreignlanguage{arabic}{فَرَض}}\ {\color{gray}\texttt{/\sffamily {{\sffamily fara(dˤ)}}/}\color{black}}\ \textsc{verb}\ [p.]\ \textbf{1.}~impose\ \ $\bullet$\ \ \setlength\topsep{0pt}\textbf{\foreignlanguage{arabic}{اِفْرِض}}\ {\color{gray}\texttt{/\sffamily {{\sffamily ʔifri(dˤ)}}/}\color{black}}\ [c.]\ \ $\bullet$\ \ \setlength\topsep{0pt}\textbf{\foreignlanguage{arabic}{يِفْرِض}}\ {\color{gray}\texttt{/\sffamily {{\sffamily jifri(dˤ)}}/}\color{black}}\ [i.]\ \color{gray}(msa. \foreignlanguage{arabic}{يَفْرِض}~\foreignlanguage{arabic}{\textbf{١.}})\color{black}\  \begin{flushright}\color{gray}\foreignlanguage{arabic}{\textbf{\underline{\foreignlanguage{arabic}{أمثلة}}}: يعني أنت فَرَضت علي حياة وتفاصيل وما اهتميت اذا هالشغلات أنا بحبها ولا لا}\end{flushright}\color{black}} \vspace{2mm}

{\setlength\topsep{0pt}\textbf{\foreignlanguage{arabic}{فَرِيضَة}}\ {\color{gray}\texttt{/\sffamily {{\sffamily fariː(dˤ)a}}/}\color{black}}\ \textsc{noun}\ [f.]\ \color{gray}(msa. \foreignlanguage{arabic}{فَريضَة}~\foreignlanguage{arabic}{\textbf{١.}})\color{black}\ \textbf{1.}~obligatory act\ \ $\bullet$\ \ \setlength\topsep{0pt}\textbf{\foreignlanguage{arabic}{فَرَائِض}}\ {\color{gray}\texttt{/\sffamily {{\sffamily faraːʔi(dˤ)}}/}\color{black}}\ [pl.]\  \begin{flushright}\color{gray}\foreignlanguage{arabic}{\textbf{\underline{\foreignlanguage{arabic}{أمثلة}}}: إِذا الفرائِض بقومش فيها}\end{flushright}\color{black}} \vspace{2mm}

{\setlength\topsep{0pt}\textbf{\foreignlanguage{arabic}{مَفْرُوض}}\ {\color{gray}\texttt{/\sffamily {{\sffamily mafruː(dˤ)}}/}\color{black}}\ \textsc{noun\textunderscore pass}\ \textbf{1.}~imposed\ \ $\bullet$\ \ \textsc{ph.} \color{gray} \foreignlanguage{arabic}{المَفْرُوض}\color{black}\ {\color{gray}\texttt{/{\sffamily ʔilmafruːdˤ}/}\color{black}}\ \color{gray} (msa. \foreignlanguage{arabic}{من المُفْتَرَض}~\foreignlanguage{arabic}{\textbf{١.}})\color{black}\ \textbf{1.}~supposedly\  \begin{flushright}\color{gray}\foreignlanguage{arabic}{\textbf{\underline{\foreignlanguage{arabic}{أمثلة}}}: المَفْروض انك أنت اللي تيجي تسأل عني. أنو الولية فينا؟}\end{flushright}\color{black}} \vspace{2mm}

\vspace{-3mm}
\markboth{\color{blue}\foreignlanguage{arabic}{ف.ر.ط}\color{blue}{}}{\color{blue}\foreignlanguage{arabic}{ف.ر.ط}\color{blue}{}}\subsection*{\color{blue}\foreignlanguage{arabic}{ف.ر.ط}\color{blue}{}\index{\color{blue}\foreignlanguage{arabic}{ف.ر.ط}\color{blue}{}}} 

{\setlength\topsep{0pt}\textbf{\foreignlanguage{arabic}{اِنْفَرَط}}\ {\color{gray}\texttt{/\sffamily {{\sffamily ʔinfaratˤ}}/}\color{black}}\ \textsc{verb}\ [p.]\ \textbf{1.}~be disjoined.  \textbf{2.}~be torn off\ \ $\bullet$\ \ \setlength\topsep{0pt}\textbf{\foreignlanguage{arabic}{اِنْفِرِط}}\ {\color{gray}\texttt{/\sffamily {{\sffamily ʔinfiritˤ}}/}\color{black}}\ [c.]\ \ $\bullet$\ \ \setlength\topsep{0pt}\textbf{\foreignlanguage{arabic}{يِنْفِرِط}}\ {\color{gray}\texttt{/\sffamily {{\sffamily jinfiritˤ}}/}\color{black}}\ [i.]\  \begin{flushright}\color{gray}\foreignlanguage{arabic}{\textbf{\underline{\foreignlanguage{arabic}{أمثلة}}}: مارضيتش أطمِّل كثير. خفت البنطلون يِنْفِرِط.}\end{flushright}\color{black}} \vspace{2mm}

{\setlength\topsep{0pt}\textbf{\foreignlanguage{arabic}{تْفَرَّط}}\ {\color{gray}\texttt{/\sffamily {{\sffamily tfarratˤ}}/}\color{black}}\ \textsc{verb}\ [p.]\ \textbf{1.}~be taken out (the leaves).  \textbf{2.}~be abandoned.  \textbf{3.}~be given up\ \ $\bullet$\ \ \setlength\topsep{0pt}\textbf{\foreignlanguage{arabic}{اِتْفَرَّط}}\ {\color{gray}\texttt{/\sffamily {{\sffamily ʔitfarratˤ}}/}\color{black}}\ [c.]\ \ $\bullet$\ \ \setlength\topsep{0pt}\textbf{\foreignlanguage{arabic}{يِتْفَرَّط}}\ {\color{gray}\texttt{/\sffamily {{\sffamily jitfarratˤ}}/}\color{black}}\ [i.]\  \begin{flushright}\color{gray}\foreignlanguage{arabic}{\textbf{\underline{\foreignlanguage{arabic}{أمثلة}}}: هذا الزلمة ابن ناس. مابيِتْفَرَّط فيه.\ $\bullet$\ \  يختي السبانخ سهلة. بتتْفَرَّط بأقل من ساعة. مش زي الملوخية.}\end{flushright}\color{black}} \vspace{2mm}

{\setlength\topsep{0pt}\textbf{\foreignlanguage{arabic}{فَارِط}}\ {\color{gray}\texttt{/\sffamily {{\sffamily faːritˤ}}/}\color{black}}\ \textsc{adj}\ [m.]\ \textbf{1.}~insignificant  \textbf{2.}~decadent  \textbf{3.}~deteriorated  \textbf{4.}~useless\ } \vspace{2mm}

{\setlength\topsep{0pt}\textbf{\foreignlanguage{arabic}{فَرَط}}\ {\color{gray}\texttt{/\sffamily {{\sffamily faratˤ}}/}\color{black}}\ \textsc{verb}\ [p.]\ \textbf{1.}~disjoin  \textbf{2.}~did not work.  \textbf{3.}~did not pay off.  \textbf{4.}~the yogurt went slimy.  \textbf{5.}~kick the bucket\ \ $\bullet$\ \ \setlength\topsep{0pt}\textbf{\foreignlanguage{arabic}{اِفْرُط}}\ {\color{gray}\texttt{/\sffamily {{\sffamily ʔifrutˤ}}/}\color{black}}\ [c.]\ \ $\bullet$\ \ \setlength\topsep{0pt}\textbf{\foreignlanguage{arabic}{يِفْرُط}}\ {\color{gray}\texttt{/\sffamily {{\sffamily jifrutˤ}}/}\color{black}}\ [i.]\ \color{gray}(msa. \foreignlanguage{arabic}{فطس}~\foreignlanguage{arabic}{\textbf{٥.}}  \foreignlanguage{arabic}{مات}~\foreignlanguage{arabic}{\textbf{٤.}}  .\foreignlanguage{arabic}{يُصبح اللبن لزج وبه كتل}~\foreignlanguage{arabic}{\textbf{٣.}}  .\foreignlanguage{arabic}{لم تثمر}~\foreignlanguage{arabic}{\textbf{٢.}}  .\foreignlanguage{arabic}{لم تنجح}~\foreignlanguage{arabic}{\textbf{١.}})\color{black}\ \ $\bullet$\ \ \textsc{ph.} \color{gray} \foreignlanguage{arabic}{فَرَط من الضُّحُك}\color{black}\ {\color{gray}\texttt{/{\sffamily faratˤ min ʔi(dˤ)(dˤ)uħuk}/}\color{black}}\ \color{gray} (msa. \foreignlanguage{arabic}{يضحك بطريقة هستيرية}~\foreignlanguage{arabic}{\textbf{١.}})\color{black}\ \textbf{1.}~laugh hysterically\ \ $\bullet$\ \ \textsc{ph.} \color{gray} \foreignlanguage{arabic}{فرطت المسبحة}\color{black}\ {\color{gray}\texttt{/{\sffamily fartˤat ʔilmasbaħa}/}\color{black}}\ \color{gray} (msa. \foreignlanguage{arabic}{فتاة تزوجت قبل صديقاتها - أخواتها}~\foreignlanguage{arabic}{\textbf{١.}})\color{black}\ \textbf{1.}~It is an idiomatic expression that means that a young lady was the first to get married from her group\  \begin{flushright}\color{gray}\foreignlanguage{arabic}{\textbf{\underline{\foreignlanguage{arabic}{أمثلة}}}: وهيك خلاص فَرََطَت المسبحة وعقبال عند كل الضبايا\ $\bullet$\ \  ماله فَرَط من كثرة الضحك؟ لسة ما حكينا شي\ $\bullet$\ \  فَرََطَت القصة كلها\ $\bullet$\ \  فَرَط اللَّبن معي وأنا بحرك فيه\ $\bullet$\ \  أبو قاعود فَرَط الله لا يرده}\end{flushright}\color{black}} \vspace{2mm}

{\setlength\topsep{0pt}\textbf{\foreignlanguage{arabic}{فَرَّط}}\ {\color{gray}\texttt{/\sffamily {{\sffamily farratˤ}}/}\color{black}}\ \textsc{verb}\ [p.]\ \textbf{1.}~take out the leaves of sth.  \textbf{2.}~abandon sb/sth.  \textbf{3.}~give sb/sth up\ \ $\bullet$\ \ \setlength\topsep{0pt}\textbf{\foreignlanguage{arabic}{فَرِّط}}\ {\color{gray}\texttt{/\sffamily {{\sffamily farritˤ}}/}\color{black}}\ [c.]\ \ $\bullet$\ \ \setlength\topsep{0pt}\textbf{\foreignlanguage{arabic}{يفَرِّط}}\ {\color{gray}\texttt{/\sffamily {{\sffamily jfarritˤ}}/}\color{black}}\ [i.]\ \color{gray}(msa. \foreignlanguage{arabic}{يَقْتَص الجزء الكبير من الورق}~\foreignlanguage{arabic}{\textbf{١.}})\color{black}\  \begin{flushright}\color{gray}\foreignlanguage{arabic}{\textbf{\underline{\foreignlanguage{arabic}{أمثلة}}}: هيّاتني عمّالي بَفَرِّط بملوخية}\end{flushright}\color{black}} \vspace{2mm}

{\setlength\topsep{0pt}\textbf{\foreignlanguage{arabic}{فْرَاطَة}}\ {\color{gray}\texttt{/\sffamily {{\sffamily fraːtˤa}}/}\color{black}}\ \textsc{noun}\ [f.]\ \textbf{1.}~change (money)\  \begin{flushright}\color{gray}\foreignlanguage{arabic}{\textbf{\underline{\foreignlanguage{arabic}{أمثلة}}}: معك فْراطَة؟}\end{flushright}\color{black}} \vspace{2mm}

{\setlength\topsep{0pt}\textbf{\foreignlanguage{arabic}{مْفَرَّط}}\ {\color{gray}\texttt{/\sffamily {{\sffamily mfarratˤ}}/}\color{black}}\ \textsc{noun\textunderscore pass}\ \textbf{1.}~be disjointed.  \textbf{2.}~be taken out (leaves)\  \begin{flushright}\color{gray}\foreignlanguage{arabic}{\textbf{\underline{\foreignlanguage{arabic}{أمثلة}}}: هاي مرة خالك بتشتريش ملوخية إِلا مْفَرَّطَة من عند العلي}\end{flushright}\color{black}} \vspace{2mm}

\vspace{-3mm}
\markboth{\color{blue}\foreignlanguage{arabic}{ف.ر.ط.ح}\color{blue}{}}{\color{blue}\foreignlanguage{arabic}{ف.ر.ط.ح}\color{blue}{}}\subsection*{\color{blue}\foreignlanguage{arabic}{ف.ر.ط.ح}\color{blue}{}\index{\color{blue}\foreignlanguage{arabic}{ف.ر.ط.ح}\color{blue}{}}} 

{\setlength\topsep{0pt}\textbf{\foreignlanguage{arabic}{تْفَرْطَح}}\ {\color{gray}\texttt{/\sffamily {{\sffamily tfartˤaħ}}/}\color{black}}\ \textsc{verb}\ [p.]\ \textbf{1.}~be flattened.  \textbf{2.}~become flat\ \ $\bullet$\ \ \setlength\topsep{0pt}\textbf{\foreignlanguage{arabic}{اِتْفَرْطَح}}\ {\color{gray}\texttt{/\sffamily {{\sffamily ʔitfartˤaħ}}/}\color{black}}\ [c.]\ \ $\bullet$\ \ \setlength\topsep{0pt}\textbf{\foreignlanguage{arabic}{يِتْفَرْطَح}}\ {\color{gray}\texttt{/\sffamily {{\sffamily jitfartˤaħ}}/}\color{black}}\ [i.]\  \begin{flushright}\color{gray}\foreignlanguage{arabic}{\textbf{\underline{\foreignlanguage{arabic}{أمثلة}}}: شوف كيف تْفَرْطَح الطحين بالمعلقة. أحسن من إنك تساويه بإيدك.}\end{flushright}\color{black}} \vspace{2mm}

{\setlength\topsep{0pt}\textbf{\foreignlanguage{arabic}{فَرْطَح}}\ {\color{gray}\texttt{/\sffamily {{\sffamily fartˤaħ}}/}\color{black}}\ \textsc{verb}\ [p.]\ \textbf{1.}~flatten\ \ $\bullet$\ \ \setlength\topsep{0pt}\textbf{\foreignlanguage{arabic}{فَرْطِح}}\ {\color{gray}\texttt{/\sffamily {{\sffamily fartˤiħ}}/}\color{black}}\ [c.]\ \ $\bullet$\ \ \setlength\topsep{0pt}\textbf{\foreignlanguage{arabic}{يفَرْطِح}}\ {\color{gray}\texttt{/\sffamily {{\sffamily jfartˤiħ}}/}\color{black}}\ [i.]\ \color{gray}(msa. \foreignlanguage{arabic}{يُسَطِّح}~\foreignlanguage{arabic}{\textbf{١.}})\color{black}\  \begin{flushright}\color{gray}\foreignlanguage{arabic}{\textbf{\underline{\foreignlanguage{arabic}{أمثلة}}}: حاول فَرْطِح الأرض بإِيدك لما تغرس البذرة}\end{flushright}\color{black}} \vspace{2mm}

{\setlength\topsep{0pt}\textbf{\foreignlanguage{arabic}{مْفَرْطَح}}\ {\color{gray}\texttt{/\sffamily {{\sffamily mfartˤaħ}}/}\color{black}}\ \textsc{adj}\ [m.]\ \color{gray}(msa. \foreignlanguage{arabic}{مُسَطَّح}~\foreignlanguage{arabic}{\textbf{١.}})\color{black}\ \textbf{1.}~flat  \textbf{2.}~flattened\ } \vspace{2mm}

\vspace{-3mm}
\markboth{\color{blue}\foreignlanguage{arabic}{ف.ر.ط.ش}\color{blue}{}}{\color{blue}\foreignlanguage{arabic}{ف.ر.ط.ش}\color{blue}{}}\subsection*{\color{blue}\foreignlanguage{arabic}{ف.ر.ط.ش}\color{blue}{}\index{\color{blue}\foreignlanguage{arabic}{ف.ر.ط.ش}\color{blue}{}}} 

{\setlength\topsep{0pt}\textbf{\foreignlanguage{arabic}{فَرْطُوشِة}}\ {\color{gray}\texttt{/\sffamily {{\sffamily fartˤuːʃe}}/}\color{black}}\ \textsc{noun}\ [m.]\ (src. \color{gray}\foreignlanguage{arabic}{الشمال}\color{black})\ \color{gray}(msa. \foreignlanguage{arabic}{طرف خيط}~\foreignlanguage{arabic}{\textbf{١.}})\color{black}\ \textbf{1.}~a lead\ \ $\bullet$\ \ \setlength\topsep{0pt}\textbf{\foreignlanguage{arabic}{فَرْطِيش}}\ {\color{gray}\texttt{/\sffamily {{\sffamily faratˤiːʃ}}/}\color{black}}\ [pl.]\ \color{gray}(msa. \foreignlanguage{arabic}{أدلة}~\foreignlanguage{arabic}{\textbf{١.}})\color{black}\ \textbf{1.}~clues\  \begin{flushright}\color{gray}\foreignlanguage{arabic}{\textbf{\underline{\foreignlanguage{arabic}{أمثلة}}}: سمعت انه لقيوا فرطوشة بقضيته}\end{flushright}\color{black}} \vspace{2mm}

\vspace{-3mm}
\markboth{\color{blue}\foreignlanguage{arabic}{ف.ر.ع}\color{blue}{}}{\color{blue}\foreignlanguage{arabic}{ف.ر.ع}\color{blue}{}}\subsection*{\color{blue}\foreignlanguage{arabic}{ف.ر.ع}\color{blue}{}\index{\color{blue}\foreignlanguage{arabic}{ف.ر.ع}\color{blue}{}}} 

{\setlength\topsep{0pt}\textbf{\foreignlanguage{arabic}{تَفْرِيعَة}}\ {\color{gray}\texttt{/\sffamily {{\sffamily tafriːʕa}}/}\color{black}}\ \textsc{noun}\ [f.]\ \color{gray}(msa. \foreignlanguage{arabic}{قميص نوم}~\foreignlanguage{arabic}{\textbf{١.}})\color{black}\ \textbf{1.}~night dress.  \textbf{2.}~lingerie\ \ $\bullet$\ \ \setlength\topsep{0pt}\textbf{\foreignlanguage{arabic}{تَفَارِيع}}\ {\color{gray}\texttt{/\sffamily {{\sffamily tafaːriːʕ}}/}\color{black}}\ [pl.]\  \begin{flushright}\color{gray}\foreignlanguage{arabic}{\textbf{\underline{\foreignlanguage{arabic}{أمثلة}}}: أنو اللي بده يطلع معي أجيب التَّفاريع قبل العرس}\end{flushright}\color{black}} \vspace{2mm}

{\setlength\topsep{0pt}\textbf{\foreignlanguage{arabic}{تْفَرَّع}}\ {\color{gray}\texttt{/\sffamily {{\sffamily tfarraʕ}}/}\color{black}}\ \textsc{verb}\ [p.]\ \textbf{1.}~branch into.  \textbf{2.}~wear lingerie\ \ $\bullet$\ \ \setlength\topsep{0pt}\textbf{\foreignlanguage{arabic}{اِتْفَرَّع}}\ {\color{gray}\texttt{/\sffamily {{\sffamily ʔitfarraʕ}}/}\color{black}}\ [c.]\ \ $\bullet$\ \ \setlength\topsep{0pt}\textbf{\foreignlanguage{arabic}{يِتْفَرَّع}}\ {\color{gray}\texttt{/\sffamily {{\sffamily jitfarraʕ}}/}\color{black}}\ [i.]\ \color{gray}(msa. \foreignlanguage{arabic}{ترتدي قميص نوم}~\foreignlanguage{arabic}{\textbf{٢.}}  \foreignlanguage{arabic}{يَتَفرَّع}~\foreignlanguage{arabic}{\textbf{١.}})\color{black}\  \begin{flushright}\color{gray}\foreignlanguage{arabic}{\textbf{\underline{\foreignlanguage{arabic}{أمثلة}}}: يختي اِتْفَرَّعي لجوزك بالدار لساتكم عرسان جدد}\end{flushright}\color{black}} \vspace{2mm}

{\setlength\topsep{0pt}\textbf{\foreignlanguage{arabic}{فَارُوعَة}}\ {\color{gray}\texttt{/\sffamily {{\sffamily faːruːʕa}}/}\color{black}}\ \textsc{noun}\ [f.]\ \textbf{1.}~a small axe with a short handle\ \ $\bullet$\ \ \setlength\topsep{0pt}\textbf{\foreignlanguage{arabic}{فَوَارِيع}}\ {\color{gray}\texttt{/\sffamily {{\sffamily fawaːriːʕ}}/}\color{black}}\ [pl.]\  \begin{flushright}\color{gray}\foreignlanguage{arabic}{\textbf{\underline{\foreignlanguage{arabic}{أمثلة}}}: امسك الفاروعة هيك وبعدين بأقوى ماعندك اضربها عالخشب}\end{flushright}\color{black}} \vspace{2mm}

{\setlength\topsep{0pt}\textbf{\foreignlanguage{arabic}{فَرِع}}\ {\color{gray}\texttt{/\sffamily {{\sffamily fariʕ}}/}\color{black}}\ \textsc{noun}\ [m.]\ \color{gray}(msa. \foreignlanguage{arabic}{فَرْع}~\foreignlanguage{arabic}{\textbf{١.}})\color{black}\ \textbf{1.}~branch\ \ $\bullet$\ \ \setlength\topsep{0pt}\textbf{\foreignlanguage{arabic}{فْرُوع}}\ {\color{gray}\texttt{/\sffamily {{\sffamily fruːʕ}}/}\color{black}}\ [pl.]\ \ $\bullet$\ \ \setlength\topsep{0pt}\textbf{\foreignlanguage{arabic}{أَفْرُع}}\ {\color{gray}\texttt{/\sffamily {{\sffamily ʔafruʕ}}/}\color{black}}\ [pl.]\  \begin{flushright}\color{gray}\foreignlanguage{arabic}{\textbf{\underline{\foreignlanguage{arabic}{أمثلة}}}: الصالحي الهم أفْرُع كثير بكل مكان\ $\bullet$\ \  افتتحوا فَرِع للمحل بقلقيليا}\end{flushright}\color{black}} \vspace{2mm}

{\setlength\topsep{0pt}\textbf{\foreignlanguage{arabic}{فَرَّع}}\ {\color{gray}\texttt{/\sffamily {{\sffamily farraʕ}}/}\color{black}}\ \textsc{verb}\ [p.]\ \textbf{1.}~take off Hijab.  \textbf{2.}~unveil\ \ $\bullet$\ \ \setlength\topsep{0pt}\textbf{\foreignlanguage{arabic}{فَرِّع}}\ {\color{gray}\texttt{/\sffamily {{\sffamily farriʕ}}/}\color{black}}\ [c.]\ \ $\bullet$\ \ \setlength\topsep{0pt}\textbf{\foreignlanguage{arabic}{يفَرِّع}}\ {\color{gray}\texttt{/\sffamily {{\sffamily jfarriʕ}}/}\color{black}}\ [i.]\ \color{gray}(msa. \foreignlanguage{arabic}{يَخْلَع الحجاب}~\foreignlanguage{arabic}{\textbf{١.}})\color{black}\  \begin{flushright}\color{gray}\foreignlanguage{arabic}{\textbf{\underline{\foreignlanguage{arabic}{أمثلة}}}: طلعت برة وفَرَّعَت ونسيت دينها وعاداتها وتقاليدها}\end{flushright}\color{black}} \vspace{2mm}

{\setlength\topsep{0pt}\textbf{\foreignlanguage{arabic}{فَرْعِي}}\ {\color{gray}\texttt{/\sffamily {{\sffamily farʕi}}/}\color{black}}\ \textsc{adj}\ [m.]\ \textbf{1.}~access  \textbf{2.}~bypass  \textbf{3.}~secondry\  \begin{flushright}\color{gray}\foreignlanguage{arabic}{\textbf{\underline{\foreignlanguage{arabic}{أمثلة}}}: في طريق فَرْعِي دايما بمر منها لما أوصل حوّارة}\end{flushright}\color{black}} \vspace{2mm}

{\setlength\topsep{0pt}\textbf{\foreignlanguage{arabic}{مْفَرِّع}}\ {\color{gray}\texttt{/\sffamily {{\sffamily mfarriʕ}}/}\color{black}}\ \textsc{adj}\ [m.]\ \color{gray}(msa. \foreignlanguage{arabic}{غير مُحجَّبة}~\foreignlanguage{arabic}{\textbf{١.}})\color{black}\ \textbf{1.}~non-Hijabi  \textbf{2.}~a lady who wears see-through clothes\ \ $\smblkdiamond$\ \ \setlength\topsep{0pt}\textbf{\foreignlanguage{arabic}{مْفَرِّع}}\ \textbf{1.}~sth that has many branches\ \ $\bullet$\ \ \textsc{ph.} \color{gray} \foreignlanguage{arabic}{مِية مْفَرِّع ومِية مْدَرِّع}\color{black}\ {\color{gray}\texttt{/{\sffamily miːt mfarriʕ wumiːt mdarriʕ}/}\color{black}}\ \textbf{1.}~many people are involved in an event\  \begin{flushright}\color{gray}\foreignlanguage{arabic}{\textbf{\underline{\foreignlanguage{arabic}{أمثلة}}}: ماخِد وحْدِة مْفَرْعَة بالرغم إِنه أهله ناس محاظين}\end{flushright}\color{black}} \vspace{2mm}

\vspace{-3mm}
\markboth{\color{blue}\foreignlanguage{arabic}{ف.ر.ع.ن}\color{blue}{}}{\color{blue}\foreignlanguage{arabic}{ف.ر.ع.ن}\color{blue}{}}\subsection*{\color{blue}\foreignlanguage{arabic}{ف.ر.ع.ن}\color{blue}{}\index{\color{blue}\foreignlanguage{arabic}{ف.ر.ع.ن}\color{blue}{}}} 

{\setlength\topsep{0pt}\textbf{\foreignlanguage{arabic}{تْفَرْعَن}}\ {\color{gray}\texttt{/\sffamily {{\sffamily tfarʕan}}/}\color{black}}\ \textsc{verb}\ [p.]\ \textbf{1.}~flex his muscles\ \ $\bullet$\ \ \setlength\topsep{0pt}\textbf{\foreignlanguage{arabic}{اِتْفَرْعَن}}\ {\color{gray}\texttt{/\sffamily {{\sffamily ʔitfarʕan}}/}\color{black}}\ [c.]\ \ $\bullet$\ \ \setlength\topsep{0pt}\textbf{\foreignlanguage{arabic}{يِتْفَرْعَن}}\ {\color{gray}\texttt{/\sffamily {{\sffamily jitfarʕan}}/}\color{black}}\ [i.]\  \begin{flushright}\color{gray}\foreignlanguage{arabic}{\textbf{\underline{\foreignlanguage{arabic}{أمثلة}}}: شايفة إنه أخوك بلش يِتْفَرْعَن وصار لازم ينحطله حد}\end{flushright}\color{black}} \vspace{2mm}

{\setlength\topsep{0pt}\textbf{\foreignlanguage{arabic}{فَرْعَن}}\ {\color{gray}\texttt{/\sffamily {{\sffamily farʕan}}/}\color{black}}\ \textsc{verb}\ [p.]\ \textbf{1.}~flex his muscles\ \ $\bullet$\ \ \setlength\topsep{0pt}\textbf{\foreignlanguage{arabic}{فَرْعِن}}\ {\color{gray}\texttt{/\sffamily {{\sffamily farʕin}}/}\color{black}}\ [c.]\ \ $\bullet$\ \ \setlength\topsep{0pt}\textbf{\foreignlanguage{arabic}{يفَرْعِن}}\ {\color{gray}\texttt{/\sffamily {{\sffamily jfarʕin}}/}\color{black}}\ [i.]\ \color{gray}(msa. \foreignlanguage{arabic}{يَستعْرِض عضلاته}~\foreignlanguage{arabic}{\textbf{١.}})\color{black}\  \begin{flushright}\color{gray}\foreignlanguage{arabic}{\textbf{\underline{\foreignlanguage{arabic}{أمثلة}}}: محمد فَرْعَن بزيادة بس راح أخوه الكبير يشتغل غربا}\end{flushright}\color{black}} \vspace{2mm}

{\setlength\topsep{0pt}\textbf{\foreignlanguage{arabic}{فَرْعَنِة}}\ {\color{gray}\texttt{/\sffamily {{\sffamily farʕane}}/}\color{black}}\ \textsc{noun}\ [f.]\ \color{gray}(msa. \foreignlanguage{arabic}{استعراض العضلات}~\foreignlanguage{arabic}{\textbf{١.}})\color{black}\ \textbf{1.}~flexing sb's muscles\  \begin{flushright}\color{gray}\foreignlanguage{arabic}{\textbf{\underline{\foreignlanguage{arabic}{أمثلة}}}: شغل الفَرْعَنِة بمشيش معي بدك تكبر راس بكسرلك راسك أنت وأكبر واحد بعيلتك.}\end{flushright}\color{black}} \vspace{2mm}

{\setlength\topsep{0pt}\textbf{\foreignlanguage{arabic}{مْفَرْعِن}}\ {\color{gray}\texttt{/\sffamily {{\sffamily mfarʕin}}/}\color{black}}\ \textsc{adj}\ [m.]\ \color{gray}(msa. \foreignlanguage{arabic}{متسلط}~\foreignlanguage{arabic}{\textbf{١.}})\color{black}\ \textbf{1.}~domineering\  \begin{flushright}\color{gray}\foreignlanguage{arabic}{\textbf{\underline{\foreignlanguage{arabic}{أمثلة}}}: أول سنة إِله بقى مْفَرْعِن وماحدا بيقدر يحكي معه كلمة. هلا لو تشوفيه نَخ وهِدِي}\end{flushright}\color{black}} \vspace{2mm}

\vspace{-3mm}
\markboth{\color{blue}\foreignlanguage{arabic}{ف.ر.غ}\color{blue}{}}{\color{blue}\foreignlanguage{arabic}{ف.ر.غ}\color{blue}{}}\subsection*{\color{blue}\foreignlanguage{arabic}{ف.ر.غ}\color{blue}{}\index{\color{blue}\foreignlanguage{arabic}{ف.ر.غ}\color{blue}{}}} 

{\setlength\topsep{0pt}\textbf{\foreignlanguage{arabic}{اِسْتَفْرَغ}}\ {\color{gray}\texttt{/\sffamily {{\sffamily ʔistafraɣ}}/}\color{black}}\ \textsc{verb}\ [p.]\ \textbf{1.}~vomit\ \ $\bullet$\ \ \setlength\topsep{0pt}\textbf{\foreignlanguage{arabic}{اِسْتَفْرِغ}}\ {\color{gray}\texttt{/\sffamily {{\sffamily ʔistafriɣ}}/}\color{black}}\ [c.]\ \ $\bullet$\ \ \setlength\topsep{0pt}\textbf{\foreignlanguage{arabic}{يِسْتَفْرِغ}}\ {\color{gray}\texttt{/\sffamily {{\sffamily jistafriɣ}}/}\color{black}}\ [i.]\ \color{gray}(msa. \foreignlanguage{arabic}{يَتَقيَّأ}~\foreignlanguage{arabic}{\textbf{١.}})\color{black}\  \begin{flushright}\color{gray}\foreignlanguage{arabic}{\textbf{\underline{\foreignlanguage{arabic}{أمثلة}}}: بحاول قد ما أقدر ما اِسْتَفْرَغ بس مش قادرة معدتي ماكلة زفت}\end{flushright}\color{black}} \vspace{2mm}

{\setlength\topsep{0pt}\textbf{\foreignlanguage{arabic}{اِسْتِفْرَاغ}}\ {\color{gray}\texttt{/\sffamily {{\sffamily ʔistifraːɣ}}/}\color{black}}\ \textsc{noun}\ [m.]\ \color{gray}(msa. \foreignlanguage{arabic}{تَقَيُّؤ}~\foreignlanguage{arabic}{\textbf{١.}})\color{black}\ \textbf{1.}~vomit\  \begin{flushright}\color{gray}\foreignlanguage{arabic}{\textbf{\underline{\foreignlanguage{arabic}{أمثلة}}}: سمعت انه الاِسْتِفْراغ بريح عشان هيك حاول اِسْتَفْرِغ}\end{flushright}\color{black}} \vspace{2mm}

{\setlength\topsep{0pt}\textbf{\foreignlanguage{arabic}{تْفَرَّغ}}\ {\color{gray}\texttt{/\sffamily {{\sffamily tfarraɣ}}/}\color{black}}\ \textsc{verb}\ [p.]\ \textbf{1.}~be empty out.  \textbf{2.}~be hollowed out.  \textbf{3.}~be free\ \ $\bullet$\ \ \setlength\topsep{0pt}\textbf{\foreignlanguage{arabic}{اِتْفَرَّغ}}\ {\color{gray}\texttt{/\sffamily {{\sffamily ʔitfarraɣ}}/}\color{black}}\ [c.]\ \ $\bullet$\ \ \setlength\topsep{0pt}\textbf{\foreignlanguage{arabic}{يِتْفَرَّغ}}\ {\color{gray}\texttt{/\sffamily {{\sffamily jitfarraɣ}}/}\color{black}}\ [i.]\  \begin{flushright}\color{gray}\foreignlanguage{arabic}{\textbf{\underline{\foreignlanguage{arabic}{أمثلة}}}: هاي الشنتة لازم الليلة تِتْفَرَّغ عشان نعبيها مونة لأختك\ $\bullet$\ \  خلصت كل الشغل اللي علي وتْفَرَّغت للطشات وشمات الهوا}\end{flushright}\color{black}} \vspace{2mm}

{\setlength\topsep{0pt}\textbf{\foreignlanguage{arabic}{فَارِغ}}\ {\color{gray}\texttt{/\sffamily {{\sffamily faːriɣ}}/}\color{black}}\ \textsc{adj}\ [m.]\ \color{gray}(msa. \foreignlanguage{arabic}{فارِغ}~\foreignlanguage{arabic}{\textbf{١.}})\color{black}\ \textbf{1.}~empty\ \ $\bullet$\ \ \textsc{ph.} \color{gray} \foreignlanguage{arabic}{عينه فَارغة}\color{black}\ {\color{gray}\texttt{/{\sffamily ʕeːno faːrɣa}/}\color{black}}\ \color{gray} (msa. \foreignlanguage{arabic}{طمّاع أو جَشِع}~\foreignlanguage{arabic}{\textbf{١.}})\color{black}\ \textbf{1.}~greedy  \textbf{2.}~covetous\  \begin{flushright}\color{gray}\foreignlanguage{arabic}{\textbf{\underline{\foreignlanguage{arabic}{أمثلة}}}: بني آدم عِينُه فارْغِة ما بملى عينه غير التراب}\end{flushright}\color{black}} \vspace{2mm}

{\setlength\topsep{0pt}\textbf{\foreignlanguage{arabic}{فَرَاغ}}\ {\color{gray}\texttt{/\sffamily {{\sffamily faraːɣ}}/}\color{black}}\ \textsc{noun}\ [m.]\ \textbf{1.}~space  \textbf{2.}~blank space.  \textbf{3.}~free time\  \begin{flushright}\color{gray}\foreignlanguage{arabic}{\textbf{\underline{\foreignlanguage{arabic}{أمثلة}}}: الفَراغ رح يقتلتي}\end{flushright}\color{black}} \vspace{2mm}

{\setlength\topsep{0pt}\textbf{\foreignlanguage{arabic}{فَرَّغ}}\ {\color{gray}\texttt{/\sffamily {{\sffamily farraɣ}}/}\color{black}}\ \textsc{verb}\ [p.]\ \textbf{1.}~empty out.  \textbf{2.}~hollow out.  \textbf{3.}~be free\ \ $\bullet$\ \ \setlength\topsep{0pt}\textbf{\foreignlanguage{arabic}{فَرِّغ}}\ {\color{gray}\texttt{/\sffamily {{\sffamily farriɣ}}/}\color{black}}\ [c.]\ \ $\bullet$\ \ \setlength\topsep{0pt}\textbf{\foreignlanguage{arabic}{يفَرِّغ}}\ {\color{gray}\texttt{/\sffamily {{\sffamily jfarriɣ}}/}\color{black}}\ [i.]\  \begin{flushright}\color{gray}\foreignlanguage{arabic}{\textbf{\underline{\foreignlanguage{arabic}{أمثلة}}}: فَرِّغلي حالك ليوم بلكي نزلنا عنابلس شفنا فساتين عرس\ $\bullet$\ \  مسك حبة الكبة وفَرِّغها من كل اللحمة وأكل الطبقة البرانية وقال شو بحب اللي برة بس}\end{flushright}\color{black}} \vspace{2mm}

{\setlength\topsep{0pt}\textbf{\foreignlanguage{arabic}{فَوَارِغ}}\ {\color{gray}\texttt{/\sffamily {{\sffamily fawaːriɣ}}/}\color{black}}\ \textsc{noun}\ [pl.]\ \textbf{1.}~stuffed tripe and intestines\  \begin{flushright}\color{gray}\foreignlanguage{arabic}{\textbf{\underline{\foreignlanguage{arabic}{أمثلة}}}: عزمونا دار أخوي عفَوارِغ}\end{flushright}\color{black}} \vspace{2mm}

{\setlength\topsep{0pt}\textbf{\foreignlanguage{arabic}{فِرِغ}}\ {\color{gray}\texttt{/\sffamily {{\sffamily firiɣ}}/}\color{black}}\ \textsc{verb}\ [p.]\ \textbf{1.}~become empty.  \textbf{2.}~become unoccupied\ \ $\bullet$\ \ \setlength\topsep{0pt}\textbf{\foreignlanguage{arabic}{اِفْرَغ}}\ {\color{gray}\texttt{/\sffamily {{\sffamily ʔifraɣ}}/}\color{black}}\ [c.]\ \ $\bullet$\ \ \setlength\topsep{0pt}\textbf{\foreignlanguage{arabic}{يِفْرَغ}}\ {\color{gray}\texttt{/\sffamily {{\sffamily jifraɣ}}/}\color{black}}\ [i.]\ \color{gray}(msa. \foreignlanguage{arabic}{يصبح فارِغ}~\foreignlanguage{arabic}{\textbf{٢.}}  \foreignlanguage{arabic}{يَفْرَغ}~\foreignlanguage{arabic}{\textbf{١.}})\color{black}\  \begin{flushright}\color{gray}\foreignlanguage{arabic}{\textbf{\underline{\foreignlanguage{arabic}{أمثلة}}}: كل ما يِفْرَغ بيرجعوا يعبوه من الحشوة الجديدة وهيك}\end{flushright}\color{black}} \vspace{2mm}

\vspace{-3mm}
\markboth{\color{blue}\foreignlanguage{arabic}{ف.ر.ف.ح}\color{blue}{}}{\color{blue}\foreignlanguage{arabic}{ف.ر.ف.ح}\color{blue}{}}\subsection*{\color{blue}\foreignlanguage{arabic}{ف.ر.ف.ح}\color{blue}{}\index{\color{blue}\foreignlanguage{arabic}{ف.ر.ف.ح}\color{blue}{}}} 

{\setlength\topsep{0pt}\textbf{\foreignlanguage{arabic}{فَرْفَح}}\ {\color{gray}\texttt{/\sffamily {{\sffamily farfaħ}}/}\color{black}}\ \textsc{verb}\ [p.]\ \textbf{1.}~make sb happy.  \textbf{2.}~gladden\ \ $\bullet$\ \ \setlength\topsep{0pt}\textbf{\foreignlanguage{arabic}{فَرْفِح}}\ {\color{gray}\texttt{/\sffamily {{\sffamily farfiħ}}/}\color{black}}\ [c.]\ \ $\bullet$\ \ \setlength\topsep{0pt}\textbf{\foreignlanguage{arabic}{يفَرْفِح}}\ {\color{gray}\texttt{/\sffamily {{\sffamily jfarfiħ}}/}\color{black}}\ [i.]\  \begin{flushright}\color{gray}\foreignlanguage{arabic}{\textbf{\underline{\foreignlanguage{arabic}{أمثلة}}}: والله فَرْفَح قلبي بس شفتها}\end{flushright}\color{black}} \vspace{2mm}

{\setlength\topsep{0pt}\textbf{\foreignlanguage{arabic}{فَرْفَحَة}}\ {\color{gray}\texttt{/\sffamily {{\sffamily farfaħa}}/}\color{black}}\ \textsc{noun}\ [f.]\ \textbf{1.}~happiness\ } \vspace{2mm}

{\setlength\topsep{0pt}\textbf{\foreignlanguage{arabic}{مْفَرْفِح}}\ {\color{gray}\texttt{/\sffamily {{\sffamily mfarfiħ}}/}\color{black}}\ \textsc{adj}\ [m.]\ \textbf{1.}~cheerful  \textbf{2.}~happy\ } \vspace{2mm}

\vspace{-3mm}
\markboth{\color{blue}\foreignlanguage{arabic}{ف.ر.ف.د}\color{blue}{}}{\color{blue}\foreignlanguage{arabic}{ف.ر.ف.د}\color{blue}{}}\subsection*{\color{blue}\foreignlanguage{arabic}{ف.ر.ف.د}\color{blue}{}\index{\color{blue}\foreignlanguage{arabic}{ف.ر.ف.د}\color{blue}{}}} 

{\setlength\topsep{0pt}\textbf{\foreignlanguage{arabic}{تْفَرْفَد}}\ {\color{gray}\texttt{/\sffamily {{\sffamily tfarfad}}/}\color{black}}\ \textsc{verb}\ [p.]\ \textbf{1.}~loosen  \textbf{2.}~spread widely.  \textbf{3.}~act freely\ \ $\bullet$\ \ \setlength\topsep{0pt}\textbf{\foreignlanguage{arabic}{اِتْفَرْفَد}}\ {\color{gray}\texttt{/\sffamily {{\sffamily ʔitfarfad}}/}\color{black}}\ [c.]\ \ $\bullet$\ \ \setlength\topsep{0pt}\textbf{\foreignlanguage{arabic}{يِتْفَرْفَد}}\ {\color{gray}\texttt{/\sffamily {{\sffamily jitfarfad}}/}\color{black}}\ [i.]\  \begin{flushright}\color{gray}\foreignlanguage{arabic}{\textbf{\underline{\foreignlanguage{arabic}{أمثلة}}}: يا زم اِتْفَرْفَد وخذ راحتك. الدار دارك.}\end{flushright}\color{black}} \vspace{2mm}

{\setlength\topsep{0pt}\textbf{\foreignlanguage{arabic}{فَرْفَد}}\ {\color{gray}\texttt{/\sffamily {{\sffamily farfad}}/}\color{black}}\ \textsc{verb}\ [p.]\ \textbf{1.}~loosen  \textbf{2.}~spread widely\ \ $\bullet$\ \ \setlength\topsep{0pt}\textbf{\foreignlanguage{arabic}{فَرْفِد}}\ {\color{gray}\texttt{/\sffamily {{\sffamily farfid}}/}\color{black}}\ [c.]\ \ $\bullet$\ \ \setlength\topsep{0pt}\textbf{\foreignlanguage{arabic}{يفَرْفِد}}\ {\color{gray}\texttt{/\sffamily {{\sffamily jfarfid}}/}\color{black}}\ [i.]\  \begin{flushright}\color{gray}\foreignlanguage{arabic}{\textbf{\underline{\foreignlanguage{arabic}{أمثلة}}}: حسيت الصيصان فَرْفَدن بس روحوا الزلام}\end{flushright}\color{black}} \vspace{2mm}

{\setlength\topsep{0pt}\textbf{\foreignlanguage{arabic}{مْفَرْفِد}}\ {\color{gray}\texttt{/\sffamily {{\sffamily mfarfid}}/}\color{black}}\ \textsc{adj}\ [m.]\ \textbf{1.}~loose  \textbf{2.}~spreading widely\  \begin{flushright}\color{gray}\foreignlanguage{arabic}{\textbf{\underline{\foreignlanguage{arabic}{أمثلة}}}: البلايز مْفَرْفِدِة ومريحيتني كثير}\end{flushright}\color{black}} \vspace{2mm}

\vspace{-3mm}
\markboth{\color{blue}\foreignlanguage{arabic}{ف.ر.ف.ر}\color{blue}{}}{\color{blue}\foreignlanguage{arabic}{ف.ر.ف.ر}\color{blue}{}}\subsection*{\color{blue}\foreignlanguage{arabic}{ف.ر.ف.ر}\color{blue}{}\index{\color{blue}\foreignlanguage{arabic}{ف.ر.ف.ر}\color{blue}{}}} 

{\setlength\topsep{0pt}\textbf{\foreignlanguage{arabic}{فَرْفَر}}\ {\color{gray}\texttt{/\sffamily {{\sffamily farfar}}/}\color{black}}\ \textsc{verb}\ [p.]\ \textbf{1.}~urinate  \textbf{2.}~go around for long hours\ \ $\bullet$\ \ \setlength\topsep{0pt}\textbf{\foreignlanguage{arabic}{فَرْفِر}}\ {\color{gray}\texttt{/\sffamily {{\sffamily farfir}}/}\color{black}}\ [c.]\ \ $\bullet$\ \ \setlength\topsep{0pt}\textbf{\foreignlanguage{arabic}{يفَرْفِر}}\ {\color{gray}\texttt{/\sffamily {{\sffamily jfarfir}}/}\color{black}}\ [i.]\ \color{gray}(msa. \foreignlanguage{arabic}{يقضي حاجته}~\foreignlanguage{arabic}{\textbf{١.}})\color{black}\  \begin{flushright}\color{gray}\foreignlanguage{arabic}{\textbf{\underline{\foreignlanguage{arabic}{أمثلة}}}: ضلينا نفَرْفِر بهالأسواق عأمل نلاقي فستان يعجب السِّت\ $\bullet$\ \  ابنك فَرْفَر عالسجاد الحقيه}\end{flushright}\color{black}} \vspace{2mm}

{\setlength\topsep{0pt}\textbf{\foreignlanguage{arabic}{فَرْفَرَة}}\ {\color{gray}\texttt{/\sffamily {{\sffamily farfara}}/}\color{black}}\ \textsc{noun}\ [f.]\ \textbf{1.}~urinating  \textbf{2.}~going around for long hours\  \begin{flushright}\color{gray}\foreignlanguage{arabic}{\textbf{\underline{\foreignlanguage{arabic}{أمثلة}}}: ما شبعتوش من فَرْفَرَة الأسواق}\end{flushright}\color{black}} \vspace{2mm}

{\setlength\topsep{0pt}\textbf{\foreignlanguage{arabic}{فَرْفُور}}\ {\color{gray}\texttt{/\sffamily {{\sffamily farfuːr}}/}\color{black}}\ \textsc{adj}\ [m.]\ \color{gray}(msa. \foreignlanguage{arabic}{مَشاغِب}~\foreignlanguage{arabic}{\textbf{١.}})\color{black}\ \textbf{1.}~naughty\ \ $\bullet$\ \ \setlength\topsep{0pt}\textbf{\foreignlanguage{arabic}{فَرَافِير}}\ {\color{gray}\texttt{/\sffamily {{\sffamily faraːfiːr}}/}\color{black}}\ [pl.]\  \begin{flushright}\color{gray}\foreignlanguage{arabic}{\textbf{\underline{\foreignlanguage{arabic}{أمثلة}}}: هاي الفَرْفورة الصغيرة اسمها جنى}\end{flushright}\color{black}} \vspace{2mm}

\vspace{-3mm}
\markboth{\color{blue}\foreignlanguage{arabic}{ف.ر.ف.ش}\color{blue}{}}{\color{blue}\foreignlanguage{arabic}{ف.ر.ف.ش}\color{blue}{}}\subsection*{\color{blue}\foreignlanguage{arabic}{ف.ر.ف.ش}\color{blue}{}\index{\color{blue}\foreignlanguage{arabic}{ف.ر.ف.ش}\color{blue}{}}} 

{\setlength\topsep{0pt}\textbf{\foreignlanguage{arabic}{فَرْفَش}}\ {\color{gray}\texttt{/\sffamily {{\sffamily farfaʃ}}/}\color{black}}\ \textsc{verb}\ [p.]\ \textbf{1.}~make sb happy.  \textbf{2.}~cheer sb up.  \textbf{3.}~enjoy  \textbf{4.}~have fun\ \ $\bullet$\ \ \setlength\topsep{0pt}\textbf{\foreignlanguage{arabic}{فَرْفِش}}\ {\color{gray}\texttt{/\sffamily {{\sffamily farfiʃ}}/}\color{black}}\ [c.]\ \ $\bullet$\ \ \setlength\topsep{0pt}\textbf{\foreignlanguage{arabic}{يفَرْفِش}}\ {\color{gray}\texttt{/\sffamily {{\sffamily jfarfiʃ}}/}\color{black}}\ [i.]\  \begin{flushright}\color{gray}\foreignlanguage{arabic}{\textbf{\underline{\foreignlanguage{arabic}{أمثلة}}}: طلعنا و فَرْفَشْنا ونسينا كل المشكلة}\end{flushright}\color{black}} \vspace{2mm}

{\setlength\topsep{0pt}\textbf{\foreignlanguage{arabic}{فَرْفَشِة}}\ {\color{gray}\texttt{/\sffamily {{\sffamily farfaʃe}}/}\color{black}}\ \textsc{noun}\ [f.]\ \textbf{1.}~happiness  \textbf{2.}~being funny\  \begin{flushright}\color{gray}\foreignlanguage{arabic}{\textbf{\underline{\foreignlanguage{arabic}{أمثلة}}}: بحب أجواء الطلعات والفَرْفَشِة بالضات وقت كنا ساكنين برام الله}\end{flushright}\color{black}} \vspace{2mm}

{\setlength\topsep{0pt}\textbf{\foreignlanguage{arabic}{فَرْفُوش}}\ {\color{gray}\texttt{/\sffamily {{\sffamily farfuːʃ}}/}\color{black}}\ \textsc{adj}\ [m.]\ \textbf{1.}~funny\ \ $\bullet$\ \ \setlength\topsep{0pt}\textbf{\foreignlanguage{arabic}{فَرَافِيش}}\ {\color{gray}\texttt{/\sffamily {{\sffamily faraːfiːʃ}}/}\color{black}}\ [pl.]\  \begin{flushright}\color{gray}\foreignlanguage{arabic}{\textbf{\underline{\foreignlanguage{arabic}{أمثلة}}}: بحب الزلمة الفَرْفوش اللي مابينكِّد عمرته أبداً}\end{flushright}\color{black}} \vspace{2mm}

{\setlength\topsep{0pt}\textbf{\foreignlanguage{arabic}{مْفَرْفِش}}\ {\color{gray}\texttt{/\sffamily {{\sffamily mfarfiʃ}}/}\color{black}}\ \textsc{adj}\ [m.]\ \textbf{1.}~happy\  \begin{flushright}\color{gray}\foreignlanguage{arabic}{\textbf{\underline{\foreignlanguage{arabic}{أمثلة}}}: شكلك اليوم مْفَرْفِشة مش زي امبارح}\end{flushright}\color{black}} \vspace{2mm}

\vspace{-3mm}
\markboth{\color{blue}\foreignlanguage{arabic}{ف.ر.ف.ط}\color{blue}{}}{\color{blue}\foreignlanguage{arabic}{ف.ر.ف.ط}\color{blue}{}}\subsection*{\color{blue}\foreignlanguage{arabic}{ف.ر.ف.ط}\color{blue}{}\index{\color{blue}\foreignlanguage{arabic}{ف.ر.ف.ط}\color{blue}{}}} 

{\setlength\topsep{0pt}\textbf{\foreignlanguage{arabic}{فَرْفَط}}\ {\color{gray}\texttt{/\sffamily {{\sffamily farfatˤ}}/}\color{black}}\ \textsc{verb}\ [p.]\ (src. \color{gray}\foreignlanguage{arabic}{جنين}\color{black})\ \textbf{1.}~get bored\ \ $\bullet$\ \ \setlength\topsep{0pt}\textbf{\foreignlanguage{arabic}{فَرْفِط}}\ {\color{gray}\texttt{/\sffamily {{\sffamily farfitˤ}}/}\color{black}}\ [c.]\ \ $\bullet$\ \ \setlength\topsep{0pt}\textbf{\foreignlanguage{arabic}{يفَرْفِط}}\ {\color{gray}\texttt{/\sffamily {{\sffamily jfarfitˤ}}/}\color{black}}\ [i.]\ \color{gray}(msa. \foreignlanguage{arabic}{يَمِل}~\foreignlanguage{arabic}{\textbf{١.}})\color{black}\  \begin{flushright}\color{gray}\foreignlanguage{arabic}{\textbf{\underline{\foreignlanguage{arabic}{أمثلة}}}: انا فرفطت روحي من هالحجر نفسي اطلع}\end{flushright}\color{black}} \vspace{2mm}

{\setlength\topsep{0pt}\textbf{\foreignlanguage{arabic}{مْفَرْفِط}}\ {\color{gray}\texttt{/\sffamily {{\sffamily mfarfitˤ}}/}\color{black}}\ \textsc{adj}\ [m.]\ \textbf{1.}~bored\  \begin{flushright}\color{gray}\foreignlanguage{arabic}{\textbf{\underline{\foreignlanguage{arabic}{أمثلة}}}: والله روحي مْفَرْفِطة شو أسوي يعني}\end{flushright}\color{black}} \vspace{2mm}

\vspace{-3mm}
\markboth{\color{blue}\foreignlanguage{arabic}{ف.ر.ف.ك}\color{blue}{}}{\color{blue}\foreignlanguage{arabic}{ف.ر.ف.ك}\color{blue}{}}\subsection*{\color{blue}\foreignlanguage{arabic}{ف.ر.ف.ك}\color{blue}{}\index{\color{blue}\foreignlanguage{arabic}{ف.ر.ف.ك}\color{blue}{}}} 

{\setlength\topsep{0pt}\textbf{\foreignlanguage{arabic}{تْفَرْفَك}}\ {\color{gray}\texttt{/\sffamily {{\sffamily tfarfak}}/}\color{black}}\ \textsc{verb}\ [p.]\ \textbf{1.}~be rubbed.  \textbf{2.}~be scrubbed\ \ $\bullet$\ \ \setlength\topsep{0pt}\textbf{\foreignlanguage{arabic}{اِتْفَرْفَك}}\ {\color{gray}\texttt{/\sffamily {{\sffamily ʔitfarfak}}/}\color{black}}\ [c.]\ \ $\bullet$\ \ \setlength\topsep{0pt}\textbf{\foreignlanguage{arabic}{يِتْفَرْفَك}}\ {\color{gray}\texttt{/\sffamily {{\sffamily jitfarfak}}/}\color{black}}\ [i.]\  \begin{flushright}\color{gray}\foreignlanguage{arabic}{\textbf{\underline{\foreignlanguage{arabic}{أمثلة}}}: نوع هالبلاط بيخزي. كل شي بيطبِّع عليه. لازم يِتْفَرْفَك منيح عشان ينظف.}\end{flushright}\color{black}} \vspace{2mm}

{\setlength\topsep{0pt}\textbf{\foreignlanguage{arabic}{فَرْفَك}}\ {\color{gray}\texttt{/\sffamily {{\sffamily farfak}}/}\color{black}}\ \textsc{verb}\ [p.]\ \textbf{1.}~rubb  \textbf{2.}~scrub\ \ $\bullet$\ \ \setlength\topsep{0pt}\textbf{\foreignlanguage{arabic}{فَرْفِك}}\ {\color{gray}\texttt{/\sffamily {{\sffamily farfik}}/}\color{black}}\ [c.]\ \ $\bullet$\ \ \setlength\topsep{0pt}\textbf{\foreignlanguage{arabic}{يفَرْفِك}}\ {\color{gray}\texttt{/\sffamily {{\sffamily jfarfik}}/}\color{black}}\ [i.]\  \begin{flushright}\color{gray}\foreignlanguage{arabic}{\textbf{\underline{\foreignlanguage{arabic}{أمثلة}}}: امسك الشريطة وبلها شوي بالمي وفَرْفِكها منيح}\end{flushright}\color{black}} \vspace{2mm}

{\setlength\topsep{0pt}\textbf{\foreignlanguage{arabic}{فَرْفَكِة}}\ {\color{gray}\texttt{/\sffamily {{\sffamily farfake}}/}\color{black}}\ \textsc{noun}\ [f.]\ \textbf{1.}~rubbing  \textbf{2.}~scrubing\ } \vspace{2mm}

\vspace{-3mm}
\markboth{\color{blue}\foreignlanguage{arabic}{ف.ر.ق}\color{blue}{}}{\color{blue}\foreignlanguage{arabic}{ف.ر.ق}\color{blue}{}}\subsection*{\color{blue}\foreignlanguage{arabic}{ف.ر.ق}\color{blue}{}\index{\color{blue}\foreignlanguage{arabic}{ف.ر.ق}\color{blue}{}}} 

{\setlength\topsep{0pt}\textbf{\foreignlanguage{arabic}{اِنْفَرَق}}\ {\color{gray}\texttt{/\sffamily {{\sffamily ʔinfara(q)}}/}\color{black}}\ \textsc{verb}\ [p.]\ \textbf{1.}~be splitted\ \ $\bullet$\ \ \setlength\topsep{0pt}\textbf{\foreignlanguage{arabic}{اِنْفَرَق}}\ {\color{gray}\texttt{/\sffamily {{\sffamily ʔinfara(q)}}/}\color{black}}\ [c.]\ \ $\bullet$\ \ \setlength\topsep{0pt}\textbf{\foreignlanguage{arabic}{يِنْفَرَق}}\ {\color{gray}\texttt{/\sffamily {{\sffamily jinfara(q)}}/}\color{black}}\ [i.]\  \begin{flushright}\color{gray}\foreignlanguage{arabic}{\textbf{\underline{\foreignlanguage{arabic}{أمثلة}}}: ولك يا هبلة أحلى الشعر يِنْفَرَق}\end{flushright}\color{black}} \vspace{2mm}

{\setlength\topsep{0pt}\textbf{\foreignlanguage{arabic}{تَفْرِقَة}}\ {\color{gray}\texttt{/\sffamily {{\sffamily tafriqa}}/}\color{black}}\ \textsc{noun}\ [m.]\ \textbf{1.}~segregation  \textbf{2.}~discrimination  \textbf{3.}~separation\ } \vspace{2mm}

{\setlength\topsep{0pt}\textbf{\foreignlanguage{arabic}{تَفْرِيق}}\ {\color{gray}\texttt{/\sffamily {{\sffamily tafriː(q)}}/}\color{black}}\ \textsc{noun}\ [m.]\ \textbf{1.}~bias  \textbf{2.}~differentiation\ } \vspace{2mm}

{\setlength\topsep{0pt}\textbf{\foreignlanguage{arabic}{تْفَرَّق}}\ {\color{gray}\texttt{/\sffamily {{\sffamily tfarra(q)}}/}\color{black}}\ \textsc{verb}\ [p.]\ \textbf{1.}~be dispersed.  \textbf{2.}~be separated\ \ $\bullet$\ \ \setlength\topsep{0pt}\textbf{\foreignlanguage{arabic}{اِتْفَرَّق}}\ {\color{gray}\texttt{/\sffamily {{\sffamily ʔitfarra(q)}}/}\color{black}}\ [c.]\ \ $\bullet$\ \ \setlength\topsep{0pt}\textbf{\foreignlanguage{arabic}{يِتْفَرَّق}}\ {\color{gray}\texttt{/\sffamily {{\sffamily jitfarra(q)}}/}\color{black}}\ [i.]\  \begin{flushright}\color{gray}\foreignlanguage{arabic}{\textbf{\underline{\foreignlanguage{arabic}{أمثلة}}}: لما تْفَرَّق بيني وبينه بتنبسط هيك؟}\end{flushright}\color{black}} \vspace{2mm}

{\setlength\topsep{0pt}\textbf{\foreignlanguage{arabic}{فَارَق}}\ {\color{gray}\texttt{/\sffamily {{\sffamily faːra(q)}}/}\color{black}}\ \textsc{verb}\ [p.]\ \textbf{1.}~separate oneself from.  \textbf{2.}~depart from\ \ $\bullet$\ \ \setlength\topsep{0pt}\textbf{\foreignlanguage{arabic}{فَارِق}}\ {\color{gray}\texttt{/\sffamily {{\sffamily faːri(q)}}/}\color{black}}\ [c.]\ \ $\bullet$\ \ \setlength\topsep{0pt}\textbf{\foreignlanguage{arabic}{يفَارِق}}\ {\color{gray}\texttt{/\sffamily {{\sffamily jfaːri(q)}}/}\color{black}}\ [i.]\  \begin{flushright}\color{gray}\foreignlanguage{arabic}{\textbf{\underline{\foreignlanguage{arabic}{أمثلة}}}: حتى تريح حالك شوي لازم تدرك انه كل شي بهالكون يا هو بيفارِقنا أو احنا بنفارقه}\end{flushright}\color{black}} \vspace{2mm}

{\setlength\topsep{0pt}\textbf{\foreignlanguage{arabic}{فَرَق}}\ {\color{gray}\texttt{/\sffamily {{\sffamily fara(q)}}/}\color{black}}\ \textsc{verb}\ [p.]\ \textbf{1.}~split  \textbf{2.}~divide  \textbf{3.}~make a difference\ \ $\bullet$\ \ \setlength\topsep{0pt}\textbf{\foreignlanguage{arabic}{اِفْرُق}}\ {\color{gray}\texttt{/\sffamily {{\sffamily ʔifru(q)}}/}\color{black}}\ [c.]\ \ $\bullet$\ \ \setlength\topsep{0pt}\textbf{\foreignlanguage{arabic}{اُفْرُق}}\ {\color{gray}\texttt{/\sffamily {{\sffamily ʔufru(q)}}/}\color{black}}\ [c.]\ \ $\bullet$\ \ \setlength\topsep{0pt}\textbf{\foreignlanguage{arabic}{اِفْرِق}}\ {\color{gray}\texttt{/\sffamily {{\sffamily ʔifri(q)}}/}\color{black}}\ [c.]\ \textbf{1.}~recognize  \textbf{2.}~identify\ \ $\bullet$\ \ \setlength\topsep{0pt}\textbf{\foreignlanguage{arabic}{يِفْرُق}}\ {\color{gray}\texttt{/\sffamily {{\sffamily jifru(q)}}/}\color{black}}\ [i.]\ \color{gray}(msa. \foreignlanguage{arabic}{يَصْنَع فَرْق}~\foreignlanguage{arabic}{\textbf{٢.}}  \foreignlanguage{arabic}{يَفْرُق}~\foreignlanguage{arabic}{\textbf{١.}})\color{black}\ \ $\bullet$\ \ \setlength\topsep{0pt}\textbf{\foreignlanguage{arabic}{يُفْرُق}}\ {\color{gray}\texttt{/\sffamily {{\sffamily jufru(q)}}/}\color{black}}\ [i.]\ \color{gray}(msa. \foreignlanguage{arabic}{يَصْنَع فَرْق}~\foreignlanguage{arabic}{\textbf{٢.}}  \foreignlanguage{arabic}{يَفْرُق}~\foreignlanguage{arabic}{\textbf{١.}})\color{black}\ \ $\bullet$\ \ \setlength\topsep{0pt}\textbf{\foreignlanguage{arabic}{يِفْرِق}}\ {\color{gray}\texttt{/\sffamily {{\sffamily jifri(q)}}/}\color{black}}\ [i.]\ \color{gray}(msa. \foreignlanguage{arabic}{يَصْنَع فَرْق}~\foreignlanguage{arabic}{\textbf{٢.}}  \foreignlanguage{arabic}{يَفْرُق}~\foreignlanguage{arabic}{\textbf{١.}})\color{black}\ \textbf{1.}~recognize  \textbf{2.}~identify\  \begin{flushright}\color{gray}\foreignlanguage{arabic}{\textbf{\underline{\foreignlanguage{arabic}{أمثلة}}}: صلاح لسة صغير يما مابيِفْرِق بعده\ $\bullet$\ \  مش رح تِفْرِق إِذا إِجى هلا أو بعدين\ $\bullet$\ \  اُفْرُق شعرك بالنص مثل عبد الله ابن الجيران ههههه}\end{flushright}\color{black}} \vspace{2mm}

{\setlength\topsep{0pt}\textbf{\foreignlanguage{arabic}{فَرِق}}\ {\color{gray}\texttt{/\sffamily {{\sffamily fari(q)}}/}\color{black}}\ \textsc{noun}\ [m.]\ \color{gray}(msa. \foreignlanguage{arabic}{فَرْق}~\foreignlanguage{arabic}{\textbf{١.}})\color{black}\ \textbf{1.}~difference\ \ $\bullet$\ \ \setlength\topsep{0pt}\textbf{\foreignlanguage{arabic}{فُرُوق}}\ {\color{gray}\texttt{/\sffamily {{\sffamily furuː(q)}}/}\color{black}}\ [pl.]\  \begin{flushright}\color{gray}\foreignlanguage{arabic}{\textbf{\underline{\foreignlanguage{arabic}{أمثلة}}}: مافي أي فَرِق بيني وبينك}\end{flushright}\color{black}} \vspace{2mm}

{\setlength\topsep{0pt}\textbf{\foreignlanguage{arabic}{فَرِيق}}\ {\color{gray}\texttt{/\sffamily {{\sffamily fariːq}}/}\color{black}}\ \textsc{noun}\ [m.]\ \textbf{1.}~team  \textbf{2.}~group  \textbf{3.}~band\ \ $\bullet$\ \ \setlength\topsep{0pt}\textbf{\foreignlanguage{arabic}{فِرَق}}\ {\color{gray}\texttt{/\sffamily {{\sffamily firaq}}/}\color{black}}\ [pl.]\ \ $\bullet$\ \ \setlength\topsep{0pt}\textbf{\foreignlanguage{arabic}{أَفْرِقَة}}\ {\color{gray}\texttt{/\sffamily {{\sffamily ʔafriqa}}/}\color{black}}\ [pl.]\  \begin{flushright}\color{gray}\foreignlanguage{arabic}{\textbf{\underline{\foreignlanguage{arabic}{أمثلة}}}: انضميت لفريق كرة القدر بالمدرسة ويوم السبت رح نبلِّش تمرين السبت}\end{flushright}\color{black}} \vspace{2mm}

{\setlength\topsep{0pt}\textbf{\foreignlanguage{arabic}{فَرَّق}}\ {\color{gray}\texttt{/\sffamily {{\sffamily farra(q)}}/}\color{black}}\ \textsc{verb}\ [p.]\ \textbf{1.}~differentiate  \textbf{2.}~make a distinction.  \textbf{3.}~distribute sth to people.  \textbf{4.}~be biased\ \ $\bullet$\ \ \setlength\topsep{0pt}\textbf{\foreignlanguage{arabic}{فَرِّق}}\ {\color{gray}\texttt{/\sffamily {{\sffamily farri(q)}}/}\color{black}}\ [c.]\ \ $\bullet$\ \ \setlength\topsep{0pt}\textbf{\foreignlanguage{arabic}{يفَرِّق}}\ {\color{gray}\texttt{/\sffamily {{\sffamily jfarri(q)}}/}\color{black}}\ [i.]\  \begin{flushright}\color{gray}\foreignlanguage{arabic}{\textbf{\underline{\foreignlanguage{arabic}{أمثلة}}}: أحيانا بصير الأب يفَرِّق بين الأخوة بالتعامل وهذا بيعمل كثير مشاكل\ $\bullet$\ \  بيجيك سؤال بالامتحان عن فَرِّق بين همزة الوصل وهمزة القطع\ $\bullet$\ \  والله انه لما طلقني فَرَّقت وربات عالعمال اللي بالمعهد}\end{flushright}\color{black}} \vspace{2mm}

{\setlength\topsep{0pt}\textbf{\foreignlanguage{arabic}{فَرْق}}\ {\color{gray}\texttt{/\sffamily {{\sffamily far(q)}}/}\color{black}}\ \textsc{noun}\ [m.]\ \color{gray}(msa. \foreignlanguage{arabic}{فَرْق}~\foreignlanguage{arabic}{\textbf{١.}})\color{black}\ \textbf{1.}~difference\ \ $\bullet$\ \ \setlength\topsep{0pt}\textbf{\foreignlanguage{arabic}{فْرُوق}}\ {\color{gray}\texttt{/\sffamily {{\sffamily fruː(q)}}/}\color{black}}\ [pl.]\ \ $\bullet$\ \ \textsc{ph.} \color{gray} \foreignlanguage{arabic}{فَرْق السمَا عن الأرض}\color{black}\ {\color{gray}\texttt{/{\sffamily far(q) ʔissama ʕan ʔilʔar(dˤ)}/}\color{black}}\ \textbf{1.}~It is completely different\ } \vspace{2mm}

{\setlength\topsep{0pt}\textbf{\foreignlanguage{arabic}{فَرْقِيِّة}}\ {\color{gray}\texttt{/\sffamily {{\sffamily far(q)ijje}}/}\color{black}}\ \textsc{noun}\ [f.]\ \textbf{1.}~the extra amount of money when sb make a payment somewhere\ } \vspace{2mm}

{\setlength\topsep{0pt}\textbf{\foreignlanguage{arabic}{فِرْقَة}}\ {\color{gray}\texttt{/\sffamily {{\sffamily fir(q)a}}/}\color{black}}\ \textsc{noun}\ [f.]\ \textbf{1.}~band\ \ $\bullet$\ \ \setlength\topsep{0pt}\textbf{\foreignlanguage{arabic}{فِرَق}}\ {\color{gray}\texttt{/\sffamily {{\sffamily fira(q)}}/}\color{black}}\ [pl.]\ \color{gray}(msa. \foreignlanguage{arabic}{فِرْقَة}~\foreignlanguage{arabic}{\textbf{١.}})\color{black}\  \begin{flushright}\color{gray}\foreignlanguage{arabic}{\textbf{\underline{\foreignlanguage{arabic}{أمثلة}}}: وقت التعليلة جاب فِرْقَة دبكة}\end{flushright}\color{black}} \vspace{2mm}

{\setlength\topsep{0pt}\textbf{\foreignlanguage{arabic}{فْرَاق}}\ {\color{gray}\texttt{/\sffamily {{\sffamily fraː(q)}}/}\color{black}}\ \textsc{noun}\ [m.]\ \color{gray}(msa. \foreignlanguage{arabic}{إِنفِصال}~\foreignlanguage{arabic}{\textbf{١.}})\color{black}\ \textbf{1.}~separation\ \ $\bullet$\ \ \textsc{ph.} \color{gray} \foreignlanguage{arabic}{فْرَاقُه عيد}\color{black}\ {\color{gray}\texttt{/{\sffamily fraːqo ʕiːd}/}\color{black}}\ \textbf{1.}~It is an expression that means that sb is very happy that he no longer will see someone\  \begin{flushright}\color{gray}\foreignlanguage{arabic}{\textbf{\underline{\foreignlanguage{arabic}{أمثلة}}}: مش زعلانة عليه بالعكس فْراقُه عيد\ $\bullet$\ \  سمعت إِنه حاتِم تِعِب كثير بعد الفْراق}\end{flushright}\color{black}} \vspace{2mm}

\vspace{-3mm}
\markboth{\color{blue}\foreignlanguage{arabic}{ف.ر.ق.ع}\color{blue}{}}{\color{blue}\foreignlanguage{arabic}{ف.ر.ق.ع}\color{blue}{}}\subsection*{\color{blue}\foreignlanguage{arabic}{ف.ر.ق.ع}\color{blue}{}\index{\color{blue}\foreignlanguage{arabic}{ف.ر.ق.ع}\color{blue}{}}} 

{\setlength\topsep{0pt}\textbf{\foreignlanguage{arabic}{فَرْقَع}}\ {\color{gray}\texttt{/\sffamily {{\sffamily far(q)aʕ}}/}\color{black}}\ \textsc{verb}\ [p.]\ \textbf{1.}~make sth burst.  \textbf{2.}~burst  \textbf{3.}~yell at sb\ \ $\bullet$\ \ \setlength\topsep{0pt}\textbf{\foreignlanguage{arabic}{فَرْقِع}}\ {\color{gray}\texttt{/\sffamily {{\sffamily far(q)iʕ}}/}\color{black}}\ [c.]\ \ $\bullet$\ \ \setlength\topsep{0pt}\textbf{\foreignlanguage{arabic}{يفَرْقِع}}\ {\color{gray}\texttt{/\sffamily {{\sffamily jfar(q)iʕ}}/}\color{black}}\ [i.]\  \begin{flushright}\color{gray}\foreignlanguage{arabic}{\textbf{\underline{\foreignlanguage{arabic}{أمثلة}}}: فَرْقَعت البلونة بوجهي}\end{flushright}\color{black}} \vspace{2mm}

{\setlength\topsep{0pt}\textbf{\foreignlanguage{arabic}{فَرْقَعَة}}\ {\color{gray}\texttt{/\sffamily {{\sffamily far(q)aʕa}}/}\color{black}}\ \textsc{noun}\ [f.]\ \textbf{1.}~making sth burst.  \textbf{2.}~bursting\  \begin{flushright}\color{gray}\foreignlanguage{arabic}{\textbf{\underline{\foreignlanguage{arabic}{أمثلة}}}: صوت الفَرْقَعَة نطَّزني}\end{flushright}\color{black}} \vspace{2mm}

{\setlength\topsep{0pt}\textbf{\foreignlanguage{arabic}{فُرْقُع}}\ {\color{gray}\texttt{/\sffamily {{\sffamily fur(q)uʕ}}/}\color{black}}\ \textsc{noun}\ [m.]\ \textbf{1.}~bursting\ \ $\bullet$\ \ \textsc{ph.} \color{gray} \foreignlanguage{arabic}{فُرْقُع لوز}\color{black}\ {\color{gray}\texttt{/{\sffamily fur(q)uʕ loːz}/}\color{black}}\ \textbf{1.}~It is an idiomatic expression that means that sb is very hyperactive\  \begin{flushright}\color{gray}\foreignlanguage{arabic}{\textbf{\underline{\foreignlanguage{arabic}{أمثلة}}}: تضلكاش تحوص هيك زي فُرْقُع لوز}\end{flushright}\color{black}} \vspace{2mm}

\vspace{-3mm}
\markboth{\color{blue}\foreignlanguage{arabic}{ف.ر.ك}\color{blue}{}}{\color{blue}\foreignlanguage{arabic}{ف.ر.ك}\color{blue}{}}\subsection*{\color{blue}\foreignlanguage{arabic}{ف.ر.ك}\color{blue}{}\index{\color{blue}\foreignlanguage{arabic}{ف.ر.ك}\color{blue}{}}} 

{\setlength\topsep{0pt}\textbf{\foreignlanguage{arabic}{اِنْفَرَك}}\ {\color{gray}\texttt{/\sffamily {{\sffamily ʔinfara(k)}}/}\color{black}}\ \textsc{verb}\ [p.]\ \textbf{1.}~be rubbed\ \ $\bullet$\ \ \setlength\topsep{0pt}\textbf{\foreignlanguage{arabic}{اِنْفِرِك}}\ {\color{gray}\texttt{/\sffamily {{\sffamily ʔinfiri(k)}}/}\color{black}}\ [c.]\ \ $\bullet$\ \ \setlength\topsep{0pt}\textbf{\foreignlanguage{arabic}{يِنْفِرِك}}\ {\color{gray}\texttt{/\sffamily {{\sffamily jinfiri(k)}}/}\color{black}}\ [i.]\ } \vspace{2mm}

{\setlength\topsep{0pt}\textbf{\foreignlanguage{arabic}{تْفَرَّك}}\ {\color{gray}\texttt{/\sffamily {{\sffamily tfarra(k)}}/}\color{black}}\ \textsc{verb}\ [p.]\ \textbf{1.}~be rubbed repeatedly\ \ $\bullet$\ \ \setlength\topsep{0pt}\textbf{\foreignlanguage{arabic}{اِتْفَرَّك}}\ {\color{gray}\texttt{/\sffamily {{\sffamily ʔitfarra(k)}}/}\color{black}}\ [c.]\ \ $\bullet$\ \ \setlength\topsep{0pt}\textbf{\foreignlanguage{arabic}{يِتْفَرَّك}}\ {\color{gray}\texttt{/\sffamily {{\sffamily jitfarra(k)}}/}\color{black}}\ [i.]\  \begin{flushright}\color{gray}\foreignlanguage{arabic}{\textbf{\underline{\foreignlanguage{arabic}{أمثلة}}}: المجلى تْفَرَّك منيح يما وصار هلا بيوج وج}\end{flushright}\color{black}} \vspace{2mm}

{\setlength\topsep{0pt}\textbf{\foreignlanguage{arabic}{فَرَايِك}}\ {\color{gray}\texttt{/\sffamily {{\sffamily faraːji(k)}}/}\color{black}}\ \textsc{noun}\ [m.]\ \textbf{1.}~ring shaped pieces of bread or biscuits that are made from many ingredients. The flour, sugar, olive oil, (ghee optional), milk are mixed together. Anise, sesame and black cumin are then added to the mixture. The dough is left to rest for one hour. After that, it is made into dough balls that are placed and flattened into a large baking tray (with a lot of oil). It is saturated with oilve oil.\ } \vspace{2mm}

{\setlength\topsep{0pt}\textbf{\foreignlanguage{arabic}{فَرَك}}\ {\color{gray}\texttt{/\sffamily {{\sffamily fara(k)}}/}\color{black}}\ \textsc{verb}\ [p.]\ \textbf{1.}~rub  \textbf{2.}~run away\ \ $\bullet$\ \ \setlength\topsep{0pt}\textbf{\foreignlanguage{arabic}{اُفْرُك}}\ {\color{gray}\texttt{/\sffamily {{\sffamily ʔufru(k)}}/}\color{black}}\ [c.]\ \ $\bullet$\ \ \setlength\topsep{0pt}\textbf{\foreignlanguage{arabic}{اِفْرُك}}\ {\color{gray}\texttt{/\sffamily {{\sffamily ʔifru(k)}}/}\color{black}}\ [c.]\ \ $\bullet$\ \ \setlength\topsep{0pt}\textbf{\foreignlanguage{arabic}{يُفْرُك}}\ {\color{gray}\texttt{/\sffamily {{\sffamily jufru(k)}}/}\color{black}}\ [i.]\ \color{gray}(msa. \foreignlanguage{arabic}{يِهْرُب}~\foreignlanguage{arabic}{\textbf{٢.}}  \foreignlanguage{arabic}{يَدْعَك}~\foreignlanguage{arabic}{\textbf{١.}})\color{black}\ \ $\bullet$\ \ \setlength\topsep{0pt}\textbf{\foreignlanguage{arabic}{يِفْرُك}}\ {\color{gray}\texttt{/\sffamily {{\sffamily jifru(k)}}/}\color{black}}\ [i.]\ \color{gray}(msa. \foreignlanguage{arabic}{يِهْرُب}~\foreignlanguage{arabic}{\textbf{٢.}}  \foreignlanguage{arabic}{يَدْعَك}~\foreignlanguage{arabic}{\textbf{١.}})\color{black}\ \ $\bullet$\ \ \textsc{ph.} \color{gray} \foreignlanguage{arabic}{إِفركهَا}\color{black}\ {\color{gray}\texttt{/{\sffamily ʔifrukha}/}\color{black}}\ \color{gray}(src. \foreignlanguage{arabic}{الضفة الغربية})\color{black}\ \color{gray} (msa. \foreignlanguage{arabic}{إِذهب من هنا}~\foreignlanguage{arabic}{\textbf{١.}})\color{black}\ \textbf{1.}~get lost\  \begin{flushright}\color{gray}\foreignlanguage{arabic}{\textbf{\underline{\foreignlanguage{arabic}{أمثلة}}}: \ $\bullet$\ \  \ $\bullet$\ \  خلِّيه يِفْرُكلي اياها مليح بالخريص\ $\bullet$\ \  وينه؟ والله فَرَك زمان\ $\bullet$\ \  فَرَكْتُه بايدي كثير منيح تراح}\end{flushright}\color{black}} \vspace{2mm}

{\setlength\topsep{0pt}\textbf{\foreignlanguage{arabic}{فَرِك}}\ {\color{gray}\texttt{/\sffamily {{\sffamily farik}}/}\color{black}}\ \textsc{noun}\ [m.]\ \textbf{1.}~rubbing\  \begin{flushright}\color{gray}\foreignlanguage{arabic}{\textbf{\underline{\foreignlanguage{arabic}{أمثلة}}}: بضبطش الها الفَرِك عشانها هشة ويمكن تنكسر بسهولة معك أو يكحت لونها لاسمح الله}\end{flushright}\color{black}} \vspace{2mm}

{\setlength\topsep{0pt}\textbf{\foreignlanguage{arabic}{فَرَّاكِة}}\ {\color{gray}\texttt{/\sffamily {{\sffamily farraːke}}/}\color{black}}\ \textsc{noun}\ [f.]\ \textbf{1.}~sth that rubs sth repeatedly\ \ $\bullet$\ \ \textsc{ph.} \color{gray} \foreignlanguage{arabic}{فَرَّاكِة سِجَّاد}\color{black}\ {\color{gray}\texttt{/{\sffamily farraːkit si(dʒ)(dʒ)aːd}/}\color{black}}\ \textbf{1.}~hand-held carpet brush sweeper\ } \vspace{2mm}

{\setlength\topsep{0pt}\textbf{\foreignlanguage{arabic}{فَرَّك}}\ {\color{gray}\texttt{/\sffamily {{\sffamily farra(k)}}/}\color{black}}\ \textsc{verb}\ [p.]\ \textbf{1.}~rub sth repeatedly\ \ $\bullet$\ \ \setlength\topsep{0pt}\textbf{\foreignlanguage{arabic}{فَرِّك}}\ {\color{gray}\texttt{/\sffamily {{\sffamily farri(k)}}/}\color{black}}\ [c.]\ \ $\bullet$\ \ \setlength\topsep{0pt}\textbf{\foreignlanguage{arabic}{يفَرِّك}}\ {\color{gray}\texttt{/\sffamily {{\sffamily jfarri(k)}}/}\color{black}}\ [i.]\ \color{gray}(msa. \foreignlanguage{arabic}{يَدْعَك بشكل متكرر}~\foreignlanguage{arabic}{\textbf{١.}})\color{black}\  \begin{flushright}\color{gray}\foreignlanguage{arabic}{\textbf{\underline{\foreignlanguage{arabic}{أمثلة}}}: ضلك فَرِّك فيها لحديت ما تنظف}\end{flushright}\color{black}} \vspace{2mm}

{\setlength\topsep{0pt}\textbf{\foreignlanguage{arabic}{فَرْكِة}}\ {\color{gray}\texttt{/\sffamily {{\sffamily farke}}/}\color{black}}\ \textsc{noun}\ [f.]\ \textbf{1.}~rubbing (one time)\ \ $\bullet$\ \ \textsc{ph.} \color{gray} \foreignlanguage{arabic}{فركة ذَان}\color{black}\ {\color{gray}\texttt{/{\sffamily farkit (d)aːn}/}\color{black}}\ \textbf{1.}~to warn sb off (sometimes in a severe way)\  \begin{flushright}\color{gray}\foreignlanguage{arabic}{\textbf{\underline{\foreignlanguage{arabic}{أمثلة}}}: المرَّة هاي عملنالك فَرْكِة ذان ان شاء الله المرَّة الجاية بنسخطك وبنسخط اللي جابك}\end{flushright}\color{black}} \vspace{2mm}

{\setlength\topsep{0pt}\textbf{\foreignlanguage{arabic}{فْرِيكِة}}\ {\color{gray}\texttt{/\sffamily {{\sffamily friː(k)e}}/}\color{black}}\ \textsc{noun}\ [f.]\ \color{gray}(msa. \foreignlanguage{arabic}{طبق طعام يتكون من حبات قمح ولحم دسم وممكن أن يقدم على شكل شوربة.}~\foreignlanguage{arabic}{\textbf{١.}})\color{black}\ \textbf{1.}~A dish consisting of wheat grains and fatty meat and could be served as a soup.\  \begin{flushright}\color{gray}\foreignlanguage{arabic}{\textbf{\underline{\foreignlanguage{arabic}{أمثلة}}}: عملنا جاج وفريكة جنبه}\end{flushright}\color{black}} \vspace{2mm}

{\setlength\topsep{0pt}\textbf{\foreignlanguage{arabic}{مُفْرَاك}}\ {\color{gray}\texttt{/\sffamily {{\sffamily mufraːk}}/}\color{black}}\ \textsc{noun}\ [m.]\ \textbf{1.}~a tool used for rubbing sth\ \ $\bullet$\ \ \textsc{ph.} \color{gray} \foreignlanguage{arabic}{مفرَاك خشب}\color{black}\ {\color{gray}\texttt{/{\sffamily mufraː(k) xaʃab}/}\color{black}}\ \color{gray} (msa. \foreignlanguage{arabic}{لتنعيم الخبيزة}~\foreignlanguage{arabic}{\textbf{١.}})\color{black}\ \textbf{1.}~wooden masher (used with Cheeseweed)\ } \vspace{2mm}

{\setlength\topsep{0pt}\textbf{\foreignlanguage{arabic}{مْفَرَّكِة}}\ {\color{gray}\texttt{/\sffamily {{\sffamily mfarrake}}/}\color{black}}\ \textsc{noun}\ [f.]\ \textbf{1.}~It is a dish that is made of fried potatoes with eggs, salt and some black pepper.\ } \vspace{2mm}

\vspace{-3mm}
\markboth{\color{blue}\foreignlanguage{arabic}{ف.ر.ك.ح}\color{blue}{}}{\color{blue}\foreignlanguage{arabic}{ف.ر.ك.ح}\color{blue}{}}\subsection*{\color{blue}\foreignlanguage{arabic}{ف.ر.ك.ح}\color{blue}{}\index{\color{blue}\foreignlanguage{arabic}{ف.ر.ك.ح}\color{blue}{}}} 

{\setlength\topsep{0pt}\textbf{\foreignlanguage{arabic}{تْفَرْكَح}}\ {\color{gray}\texttt{/\sffamily {{\sffamily tfarkaħ}}/}\color{black}}\ \textsc{verb}\ [p.]\ \textbf{1.}~limp  \textbf{2.}~tumble  \textbf{3.}~trip over\ \ $\bullet$\ \ \setlength\topsep{0pt}\textbf{\foreignlanguage{arabic}{اِتْفَرْكَح}}\ {\color{gray}\texttt{/\sffamily {{\sffamily ʔitfarkaħ}}/}\color{black}}\ [c.]\ \ $\bullet$\ \ \setlength\topsep{0pt}\textbf{\foreignlanguage{arabic}{يِتْفَرْكَح}}\ {\color{gray}\texttt{/\sffamily {{\sffamily jitfarkaħ}}/}\color{black}}\ [i.]\ \color{gray}(msa. \foreignlanguage{arabic}{يَتَعَثَّر}~\foreignlanguage{arabic}{\textbf{٢.}}  \foreignlanguage{arabic}{يَعْرُج}~\foreignlanguage{arabic}{\textbf{١.}})\color{black}\  \begin{flushright}\color{gray}\foreignlanguage{arabic}{\textbf{\underline{\foreignlanguage{arabic}{أمثلة}}}: يا حرام خبطتها سيارة وصارت تِتفَرْكَح\ $\bullet$\ \  الله يخزيها كانت بتطقطق بالكعب وهي ماشية فتْفَرْكَحت والناس صارت تتطلع عليها}\end{flushright}\color{black}} \vspace{2mm}

{\setlength\topsep{0pt}\textbf{\foreignlanguage{arabic}{فَرْكَحَة}}\ {\color{gray}\texttt{/\sffamily {{\sffamily farkaħa}}/}\color{black}}\ \textsc{noun}\ [f.]\ \color{gray}(msa. \foreignlanguage{arabic}{عَرْجَة بالمشي}~\foreignlanguage{arabic}{\textbf{١.}})\color{black}\ \textbf{1.}~limp\ } \vspace{2mm}

{\setlength\topsep{0pt}\textbf{\foreignlanguage{arabic}{مْفَرْكَح}}\ {\color{gray}\texttt{/\sffamily {{\sffamily mfarkaħ}}/}\color{black}}\ \textsc{adj}\ [m.]\ \textbf{1.}~sb who limps\  \begin{flushright}\color{gray}\foreignlanguage{arabic}{\textbf{\underline{\foreignlanguage{arabic}{أمثلة}}}: إِياد لما إِجى يخطب مالقى يُخْطُب غير وحدة مْفَرْكَحَة!}\end{flushright}\color{black}} \vspace{2mm}

\vspace{-3mm}
\markboth{\color{blue}\foreignlanguage{arabic}{ف.ر.ك.ش}\color{blue}{}}{\color{blue}\foreignlanguage{arabic}{ف.ر.ك.ش}\color{blue}{}}\subsection*{\color{blue}\foreignlanguage{arabic}{ف.ر.ك.ش}\color{blue}{}\index{\color{blue}\foreignlanguage{arabic}{ف.ر.ك.ش}\color{blue}{}}} 

{\setlength\topsep{0pt}\textbf{\foreignlanguage{arabic}{تْفَرْكَش}}\ {\color{gray}\texttt{/\sffamily {{\sffamily tfarkaʃ}}/}\color{black}}\ \textsc{verb}\ [p.]\ \textbf{1.}~did not work.  \textbf{2.}~fail\ \ $\bullet$\ \ \setlength\topsep{0pt}\textbf{\foreignlanguage{arabic}{اِتْفَرْكَش}}\ {\color{gray}\texttt{/\sffamily {{\sffamily ʔitfarkaʃ}}/}\color{black}}\ [c.]\ \ $\bullet$\ \ \setlength\topsep{0pt}\textbf{\foreignlanguage{arabic}{يِتْفَرْكَش}}\ {\color{gray}\texttt{/\sffamily {{\sffamily jitfarkaʃ}}/}\color{black}}\ [i.]\  \begin{flushright}\color{gray}\foreignlanguage{arabic}{\textbf{\underline{\foreignlanguage{arabic}{أمثلة}}}: تْفَرْكَش العرس قبل الموعد بيوم}\end{flushright}\color{black}} \vspace{2mm}

{\setlength\topsep{0pt}\textbf{\foreignlanguage{arabic}{فَرْكَش}}\ {\color{gray}\texttt{/\sffamily {{\sffamily farkaʃ}}/}\color{black}}\ \textsc{verb}\ [p.]\ \textbf{1.}~did not work.  \textbf{2.}~fail  \textbf{3.}~make sth unsuccessful\ \ $\bullet$\ \ \setlength\topsep{0pt}\textbf{\foreignlanguage{arabic}{فَرْكِش}}\ {\color{gray}\texttt{/\sffamily {{\sffamily farkiʃ}}/}\color{black}}\ [c.]\ \ $\bullet$\ \ \setlength\topsep{0pt}\textbf{\foreignlanguage{arabic}{يفَرْكِش}}\ {\color{gray}\texttt{/\sffamily {{\sffamily jfarkiʃ}}/}\color{black}}\ [i.]\  \begin{flushright}\color{gray}\foreignlanguage{arabic}{\textbf{\underline{\foreignlanguage{arabic}{أمثلة}}}: والله حاول يفَرْكِش الموضوع بس ماطلع بايده\ $\bullet$\ \  فَرْكَشت الخطبة بسبب شغل أبوها}\end{flushright}\color{black}} \vspace{2mm}

{\setlength\topsep{0pt}\textbf{\foreignlanguage{arabic}{فَرْكَشِة}}\ {\color{gray}\texttt{/\sffamily {{\sffamily farkaʃe}}/}\color{black}}\ \textsc{noun}\ [f.]\ \textbf{1.}~failure\  \begin{flushright}\color{gray}\foreignlanguage{arabic}{\textbf{\underline{\foreignlanguage{arabic}{أمثلة}}}: فَرْكَشِة الجيزة قبل العرس واردة جداً}\end{flushright}\color{black}} \vspace{2mm}

\vspace{-3mm}
\markboth{\color{blue}\foreignlanguage{arabic}{ف.ر.م}\color{blue}{}}{\color{blue}\foreignlanguage{arabic}{ف.ر.م}\color{blue}{}}\subsection*{\color{blue}\foreignlanguage{arabic}{ف.ر.م}\color{blue}{}\index{\color{blue}\foreignlanguage{arabic}{ف.ر.م}\color{blue}{}}} 

{\setlength\topsep{0pt}\textbf{\foreignlanguage{arabic}{اِنْفَرَم}}\ {\color{gray}\texttt{/\sffamily {{\sffamily ʔinfaram}}/}\color{black}}\ \textsc{verb}\ [p.]\ \textbf{1.}~be ground (meat, chicken, etc)\ \ $\bullet$\ \ \setlength\topsep{0pt}\textbf{\foreignlanguage{arabic}{اِنْفِرِم}}\ {\color{gray}\texttt{/\sffamily {{\sffamily ʔinfirim}}/}\color{black}}\ [c.]\ \ $\bullet$\ \ \setlength\topsep{0pt}\textbf{\foreignlanguage{arabic}{اِنِفْرِم}}\ {\color{gray}\texttt{/\sffamily {{\sffamily ʔinifrim}}/}\color{black}}\ [c.]\ \ $\bullet$\ \ \setlength\topsep{0pt}\textbf{\foreignlanguage{arabic}{يِنْفِرِم}}\ {\color{gray}\texttt{/\sffamily {{\sffamily jinfirim}}/}\color{black}}\ [i.]\ \color{gray}(msa. \foreignlanguage{arabic}{يُفْرَم}~\foreignlanguage{arabic}{\textbf{١.}})\color{black}\ \ $\bullet$\ \ \setlength\topsep{0pt}\textbf{\foreignlanguage{arabic}{يِنِفْرِم}}\ {\color{gray}\texttt{/\sffamily {{\sffamily jinifrim}}/}\color{black}}\ [i.]\  \begin{flushright}\color{gray}\foreignlanguage{arabic}{\textbf{\underline{\foreignlanguage{arabic}{أمثلة}}}: بحبس البقدونس يِنِفْرِم هالقد للتبولة}\end{flushright}\color{black}} \vspace{2mm}

{\setlength\topsep{0pt}\textbf{\foreignlanguage{arabic}{فَرَم}}\ {\color{gray}\texttt{/\sffamily {{\sffamily faram}}/}\color{black}}\ \textsc{verb}\ [p.]\ \textbf{1.}~grind (meat, chicken, etc)\ \ $\bullet$\ \ \setlength\topsep{0pt}\textbf{\foreignlanguage{arabic}{اُفْرُم}}\ {\color{gray}\texttt{/\sffamily {{\sffamily ʔufrum}}/}\color{black}}\ [c.]\ \ $\bullet$\ \ \setlength\topsep{0pt}\textbf{\foreignlanguage{arabic}{يُفْرُم}}\ {\color{gray}\texttt{/\sffamily {{\sffamily jufrum}}/}\color{black}}\ [i.]\ \color{gray}(msa. \foreignlanguage{arabic}{يَفْرُم}~\foreignlanguage{arabic}{\textbf{١.}})\color{black}\  \begin{flushright}\color{gray}\foreignlanguage{arabic}{\textbf{\underline{\foreignlanguage{arabic}{أمثلة}}}: اُفْرُم البصل فَرِم ناعم}\end{flushright}\color{black}} \vspace{2mm}

{\setlength\topsep{0pt}\textbf{\foreignlanguage{arabic}{فَرِم}}\ {\color{gray}\texttt{/\sffamily {{\sffamily farim}}/}\color{black}}\ \textsc{noun}\ [m.]\ \textbf{1.}~grinding sth (meat, chicken, etc)\  \begin{flushright}\color{gray}\foreignlanguage{arabic}{\textbf{\underline{\foreignlanguage{arabic}{أمثلة}}}: بتقدر تفرملي السلطة فَرِم ناعم؟}\end{flushright}\color{black}} \vspace{2mm}

{\setlength\topsep{0pt}\textbf{\foreignlanguage{arabic}{فَرَّامِة}}\ {\color{gray}\texttt{/\sffamily {{\sffamily farraːme}}/}\color{black}}\ \textsc{noun}\ [f.]\ \textbf{1.}~meat grinder.  \textbf{2.}~grinder  \textbf{3.}~cutting board\ \ $\bullet$\ \ \textsc{ph.} \color{gray} \foreignlanguage{arabic}{فَرَّامِة ملوخِيِّة}\color{black}\ {\color{gray}\texttt{/{\sffamily farraːmit mluːxijje}/}\color{black}}\ \textbf{1.}~a croissant-shaped knife that is used to grind Mulukhiyah\  \begin{flushright}\color{gray}\foreignlanguage{arabic}{\textbf{\underline{\foreignlanguage{arabic}{أمثلة}}}: جبت فَرّامِة لحمة أحسن وأرخص من اللي عندك}\end{flushright}\color{black}} \vspace{2mm}

{\setlength\topsep{0pt}\textbf{\foreignlanguage{arabic}{فَرَّم}}\ {\color{gray}\texttt{/\sffamily {{\sffamily farram}}/}\color{black}}\ \textsc{verb}\ [p.]\ \textbf{1.}~grind (meat, chicken, etc)\ \ $\bullet$\ \ \setlength\topsep{0pt}\textbf{\foreignlanguage{arabic}{فَرِّم}}\ {\color{gray}\texttt{/\sffamily {{\sffamily farrim}}/}\color{black}}\ [c.]\ \ $\bullet$\ \ \setlength\topsep{0pt}\textbf{\foreignlanguage{arabic}{يفَرِّم}}\ {\color{gray}\texttt{/\sffamily {{\sffamily jfarrim}}/}\color{black}}\ [i.]\  \begin{flushright}\color{gray}\foreignlanguage{arabic}{\textbf{\underline{\foreignlanguage{arabic}{أمثلة}}}: هو فَرَّم اللحمة بطريقة غير المتعودين عليها بس مش مشكلة}\end{flushright}\color{black}} \vspace{2mm}

{\setlength\topsep{0pt}\textbf{\foreignlanguage{arabic}{مَفْرَمِة}}\ {\color{gray}\texttt{/\sffamily {{\sffamily maframe}}/}\color{black}}\ \textsc{noun}\ [f.]\ \textbf{1.}~meat grinder.  \textbf{2.}~grinder  \textbf{3.}~cutting board\ \ $\bullet$\ \ \setlength\topsep{0pt}\textbf{\foreignlanguage{arabic}{مَفَارِم}}\ {\color{gray}\texttt{/\sffamily {{\sffamily mafaːrim}}/}\color{black}}\ [pl.]\ } \vspace{2mm}

{\setlength\topsep{0pt}\textbf{\foreignlanguage{arabic}{مَفْرُوم}}\ {\color{gray}\texttt{/\sffamily {{\sffamily mafruːm}}/}\color{black}}\ \textsc{noun\textunderscore pass}\ \textbf{1.}~ground\  \begin{flushright}\color{gray}\foreignlanguage{arabic}{\textbf{\underline{\foreignlanguage{arabic}{أمثلة}}}: جيبلي كيلو لحمة مفْرومة عشان أعمل جنب الملوخية صينية كفتة}\end{flushright}\color{black}} \vspace{2mm}

\vspace{-3mm}
\markboth{\color{blue}\foreignlanguage{arabic}{ف.ر.م.ش}\color{blue}{}}{\color{blue}\foreignlanguage{arabic}{ف.ر.م.ش}\color{blue}{}}\subsection*{\color{blue}\foreignlanguage{arabic}{ف.ر.م.ش}\color{blue}{}\index{\color{blue}\foreignlanguage{arabic}{ف.ر.م.ش}\color{blue}{}}} 

{\setlength\topsep{0pt}\textbf{\foreignlanguage{arabic}{فَرْمَشَاني}}\ {\color{gray}\texttt{/\sffamily {{\sffamily farmaʃaːni}}/}\color{black}}\ \textsc{noun}\ [m.]\ \textbf{1.}~pharmacist\ } \vspace{2mm}

{\setlength\topsep{0pt}\textbf{\foreignlanguage{arabic}{فَرْمَشِيِّة}}\ {\color{gray}\texttt{/\sffamily {{\sffamily farmaʃijje}}/}\color{black}}\ \textsc{noun}\ [f.]\ \color{gray}(msa. \foreignlanguage{arabic}{صيدليِّة}~\foreignlanguage{arabic}{\textbf{١.}})\color{black}\ \textbf{1.}~pharmacy\  \begin{flushright}\color{gray}\foreignlanguage{arabic}{\textbf{\underline{\foreignlanguage{arabic}{أمثلة}}}: عندي مشوار مهم. بدي أصَل الفَرْمَشِيِّة دقيقتين مش مطولة}\end{flushright}\color{black}} \vspace{2mm}

\vspace{-3mm}
\markboth{\color{blue}\foreignlanguage{arabic}{ف.ر.م.ش}\color{blue}{ (ntws)}}{\color{blue}\foreignlanguage{arabic}{ف.ر.م.ش}\color{blue}{ (ntws)}}\subsection*{\color{blue}\foreignlanguage{arabic}{ف.ر.م.ش}\color{blue}{ (ntws)}\index{\color{blue}\foreignlanguage{arabic}{ف.ر.م.ش}\color{blue}{ (ntws)}}} 

{\setlength\topsep{0pt}\textbf{\foreignlanguage{arabic}{فَرْمَشَونِة}}\ {\color{gray}\texttt{/\sffamily {{\sffamily farʃamoːne}}/}\color{black}}\ \textsc{noun}\ [f.]\ (src. \color{gray}\foreignlanguage{arabic}{رامين}\color{black})\ \color{gray}(msa. \foreignlanguage{arabic}{صيدليِّة}~\foreignlanguage{arabic}{\textbf{١.}})\color{black}\ \textbf{1.}~pharmacy\  \begin{flushright}\color{gray}\foreignlanguage{arabic}{\textbf{\underline{\foreignlanguage{arabic}{أمثلة}}}: بطني بوجعني روح جبلي دوا من الفَرْْشَمونِة}\end{flushright}\color{black}} \vspace{2mm}

\vspace{-3mm}
\markboth{\color{blue}\foreignlanguage{arabic}{ف.ر.ن}\color{blue}{}}{\color{blue}\foreignlanguage{arabic}{ف.ر.ن}\color{blue}{}}\subsection*{\color{blue}\foreignlanguage{arabic}{ف.ر.ن}\color{blue}{}\index{\color{blue}\foreignlanguage{arabic}{ف.ر.ن}\color{blue}{}}} 

{\setlength\topsep{0pt}\textbf{\foreignlanguage{arabic}{فَرَّان}}\ {\color{gray}\texttt{/\sffamily {{\sffamily farraːn}}/}\color{black}}\ \textsc{noun}\ [m.]\ \textbf{1.}~the oven man.  \textbf{2.}~a person who works in a place like a bakery in which there is a large oven where all the people living in the neighbourhood to bake their food in it.\  \begin{flushright}\color{gray}\foreignlanguage{arabic}{\textbf{\underline{\foreignlanguage{arabic}{أمثلة}}}: بقن نسوان الحارة كلهن يطقن حنك مع الفَرّان اللي بالحارة}\end{flushright}\color{black}} \vspace{2mm}

{\setlength\topsep{0pt}\textbf{\foreignlanguage{arabic}{فُرُن}}\ {\color{gray}\texttt{/\sffamily {{\sffamily furun}}/}\color{black}}\ \textsc{adj}\ [m.]\ \color{gray}(msa. \foreignlanguage{arabic}{حر خانق}~\foreignlanguage{arabic}{\textbf{١.}})\color{black}\ \textbf{1.}~suffocatingly hot\  \begin{flushright}\color{gray}\foreignlanguage{arabic}{\textbf{\underline{\foreignlanguage{arabic}{أمثلة}}}: يا الله الدنيا عنا فُرُن}\end{flushright}\color{black}} \vspace{2mm}

{\setlength\topsep{0pt}\textbf{\foreignlanguage{arabic}{فُرُن}}\ {\color{gray}\texttt{/\sffamily {{\sffamily furun}}/}\color{black}}\ \textsc{noun}\ [m.]\ \color{gray}(msa. \foreignlanguage{arabic}{فُرْن}~\foreignlanguage{arabic}{\textbf{١.}})\color{black}\ \textbf{1.}~oven\ \ $\bullet$\ \ \setlength\topsep{0pt}\textbf{\foreignlanguage{arabic}{فْرُونِة}}\ {\color{gray}\texttt{/\sffamily {{\sffamily fruːne}}/}\color{black}}\ [pl.]\  \begin{flushright}\color{gray}\foreignlanguage{arabic}{\textbf{\underline{\foreignlanguage{arabic}{أمثلة}}}: حمريه بالفُرُن شوي ماتطوليش عليه}\end{flushright}\color{black}} \vspace{2mm}

{\setlength\topsep{0pt}\textbf{\foreignlanguage{arabic}{فُرْنِينِة}}\ {\color{gray}\texttt{/\sffamily {{\sffamily furniːne}}/}\color{black}}\ \textsc{noun}\ [f.]\ \color{gray}(msa. \foreignlanguage{arabic}{لعبة دولاب الهواء للصغار}~\foreignlanguage{arabic}{\textbf{١.}})\color{black}\ \textbf{1.}~pinwheel / windmill\  \begin{flushright}\color{gray}\foreignlanguage{arabic}{\textbf{\underline{\foreignlanguage{arabic}{أمثلة}}}: أبوي جابلي فُرْنِينِة أنت أبوك ما جبلك ننننا}\end{flushright}\color{black}} \vspace{2mm}

{\setlength\topsep{0pt}\textbf{\foreignlanguage{arabic}{فْرَين}}\ {\color{gray}\texttt{/\sffamily {{\sffamily freːn}}/}\color{black}}\ \textsc{noun}\ [m.]\ \color{gray}(msa. \foreignlanguage{arabic}{مكبح السيارة}~\foreignlanguage{arabic}{\textbf{١.}})\color{black}\ \textbf{1.}~brake\  \begin{flushright}\color{gray}\foreignlanguage{arabic}{\textbf{\underline{\foreignlanguage{arabic}{أمثلة}}}: ادعس عالفْرَين زي الناس}\end{flushright}\color{black}} \vspace{2mm}

\vspace{-3mm}
\markboth{\color{blue}\foreignlanguage{arabic}{ف.ر.ن.ج.ي}\color{blue}{ (ntws)}}{\color{blue}\foreignlanguage{arabic}{ف.ر.ن.ج.ي}\color{blue}{ (ntws)}}\subsection*{\color{blue}\foreignlanguage{arabic}{ف.ر.ن.ج.ي}\color{blue}{ (ntws)}\index{\color{blue}\foreignlanguage{arabic}{ف.ر.ن.ج.ي}\color{blue}{ (ntws)}}} 

{\setlength\topsep{0pt}\textbf{\foreignlanguage{arabic}{إِفْرَنْجِي}}\ {\color{gray}\texttt{/\sffamily {{\sffamily ʔifran(dʒ)i}}/}\color{black}}\ \textsc{adj}\ [m.]\ \color{gray}(msa. \foreignlanguage{arabic}{أوروبي}~\foreignlanguage{arabic}{\textbf{٢.}}  \foreignlanguage{arabic}{إِفْرَنْجِي}~\foreignlanguage{arabic}{\textbf{١.}})\color{black}\ \textbf{1.}~European  \textbf{2.}~non-Arab\  \begin{flushright}\color{gray}\foreignlanguage{arabic}{\textbf{\underline{\foreignlanguage{arabic}{أمثلة}}}: الحمام اللي عندكم عربي ولا إِفْرَنْجِي؟}\end{flushright}\color{black}} \vspace{2mm}

\vspace{-3mm}
\markboth{\color{blue}\foreignlanguage{arabic}{ف.ر.و}\color{blue}{}}{\color{blue}\foreignlanguage{arabic}{ف.ر.و}\color{blue}{}}\subsection*{\color{blue}\foreignlanguage{arabic}{ف.ر.و}\color{blue}{}\index{\color{blue}\foreignlanguage{arabic}{ف.ر.و}\color{blue}{}}} 

{\setlength\topsep{0pt}\textbf{\foreignlanguage{arabic}{فَرُو}}\ {\color{gray}\texttt{/\sffamily {{\sffamily faru}}/}\color{black}}\ \textsc{noun}\ [m.]\ \color{gray}(msa. \foreignlanguage{arabic}{فِراء}~\foreignlanguage{arabic}{\textbf{١.}})\color{black}\ \textbf{1.}~furr\  \begin{flushright}\color{gray}\foreignlanguage{arabic}{\textbf{\underline{\foreignlanguage{arabic}{أمثلة}}}: شفتها لابسة فَرُو  مثل الخواجات}\end{flushright}\color{black}} \vspace{2mm}

{\setlength\topsep{0pt}\textbf{\foreignlanguage{arabic}{فَرْوِة}}\ {\color{gray}\texttt{/\sffamily {{\sffamily farwe}}/}\color{black}}\ \textsc{noun}\ [f.]\ \color{gray}(msa. \foreignlanguage{arabic}{فأس صغير لتقطيع الخشب}~\foreignlanguage{arabic}{\textbf{١.}})\color{black}\ \textbf{1.}~a small axe to chop wood\ \ $\bullet$\ \ \setlength\topsep{0pt}\textbf{\foreignlanguage{arabic}{فَرَاوي}}\ {\color{gray}\texttt{/\sffamily {{\sffamily faraːwi}}/}\color{black}}\ [pl.]\ \ $\bullet$\ \ \setlength\topsep{0pt}\textbf{\foreignlanguage{arabic}{فوَاري}}\ {\color{gray}\texttt{/\sffamily {{\sffamily fawaːri}}/}\color{black}}\ [pl.]\  \begin{flushright}\color{gray}\foreignlanguage{arabic}{\textbf{\underline{\foreignlanguage{arabic}{أمثلة}}}: إِجوا زلام كثير لابسين فواري عشان الدنيا بقت ثَلِج}\end{flushright}\color{black}} \vspace{2mm}

\vspace{-3mm}
\markboth{\color{blue}\foreignlanguage{arabic}{ف.ر.و.ل}\color{blue}{}}{\color{blue}\foreignlanguage{arabic}{ف.ر.و.ل}\color{blue}{}}\subsection*{\color{blue}\foreignlanguage{arabic}{ف.ر.و.ل}\color{blue}{}\index{\color{blue}\foreignlanguage{arabic}{ف.ر.و.ل}\color{blue}{}}} 

{\setlength\topsep{0pt}\textbf{\foreignlanguage{arabic}{فَرَاوْلَايِة}}\footnote{Unit noun}\ \ {\color{gray}\texttt{/\sffamily {{\sffamily faraːwlaːje}}/}\color{black}}\ \textsc{noun}\ [f.]\ \textbf{1.}~one strawberry\  \begin{flushright}\color{gray}\foreignlanguage{arabic}{\textbf{\underline{\foreignlanguage{arabic}{أمثلة}}}: اشتهيتلك هالفَراولِايِة خذي}\end{flushright}\color{black}} \vspace{2mm}

{\setlength\topsep{0pt}\textbf{\foreignlanguage{arabic}{فَرَاوْلِة}}\footnote{Collective noun}\ \ {\color{gray}\texttt{/\sffamily {{\sffamily faraːwle}}/}\color{black}}\ \textsc{noun}\ [f.]\ \textbf{1.}~strawberry\ } \vspace{2mm}

{\setlength\topsep{0pt}\textbf{\foreignlanguage{arabic}{فَرْوَل}}\ {\color{gray}\texttt{/\sffamily {{\sffamily farwal}}/}\color{black}}\ \textsc{verb}\ [p.]\ \textbf{1.}~be crumbly\ \ $\bullet$\ \ \setlength\topsep{0pt}\textbf{\foreignlanguage{arabic}{فَرْوِل}}\ {\color{gray}\texttt{/\sffamily {{\sffamily farwil}}/}\color{black}}\ [c.]\ \ $\bullet$\ \ \setlength\topsep{0pt}\textbf{\foreignlanguage{arabic}{يفَرْوِل}}\ {\color{gray}\texttt{/\sffamily {{\sffamily jfarwil}}/}\color{black}}\ [i.]\  \begin{flushright}\color{gray}\foreignlanguage{arabic}{\textbf{\underline{\foreignlanguage{arabic}{أمثلة}}}: تكثريش زيت عليهم ولا بيفَرْوِلِن}\end{flushright}\color{black}} \vspace{2mm}

{\setlength\topsep{0pt}\textbf{\foreignlanguage{arabic}{مْفَرْوِل}}\ {\color{gray}\texttt{/\sffamily {{\sffamily mfarwil}}/}\color{black}}\ \textsc{adj}\ [m.]\ \textbf{1.}~be crumbly\  \begin{flushright}\color{gray}\foreignlanguage{arabic}{\textbf{\underline{\foreignlanguage{arabic}{أمثلة}}}: القراص مْفَرْوِلات بضبطش أفرِّزهن}\end{flushright}\color{black}} \vspace{2mm}

\vspace{-3mm}
\markboth{\color{blue}\foreignlanguage{arabic}{ف.ر.ي}\color{blue}{}}{\color{blue}\foreignlanguage{arabic}{ف.ر.ي}\color{blue}{}}\subsection*{\color{blue}\foreignlanguage{arabic}{ف.ر.ي}\color{blue}{}\index{\color{blue}\foreignlanguage{arabic}{ف.ر.ي}\color{blue}{}}} 

{\setlength\topsep{0pt}\textbf{\foreignlanguage{arabic}{إِفْتِرَاء}}\ {\color{gray}\texttt{/\sffamily {{\sffamily ʔiftiraː}}/}\color{black}}\ \textsc{noun}\ [m.]\ \color{gray}(msa. \foreignlanguage{arabic}{مُبالَغَة}~\foreignlanguage{arabic}{\textbf{٢.}}  \foreignlanguage{arabic}{إِفْتِراء}~\foreignlanguage{arabic}{\textbf{١.}})\color{black}\ \textbf{1.}~slandering  \textbf{2.}~exaggeration\  \begin{flushright}\color{gray}\foreignlanguage{arabic}{\textbf{\underline{\foreignlanguage{arabic}{أمثلة}}}: هالحكي كله عبعض إِفْتِراء بإِفْتِراء}\end{flushright}\color{black}} \vspace{2mm}

{\setlength\topsep{0pt}\textbf{\foreignlanguage{arabic}{اِفْتَرَى}}\ {\color{gray}\texttt{/\sffamily {{\sffamily ʔiftara}}/}\color{black}}\ \textsc{verb}\ [p.]\ \textbf{1.}~slander  \textbf{2.}~lie  \textbf{3.}~exaggerate  \textbf{4.}~do sth excessively\ \ $\bullet$\ \ \setlength\topsep{0pt}\textbf{\foreignlanguage{arabic}{اِفْتَرِي}}\ {\color{gray}\texttt{/\sffamily {{\sffamily ʔiftari}}/}\color{black}}\ [c.]\ \ $\bullet$\ \ \setlength\topsep{0pt}\textbf{\foreignlanguage{arabic}{يِفْتَرِي}}\ {\color{gray}\texttt{/\sffamily {{\sffamily jiftari}}/}\color{black}}\ [i.]\ \color{gray}(msa. \foreignlanguage{arabic}{يقوم بفعل شيء بشكل مبالغ فيه}~\foreignlanguage{arabic}{\textbf{٤.}}  \foreignlanguage{arabic}{يبالِغ}~\foreignlanguage{arabic}{\textbf{٣.}}  \foreignlanguage{arabic}{يَكْذِب}~\foreignlanguage{arabic}{\textbf{٢.}}  \foreignlanguage{arabic}{يَفْتَرِي}~\foreignlanguage{arabic}{\textbf{١.}})\color{black}\  \begin{flushright}\color{gray}\foreignlanguage{arabic}{\textbf{\underline{\foreignlanguage{arabic}{أمثلة}}}: أنت هيك بتفتري عالمخلوقة حرام عليك\ $\bullet$\ \  ياولدي قديش افتريت بالأكل}\end{flushright}\color{black}} \vspace{2mm}

{\setlength\topsep{0pt}\textbf{\foreignlanguage{arabic}{اِنْفَرَى}}\ {\color{gray}\texttt{/\sffamily {{\sffamily ʔinfara}}/}\color{black}}\ \textsc{verb}\ [p.]\ \textbf{1.}~cry a lot until sb is exhausted.  \textbf{2.}~be exhausted\ \ $\bullet$\ \ \setlength\topsep{0pt}\textbf{\foreignlanguage{arabic}{اِنْفِرِي}}\ {\color{gray}\texttt{/\sffamily {{\sffamily ʔinfiri}}/}\color{black}}\ [c.]\ \ $\bullet$\ \ \setlength\topsep{0pt}\textbf{\foreignlanguage{arabic}{يِنْفِرِي}}\ {\color{gray}\texttt{/\sffamily {{\sffamily jinfiri}}/}\color{black}}\ [i.]\ \ $\bullet$\ \ \textsc{ph.} \color{gray} \foreignlanguage{arabic}{اِنفرت كليتي}\color{black}\ {\color{gray}\texttt{/{\sffamily ʔinfarat kiljiti}/}\color{black}}\ \color{gray} (msa. \foreignlanguage{arabic}{يمل من القيام بشيء ما}~\foreignlanguage{arabic}{\textbf{١.}})\color{black}\ \textbf{1.}~be sick of doing sth.  \textbf{2.}~feel exhausted\  \begin{flushright}\color{gray}\foreignlanguage{arabic}{\textbf{\underline{\foreignlanguage{arabic}{أمثلة}}}: انفرت كليتي\ $\bullet$\ \  والله انفريت وأنا أحكيله بدي عصير}\end{flushright}\color{black}} \vspace{2mm}

{\setlength\topsep{0pt}\textbf{\foreignlanguage{arabic}{فَرَى}}\ {\color{gray}\texttt{/\sffamily {{\sffamily fara}}/}\color{black}}\ \textsc{verb}\ [p.]\ \textbf{1.}~pick (rabbits) up by their ears or necks\ \ $\bullet$\ \ \setlength\topsep{0pt}\textbf{\foreignlanguage{arabic}{اِفْرِي}}\ {\color{gray}\texttt{/\sffamily {{\sffamily ʔifri}}/}\color{black}}\ [c.]\ \ $\bullet$\ \ \setlength\topsep{0pt}\textbf{\foreignlanguage{arabic}{يِفْرِي}}\ {\color{gray}\texttt{/\sffamily {{\sffamily jifri}}/}\color{black}}\ [i.]\ \ $\bullet$\ \ \textsc{ph.} \color{gray} \foreignlanguage{arabic}{فَرَى مَاسَاتُه}\color{black}\ {\color{gray}\texttt{/{\sffamily fara masaːto}/}\color{black}}\ \textbf{1.}~turn sb (the baby) head down position\  \begin{flushright}\color{gray}\foreignlanguage{arabic}{\textbf{\underline{\foreignlanguage{arabic}{أمثلة}}}: فَرَى ماساتُه الله ستر ما استفرغ\ $\bullet$\ \  واحنا صغار كان في عند عمي أبو خالد أرانب. بقينا نحب نفريهن وعمي يضل يسبسب ويكفِّر}\end{flushright}\color{black}} \vspace{2mm}

{\setlength\topsep{0pt}\textbf{\foreignlanguage{arabic}{مَفْرِي}}\ {\color{gray}\texttt{/\sffamily {{\sffamily mafri}}/}\color{black}}\ \textsc{adj}\ [m.]\ \color{gray}(msa. \foreignlanguage{arabic}{مُنْهَك}~\foreignlanguage{arabic}{\textbf{١.}})\color{black}\ \textbf{1.}~exhausted\  \begin{flushright}\color{gray}\foreignlanguage{arabic}{\textbf{\underline{\foreignlanguage{arabic}{أمثلة}}}: رجعت من الجسر مَفْرِي من التعب}\end{flushright}\color{black}} \vspace{2mm}

{\setlength\topsep{0pt}\textbf{\foreignlanguage{arabic}{مُفْتَرِي}}\ {\color{gray}\texttt{/\sffamily {{\sffamily muftari}}/}\color{black}}\ \textsc{adj}\ [m.]\ \color{gray}(msa. \foreignlanguage{arabic}{دجّال}~\foreignlanguage{arabic}{\textbf{٢.}}  \foreignlanguage{arabic}{كَذّاب}~\foreignlanguage{arabic}{\textbf{١.}})\color{black}\ \textbf{1.}~liar  \textbf{2.}~imposter\  \begin{flushright}\color{gray}\foreignlanguage{arabic}{\textbf{\underline{\foreignlanguage{arabic}{أمثلة}}}: هاد واحد مُفْتَرِي وما بخاف الله}\end{flushright}\color{black}} \vspace{2mm}

\vspace{-3mm}
\markboth{\color{blue}\foreignlanguage{arabic}{ف.ز.ر}\color{blue}{}}{\color{blue}\foreignlanguage{arabic}{ف.ز.ر}\color{blue}{}}\subsection*{\color{blue}\foreignlanguage{arabic}{ف.ز.ر}\color{blue}{}\index{\color{blue}\foreignlanguage{arabic}{ف.ز.ر}\color{blue}{}}} 

{\setlength\topsep{0pt}\textbf{\foreignlanguage{arabic}{اِنْفَزَر}}\ {\color{gray}\texttt{/\sffamily {{\sffamily ʔinfazar}}/}\color{black}}\ \textsc{verb}\ [p.]\ \textbf{1.}~be burst open.  \textbf{2.}~be full.  \textbf{3.}~get very annoyed\ \ $\bullet$\ \ \setlength\topsep{0pt}\textbf{\foreignlanguage{arabic}{اِنْفِزِر}}\ {\color{gray}\texttt{/\sffamily {{\sffamily ʔinfizir}}/}\color{black}}\ [c.]\ \ $\bullet$\ \ \setlength\topsep{0pt}\textbf{\foreignlanguage{arabic}{يِنْفِزِر}}\ {\color{gray}\texttt{/\sffamily {{\sffamily jinfizir}}/}\color{black}}\ [i.]\  \begin{flushright}\color{gray}\foreignlanguage{arabic}{\textbf{\underline{\foreignlanguage{arabic}{أمثلة}}}: هو رح يِنْفِزِر منها قد مابتطلب منه طلبات\ $\bullet$\ \  أخوي انفَزَر من كثر ما أكل\ $\bullet$\ \  انفَزَر الكيس قد ماهو محشَّى أشياء}\end{flushright}\color{black}} \vspace{2mm}

{\setlength\topsep{0pt}\textbf{\foreignlanguage{arabic}{تْفَزَّر}}\ {\color{gray}\texttt{/\sffamily {{\sffamily tfazzar}}/}\color{black}}\ \textsc{verb}\ [p.]\ \textbf{1.}~be burst open.  \textbf{2.}~be torn.  \textbf{3.}~be ripped.  \textbf{4.}~be too tight that sth is going to burst open\ \ $\bullet$\ \ \setlength\topsep{0pt}\textbf{\foreignlanguage{arabic}{اِتْفَزَّر}}\ {\color{gray}\texttt{/\sffamily {{\sffamily ʔitfazzar}}/}\color{black}}\ [c.]\ \ $\bullet$\ \ \setlength\topsep{0pt}\textbf{\foreignlanguage{arabic}{يِتْفَزَّر}}\ {\color{gray}\texttt{/\sffamily {{\sffamily jitfazzar}}/}\color{black}}\ [i.]\ } \vspace{2mm}

{\setlength\topsep{0pt}\textbf{\foreignlanguage{arabic}{فَزَر}}\ {\color{gray}\texttt{/\sffamily {{\sffamily fazar}}/}\color{black}}\ \textsc{verb}\ [p.]\ \textbf{1.}~burst sth open.  \textbf{2.}~make sb full.  \textbf{3.}~make sb very annoyed\ \ $\bullet$\ \ \setlength\topsep{0pt}\textbf{\foreignlanguage{arabic}{اِفْزُر}}\ {\color{gray}\texttt{/\sffamily {{\sffamily ʔifzur}}/}\color{black}}\ [c.]\ \textbf{1.}~burst  \textbf{2.}~tear sth off.  \textbf{3.}~rip sth off\ \ $\bullet$\ \ \setlength\topsep{0pt}\textbf{\foreignlanguage{arabic}{اُفْزُر}}\ {\color{gray}\texttt{/\sffamily {{\sffamily ʔufzur}}/}\color{black}}\ [c.]\ \ $\bullet$\ \ \setlength\topsep{0pt}\textbf{\foreignlanguage{arabic}{يِفْزُر}}\ {\color{gray}\texttt{/\sffamily {{\sffamily jifzur}}/}\color{black}}\ [i.]\ \textbf{1.}~burst  \textbf{2.}~tear sth off.  \textbf{3.}~rip sth off\ \ $\bullet$\ \ \setlength\topsep{0pt}\textbf{\foreignlanguage{arabic}{يُفْزُر}}\ {\color{gray}\texttt{/\sffamily {{\sffamily jufzur}}/}\color{black}}\ [i.]\ \ $\bullet$\ \ \textsc{ph.} \color{gray} \foreignlanguage{arabic}{فَزَر اللي يِفْزُرك}\color{black}\ {\color{gray}\texttt{/{\sffamily fazar ʔilli jufzurak}/}\color{black}}\ \textbf{1.}~It is an expression that means that sb is very angry with someone that he wants him to burst like a balloon\  \begin{flushright}\color{gray}\foreignlanguage{arabic}{\textbf{\underline{\foreignlanguage{arabic}{أمثلة}}}: فَزَر اللي يِفْزُرك ان شاء الله رد\ $\bullet$\ \  صير عبي المي بكياس وعلقهم عالشباك وصير اُفْزُرهم قبل ما تطلع عالشغل\ $\bullet$\ \  ما فزرني غير لما جاب مرته الجديدة خاوا عالدار\ $\bullet$\ \  أكلت سندويشة كبيرة فزرتني}\end{flushright}\color{black}} \vspace{2mm}

{\setlength\topsep{0pt}\textbf{\foreignlanguage{arabic}{فَزِر}}\ {\color{gray}\texttt{/\sffamily {{\sffamily fazir}}/}\color{black}}\ \textsc{interj}\ \textbf{1.}~It is an expression that means that sb is very angry with someone that he wants him to burst like a balloon\ } \vspace{2mm}

{\setlength\topsep{0pt}\textbf{\foreignlanguage{arabic}{فَزَّر}}\ {\color{gray}\texttt{/\sffamily {{\sffamily fazzar}}/}\color{black}}\ \textsc{verb}\ [p.]\ \textbf{1.}~burst sth open (repeatedly with great force)\ \ $\bullet$\ \ \setlength\topsep{0pt}\textbf{\foreignlanguage{arabic}{فَزِّر}}\ {\color{gray}\texttt{/\sffamily {{\sffamily fazzir}}/}\color{black}}\ [c.]\ \ $\bullet$\ \ \setlength\topsep{0pt}\textbf{\foreignlanguage{arabic}{يفَزِّر}}\ {\color{gray}\texttt{/\sffamily {{\sffamily jfazzir}}/}\color{black}}\ [i.]\ \ $\bullet$\ \ \textsc{ph.} \color{gray} \foreignlanguage{arabic}{حَزِّر فَزِّر}\color{black}\ {\color{gray}\texttt{/{\sffamily ħazzar fazzar}/}\color{black}}\ \textbf{1.}~guess what!.  \textbf{2.}~guess whom!\  \begin{flushright}\color{gray}\foreignlanguage{arabic}{\textbf{\underline{\foreignlanguage{arabic}{أمثلة}}}: حَزِّر فَزِّر مين اجى عنا اليوم؟\ $\bullet$\ \  مسك السكين وضله يفَزِّر بالكيس عشان يطلع منه الهوا}\end{flushright}\color{black}} \vspace{2mm}

{\setlength\topsep{0pt}\textbf{\foreignlanguage{arabic}{فَزُّورَة}}\ {\color{gray}\texttt{/\sffamily {{\sffamily fazzuːra}}/}\color{black}}\ \textsc{noun}\ [f.]\ \textbf{1.}~riddle\ \ $\bullet$\ \ \setlength\topsep{0pt}\textbf{\foreignlanguage{arabic}{فوَازِير}}\ {\color{gray}\texttt{/\sffamily {{\sffamily fawaːziːr}}/}\color{black}}\ [pl.]\  \begin{flushright}\color{gray}\foreignlanguage{arabic}{\textbf{\underline{\foreignlanguage{arabic}{أمثلة}}}: كل الفوازِير اللي حكوها اليوم قديمة، بدنا شي جديد}\end{flushright}\color{black}} \vspace{2mm}

{\setlength\topsep{0pt}\textbf{\foreignlanguage{arabic}{مَفْزُور}}\ {\color{gray}\texttt{/\sffamily {{\sffamily mafzuːr}}/}\color{black}}\ \textsc{adj}\ [m.]\ \color{gray}(msa. \foreignlanguage{arabic}{ممزَّق}~\foreignlanguage{arabic}{\textbf{١.}})\color{black}\ \textbf{1.}~torn/ripped plastic bag\ \ $\smblkdiamond$\ \ \setlength\topsep{0pt}\textbf{\foreignlanguage{arabic}{مَفْزُور}}\ \color{gray}(msa. \foreignlanguage{arabic}{غاضِب جداً}~\foreignlanguage{arabic}{\textbf{٢.}}  .\foreignlanguage{arabic}{شبعان حد التُّخْمَة}~\foreignlanguage{arabic}{\textbf{١.}})\color{black}\ \textbf{1.}~full (sb has eaten so much food that he cannot eat any more)satiated.  \textbf{2.}~very upset\  \begin{flushright}\color{gray}\foreignlanguage{arabic}{\textbf{\underline{\foreignlanguage{arabic}{أمثلة}}}: أنا بقيت مَفْزُور منك بس مش راضي أحكيلك\ $\bullet$\ \  مَفْزُور من الأكل\ $\bullet$\ \  دير بالك الكيس مَفْزُورْ}\end{flushright}\color{black}} \vspace{2mm}

\vspace{-3mm}
\markboth{\color{blue}\foreignlanguage{arabic}{ف.ز.ز}\color{blue}{}}{\color{blue}\foreignlanguage{arabic}{ف.ز.ز}\color{blue}{}}\subsection*{\color{blue}\foreignlanguage{arabic}{ف.ز.ز}\color{blue}{}\index{\color{blue}\foreignlanguage{arabic}{ف.ز.ز}\color{blue}{}}} 

{\setlength\topsep{0pt}\textbf{\foreignlanguage{arabic}{فَازِز}}\ {\color{gray}\texttt{/\sffamily {{\sffamily faːziz}}/}\color{black}}\ \textsc{noun\textunderscore act}\ [m.]\ \textbf{1.}~getting up\  \begin{flushright}\color{gray}\foreignlanguage{arabic}{\textbf{\underline{\foreignlanguage{arabic}{أمثلة}}}: هيك فازِز عحيلك وفش فيك شي صلاة محمد}\end{flushright}\color{black}} \vspace{2mm}

{\setlength\topsep{0pt}\textbf{\foreignlanguage{arabic}{فَزّ}}\ {\color{gray}\texttt{/\sffamily {{\sffamily fazz}}/}\color{black}}\ \textsc{verb}\ [p.]\ \textbf{1.}~get up\ \ $\bullet$\ \ \setlength\topsep{0pt}\textbf{\foreignlanguage{arabic}{فِزّ}}\ {\color{gray}\texttt{/\sffamily {{\sffamily fizz}}/}\color{black}}\ [c.]\ \ $\bullet$\ \ \setlength\topsep{0pt}\textbf{\foreignlanguage{arabic}{يفِزّ}}\ {\color{gray}\texttt{/\sffamily {{\sffamily jfizz}}/}\color{black}}\ [i.]\ \color{gray}(msa. \foreignlanguage{arabic}{يَنْهَض}~\foreignlanguage{arabic}{\textbf{١.}})\color{black}\  \begin{flushright}\color{gray}\foreignlanguage{arabic}{\textbf{\underline{\foreignlanguage{arabic}{أمثلة}}}: ولك فِز بشرعة خالتو بديعة جاي عنا}\end{flushright}\color{black}} \vspace{2mm}

\vspace{-3mm}
\markboth{\color{blue}\foreignlanguage{arabic}{ف.ز.ع}\color{blue}{}}{\color{blue}\foreignlanguage{arabic}{ف.ز.ع}\color{blue}{}}\subsection*{\color{blue}\foreignlanguage{arabic}{ف.ز.ع}\color{blue}{}\index{\color{blue}\foreignlanguage{arabic}{ف.ز.ع}\color{blue}{}}} 

{\setlength\topsep{0pt}\textbf{\foreignlanguage{arabic}{أَفْزَع}}\ {\color{gray}\texttt{/\sffamily {{\sffamily ʔafzaʕ}}/}\color{black}}\ \textsc{verb}\ [p.]\ \textbf{1.}~frighten sb.  \textbf{2.}~intimidate sb\ \ $\bullet$\ \ \setlength\topsep{0pt}\textbf{\foreignlanguage{arabic}{اِفْزِع}}\ {\color{gray}\texttt{/\sffamily {{\sffamily ʔifziʕ}}/}\color{black}}\ [c.]\ \ $\bullet$\ \ \setlength\topsep{0pt}\textbf{\foreignlanguage{arabic}{يِفْزِع}}\ {\color{gray}\texttt{/\sffamily {{\sffamily jifziʕ}}/}\color{black}}\ [i.]\  \begin{flushright}\color{gray}\foreignlanguage{arabic}{\textbf{\underline{\foreignlanguage{arabic}{أمثلة}}}: بديش أصحيها بدفاشة هيك وأفْزِعها   حرام}\end{flushright}\color{black}} \vspace{2mm}

{\setlength\topsep{0pt}\textbf{\foreignlanguage{arabic}{اِنْفَزَع}}\ {\color{gray}\texttt{/\sffamily {{\sffamily ʔinfazaʕ}}/}\color{black}}\ \textsc{verb}\ [p.]\ \textbf{1.}~be intimidated\ \ $\bullet$\ \ \setlength\topsep{0pt}\textbf{\foreignlanguage{arabic}{اِنْفِزِع}}\ {\color{gray}\texttt{/\sffamily {{\sffamily ʔinfiziʕ}}/}\color{black}}\ [c.]\ \ $\bullet$\ \ \setlength\topsep{0pt}\textbf{\foreignlanguage{arabic}{يِنْفِزِع}}\ {\color{gray}\texttt{/\sffamily {{\sffamily jinfiziʕ}}/}\color{black}}\ [i.]\ } \vspace{2mm}

{\setlength\topsep{0pt}\textbf{\foreignlanguage{arabic}{فَزَع}}\ {\color{gray}\texttt{/\sffamily {{\sffamily fazaʕ}}/}\color{black}}\ \textsc{verb}\ [p.]\ \textbf{1.}~side with sb in a fight.  \textbf{2.}~support sb in a fight (people who get behind sb when he is in a fight)\ \ $\bullet$\ \ \setlength\topsep{0pt}\textbf{\foreignlanguage{arabic}{اِفْزَع}}\ {\color{gray}\texttt{/\sffamily {{\sffamily ʔifzaʕ}}/}\color{black}}\ [c.]\ \ $\bullet$\ \ \setlength\topsep{0pt}\textbf{\foreignlanguage{arabic}{يِفْزَع}}\ {\color{gray}\texttt{/\sffamily {{\sffamily jifzaʕ}}/}\color{black}}\ [i.]\  \begin{flushright}\color{gray}\foreignlanguage{arabic}{\textbf{\underline{\foreignlanguage{arabic}{أمثلة}}}: صارت طوشة بيننا وبين شباب مخيم العروب واجى سروجي يفزعلنا}\end{flushright}\color{black}} \vspace{2mm}

{\setlength\topsep{0pt}\textbf{\foreignlanguage{arabic}{فَزَّاعَة}}\ {\color{gray}\texttt{/\sffamily {{\sffamily fazzaːʕa}}/}\color{black}}\ \textsc{noun}\ [f.]\ \color{gray}(msa. \foreignlanguage{arabic}{فَزّاعة}~\foreignlanguage{arabic}{\textbf{١.}})\color{black}\ \textbf{1.}~scarecrow\  \begin{flushright}\color{gray}\foreignlanguage{arabic}{\textbf{\underline{\foreignlanguage{arabic}{أمثلة}}}: شكلك بهالدماية بيفرط ضحك مثل الفَزّاعة}\end{flushright}\color{black}} \vspace{2mm}

{\setlength\topsep{0pt}\textbf{\foreignlanguage{arabic}{فَزَّع}}\ {\color{gray}\texttt{/\sffamily {{\sffamily fazzaʕ}}/}\color{black}}\ \textsc{verb}\ [p.]\ \textbf{1.}~frighten sb.  \textbf{2.}~intimidate sb\ \ $\bullet$\ \ \setlength\topsep{0pt}\textbf{\foreignlanguage{arabic}{فَزِّع}}\ {\color{gray}\texttt{/\sffamily {{\sffamily fazziʕ}}/}\color{black}}\ [c.]\ \ $\bullet$\ \ \setlength\topsep{0pt}\textbf{\foreignlanguage{arabic}{يفَزِّع}}\ {\color{gray}\texttt{/\sffamily {{\sffamily jfazziʕ}}/}\color{black}}\ [i.]\ \ $\bullet$\ \ \textsc{ph.} \color{gray} \foreignlanguage{arabic}{فَزَّع الدنيَا}\color{black}\ {\color{gray}\texttt{/{\sffamily fazzaʕ ʔiddinja}/}\color{black}}\ \color{gray} (msa. \foreignlanguage{arabic}{أخبَر الجميع بهذا الأمر}~\foreignlanguage{arabic}{\textbf{١.}})\color{black}\ \textbf{1.}~let the cat out of the bag\  \begin{flushright}\color{gray}\foreignlanguage{arabic}{\textbf{\underline{\foreignlanguage{arabic}{أمثلة}}}: فَزَّع الدِّنْيا انه هو رايح عغربا بالأخير كحشوه\ $\bullet$\ \  صار يصيح بقوة عالساعة ثلاثة فَزَّعنا}\end{flushright}\color{black}} \vspace{2mm}

{\setlength\topsep{0pt}\textbf{\foreignlanguage{arabic}{فَزِّيع}}\ {\color{gray}\texttt{/\sffamily {{\sffamily fazziːʕ}}/}\color{black}}\ \textsc{adj}\ [m.]\ \textbf{1.}~sb who sides with someone in a fight.  \textbf{2.}~sb who supports someone in a fight (people who get behind sb when he is in a fight)\  \begin{flushright}\color{gray}\foreignlanguage{arabic}{\textbf{\underline{\foreignlanguage{arabic}{أمثلة}}}: ما شاء الله فَزِّيع متى ما احتجناه بنلاقيه}\end{flushright}\color{black}} \vspace{2mm}

{\setlength\topsep{0pt}\textbf{\foreignlanguage{arabic}{فَزْعَة}}\ {\color{gray}\texttt{/\sffamily {{\sffamily fazʕa}}/}\color{black}}\ \textsc{noun}\ [f.]\ \color{gray}(msa. \foreignlanguage{arabic}{مساعدة من ناس في مشكلة}~\foreignlanguage{arabic}{\textbf{١.}})\color{black}\ \textbf{1.}~support (people who get behind sb when he is in a fight)\  \begin{flushright}\color{gray}\foreignlanguage{arabic}{\textbf{\underline{\foreignlanguage{arabic}{أمثلة}}}: بدنا فَزْعَة يا شباب}\end{flushright}\color{black}} \vspace{2mm}

{\setlength\topsep{0pt}\textbf{\foreignlanguage{arabic}{فِزِع}}\ {\color{gray}\texttt{/\sffamily {{\sffamily fiziʕ}}/}\color{black}}\ \textsc{verb}\ [p.]\ \textbf{1.}~be intimidated\ \ $\bullet$\ \ \setlength\topsep{0pt}\textbf{\foreignlanguage{arabic}{اِفْزَع}}\ {\color{gray}\texttt{/\sffamily {{\sffamily ʔifzaʕ}}/}\color{black}}\ [c.]\ \ $\bullet$\ \ \setlength\topsep{0pt}\textbf{\foreignlanguage{arabic}{يِفْزَع}}\ {\color{gray}\texttt{/\sffamily {{\sffamily jifzaʕ}}/}\color{black}}\ [i.]\  \begin{flushright}\color{gray}\foreignlanguage{arabic}{\textbf{\underline{\foreignlanguage{arabic}{أمثلة}}}: لما خالتو ندية صحيت الساعة 3 الفجر فِزعت بس شافتني اني رجعت من غربا بهيك وقت}\end{flushright}\color{black}} \vspace{2mm}

{\setlength\topsep{0pt}\textbf{\foreignlanguage{arabic}{مَفْزُوع}}\ {\color{gray}\texttt{/\sffamily {{\sffamily mafzuːʕ}}/}\color{black}}\ \textsc{adj}\ [m.]\ \textbf{1.}~intimidated\  \begin{flushright}\color{gray}\foreignlanguage{arabic}{\textbf{\underline{\foreignlanguage{arabic}{أمثلة}}}: صحيت مَفْزُوع شفت منام مش مليح أبدا}\end{flushright}\color{black}} \vspace{2mm}

{\setlength\topsep{0pt}\textbf{\foreignlanguage{arabic}{مُفْزِع}}\ {\color{gray}\texttt{/\sffamily {{\sffamily mufziʕ}}/}\color{black}}\ \textsc{adj}\ [m.]\ \textbf{1.}~frightening  \textbf{2.}~intimidating\  \begin{flushright}\color{gray}\foreignlanguage{arabic}{\textbf{\underline{\foreignlanguage{arabic}{أمثلة}}}: منظر السيارة وهي بتطلع دُخّان مُفْزِع جداً}\end{flushright}\color{black}} \vspace{2mm}

\vspace{-3mm}
\markboth{\color{blue}\foreignlanguage{arabic}{ف.ز.ل.ك}\color{blue}{}}{\color{blue}\foreignlanguage{arabic}{ف.ز.ل.ك}\color{blue}{}}\subsection*{\color{blue}\foreignlanguage{arabic}{ف.ز.ل.ك}\color{blue}{}\index{\color{blue}\foreignlanguage{arabic}{ف.ز.ل.ك}\color{blue}{}}} 

{\setlength\topsep{0pt}\textbf{\foreignlanguage{arabic}{تْفَزْلَك}}\ {\color{gray}\texttt{/\sffamily {{\sffamily tfazlak}}/}\color{black}}\ \textsc{verb}\ [p.]\ \textbf{1.}~show off.  \textbf{2.}~fake\ \ $\bullet$\ \ \setlength\topsep{0pt}\textbf{\foreignlanguage{arabic}{اِتْفَزْلَك}}\ {\color{gray}\texttt{/\sffamily {{\sffamily ʔitfazlak}}/}\color{black}}\ [c.]\ \ $\bullet$\ \ \setlength\topsep{0pt}\textbf{\foreignlanguage{arabic}{يِتْفَزْلَك}}\ {\color{gray}\texttt{/\sffamily {{\sffamily jitfazlak}}/}\color{black}}\ [i.]\ \color{gray}(msa. \foreignlanguage{arabic}{يَتَصَنَّع}~\foreignlanguage{arabic}{\textbf{٢.}}  \foreignlanguage{arabic}{يَتَباهَى}~\foreignlanguage{arabic}{\textbf{١.}})\color{black}\  \begin{flushright}\color{gray}\foreignlanguage{arabic}{\textbf{\underline{\foreignlanguage{arabic}{أمثلة}}}: تِتفزلكِش ولا انطَم واسكت}\end{flushright}\color{black}} \vspace{2mm}

{\setlength\topsep{0pt}\textbf{\foreignlanguage{arabic}{فَزْلَكِة}}\ {\color{gray}\texttt{/\sffamily {{\sffamily fazlake}}/}\color{black}}\ \textsc{noun}\ [f.]\ \color{gray}(msa. \foreignlanguage{arabic}{مُباهاة}~\foreignlanguage{arabic}{\textbf{١.}})\color{black}\ \textbf{1.}~show-off\ } \vspace{2mm}

{\setlength\topsep{0pt}\textbf{\foreignlanguage{arabic}{مْفَزْلَك}}\ {\color{gray}\texttt{/\sffamily {{\sffamily mfazlak}}/}\color{black}}\ \textsc{adj}\ [m.]\ \color{gray}(msa. \foreignlanguage{arabic}{متصنع ومدَّعي الأهمية}~\foreignlanguage{arabic}{\textbf{١.}})\color{black}\ \textbf{1.}~pretentious\  \begin{flushright}\color{gray}\foreignlanguage{arabic}{\textbf{\underline{\foreignlanguage{arabic}{أمثلة}}}: اه بعرف ابنها الكبير مْفَزْلَك كثير}\end{flushright}\color{black}} \vspace{2mm}

\vspace{-3mm}
\markboth{\color{blue}\foreignlanguage{arabic}{ف.ز.ل.ن}\color{blue}{ (ntws)}}{\color{blue}\foreignlanguage{arabic}{ف.ز.ل.ن}\color{blue}{ (ntws)}}\subsection*{\color{blue}\foreignlanguage{arabic}{ف.ز.ل.ن}\color{blue}{ (ntws)}\index{\color{blue}\foreignlanguage{arabic}{ف.ز.ل.ن}\color{blue}{ (ntws)}}} 

{\setlength\topsep{0pt}\textbf{\foreignlanguage{arabic}{فَازْلِين}}\ {\color{gray}\texttt{/\sffamily {{\sffamily vaːzliːn}}/}\color{black}}\ \textsc{noun}\ [m.]\ \textbf{1.}~vaseline\ } \vspace{2mm}

\vspace{-3mm}
\markboth{\color{blue}\foreignlanguage{arabic}{ف.س.ت.ق}\color{blue}{}}{\color{blue}\foreignlanguage{arabic}{ف.س.ت.ق}\color{blue}{}}\subsection*{\color{blue}\foreignlanguage{arabic}{ف.س.ت.ق}\color{blue}{}\index{\color{blue}\foreignlanguage{arabic}{ف.س.ت.ق}\color{blue}{}}} 

{\setlength\topsep{0pt}\textbf{\foreignlanguage{arabic}{فُسْتُق}}\ {\color{gray}\texttt{/\sffamily {{\sffamily fustu(q)}}/}\color{black}}\ \textsc{noun}\ [m.]\ \textbf{1.}~Pistachios  \textbf{2.}~peanuts\ \ $\bullet$\ \ \textsc{ph.} \color{gray} \foreignlanguage{arabic}{فُسْتُق فَاضي}\color{black}\ {\color{gray}\texttt{/{\sffamily fustu(q) faː(dˤ)i}/}\color{black}}\ \color{gray} (msa. \foreignlanguage{arabic}{مبالغ فيه}~\foreignlanguage{arabic}{\textbf{١.}})\color{black}\ \textbf{1.}~overrated\ \ $\bullet$\ \ \textsc{ph.} \color{gray} \foreignlanguage{arabic}{فُسْتُق عبيد}\color{black}\ {\color{gray}\texttt{/{\sffamily fustuq ʕabiːd}/}\color{black}}\ \color{gray} (msa. \foreignlanguage{arabic}{فول سوداني}~\foreignlanguage{arabic}{\textbf{١.}})\color{black}\ \textbf{1.}~Peanuts\  \begin{flushright}\color{gray}\foreignlanguage{arabic}{\textbf{\underline{\foreignlanguage{arabic}{أمثلة}}}: أصلا معروف إِنُّه هاي الجامعة فُسْتُق فاضِي}\end{flushright}\color{black}} \vspace{2mm}

{\setlength\topsep{0pt}\textbf{\foreignlanguage{arabic}{فُسْتُقَة}}\ {\color{gray}\texttt{/\sffamily {{\sffamily fustu(q)a}}/}\color{black}}\ \textsc{noun}\ [f.]\ \textbf{1.}~one piece of Pistachios.  \textbf{2.}~peanuts\ \ $\bullet$\ \ \textsc{ph.} \color{gray} \foreignlanguage{arabic}{قد الفُسْتُقَة}\color{black}\ {\color{gray}\texttt{/{\sffamily (q)addil fustu(q)a}/}\color{black}}\ \color{gray} (msa. \foreignlanguage{arabic}{صغير جدا}~\foreignlanguage{arabic}{\textbf{١.}})\color{black}\ \textbf{1.}~like a pistachio (It is an idiomatic expression that means sth is very small)\  \begin{flushright}\color{gray}\foreignlanguage{arabic}{\textbf{\underline{\foreignlanguage{arabic}{أمثلة}}}: ثمها قد الفُسْتُقَة لما تضحك يادوب تبين سنانها}\end{flushright}\color{black}} \vspace{2mm}

{\setlength\topsep{0pt}\textbf{\foreignlanguage{arabic}{مْفَسْتِق}}\ {\color{gray}\texttt{/\sffamily {{\sffamily mfasti(q)}}/}\color{black}}\ \textsc{adj}\ [m.]\ \color{gray}(msa. \foreignlanguage{arabic}{متعب / لايمكنه القيام باي شيء}~\foreignlanguage{arabic}{\textbf{١.}})\color{black}\ \textbf{1.}~exhausted\ } \vspace{2mm}

\vspace{-3mm}
\markboth{\color{blue}\foreignlanguage{arabic}{ف.س.ت.ن}\color{blue}{}}{\color{blue}\foreignlanguage{arabic}{ف.س.ت.ن}\color{blue}{}}\subsection*{\color{blue}\foreignlanguage{arabic}{ف.س.ت.ن}\color{blue}{}\index{\color{blue}\foreignlanguage{arabic}{ف.س.ت.ن}\color{blue}{}}} 

{\setlength\topsep{0pt}\textbf{\foreignlanguage{arabic}{فُسْتَان}}\ {\color{gray}\texttt{/\sffamily {{\sffamily fusˤtˤaːn}}/}\color{black}}\ \textsc{noun}\ [m.]\ \color{gray}(msa. \foreignlanguage{arabic}{ثوب}~\foreignlanguage{arabic}{\textbf{١.}})\color{black}\ \textbf{1.}~dress\ \ $\bullet$\ \ \setlength\topsep{0pt}\textbf{\foreignlanguage{arabic}{فَسَاتِين}}\ {\color{gray}\texttt{/\sffamily {{\sffamily fasˤaːtˤiːn}}/}\color{black}}\ [pl.]\  \begin{flushright}\color{gray}\foreignlanguage{arabic}{\textbf{\underline{\foreignlanguage{arabic}{أمثلة}}}: كل فَساتِينها كاشحة بيضبطش هيك مع دار حماها}\end{flushright}\color{black}} \vspace{2mm}

\vspace{-3mm}
\markboth{\color{blue}\foreignlanguage{arabic}{ف.س.ح}\color{blue}{}}{\color{blue}\foreignlanguage{arabic}{ف.س.ح}\color{blue}{}}\subsection*{\color{blue}\foreignlanguage{arabic}{ف.س.ح}\color{blue}{}\index{\color{blue}\foreignlanguage{arabic}{ف.س.ح}\color{blue}{}}} 

{\setlength\topsep{0pt}\textbf{\foreignlanguage{arabic}{تْفَسَّح}}\ {\color{gray}\texttt{/\sffamily {{\sffamily tfassaħ}}/}\color{black}}\ \textsc{verb}\ [p.]\ \textbf{1.}~go on a pleasurable stroll\ \ $\bullet$\ \ \setlength\topsep{0pt}\textbf{\foreignlanguage{arabic}{اِتْفَسَّح}}\ {\color{gray}\texttt{/\sffamily {{\sffamily ʔitfassaħ}}/}\color{black}}\ [c.]\ \ $\bullet$\ \ \setlength\topsep{0pt}\textbf{\foreignlanguage{arabic}{يِتْفَسَّح}}\ {\color{gray}\texttt{/\sffamily {{\sffamily jitfassaħ}}/}\color{black}}\ [i.]\ } \vspace{2mm}

{\setlength\topsep{0pt}\textbf{\foreignlanguage{arabic}{فَسَح}}\ {\color{gray}\texttt{/\sffamily {{\sffamily fasaħ}}/}\color{black}}\ \textsc{verb}\ [p.]\ \textbf{1.}~make room for sth\ \ $\bullet$\ \ \setlength\topsep{0pt}\textbf{\foreignlanguage{arabic}{اِفْسِح}}\ {\color{gray}\texttt{/\sffamily {{\sffamily ʔifsiħ}}/}\color{black}}\ [c.]\ \ $\bullet$\ \ \setlength\topsep{0pt}\textbf{\foreignlanguage{arabic}{يِفْسِح}}\ {\color{gray}\texttt{/\sffamily {{\sffamily jifsiħ}}/}\color{black}}\ [i.]\ \color{gray}(msa. \foreignlanguage{arabic}{يُفْسِح}~\foreignlanguage{arabic}{\textbf{١.}})\color{black}\  \begin{flushright}\color{gray}\foreignlanguage{arabic}{\textbf{\underline{\foreignlanguage{arabic}{أمثلة}}}: اِفْسِح المجال لغيرك إِنُّه يقدِّم أفضل ماعنده}\end{flushright}\color{black}} \vspace{2mm}

{\setlength\topsep{0pt}\textbf{\foreignlanguage{arabic}{فَسَّح}}\ {\color{gray}\texttt{/\sffamily {{\sffamily fassaħ}}/}\color{black}}\ \textsc{verb}\ [p.]\ \textbf{1.}~take sb for a pleasurable stroll\ \ $\bullet$\ \ \setlength\topsep{0pt}\textbf{\foreignlanguage{arabic}{فَسِّح}}\ {\color{gray}\texttt{/\sffamily {{\sffamily fassiħ}}/}\color{black}}\ [c.]\ \ $\bullet$\ \ \setlength\topsep{0pt}\textbf{\foreignlanguage{arabic}{يفَسِّح}}\ {\color{gray}\texttt{/\sffamily {{\sffamily jfassiħ}}/}\color{black}}\ [i.]\  \begin{flushright}\color{gray}\foreignlanguage{arabic}{\textbf{\underline{\foreignlanguage{arabic}{أمثلة}}}: ياعمي هاي مرتك. خذها، جيبها، فسِّحها، شممها الهوا!}\end{flushright}\color{black}} \vspace{2mm}

{\setlength\topsep{0pt}\textbf{\foreignlanguage{arabic}{فَسِيح}}\ {\color{gray}\texttt{/\sffamily {{\sffamily fasiːħ}}/}\color{black}}\ \textsc{adj}\ [m.]\ \color{gray}(msa. \foreignlanguage{arabic}{واِع}~\foreignlanguage{arabic}{\textbf{١.}})\color{black}\ \textbf{1.}~wide  \textbf{2.}~broad  \textbf{3.}~capacious\  \begin{flushright}\color{gray}\foreignlanguage{arabic}{\textbf{\underline{\foreignlanguage{arabic}{أمثلة}}}: الله يرحمه ويغفرله ويسكنه فَسِيح جناته}\end{flushright}\color{black}} \vspace{2mm}

{\setlength\topsep{0pt}\textbf{\foreignlanguage{arabic}{فُسْحَة}}\ {\color{gray}\texttt{/\sffamily {{\sffamily fusħa}}/}\color{black}}\ \textsc{noun}\ [f.]\ \textbf{1.}~a pleasurable stroll\ \ $\bullet$\ \ \setlength\topsep{0pt}\textbf{\foreignlanguage{arabic}{فُسَح}}\ {\color{gray}\texttt{/\sffamily {{\sffamily fusaħ}}/}\color{black}}\ [pl.]\  \begin{flushright}\color{gray}\foreignlanguage{arabic}{\textbf{\underline{\foreignlanguage{arabic}{أمثلة}}}: هي ماخذة روحة المستشفى فُسْحَة؟ الله يثبت علينا العقل والدين!}\end{flushright}\color{black}} \vspace{2mm}

\vspace{-3mm}
\markboth{\color{blue}\foreignlanguage{arabic}{ف.س.خ}\color{blue}{}}{\color{blue}\foreignlanguage{arabic}{ف.س.خ}\color{blue}{}}\subsection*{\color{blue}\foreignlanguage{arabic}{ف.س.خ}\color{blue}{}\index{\color{blue}\foreignlanguage{arabic}{ف.س.خ}\color{blue}{}}} 

{\setlength\topsep{0pt}\textbf{\foreignlanguage{arabic}{تْفَسَّخ}}\ {\color{gray}\texttt{/\sffamily {{\sffamily tfassax}}/}\color{black}}\ \textsc{verb}\ [p.]\ \textbf{1.}~be torn apart.  \textbf{2.}~be ripped off\ \ $\bullet$\ \ \setlength\topsep{0pt}\textbf{\foreignlanguage{arabic}{اِتْفَسَّخ}}\ {\color{gray}\texttt{/\sffamily {{\sffamily ʔitfassax}}/}\color{black}}\ [c.]\ \ $\bullet$\ \ \setlength\topsep{0pt}\textbf{\foreignlanguage{arabic}{يِتْفَسَّخ}}\ {\color{gray}\texttt{/\sffamily {{\sffamily jitfassax}}/}\color{black}}\ [i.]\  \begin{flushright}\color{gray}\foreignlanguage{arabic}{\textbf{\underline{\foreignlanguage{arabic}{أمثلة}}}: تْفَسَّخ الكيس وأنا بالسوق. شو بدي أعمل}\end{flushright}\color{black}} \vspace{2mm}

{\setlength\topsep{0pt}\textbf{\foreignlanguage{arabic}{فَاسِخ}}\ {\color{gray}\texttt{/\sffamily {{\sffamily faːsix}}/}\color{black}}\ \textsc{adj}\ [m.]\ \textbf{1.}~divorced (engagement)\  \begin{flushright}\color{gray}\foreignlanguage{arabic}{\textbf{\underline{\foreignlanguage{arabic}{أمثلة}}}: والله يا خالتو أهلي مابعطوا واحد كان خاطِب وفاسِخ}\end{flushright}\color{black}} \vspace{2mm}

{\setlength\topsep{0pt}\textbf{\foreignlanguage{arabic}{فَاسِخ}}\ {\color{gray}\texttt{/\sffamily {{\sffamily faːsix}}/}\color{black}}\ \textsc{noun\textunderscore act}\ \textbf{1.}~splitting sth into two halves\  \begin{flushright}\color{gray}\foreignlanguage{arabic}{\textbf{\underline{\foreignlanguage{arabic}{أمثلة}}}: باقي فاسِخ الجاجة من إِجريها}\end{flushright}\color{black}} \vspace{2mm}

{\setlength\topsep{0pt}\textbf{\foreignlanguage{arabic}{فَسَخ}}\ {\color{gray}\texttt{/\sffamily {{\sffamily fasax}}/}\color{black}}\ \textsc{verb}\ [p.]\ \textbf{1.}~terminate a contract.  \textbf{2.}~divorce sb (engagement).  \textbf{3.}~spread one's legs wide open.  \textbf{4.}~split sth into two halves\ \ $\bullet$\ \ \setlength\topsep{0pt}\textbf{\foreignlanguage{arabic}{اِفْسَخ}}\ {\color{gray}\texttt{/\sffamily {{\sffamily ʔifsax}}/}\color{black}}\ [c.]\ \textbf{1.}~run away!\ \ $\bullet$\ \ \setlength\topsep{0pt}\textbf{\foreignlanguage{arabic}{يِفْسَخ}}\ {\color{gray}\texttt{/\sffamily {{\sffamily jifsax}}/}\color{black}}\ [i.]\  \begin{flushright}\color{gray}\foreignlanguage{arabic}{\textbf{\underline{\foreignlanguage{arabic}{أمثلة}}}: المدرب عنا بيعرف يِفْسَخ إِجريه زي المسطرة\ $\bullet$\ \  كاظم بقى بده يِفْسَخ الخطوبة من زمان بس كان بدوزنها براسه\ $\bullet$\ \  نصيحة اِفْسَخ ولا بتعرفش شو ممكن يصيرلك\ $\bullet$\ \  الوكالة فَسَخت عقدي أنا واثنين مهندسين من بيت ساحور}\end{flushright}\color{black}} \vspace{2mm}

{\setlength\topsep{0pt}\textbf{\foreignlanguage{arabic}{فَسِخ}}\ {\color{gray}\texttt{/\sffamily {{\sffamily fasix}}/}\color{black}}\ \textsc{noun}\ [m.]\ \textbf{1.}~termination  \textbf{2.}~divorce (engagement)\  \begin{flushright}\color{gray}\foreignlanguage{arabic}{\textbf{\underline{\foreignlanguage{arabic}{أمثلة}}}: فَسخ العقد ماعليهوش تعويض للأسف}\end{flushright}\color{black}} \vspace{2mm}

{\setlength\topsep{0pt}\textbf{\foreignlanguage{arabic}{فَسَّخ}}\ {\color{gray}\texttt{/\sffamily {{\sffamily fassax}}/}\color{black}}\ \textsc{verb}\ [p.]\ \textbf{1.}~tear sth apart with force\ \ $\bullet$\ \ \setlength\topsep{0pt}\textbf{\foreignlanguage{arabic}{فَسِّخ}}\ {\color{gray}\texttt{/\sffamily {{\sffamily fassix}}/}\color{black}}\ [c.]\ \ $\bullet$\ \ \setlength\topsep{0pt}\textbf{\foreignlanguage{arabic}{يفَسِّخ}}\ {\color{gray}\texttt{/\sffamily {{\sffamily jfassix}}/}\color{black}}\ [i.]\  \begin{flushright}\color{gray}\foreignlanguage{arabic}{\textbf{\underline{\foreignlanguage{arabic}{أمثلة}}}: فَسَّخلي الكيس بالميزان تبعه}\end{flushright}\color{black}} \vspace{2mm}

{\setlength\topsep{0pt}\textbf{\foreignlanguage{arabic}{مَفْسُوخ}}\ {\color{gray}\texttt{/\sffamily {{\sffamily mafsuːx}}/}\color{black}}\ \textsc{noun\textunderscore pass}\ \textbf{1.}~terminated  \textbf{2.}~torn apart\  \begin{flushright}\color{gray}\foreignlanguage{arabic}{\textbf{\underline{\foreignlanguage{arabic}{أمثلة}}}: العقد هيك بيكون مَفْسوخ. بتقدر تتفضل!}\end{flushright}\color{black}} \vspace{2mm}

\vspace{-3mm}
\markboth{\color{blue}\foreignlanguage{arabic}{ف.س.د}\color{blue}{}}{\color{blue}\foreignlanguage{arabic}{ف.س.د}\color{blue}{}}\subsection*{\color{blue}\foreignlanguage{arabic}{ف.س.د}\color{blue}{}\index{\color{blue}\foreignlanguage{arabic}{ف.س.د}\color{blue}{}}} 

{\setlength\topsep{0pt}\textbf{\foreignlanguage{arabic}{أَفْسَد}}\ {\color{gray}\texttt{/\sffamily {{\sffamily ʔafsad}}/}\color{black}}\ \textsc{verb}\ [p.]\ \textbf{1.}~spoil  \textbf{2.}~corrupt\ \ $\bullet$\ \ \setlength\topsep{0pt}\textbf{\foreignlanguage{arabic}{اِفْسِد}}\ {\color{gray}\texttt{/\sffamily {{\sffamily ʔifsid}}/}\color{black}}\ [c.]\ \ $\bullet$\ \ \setlength\topsep{0pt}\textbf{\foreignlanguage{arabic}{يِفْسِد}}\ {\color{gray}\texttt{/\sffamily {{\sffamily jifsid}}/}\color{black}}\ [i.]\  \begin{flushright}\color{gray}\foreignlanguage{arabic}{\textbf{\underline{\foreignlanguage{arabic}{أمثلة}}}: روح اِفْسِد عليهم اللحظات الحلوة وخليهم يتنكدوا}\end{flushright}\color{black}} \vspace{2mm}

{\setlength\topsep{0pt}\textbf{\foreignlanguage{arabic}{تَفْسِيد}}\ {\color{gray}\texttt{/\sffamily {{\sffamily tafsiːd}}/}\color{black}}\ \textsc{noun}\ [m.]\ \color{gray}(msa. \foreignlanguage{arabic}{إِفشاء الأسرار}~\foreignlanguage{arabic}{\textbf{١.}})\color{black}\ \textbf{1.}~divulging secrets\ } \vspace{2mm}

{\setlength\topsep{0pt}\textbf{\foreignlanguage{arabic}{فَاسِد}}\ {\color{gray}\texttt{/\sffamily {{\sffamily faːsid}}/}\color{black}}\ \textsc{adj}\ [m.]\ \color{gray}(msa. \foreignlanguage{arabic}{فاسِد}~\foreignlanguage{arabic}{\textbf{١.}})\color{black}\ \textbf{1.}~corrupt  \textbf{2.}~spoiled  \textbf{3.}~rotten\  \begin{flushright}\color{gray}\foreignlanguage{arabic}{\textbf{\underline{\foreignlanguage{arabic}{أمثلة}}}: نِضال فاسِد وانحبس بتهمة الفَساد}\end{flushright}\color{black}} \vspace{2mm}

{\setlength\topsep{0pt}\textbf{\foreignlanguage{arabic}{فَسَاد}}\ {\color{gray}\texttt{/\sffamily {{\sffamily fasaːd}}/}\color{black}}\ \textsc{noun}\ [m.]\ \color{gray}(msa. \foreignlanguage{arabic}{فَساد}~\foreignlanguage{arabic}{\textbf{١.}})\color{black}\ \textbf{1.}~curroption\  \begin{flushright}\color{gray}\foreignlanguage{arabic}{\textbf{\underline{\foreignlanguage{arabic}{أمثلة}}}: همي طحوه من شغله عشان كشفوا قضايا فَساد عليه}\end{flushright}\color{black}} \vspace{2mm}

{\setlength\topsep{0pt}\textbf{\foreignlanguage{arabic}{فَسَد}}\ {\color{gray}\texttt{/\sffamily {{\sffamily fasad}}/}\color{black}}\ \textsc{verb}\ [p.]\ \textbf{1.}~become corrupt.  \textbf{2.}~go off.  \textbf{3.}~spoil  \textbf{4.}~divulge\ \ $\bullet$\ \ \setlength\topsep{0pt}\textbf{\foreignlanguage{arabic}{اِفْسِد}}\ {\color{gray}\texttt{/\sffamily {{\sffamily ʔifsid}}/}\color{black}}\ [c.]\ \ $\bullet$\ \ \setlength\topsep{0pt}\textbf{\foreignlanguage{arabic}{يِفْسِد}}\ {\color{gray}\texttt{/\sffamily {{\sffamily jifsid}}/}\color{black}}\ [i.]\ \ $\bullet$\ \ \textsc{ph.} \color{gray} \foreignlanguage{arabic}{حليبهم فسد}\color{black}\ {\color{gray}\texttt{/{\sffamily ħaliːbhum fasad}/}\color{black}}\ \color{gray} (msa. \foreignlanguage{arabic}{عبارة تقال كناية عن عقوق الوالدين.}~\foreignlanguage{arabic}{\textbf{١.}})\color{black}\ \textbf{1.}~A metaphor for parental disobedience.\  \begin{flushright}\color{gray}\foreignlanguage{arabic}{\textbf{\underline{\foreignlanguage{arabic}{أمثلة}}}: هدول الشباب حليبهم فسد ومش مربيين\ $\bullet$\ \  خايفة أخلاقك تِفْسِد\ $\bullet$\ \  في حدا واطي فَسَد علي لأهلي}\end{flushright}\color{black}} \vspace{2mm}

{\setlength\topsep{0pt}\textbf{\foreignlanguage{arabic}{فَسَّاد}}\ {\color{gray}\texttt{/\sffamily {{\sffamily fassaːd}}/}\color{black}}\ \textsc{noun}\ [m.]\ \textbf{1.}~sb who divulges secrets\  \begin{flushright}\color{gray}\foreignlanguage{arabic}{\textbf{\underline{\foreignlanguage{arabic}{أمثلة}}}: أخوك فَسّاد وأنا بأمنلوش}\end{flushright}\color{black}} \vspace{2mm}

{\setlength\topsep{0pt}\textbf{\foreignlanguage{arabic}{فَسَّد}}\ {\color{gray}\texttt{/\sffamily {{\sffamily fassad}}/}\color{black}}\ \textsc{verb}\ [p.]\ \textbf{1.}~divulge\ \ $\bullet$\ \ \setlength\topsep{0pt}\textbf{\foreignlanguage{arabic}{فَسِّد}}\ {\color{gray}\texttt{/\sffamily {{\sffamily fassid}}/}\color{black}}\ [c.]\ \ $\bullet$\ \ \setlength\topsep{0pt}\textbf{\foreignlanguage{arabic}{يفَسِّد}}\ {\color{gray}\texttt{/\sffamily {{\sffamily jfassid}}/}\color{black}}\ [i.]\ \color{gray}(msa. \foreignlanguage{arabic}{يُفْشِي سِر}~\foreignlanguage{arabic}{\textbf{١.}})\color{black}\  \begin{flushright}\color{gray}\foreignlanguage{arabic}{\textbf{\underline{\foreignlanguage{arabic}{أمثلة}}}: أنو اللي حاول يفَسِّد أخلاقك؟\ $\bullet$\ \  في حدا فَسَّد علي الله لايوفقه}\end{flushright}\color{black}} \vspace{2mm}

{\setlength\topsep{0pt}\textbf{\foreignlanguage{arabic}{فُسَّيديِّة}}\ {\color{gray}\texttt{/\sffamily {{\sffamily fusseːdijje}}/}\color{black}}\ \textsc{noun}\ [f.]\ \textbf{1.}~a piece of news that has been divulged\  \begin{flushright}\color{gray}\foreignlanguage{arabic}{\textbf{\underline{\foreignlanguage{arabic}{أمثلة}}}: عندي فُسَّيديِّة صغيرة تعالي بسرعة}\end{flushright}\color{black}} \vspace{2mm}

\vspace{-3mm}
\markboth{\color{blue}\foreignlanguage{arabic}{ف.س.ر}\color{blue}{}}{\color{blue}\foreignlanguage{arabic}{ف.س.ر}\color{blue}{}}\subsection*{\color{blue}\foreignlanguage{arabic}{ف.س.ر}\color{blue}{}\index{\color{blue}\foreignlanguage{arabic}{ف.س.ر}\color{blue}{}}} 

{\setlength\topsep{0pt}\textbf{\foreignlanguage{arabic}{اِسْتَفْسَر}}\ {\color{gray}\texttt{/\sffamily {{\sffamily ʔistafsar}}/}\color{black}}\ \textsc{verb}\ [p.]\ \textbf{1.}~inquire about sth\ \ $\bullet$\ \ \setlength\topsep{0pt}\textbf{\foreignlanguage{arabic}{اِسْتَفْسِر}}\ {\color{gray}\texttt{/\sffamily {{\sffamily ʔistafsir}}/}\color{black}}\ [c.]\ \ $\bullet$\ \ \setlength\topsep{0pt}\textbf{\foreignlanguage{arabic}{يِسْتَفْسِر}}\ {\color{gray}\texttt{/\sffamily {{\sffamily jistafsir}}/}\color{black}}\ [i.]\ \color{gray}(msa. \foreignlanguage{arabic}{يَسْتَفْسِر}~\foreignlanguage{arabic}{\textbf{١.}})\color{black}\  \begin{flushright}\color{gray}\foreignlanguage{arabic}{\textbf{\underline{\foreignlanguage{arabic}{أمثلة}}}: بدي أسْتَفْسِر عن عرض الخمسة كيلو جاج ب100 شيكل. لسة عنكم؟}\end{flushright}\color{black}} \vspace{2mm}

{\setlength\topsep{0pt}\textbf{\foreignlanguage{arabic}{اِسْتِفْسَار}}\ {\color{gray}\texttt{/\sffamily {{\sffamily ʔistifsaːr}}/}\color{black}}\ \textsc{noun}\ [m.]\ \color{gray}(msa. \foreignlanguage{arabic}{اِسْتِفْسار}~\foreignlanguage{arabic}{\textbf{١.}})\color{black}\ \textbf{1.}~inquiry\  \begin{flushright}\color{gray}\foreignlanguage{arabic}{\textbf{\underline{\foreignlanguage{arabic}{أمثلة}}}: عندي اِسْتِفْسار بخصوص برنامج المهني. بقدر طالب راسب توجيهي إِنه يسجِّل عندكم؟}\end{flushright}\color{black}} \vspace{2mm}

{\setlength\topsep{0pt}\textbf{\foreignlanguage{arabic}{تَفْسِير}}\ {\color{gray}\texttt{/\sffamily {{\sffamily tafsiːr}}/}\color{black}}\ \textsc{noun}\ [m.]\ \color{gray}(msa. \foreignlanguage{arabic}{تفسير}~\foreignlanguage{arabic}{\textbf{١.}})\color{black}\ \textbf{1.}~interpretation\  \begin{flushright}\color{gray}\foreignlanguage{arabic}{\textbf{\underline{\foreignlanguage{arabic}{أمثلة}}}: دورت عسبب كراهيتها الي بس مالقيت أي تفسير}\end{flushright}\color{black}} \vspace{2mm}

{\setlength\topsep{0pt}\textbf{\foreignlanguage{arabic}{تْفَسَّر}}\ {\color{gray}\texttt{/\sffamily {{\sffamily tfassar}}/}\color{black}}\ \textsc{verb}\ [p.]\ \textbf{1.}~be interpreted\ \ $\bullet$\ \ \setlength\topsep{0pt}\textbf{\foreignlanguage{arabic}{اِتْفَسَّر}}\ {\color{gray}\texttt{/\sffamily {{\sffamily ʔitfassar}}/}\color{black}}\ [c.]\ \ $\bullet$\ \ \setlength\topsep{0pt}\textbf{\foreignlanguage{arabic}{يِتْفَسَّر}}\ {\color{gray}\texttt{/\sffamily {{\sffamily jitfassar}}/}\color{black}}\ [i.]\ \ $\bullet$\ \ \textsc{ph.} \color{gray} \foreignlanguage{arabic}{وجهه مَا بيتْفَسَّر}\color{black}\ {\color{gray}\texttt{/{\sffamily wi(dʒ)ho maː bjitfassar}/}\color{black}}\ \textbf{1.}~expressionless  \textbf{2.}~shocked\  \begin{flushright}\color{gray}\foreignlanguage{arabic}{\textbf{\underline{\foreignlanguage{arabic}{أمثلة}}}: هل ممكن هيك تصرف بدر مني إِنه يتْفَسَّر على إِنه وقاحة أو قلَّة إِحتِرام؟}\end{flushright}\color{black}} \vspace{2mm}

{\setlength\topsep{0pt}\textbf{\foreignlanguage{arabic}{فَسَّر}}\ {\color{gray}\texttt{/\sffamily {{\sffamily fassar}}/}\color{black}}\ \textsc{verb}\ [p.]\ \textbf{1.}~interpret  \textbf{2.}~explain\ \ $\bullet$\ \ \setlength\topsep{0pt}\textbf{\foreignlanguage{arabic}{فَسِّر}}\ {\color{gray}\texttt{/\sffamily {{\sffamily fassir}}/}\color{black}}\ [c.]\ \ $\bullet$\ \ \setlength\topsep{0pt}\textbf{\foreignlanguage{arabic}{يفَسِّر}}\ {\color{gray}\texttt{/\sffamily {{\sffamily jfassir}}/}\color{black}}\ [i.]\ \color{gray}(msa. \foreignlanguage{arabic}{يُفَسِّر}~\foreignlanguage{arabic}{\textbf{١.}})\color{black}\  \begin{flushright}\color{gray}\foreignlanguage{arabic}{\textbf{\underline{\foreignlanguage{arabic}{أمثلة}}}: أنت أدرى بزياد وإِخوته. كل واحد فيهم بيفَسِّر الموضوع عهواه}\end{flushright}\color{black}} \vspace{2mm}

\vspace{-3mm}
\markboth{\color{blue}\foreignlanguage{arabic}{ف.س.س}\color{blue}{}}{\color{blue}\foreignlanguage{arabic}{ف.س.س}\color{blue}{}}\subsection*{\color{blue}\foreignlanguage{arabic}{ف.س.س}\color{blue}{}\index{\color{blue}\foreignlanguage{arabic}{ف.س.س}\color{blue}{}}} 

{\setlength\topsep{0pt}\textbf{\foreignlanguage{arabic}{فِسِّة}}\ {\color{gray}\texttt{/\sffamily {{\sffamily fisse}}/}\color{black}}\ \textsc{noun}\ [f.]\ \textbf{1.}~see phrase\ \ $\bullet$\ \ \textsc{ph.} \color{gray} \foreignlanguage{arabic}{قَدّ الفِسِّة}\color{black}\ {\color{gray}\texttt{/{\sffamily ʔadd ʔilfisse}/}\color{black}}\ \textbf{1.}~very small\  \begin{flushright}\color{gray}\foreignlanguage{arabic}{\textbf{\underline{\foreignlanguage{arabic}{أمثلة}}}: خطها قد الفِسِّة\ $\bullet$\ \  خطها قد الفِسِّة}\end{flushright}\color{black}} \vspace{2mm}

{\setlength\topsep{0pt}\textbf{\foreignlanguage{arabic}{فْسَيسِي}}\ {\color{gray}\texttt{/\sffamily {{\sffamily fseːsi}}/}\color{black}}\ \textsc{noun}\ [m.]\ \color{gray}(msa. \foreignlanguage{arabic}{طائر الدوري}~\foreignlanguage{arabic}{\textbf{١.}})\color{black}\ \textbf{1.}~sparrow\  \begin{flushright}\color{gray}\foreignlanguage{arabic}{\textbf{\underline{\foreignlanguage{arabic}{أمثلة}}}: ما أقواه كيق قدر يمسك الفْسِيسِي؟}\end{flushright}\color{black}} \vspace{2mm}

\vspace{-3mm}
\markboth{\color{blue}\foreignlanguage{arabic}{ف.س.ط}\color{blue}{ (ntws)}}{\color{blue}\foreignlanguage{arabic}{ف.س.ط}\color{blue}{ (ntws)}}\subsection*{\color{blue}\foreignlanguage{arabic}{ف.س.ط}\color{blue}{ (ntws)}\index{\color{blue}\foreignlanguage{arabic}{ف.س.ط}\color{blue}{ (ntws)}}} 

{\setlength\topsep{0pt}\textbf{\foreignlanguage{arabic}{فَوسْطَة}}\ {\color{gray}\texttt{/\sffamily {{\sffamily foːsˤtˤa}}/}\color{black}}\ \textsc{noun}\ [f.]\ (src. \color{gray}\foreignlanguage{arabic}{القدس}\color{black})\ \color{gray}(msa. \foreignlanguage{arabic}{ثوب}~\foreignlanguage{arabic}{\textbf{١.}})\color{black}\ \textbf{1.}~dress\  \begin{flushright}\color{gray}\foreignlanguage{arabic}{\textbf{\underline{\foreignlanguage{arabic}{أمثلة}}}: لابسة هالفوسْطَة الشلبي}\end{flushright}\color{black}} \vspace{2mm}

\vspace{-3mm}
\markboth{\color{blue}\foreignlanguage{arabic}{ف.س.ف.س}\color{blue}{}}{\color{blue}\foreignlanguage{arabic}{ف.س.ف.س}\color{blue}{}}\subsection*{\color{blue}\foreignlanguage{arabic}{ف.س.ف.س}\color{blue}{}\index{\color{blue}\foreignlanguage{arabic}{ف.س.ف.س}\color{blue}{}}} 

{\setlength\topsep{0pt}\textbf{\foreignlanguage{arabic}{فَسْفَس}}\ {\color{gray}\texttt{/\sffamily {{\sffamily fasfas}}/}\color{black}}\ \textsc{verb}\ [p.]\ \textbf{1.}~gossip  \textbf{2.}~divulge a secret.  \textbf{3.}~sew a sedition\ \ $\bullet$\ \ \setlength\topsep{0pt}\textbf{\foreignlanguage{arabic}{فَسْفِس}}\ {\color{gray}\texttt{/\sffamily {{\sffamily fasfis}}/}\color{black}}\ [c.]\ \ $\bullet$\ \ \setlength\topsep{0pt}\textbf{\foreignlanguage{arabic}{يفَسْفِس}}\ {\color{gray}\texttt{/\sffamily {{\sffamily jfasfis}}/}\color{black}}\ [i.]\  \begin{flushright}\color{gray}\foreignlanguage{arabic}{\textbf{\underline{\foreignlanguage{arabic}{أمثلة}}}: أنا متأكِّد إِنه في حدا كا بيفَسْفِس له كل شي بغيابنا}\end{flushright}\color{black}} \vspace{2mm}

{\setlength\topsep{0pt}\textbf{\foreignlanguage{arabic}{فَسْفَسِة}}\ {\color{gray}\texttt{/\sffamily {{\sffamily fasfase}}/}\color{black}}\ \textsc{noun}\ [f.]\ \textbf{1.}~gossip  \textbf{2.}~divulge a secret.  \textbf{3.}~sew a sedition\ } \vspace{2mm}

\vspace{-3mm}
\markboth{\color{blue}\foreignlanguage{arabic}{ف.س.ق}\color{blue}{}}{\color{blue}\foreignlanguage{arabic}{ف.س.ق}\color{blue}{}}\subsection*{\color{blue}\foreignlanguage{arabic}{ف.س.ق}\color{blue}{}\index{\color{blue}\foreignlanguage{arabic}{ف.س.ق}\color{blue}{}}} 

{\setlength\topsep{0pt}\textbf{\foreignlanguage{arabic}{فَاسِق}}\ {\color{gray}\texttt{/\sffamily {{\sffamily faːsiq}}/}\color{black}}\ \textsc{adj}\ [m.]\ \color{gray}(msa. \foreignlanguage{arabic}{فاجِر}~\foreignlanguage{arabic}{\textbf{٢.}}  \foreignlanguage{arabic}{فاسِق}~\foreignlanguage{arabic}{\textbf{١.}})\color{black}\ \textbf{1.}~licentious and depraved\ \ $\bullet$\ \ \setlength\topsep{0pt}\textbf{\foreignlanguage{arabic}{فَسَقَة}}\ {\color{gray}\texttt{/\sffamily {{\sffamily fasaqa}}/}\color{black}}\ [pl.]\  \begin{flushright}\color{gray}\foreignlanguage{arabic}{\textbf{\underline{\foreignlanguage{arabic}{أمثلة}}}: شو بده منك هذا الفاسِق؟}\end{flushright}\color{black}} \vspace{2mm}

{\setlength\topsep{0pt}\textbf{\foreignlanguage{arabic}{فَسَق}}\ {\color{gray}\texttt{/\sffamily {{\sffamily fasaq}}/}\color{black}}\ \textsc{verb}\ [p.]\ \textbf{1.}~become licentious.  \textbf{2.}~act licentiously\ \ $\bullet$\ \ \setlength\topsep{0pt}\textbf{\foreignlanguage{arabic}{اُفْسُق}}\ {\color{gray}\texttt{/\sffamily {{\sffamily ʔufsuq}}/}\color{black}}\ [c.]\ \ $\bullet$\ \ \setlength\topsep{0pt}\textbf{\foreignlanguage{arabic}{اِفْسُق}}\ {\color{gray}\texttt{/\sffamily {{\sffamily ʔifsuq}}/}\color{black}}\ [c.]\ \ $\bullet$\ \ \setlength\topsep{0pt}\textbf{\foreignlanguage{arabic}{اِفْسِق}}\ {\color{gray}\texttt{/\sffamily {{\sffamily ʔifsiq}}/}\color{black}}\ [c.]\ \ $\bullet$\ \ \setlength\topsep{0pt}\textbf{\foreignlanguage{arabic}{يِفْسُق}}\ {\color{gray}\texttt{/\sffamily {{\sffamily jifsuq}}/}\color{black}}\ [i.]\ \ $\bullet$\ \ \setlength\topsep{0pt}\textbf{\foreignlanguage{arabic}{يُفْسُق}}\ {\color{gray}\texttt{/\sffamily {{\sffamily jufsuq}}/}\color{black}}\ [i.]\ \ $\bullet$\ \ \setlength\topsep{0pt}\textbf{\foreignlanguage{arabic}{يِفْسِق}}\ {\color{gray}\texttt{/\sffamily {{\sffamily jifsiq}}/}\color{black}}\ [i.]\  \begin{flushright}\color{gray}\foreignlanguage{arabic}{\textbf{\underline{\foreignlanguage{arabic}{أمثلة}}}: أبوه خاف عليه يهود عغربا ويُفْسُق هناك مع البنات والهمل\ $\bullet$\ \  اُفْسُق براحتك ماحدا داري عنك}\end{flushright}\color{black}} \vspace{2mm}

{\setlength\topsep{0pt}\textbf{\foreignlanguage{arabic}{فَسَّق}}\ {\color{gray}\texttt{/\sffamily {{\sffamily fassaq}}/}\color{black}}\ \textsc{verb}\ [p.]\ \textbf{1.}~cause sb to be licentious.  \textbf{2.}~make sb licentious and depraved\ \ $\bullet$\ \ \setlength\topsep{0pt}\textbf{\foreignlanguage{arabic}{فَسِّق}}\ {\color{gray}\texttt{/\sffamily {{\sffamily fassiq}}/}\color{black}}\ [c.]\ \ $\bullet$\ \ \setlength\topsep{0pt}\textbf{\foreignlanguage{arabic}{يفَسِّق}}\ {\color{gray}\texttt{/\sffamily {{\sffamily jfassiq}}/}\color{black}}\ [i.]\  \begin{flushright}\color{gray}\foreignlanguage{arabic}{\textbf{\underline{\foreignlanguage{arabic}{أمثلة}}}: أفهم من كلامك انه بنتك العقيقة الطاهرة وأنا اللي فَسَّقتها وخربت أخلاقها؟}\end{flushright}\color{black}} \vspace{2mm}

{\setlength\topsep{0pt}\textbf{\foreignlanguage{arabic}{فُسُوق}}\ {\color{gray}\texttt{/\sffamily {{\sffamily fusuːq}}/}\color{black}}\ \textsc{noun}\ [m.]\ \color{gray}(msa. \foreignlanguage{arabic}{فُسوق}~\foreignlanguage{arabic}{\textbf{٢.}}  \foreignlanguage{arabic}{فِسْق}~\foreignlanguage{arabic}{\textbf{١.}})\color{black}\ \textbf{1.}~debauchery  \textbf{2.}~licentiousness\  \begin{flushright}\color{gray}\foreignlanguage{arabic}{\textbf{\underline{\foreignlanguage{arabic}{أمثلة}}}: الفُسوق اللي شفته بتركيا والله بيخوف. الله يستر ماتنخسف فينا الأرض بسس هذول الفاسقين}\end{flushright}\color{black}} \vspace{2mm}

{\setlength\topsep{0pt}\textbf{\foreignlanguage{arabic}{فِسِق}}\ {\color{gray}\texttt{/\sffamily {{\sffamily fisiq}}/}\color{black}}\ \textsc{noun}\ [m.]\ \color{gray}(msa. \foreignlanguage{arabic}{فُسوق}~\foreignlanguage{arabic}{\textbf{٢.}}  \foreignlanguage{arabic}{فِسْق}~\foreignlanguage{arabic}{\textbf{١.}})\color{black}\ \textbf{1.}~debauchery  \textbf{2.}~licentiousness\  \begin{flushright}\color{gray}\foreignlanguage{arabic}{\textbf{\underline{\foreignlanguage{arabic}{أمثلة}}}: لليش رايح عالحفلة؟ كلها فِسِق وفجور والعياذ بالله}\end{flushright}\color{black}} \vspace{2mm}

\vspace{-3mm}
\markboth{\color{blue}\foreignlanguage{arabic}{ف.س.ق.ل}\color{blue}{}}{\color{blue}\foreignlanguage{arabic}{ف.س.ق.ل}\color{blue}{}}\subsection*{\color{blue}\foreignlanguage{arabic}{ف.س.ق.ل}\color{blue}{}\index{\color{blue}\foreignlanguage{arabic}{ف.س.ق.ل}\color{blue}{}}} 

{\setlength\topsep{0pt}\textbf{\foreignlanguage{arabic}{فَسْقَل}}\ {\color{gray}\texttt{/\sffamily {{\sffamily fasqal}}/}\color{black}}\ \textsc{verb}\ [p.]\ \textbf{1.}~tear sth apart with force.  \textbf{2.}~rip sth with force\ \ $\bullet$\ \ \setlength\topsep{0pt}\textbf{\foreignlanguage{arabic}{فَسْقِل}}\ {\color{gray}\texttt{/\sffamily {{\sffamily fasqil}}/}\color{black}}\ [c.]\ \ $\bullet$\ \ \setlength\topsep{0pt}\textbf{\foreignlanguage{arabic}{يفَسْقِل}}\ {\color{gray}\texttt{/\sffamily {{\sffamily jfasqil}}/}\color{black}}\ [i.]\  \begin{flushright}\color{gray}\foreignlanguage{arabic}{\textbf{\underline{\foreignlanguage{arabic}{أمثلة}}}: المجمرم مسك دفتر صاحبتي باله دفتري. إِجى فَسْقَله هيك شقفتين.}\end{flushright}\color{black}} \vspace{2mm}

{\setlength\topsep{0pt}\textbf{\foreignlanguage{arabic}{فَسْقَلِة}}\ {\color{gray}\texttt{/\sffamily {{\sffamily fasqale}}/}\color{black}}\ \textsc{noun}\ [f.]\ \textbf{1.}~tearing sth apart with force.  \textbf{2.}~ripping sth with force\ } \vspace{2mm}

{\setlength\topsep{0pt}\textbf{\foreignlanguage{arabic}{مْفَسْقَل}}\ {\color{gray}\texttt{/\sffamily {{\sffamily mfasqal}}/}\color{black}}\ \textsc{adj}\ [m.]\ \textbf{1.}~torn off.  \textbf{2.}~ripped off\  \begin{flushright}\color{gray}\foreignlanguage{arabic}{\textbf{\underline{\foreignlanguage{arabic}{أمثلة}}}: ليش معطيني الشادر مْفَسْقَل هيك؟}\end{flushright}\color{black}} \vspace{2mm}

\vspace{-3mm}
\markboth{\color{blue}\foreignlanguage{arabic}{ف.س.ي}\color{blue}{}}{\color{blue}\foreignlanguage{arabic}{ف.س.ي}\color{blue}{}}\subsection*{\color{blue}\foreignlanguage{arabic}{ف.س.ي}\color{blue}{}\index{\color{blue}\foreignlanguage{arabic}{ف.س.ي}\color{blue}{}}} 

{\setlength\topsep{0pt}\textbf{\foreignlanguage{arabic}{فَاسِي}}\ {\color{gray}\texttt{/\sffamily {{\sffamily faːsi}}/}\color{black}}\ \textsc{noun\textunderscore act}\ [m.]\ \textbf{1.}~breaking wind\  \begin{flushright}\color{gray}\foreignlanguage{arabic}{\textbf{\underline{\foreignlanguage{arabic}{أمثلة}}}: في حدا فاسِي الله يقرفكم}\end{flushright}\color{black}} \vspace{2mm}

{\setlength\topsep{0pt}\textbf{\foreignlanguage{arabic}{فَسو}}\ {\color{gray}\texttt{/\sffamily {{\sffamily fasu}}/}\color{black}}\ \textsc{noun}\ [m.]\ \textbf{1.}~breaking wind\  \begin{flushright}\color{gray}\foreignlanguage{arabic}{\textbf{\underline{\foreignlanguage{arabic}{أمثلة}}}: حدا شامِم ريحة فَسو بالغرفة؟}\end{flushright}\color{black}} \vspace{2mm}

{\setlength\topsep{0pt}\textbf{\foreignlanguage{arabic}{فَسَى}}\ {\color{gray}\texttt{/\sffamily {{\sffamily fasa}}/}\color{black}}\ \textsc{verb}\ [p.]\ \textbf{1.}~break wind\ \ $\bullet$\ \ \setlength\topsep{0pt}\textbf{\foreignlanguage{arabic}{اِفْسِي}}\ {\color{gray}\texttt{/\sffamily {{\sffamily ʔifsi}}/}\color{black}}\ [c.]\ \ $\bullet$\ \ \setlength\topsep{0pt}\textbf{\foreignlanguage{arabic}{يِفْسِي}}\ {\color{gray}\texttt{/\sffamily {{\sffamily jifsi}}/}\color{black}}\ [i.]\ \color{gray}(msa. \foreignlanguage{arabic}{يُطْلِق ريح}~\foreignlanguage{arabic}{\textbf{١.}})\color{black}\ } \vspace{2mm}

{\setlength\topsep{0pt}\textbf{\foreignlanguage{arabic}{فَسَّى}}\ {\color{gray}\texttt{/\sffamily {{\sffamily fassa}}/}\color{black}}\ \textsc{verb}\ [p.]\ \textbf{1.}~break wind\ \ $\bullet$\ \ \setlength\topsep{0pt}\textbf{\foreignlanguage{arabic}{فَسِّي}}\ {\color{gray}\texttt{/\sffamily {{\sffamily fassi}}/}\color{black}}\ [c.]\ \ $\bullet$\ \ \setlength\topsep{0pt}\textbf{\foreignlanguage{arabic}{يفَسِّي}}\ {\color{gray}\texttt{/\sffamily {{\sffamily jfassi}}/}\color{black}}\ [i.]\ \color{gray}(msa. \foreignlanguage{arabic}{يُطْلِق ريح}~\foreignlanguage{arabic}{\textbf{١.}})\color{black}\ } \vspace{2mm}

\vspace{-3mm}
\markboth{\color{blue}\foreignlanguage{arabic}{ف.ش.خ}\color{blue}{}}{\color{blue}\foreignlanguage{arabic}{ف.ش.خ}\color{blue}{}}\subsection*{\color{blue}\foreignlanguage{arabic}{ف.ش.خ}\color{blue}{}\index{\color{blue}\foreignlanguage{arabic}{ف.ش.خ}\color{blue}{}}} 

{\setlength\topsep{0pt}\textbf{\foreignlanguage{arabic}{اِنْفَشَخ}}\ {\color{gray}\texttt{/\sffamily {{\sffamily ʔinfaʃax}}/}\color{black}}\ \textsc{verb}\ [p.]\ \textbf{1.}~be spread.  \textbf{2.}~be opened widely.  \textbf{3.}~be torn off.  \textbf{4.}~be stoned and injured\ \ $\bullet$\ \ \setlength\topsep{0pt}\textbf{\foreignlanguage{arabic}{اِنْفِشِخ}}\ {\color{gray}\texttt{/\sffamily {{\sffamily ʔinfiʃix}}/}\color{black}}\ [c.]\ \ $\bullet$\ \ \setlength\topsep{0pt}\textbf{\foreignlanguage{arabic}{يِنْفِشِخ}}\ {\color{gray}\texttt{/\sffamily {{\sffamily jinfiʃix}}/}\color{black}}\ [i.]\  \begin{flushright}\color{gray}\foreignlanguage{arabic}{\textbf{\underline{\foreignlanguage{arabic}{أمثلة}}}: دير بالك ما يِنْفِشِخ راسك}\end{flushright}\color{black}} \vspace{2mm}

{\setlength\topsep{0pt}\textbf{\foreignlanguage{arabic}{فَشَخ}}\ {\color{gray}\texttt{/\sffamily {{\sffamily faʃax}}/}\color{black}}\ \textsc{verb}\ [p.]\ \textbf{1.}~spread sth open.  \textbf{2.}~open sth widely.  \textbf{3.}~tear sth off.  \textbf{4.}~stone sb and cause him a bad injury\ \ $\bullet$\ \ \setlength\topsep{0pt}\textbf{\foreignlanguage{arabic}{اِفْشَخ}}\ {\color{gray}\texttt{/\sffamily {{\sffamily ʔifʃax}}/}\color{black}}\ [c.]\ \ $\bullet$\ \ \setlength\topsep{0pt}\textbf{\foreignlanguage{arabic}{يِفْشَخ}}\ {\color{gray}\texttt{/\sffamily {{\sffamily jifʃax}}/}\color{black}}\ [i.]\ \color{gray}(msa. \foreignlanguage{arabic}{يرجم شخص بالحجار ويسبب له أذى}~\foreignlanguage{arabic}{\textbf{٢.}}  .\foreignlanguage{arabic}{يفتح شيء بشكل واسع}~\foreignlanguage{arabic}{\textbf{١.}})\color{black}\  \begin{flushright}\color{gray}\foreignlanguage{arabic}{\textbf{\underline{\foreignlanguage{arabic}{أمثلة}}}: اِفْشَخ اجريك خليني أشوف وين مفروط\ $\bullet$\ \  فَشَخ راس أخوه بالحجر}\end{flushright}\color{black}} \vspace{2mm}

{\setlength\topsep{0pt}\textbf{\foreignlanguage{arabic}{فَشْخَة}}\ {\color{gray}\texttt{/\sffamily {{\sffamily faʃxa}}/}\color{black}}\ \textsc{noun}\ [f.]\ \color{gray}(msa. \foreignlanguage{arabic}{خُطْوَة}~\foreignlanguage{arabic}{\textbf{١.}})\color{black}\ \textbf{1.}~step\  \begin{flushright}\color{gray}\foreignlanguage{arabic}{\textbf{\underline{\foreignlanguage{arabic}{أمثلة}}}: البيت فَشْخَتين من هون}\end{flushright}\color{black}} \vspace{2mm}

{\setlength\topsep{0pt}\textbf{\foreignlanguage{arabic}{مَفْشُوخ}}\ {\color{gray}\texttt{/\sffamily {{\sffamily mafʃuːx}}/}\color{black}}\ \textsc{noun\textunderscore pass}\ \textbf{1.}~be hurt.  \textbf{2.}~be torn off.  \textbf{3.}~be opened widely\  \begin{flushright}\color{gray}\foreignlanguage{arabic}{\textbf{\underline{\foreignlanguage{arabic}{أمثلة}}}: راسي مَفْشوخ الله لايوفقهم}\end{flushright}\color{black}} \vspace{2mm}

\vspace{-3mm}
\markboth{\color{blue}\foreignlanguage{arabic}{ف.ش.خ.ر}\color{blue}{}}{\color{blue}\foreignlanguage{arabic}{ف.ش.خ.ر}\color{blue}{}}\subsection*{\color{blue}\foreignlanguage{arabic}{ف.ش.خ.ر}\color{blue}{}\index{\color{blue}\foreignlanguage{arabic}{ف.ش.خ.ر}\color{blue}{}}} 

{\setlength\topsep{0pt}\textbf{\foreignlanguage{arabic}{تْفَشْخَر}}\ {\color{gray}\texttt{/\sffamily {{\sffamily tfaʃxar}}/}\color{black}}\ \textsc{verb}\ [p.]\ \textbf{1.}~boast  \textbf{2.}~show off\ \ $\bullet$\ \ \setlength\topsep{0pt}\textbf{\foreignlanguage{arabic}{اِتْفَشْخَر}}\ {\color{gray}\texttt{/\sffamily {{\sffamily ʔitfaʃxar}}/}\color{black}}\ [c.]\ \ $\bullet$\ \ \setlength\topsep{0pt}\textbf{\foreignlanguage{arabic}{يِتْفَشْخَر}}\ {\color{gray}\texttt{/\sffamily {{\sffamily jitfaʃxar}}/}\color{black}}\ [i.]\ \color{gray}(msa. \foreignlanguage{arabic}{يَتَباهَى}~\foreignlanguage{arabic}{\textbf{١.}})\color{black}\  \begin{flushright}\color{gray}\foreignlanguage{arabic}{\textbf{\underline{\foreignlanguage{arabic}{أمثلة}}}: إِجى بده يِتْفَشْخَر علي قمت فنسته}\end{flushright}\color{black}} \vspace{2mm}

{\setlength\topsep{0pt}\textbf{\foreignlanguage{arabic}{فَشْخَرَة}}\ {\color{gray}\texttt{/\sffamily {{\sffamily faʃxara}}/}\color{black}}\ \textsc{noun}\ [f.]\ \textbf{1.}~boasting  \textbf{2.}~showing off\  \begin{flushright}\color{gray}\foreignlanguage{arabic}{\textbf{\underline{\foreignlanguage{arabic}{أمثلة}}}: بموتوا عالفَشْخَرة وشوفة الحال}\end{flushright}\color{black}} \vspace{2mm}

{\setlength\topsep{0pt}\textbf{\foreignlanguage{arabic}{فَشْخَرْجِي}}\ {\color{gray}\texttt{/\sffamily {{\sffamily faʃxar(dʒ)i}}/}\color{black}}\ \textsc{adj}\ [m.]\ \textbf{1.}~boastful  \textbf{2.}~ostentatious\ } \vspace{2mm}

{\setlength\topsep{0pt}\textbf{\foreignlanguage{arabic}{مْفَشْخَر}}\ {\color{gray}\texttt{/\sffamily {{\sffamily mfaʃxar}}/}\color{black}}\ \textsc{adj}\ [m.]\ \textbf{1.}~boastful  \textbf{2.}~ostentatious\  \begin{flushright}\color{gray}\foreignlanguage{arabic}{\textbf{\underline{\foreignlanguage{arabic}{أمثلة}}}: أكره ماعلي الزلمة المْفَشْخَر}\end{flushright}\color{black}} \vspace{2mm}

\vspace{-3mm}
\markboth{\color{blue}\foreignlanguage{arabic}{ف.ش.ر}\color{blue}{}}{\color{blue}\foreignlanguage{arabic}{ف.ش.ر}\color{blue}{}}\subsection*{\color{blue}\foreignlanguage{arabic}{ف.ش.ر}\color{blue}{}\index{\color{blue}\foreignlanguage{arabic}{ف.ش.ر}\color{blue}{}}} 

{\setlength\topsep{0pt}\textbf{\foreignlanguage{arabic}{فَشَر}}\ {\color{gray}\texttt{/\sffamily {{\sffamily faʃar}}/}\color{black}}\ \textsc{interj}\ \textbf{1.}~nonsense!\  \begin{flushright}\color{gray}\foreignlanguage{arabic}{\textbf{\underline{\foreignlanguage{arabic}{أمثلة}}}: فَشَر! والله ما أخليه ياخذها على جثتي}\end{flushright}\color{black}} \vspace{2mm}

{\setlength\topsep{0pt}\textbf{\foreignlanguage{arabic}{فَشِر}}\ {\color{gray}\texttt{/\sffamily {{\sffamily faʃir}}/}\color{black}}\ \textsc{noun}\ [m.]\ \color{gray}(msa. \foreignlanguage{arabic}{المُباهاة}~\foreignlanguage{arabic}{\textbf{٢.}}  \foreignlanguage{arabic}{الكذب}~\foreignlanguage{arabic}{\textbf{١.}})\color{black}\ \textbf{1.}~lying  \textbf{2.}~bragging\ } \vspace{2mm}

{\setlength\topsep{0pt}\textbf{\foreignlanguage{arabic}{فَشَّار}}\ {\color{gray}\texttt{/\sffamily {{\sffamily faʃʃaːr}}/}\color{black}}\ \textsc{adj}\ [m.]\ \color{gray}(msa. \foreignlanguage{arabic}{كَذّاب}~\foreignlanguage{arabic}{\textbf{٢.}}  \foreignlanguage{arabic}{مُباهِي}~\foreignlanguage{arabic}{\textbf{١.}})\color{black}\ \textbf{1.}~braggart  \textbf{2.}~liar\  \begin{flushright}\color{gray}\foreignlanguage{arabic}{\textbf{\underline{\foreignlanguage{arabic}{أمثلة}}}: أنت واحد فَشّار وهذا وجهي إِذا بسمع كلامك مرة ثانية}\end{flushright}\color{black}} \vspace{2mm}

{\setlength\topsep{0pt}\textbf{\foreignlanguage{arabic}{فَشَّر}}\ {\color{gray}\texttt{/\sffamily {{\sffamily faʃʃar}}/}\color{black}}\ \textsc{verb}\ [p.]\ \textbf{1.}~lie  \textbf{2.}~brag\ \ $\bullet$\ \ \setlength\topsep{0pt}\textbf{\foreignlanguage{arabic}{فَشِّر}}\ {\color{gray}\texttt{/\sffamily {{\sffamily faʃʃir}}/}\color{black}}\ [c.]\ \ $\bullet$\ \ \setlength\topsep{0pt}\textbf{\foreignlanguage{arabic}{يفَشِّر}}\ {\color{gray}\texttt{/\sffamily {{\sffamily jfaʃʃir}}/}\color{black}}\ [i.]\ \color{gray}(msa. \foreignlanguage{arabic}{يُباهِي}~\foreignlanguage{arabic}{\textbf{٢.}}  \foreignlanguage{arabic}{يكذِب}~\foreignlanguage{arabic}{\textbf{١.}})\color{black}\  \begin{flushright}\color{gray}\foreignlanguage{arabic}{\textbf{\underline{\foreignlanguage{arabic}{أمثلة}}}: إِجى يفَشِّر علينا بالورثة والسفر}\end{flushright}\color{black}} \vspace{2mm}

\vspace{-3mm}
\markboth{\color{blue}\foreignlanguage{arabic}{ف.ش.ش}\color{blue}{}}{\color{blue}\foreignlanguage{arabic}{ف.ش.ش}\color{blue}{}}\subsection*{\color{blue}\foreignlanguage{arabic}{ف.ش.ش}\color{blue}{}\index{\color{blue}\foreignlanguage{arabic}{ف.ش.ش}\color{blue}{}}} 

{\setlength\topsep{0pt}\textbf{\foreignlanguage{arabic}{تْفَشَّش}}\ {\color{gray}\texttt{/\sffamily {{\sffamily tfaʃʃaʃ}}/}\color{black}}\ \textsc{verb}\ [p.]\ \textbf{1.}~to fight/quarrel with sb\ \ $\bullet$\ \ \setlength\topsep{0pt}\textbf{\foreignlanguage{arabic}{اِتْفَشَّش}}\ {\color{gray}\texttt{/\sffamily {{\sffamily ʔitfaʃʃaʃ}}/}\color{black}}\ [c.]\ \textbf{1.}~sb tries to kill time\ \ $\bullet$\ \ \setlength\topsep{0pt}\textbf{\foreignlanguage{arabic}{يِتْفَشَّش}}\ {\color{gray}\texttt{/\sffamily {{\sffamily jitfaʃʃaʃ}}/}\color{black}}\ [i.]\ \color{gray}(msa. \foreignlanguage{arabic}{يتشاجر مع شخص}~\foreignlanguage{arabic}{\textbf{١.}})\color{black}\  \begin{flushright}\color{gray}\foreignlanguage{arabic}{\textbf{\underline{\foreignlanguage{arabic}{أمثلة}}}: أربع وعشرين ساعة بتْفَشَّشوا ببعض مثل الكلاب الصعرانة\ $\bullet$\ \  سعيد زَمْقان و بِتْْفَشَّش بأي شي}\end{flushright}\color{black}} \vspace{2mm}

{\setlength\topsep{0pt}\textbf{\foreignlanguage{arabic}{فَاشُوش}}\ {\color{gray}\texttt{/\sffamily {{\sffamily faːʃuːʃ}}/}\color{black}}\ \textsc{noun}\ [m.]\ \textbf{1.}~nothingness\ \ $\bullet$\ \ \textsc{ph.} \color{gray} \foreignlanguage{arabic}{على فَاشُوش}\color{black}\ {\color{gray}\texttt{/{\sffamily ʕala faːʃuːʃ}/}\color{black}}\ \textbf{1.}~for no good reason.  \textbf{2.}~for no reason\  \begin{flushright}\color{gray}\foreignlanguage{arabic}{\textbf{\underline{\foreignlanguage{arabic}{أمثلة}}}: والله بتاكلش شي المرة بس ياحرام كل مالها رايحة بالعرض. بتنصح على فاشوش!}\end{flushright}\color{black}} \vspace{2mm}

{\setlength\topsep{0pt}\textbf{\foreignlanguage{arabic}{فَشّ}}\ {\color{gray}\texttt{/\sffamily {{\sffamily faʃʃ}}/}\color{black}}\ \textsc{verb}\ [p.]\ \textbf{1.}~release (negative feelings e.g. anger or sadness)\ \ $\bullet$\ \ \setlength\topsep{0pt}\textbf{\foreignlanguage{arabic}{فِشّ}}\ {\color{gray}\texttt{/\sffamily {{\sffamily fiʃʃ}}/}\color{black}}\ [c.]\ \ $\bullet$\ \ \setlength\topsep{0pt}\textbf{\foreignlanguage{arabic}{يفِشّ}}\ {\color{gray}\texttt{/\sffamily {{\sffamily jfiʃʃ}}/}\color{black}}\ [i.]\ \ $\bullet$\ \ \textsc{ph.} \color{gray} \foreignlanguage{arabic}{بفِشِّش العِلِّة}\color{black}\ {\color{gray}\texttt{/{\sffamily bifiʃʃiʃ ʔilʕille}/}\color{black}}\ \color{gray} (msa. \foreignlanguage{arabic}{بليد}~\foreignlanguage{arabic}{\textbf{١.}})\color{black}\ \textbf{1.}~sluggish\  \begin{flushright}\color{gray}\foreignlanguage{arabic}{\textbf{\underline{\foreignlanguage{arabic}{أمثلة}}}: ابنها الكبير بِفشِّش العِلَّة  بتبعثه عالمكان برجع وايديه فاضية\ $\bullet$\ \  يا حبيبتي مش مشكلة فِشِّي كل اللي بقلبك}\end{flushright}\color{black}} \vspace{2mm}

{\setlength\topsep{0pt}\textbf{\foreignlanguage{arabic}{فَشِّة}}\ {\color{gray}\texttt{/\sffamily {{\sffamily faʃʃe}}/}\color{black}}\ \textsc{noun}\ [f.]\ \textbf{1.}~release (negative feelings e.g. anger or sadness)\ \ $\bullet$\ \ \textsc{ph.} \color{gray} \foreignlanguage{arabic}{فشة خلق}\color{black}\ {\color{gray}\texttt{/{\sffamily faʃʃit xulu(q)}/}\color{black}}\ \textbf{1.}~pouring out your soul / heart\  \begin{flushright}\color{gray}\foreignlanguage{arabic}{\textbf{\underline{\foreignlanguage{arabic}{أمثلة}}}: الموضوع كله فشِّة خُلُق ما تقلقل رح أصير منيحة بعد ما يطلعوا الجماعة\ $\bullet$\ \  كنت معصبة وأنت طلعت بوجهي فطلعت كل الفَشِّة فيه}\end{flushright}\color{black}} \vspace{2mm}

{\setlength\topsep{0pt}\textbf{\foreignlanguage{arabic}{فِشِّة}}\ {\color{gray}\texttt{/\sffamily {{\sffamily fiʃʃe}}/}\color{black}}\ \textsc{noun}\ [f.]\ \color{gray}(msa. \foreignlanguage{arabic}{رئتين الخروف}~\foreignlanguage{arabic}{\textbf{١.}})\color{black}\ \textbf{1.}~sheep lungs\ \ $\bullet$\ \ \setlength\topsep{0pt}\textbf{\foreignlanguage{arabic}{فِشَش}}\ {\color{gray}\texttt{/\sffamily {{\sffamily fiʃaʃ}}/}\color{black}}\ [pl.]\  \begin{flushright}\color{gray}\foreignlanguage{arabic}{\textbf{\underline{\foreignlanguage{arabic}{أمثلة}}}: حدا بتعشى فِشِّة الساعة 1 بالليل؟}\end{flushright}\color{black}} \vspace{2mm}

\vspace{-3mm}
\markboth{\color{blue}\foreignlanguage{arabic}{ف.ش.ف.ش}\color{blue}{}}{\color{blue}\foreignlanguage{arabic}{ف.ش.ف.ش}\color{blue}{}}\subsection*{\color{blue}\foreignlanguage{arabic}{ف.ش.ف.ش}\color{blue}{}\index{\color{blue}\foreignlanguage{arabic}{ف.ش.ف.ش}\color{blue}{}}} 

{\setlength\topsep{0pt}\textbf{\foreignlanguage{arabic}{فَشْفَش}}\ {\color{gray}\texttt{/\sffamily {{\sffamily faʃfaʃ}}/}\color{black}}\ \textsc{verb}\ [p.]\ \textbf{1.}~get wrinkly in water\ \ $\bullet$\ \ \setlength\topsep{0pt}\textbf{\foreignlanguage{arabic}{فَشْفِش}}\ {\color{gray}\texttt{/\sffamily {{\sffamily faʃfiʃ}}/}\color{black}}\ [c.]\ \ $\bullet$\ \ \setlength\topsep{0pt}\textbf{\foreignlanguage{arabic}{يفَشْفِش}}\ {\color{gray}\texttt{/\sffamily {{\sffamily jfaʃfiʃ}}/}\color{black}}\ [i.]\  \begin{flushright}\color{gray}\foreignlanguage{arabic}{\textbf{\underline{\foreignlanguage{arabic}{أمثلة}}}: فَشْفَش جلدي من كثر ما سبحت اليوم}\end{flushright}\color{black}} \vspace{2mm}

{\setlength\topsep{0pt}\textbf{\foreignlanguage{arabic}{فِشْفَاش}}\ {\color{gray}\texttt{/\sffamily {{\sffamily fiʃfaːʃ}}/}\color{black}}\ \textsc{noun}\ [m.]\ \color{gray}(msa. \foreignlanguage{arabic}{رقائق الذرة المنكهة}~\foreignlanguage{arabic}{\textbf{١.}})\color{black}\ \textbf{1.}~flavoured puffcorn\ } \vspace{2mm}

{\setlength\topsep{0pt}\textbf{\foreignlanguage{arabic}{مْفَشْفِش}}\ {\color{gray}\texttt{/\sffamily {{\sffamily mfaʃfiʃ}}/}\color{black}}\ \textsc{adj}\ [m.]\ \textbf{1.}~be wrinkly in water\ \ $\smblkdiamond$\ \ \setlength\topsep{0pt}\textbf{\foreignlanguage{arabic}{مْفَشْفِش}}\ \textbf{1.}~very weak and petite\  \begin{flushright}\color{gray}\foreignlanguage{arabic}{\textbf{\underline{\foreignlanguage{arabic}{أمثلة}}}: ماخذيتلي واحد مْفَشْفِش وحاسبتيه عالرجال\ $\bullet$\ \  ايديك وغريك مْفَشْفِشات بطريقة بتخوف.}\end{flushright}\color{black}} \vspace{2mm}

\vspace{-3mm}
\markboth{\color{blue}\foreignlanguage{arabic}{ف.ش.ق}\color{blue}{}}{\color{blue}\foreignlanguage{arabic}{ف.ش.ق}\color{blue}{}}\subsection*{\color{blue}\foreignlanguage{arabic}{ف.ش.ق}\color{blue}{}\index{\color{blue}\foreignlanguage{arabic}{ف.ش.ق}\color{blue}{}}} 

{\setlength\topsep{0pt}\textbf{\foreignlanguage{arabic}{فَشَق}}\ {\color{gray}\texttt{/\sffamily {{\sffamily faʃa(q)}}/}\color{black}}\ \textsc{verb}\ [p.]\ \textbf{1.}~skip\ \ $\bullet$\ \ \setlength\topsep{0pt}\textbf{\foreignlanguage{arabic}{اُفْشُق}}\ {\color{gray}\texttt{/\sffamily {{\sffamily ʔufʃu(q)}}/}\color{black}}\ [c.]\ \ $\bullet$\ \ \setlength\topsep{0pt}\textbf{\foreignlanguage{arabic}{اِفْشُق}}\ {\color{gray}\texttt{/\sffamily {{\sffamily ʔifʃu(q)}}/}\color{black}}\ [c.]\ \ $\bullet$\ \ \setlength\topsep{0pt}\textbf{\foreignlanguage{arabic}{يُفْشُق}}\ {\color{gray}\texttt{/\sffamily {{\sffamily jufʃu(q)}}/}\color{black}}\ [i.]\ \color{gray}(msa. \foreignlanguage{arabic}{يتخطَّى}~\foreignlanguage{arabic}{\textbf{١.}})\color{black}\ \ $\bullet$\ \ \setlength\topsep{0pt}\textbf{\foreignlanguage{arabic}{يِفْشُق}}\ {\color{gray}\texttt{/\sffamily {{\sffamily jifʃu(q)}}/}\color{black}}\ [i.]\ \color{gray}(msa. \foreignlanguage{arabic}{يتخطَّى}~\foreignlanguage{arabic}{\textbf{١.}})\color{black}\  \begin{flushright}\color{gray}\foreignlanguage{arabic}{\textbf{\underline{\foreignlanguage{arabic}{أمثلة}}}: أي شي بتحسه مش مناسب لسياسات الوكالة اُفْشُق عنه}\end{flushright}\color{black}} \vspace{2mm}

{\setlength\topsep{0pt}\textbf{\foreignlanguage{arabic}{فَشَّق}}\ {\color{gray}\texttt{/\sffamily {{\sffamily faʃʃa(q)}}/}\color{black}}\ \textsc{verb}\ [p.]\ \textbf{1.}~skip\ \ $\bullet$\ \ \setlength\topsep{0pt}\textbf{\foreignlanguage{arabic}{فَشِّق}}\ {\color{gray}\texttt{/\sffamily {{\sffamily faʃʃi(q)}}/}\color{black}}\ [c.]\ \ $\bullet$\ \ \setlength\topsep{0pt}\textbf{\foreignlanguage{arabic}{يفَشِّق}}\ {\color{gray}\texttt{/\sffamily {{\sffamily jfaʃʃi(q)}}/}\color{black}}\ [i.]\ \color{gray}(msa. \foreignlanguage{arabic}{يتخطَّى}~\foreignlanguage{arabic}{\textbf{١.}})\color{black}\  \begin{flushright}\color{gray}\foreignlanguage{arabic}{\textbf{\underline{\foreignlanguage{arabic}{أمثلة}}}: أوعك تفَشِّق عالقرآن. شيله وبوسه وحطه عالخزانة.\ $\bullet$\ \  حدا انتبه إِنُّه فَشَّق عن نص الصفحات ومسك بآخر شي انحكى؟}\end{flushright}\color{black}} \vspace{2mm}

\vspace{-3mm}
\markboth{\color{blue}\foreignlanguage{arabic}{ف.ش.ك}\color{blue}{}}{\color{blue}\foreignlanguage{arabic}{ف.ش.ك}\color{blue}{}}\subsection*{\color{blue}\foreignlanguage{arabic}{ف.ش.ك}\color{blue}{}\index{\color{blue}\foreignlanguage{arabic}{ف.ش.ك}\color{blue}{}}} 

{\setlength\topsep{0pt}\textbf{\foreignlanguage{arabic}{فَشَكِة}}\ {\color{gray}\texttt{/\sffamily {{\sffamily faʃake}}/}\color{black}}\ \textsc{noun}\ [f.]\ \color{gray}(msa. \foreignlanguage{arabic}{طَلَقة}~\foreignlanguage{arabic}{\textbf{١.}})\color{black}\ \textbf{1.}~bullet\  \begin{flushright}\color{gray}\foreignlanguage{arabic}{\textbf{\underline{\foreignlanguage{arabic}{أمثلة}}}: البارودة ما فيها ولا فَشَكِة}\end{flushright}\color{black}} \vspace{2mm}

\vspace{-3mm}
\markboth{\color{blue}\foreignlanguage{arabic}{ف.ش.ك.ل}\color{blue}{}}{\color{blue}\foreignlanguage{arabic}{ف.ش.ك.ل}\color{blue}{}}\subsection*{\color{blue}\foreignlanguage{arabic}{ف.ش.ك.ل}\color{blue}{}\index{\color{blue}\foreignlanguage{arabic}{ف.ش.ك.ل}\color{blue}{}}} 

{\setlength\topsep{0pt}\textbf{\foreignlanguage{arabic}{فَشْكَل}}\ {\color{gray}\texttt{/\sffamily {{\sffamily faʃkal}}/}\color{black}}\ \textsc{verb}\ [p.]\ \textbf{1.}~stumble  \textbf{2.}~sth that does not work.  \textbf{3.}~sth that is messed up\ \ $\bullet$\ \ \setlength\topsep{0pt}\textbf{\foreignlanguage{arabic}{فَشْكِل}}\ {\color{gray}\texttt{/\sffamily {{\sffamily faʃkil}}/}\color{black}}\ [c.]\ \ $\bullet$\ \ \setlength\topsep{0pt}\textbf{\foreignlanguage{arabic}{يفَشْكِل}}\ {\color{gray}\texttt{/\sffamily {{\sffamily jfaʃkil}}/}\color{black}}\ [i.]\ \color{gray}(msa. \foreignlanguage{arabic}{يخرب}~\foreignlanguage{arabic}{\textbf{٢.}}  \foreignlanguage{arabic}{يتَعَثَّر}~\foreignlanguage{arabic}{\textbf{١.}})\color{black}\  \begin{flushright}\color{gray}\foreignlanguage{arabic}{\textbf{\underline{\foreignlanguage{arabic}{أمثلة}}}: فَشْكِل هالجيزة طنيب عولاياك\ $\bullet$\ \  شفتيها لما فشكلت بالأرض؟}\end{flushright}\color{black}} \vspace{2mm}

\vspace{-3mm}
\markboth{\color{blue}\foreignlanguage{arabic}{ف.ش.ل}\color{blue}{}}{\color{blue}\foreignlanguage{arabic}{ف.ش.ل}\color{blue}{}}\subsection*{\color{blue}\foreignlanguage{arabic}{ف.ش.ل}\color{blue}{}\index{\color{blue}\foreignlanguage{arabic}{ف.ش.ل}\color{blue}{}}} 

{\setlength\topsep{0pt}\textbf{\foreignlanguage{arabic}{تْفَشَّل}}\ {\color{gray}\texttt{/\sffamily {{\sffamily tfaʃʃal}}/}\color{black}}\ \textsc{verb}\ [p.]\ \textbf{1.}~be disappointed.  \textbf{2.}~be embarrassed\ \ $\bullet$\ \ \setlength\topsep{0pt}\textbf{\foreignlanguage{arabic}{اِتْفَشَّل}}\ {\color{gray}\texttt{/\sffamily {{\sffamily ʔitfaʃʃal}}/}\color{black}}\ [c.]\ \ $\bullet$\ \ \setlength\topsep{0pt}\textbf{\foreignlanguage{arabic}{يِتْفَشَّل}}\ {\color{gray}\texttt{/\sffamily {{\sffamily jitfaʃʃal}}/}\color{black}}\ [i.]\  \begin{flushright}\color{gray}\foreignlanguage{arabic}{\textbf{\underline{\foreignlanguage{arabic}{أمثلة}}}: ياحرام لو تشوفوا كيف المسكين تْفَشَّل قدام المعازيم بسبب مرته الكرنيبة}\end{flushright}\color{black}} \vspace{2mm}

{\setlength\topsep{0pt}\textbf{\foreignlanguage{arabic}{فَاشِل}}\ {\color{gray}\texttt{/\sffamily {{\sffamily faːʃil}}/}\color{black}}\ \textsc{adj}\ [m.]\ \color{gray}(msa. \foreignlanguage{arabic}{لايجيد فعل شيء ما}~\foreignlanguage{arabic}{\textbf{٢.}}  \foreignlanguage{arabic}{فاشِل}~\foreignlanguage{arabic}{\textbf{١.}})\color{black}\ \textbf{1.}~failure  \textbf{2.}~not good at sth\  \begin{flushright}\color{gray}\foreignlanguage{arabic}{\textbf{\underline{\foreignlanguage{arabic}{أمثلة}}}: أنا فاشلِة بالطبخ يعني احتمال بس توكلوا من طبخي تتسمموا وتموتوا}\end{flushright}\color{black}} \vspace{2mm}

{\setlength\topsep{0pt}\textbf{\foreignlanguage{arabic}{فَشَل}}\ {\color{gray}\texttt{/\sffamily {{\sffamily faʃal}}/}\color{black}}\ \textsc{noun}\ [m.]\ \textbf{1.}~failure\ } \vspace{2mm}

{\setlength\topsep{0pt}\textbf{\foreignlanguage{arabic}{فَشَّل}}\ {\color{gray}\texttt{/\sffamily {{\sffamily faʃʃal}}/}\color{black}}\ \textsc{verb}\ [p.]\ \textbf{1.}~make sth fail.  \textbf{2.}~fail sth.  \textbf{3.}~disappoint  \textbf{4.}~embarrass\ \ $\bullet$\ \ \setlength\topsep{0pt}\textbf{\foreignlanguage{arabic}{فَشِّل}}\ {\color{gray}\texttt{/\sffamily {{\sffamily faʃʃil}}/}\color{black}}\ [c.]\ \ $\bullet$\ \ \setlength\topsep{0pt}\textbf{\foreignlanguage{arabic}{يفَشِّل}}\ {\color{gray}\texttt{/\sffamily {{\sffamily jfaʃʃil}}/}\color{black}}\ [i.]\  \begin{flushright}\color{gray}\foreignlanguage{arabic}{\textbf{\underline{\foreignlanguage{arabic}{أمثلة}}}: والله شارب قهوة عند دار أبو تحسين بس مابدي أفِّشلك وهي كما بشرب عندكم\ $\bullet$\ \  هو اللي فَشَّل المشروع بغباؤه}\end{flushright}\color{black}} \vspace{2mm}

{\setlength\topsep{0pt}\textbf{\foreignlanguage{arabic}{فِشِل}}\ {\color{gray}\texttt{/\sffamily {{\sffamily fishil}}/}\color{black}}\ \textsc{verb}\ [p.]\ \textbf{1.}~fail\ \ $\bullet$\ \ \setlength\topsep{0pt}\textbf{\foreignlanguage{arabic}{اِفْشَل}}\ {\color{gray}\texttt{/\sffamily {{\sffamily ʔifʃal}}/}\color{black}}\ [c.]\ \ $\bullet$\ \ \setlength\topsep{0pt}\textbf{\foreignlanguage{arabic}{يِفْشَل}}\ {\color{gray}\texttt{/\sffamily {{\sffamily jifʃal}}/}\color{black}}\ [i.]\ \color{gray}(msa. \foreignlanguage{arabic}{يَفْشَل}~\foreignlanguage{arabic}{\textbf{١.}})\color{black}\  \begin{flushright}\color{gray}\foreignlanguage{arabic}{\textbf{\underline{\foreignlanguage{arabic}{أمثلة}}}: فْشِلت بتربايتكم عشلن هيك طلعتوا هيك}\end{flushright}\color{black}} \vspace{2mm}

\vspace{-3mm}
\markboth{\color{blue}\foreignlanguage{arabic}{ف.ش.ي}\color{blue}{}}{\color{blue}\foreignlanguage{arabic}{ف.ش.ي}\color{blue}{}}\subsection*{\color{blue}\foreignlanguage{arabic}{ف.ش.ي}\color{blue}{}\index{\color{blue}\foreignlanguage{arabic}{ف.ش.ي}\color{blue}{}}} 

{\setlength\topsep{0pt}\textbf{\foreignlanguage{arabic}{أَفْشَى}}\ {\color{gray}\texttt{/\sffamily {{\sffamily ʔafʃa}}/}\color{black}}\ \textsc{verb}\ [p.]\ \textbf{1.}~divulge  \textbf{2.}~spread\ \ $\bullet$\ \ \setlength\topsep{0pt}\textbf{\foreignlanguage{arabic}{اِفْشِي}}\ {\color{gray}\texttt{/\sffamily {{\sffamily ʔifʃi}}/}\color{black}}\ [c.]\ \ $\bullet$\ \ \setlength\topsep{0pt}\textbf{\foreignlanguage{arabic}{يِفْشِي}}\ {\color{gray}\texttt{/\sffamily {{\sffamily jifʃi}}/}\color{black}}\ [i.]\ \color{gray}(msa. \foreignlanguage{arabic}{يَفْشِي}~\foreignlanguage{arabic}{\textbf{١.}})\color{black}\  \begin{flushright}\color{gray}\foreignlanguage{arabic}{\textbf{\underline{\foreignlanguage{arabic}{أمثلة}}}: بس تفوت عأي مكان اِفْشِي السلام تعلم انك تكون مليح وبشوش مع العالم}\end{flushright}\color{black}} \vspace{2mm}

{\setlength\topsep{0pt}\textbf{\foreignlanguage{arabic}{إِفْشَاء}}\ {\color{gray}\texttt{/\sffamily {{\sffamily ʔifʃaːʔ}}/}\color{black}}\ \textsc{noun}\ [m.]\ \textbf{1.}~divulging  \textbf{2.}~spreading\  \begin{flushright}\color{gray}\foreignlanguage{arabic}{\textbf{\underline{\foreignlanguage{arabic}{أمثلة}}}: ديننا حثنا عالصيام، الصلاة، الاحترام، إِفْشاء السلام. مش إِفْشاء الأسرار!}\end{flushright}\color{black}} \vspace{2mm}

{\setlength\topsep{0pt}\textbf{\foreignlanguage{arabic}{فَاشِي}}\ {\color{gray}\texttt{/\sffamily {{\sffamily faːʃi}}/}\color{black}}\ \textsc{adj}\ [m.]\ \textbf{1.}~Fascist\  \begin{flushright}\color{gray}\foreignlanguage{arabic}{\textbf{\underline{\foreignlanguage{arabic}{أمثلة}}}: النظام الاسرائيلي كان نظام فاشِي مستبد}\end{flushright}\color{black}} \vspace{2mm}

{\setlength\topsep{0pt}\textbf{\foreignlanguage{arabic}{فَاشِيِّة}}\ {\color{gray}\texttt{/\sffamily {{\sffamily faːʃijje}}/}\color{black}}\ \textsc{noun}\ [f.]\ \textbf{1.}~Fascism\ } \vspace{2mm}

{\setlength\topsep{0pt}\textbf{\foreignlanguage{arabic}{فَشَى}}\ {\color{gray}\texttt{/\sffamily {{\sffamily faʃa}}/}\color{black}}\ \textsc{verb}\ [p.]\ \textbf{1.}~divulge  \textbf{2.}~spread\ \ $\bullet$\ \ \setlength\topsep{0pt}\textbf{\foreignlanguage{arabic}{اِفْشِي}}\ {\color{gray}\texttt{/\sffamily {{\sffamily ʔifʃi}}/}\color{black}}\ [c.]\ \ $\bullet$\ \ \setlength\topsep{0pt}\textbf{\foreignlanguage{arabic}{يِفْشِي}}\ {\color{gray}\texttt{/\sffamily {{\sffamily jifʃi}}/}\color{black}}\ [i.]\  \begin{flushright}\color{gray}\foreignlanguage{arabic}{\textbf{\underline{\foreignlanguage{arabic}{أمثلة}}}: أنو اللي فَشَى أسراري، مش أنت يا حيوان؟}\end{flushright}\color{black}} \vspace{2mm}

\vspace{-3mm}
\markboth{\color{blue}\foreignlanguage{arabic}{ف.ص.ح}\color{blue}{}}{\color{blue}\foreignlanguage{arabic}{ف.ص.ح}\color{blue}{}}\subsection*{\color{blue}\foreignlanguage{arabic}{ف.ص.ح}\color{blue}{}\index{\color{blue}\foreignlanguage{arabic}{ف.ص.ح}\color{blue}{}}} 

{\setlength\topsep{0pt}\textbf{\foreignlanguage{arabic}{أَفْصَح}}\ {\color{gray}\texttt{/\sffamily {{\sffamily ʔafsˤaħ}}/}\color{black}}\ \textsc{adj\textunderscore comp}\ \textbf{1.}~most eloquent.  \textbf{2.}~smartest  \textbf{3.}~most worldly-wise.  \textbf{4.}~most hard-bitten\  \begin{flushright}\color{gray}\foreignlanguage{arabic}{\textbf{\underline{\foreignlanguage{arabic}{أمثلة}}}: ما أفْصَحها وما أحلاها فروحة!}\end{flushright}\color{black}} \vspace{2mm}

{\setlength\topsep{0pt}\textbf{\foreignlanguage{arabic}{تْفَصَّح}}\ {\color{gray}\texttt{/\sffamily {{\sffamily tfasˤsˤaħ}}/}\color{black}}\ \textsc{verb}\ [p.]\ \textbf{1.}~become worldly-wise.  \textbf{2.}~become hard-bitten\ \ $\bullet$\ \ \setlength\topsep{0pt}\textbf{\foreignlanguage{arabic}{اِتْفَصَّح}}\ {\color{gray}\texttt{/\sffamily {{\sffamily ʔitfasˤsˤaħ}}/}\color{black}}\ [c.]\ \ $\bullet$\ \ \setlength\topsep{0pt}\textbf{\foreignlanguage{arabic}{يِتْفَصَّح}}\ {\color{gray}\texttt{/\sffamily {{\sffamily jitfasˤsˤaħ}}/}\color{black}}\ [i.]\  \begin{flushright}\color{gray}\foreignlanguage{arabic}{\textbf{\underline{\foreignlanguage{arabic}{أمثلة}}}: شوفي كيف ريما تْفَصَّحت وصارت تعرف للأسواق}\end{flushright}\color{black}} \vspace{2mm}

{\setlength\topsep{0pt}\textbf{\foreignlanguage{arabic}{تْفَصْحَن}}\ {\color{gray}\texttt{/\sffamily {{\sffamily tfasˤħan}}/}\color{black}}\ \textsc{verb}\ [p.]\ \textbf{1.}~pontificate on/about sth and try to be idealistic\ \ $\bullet$\ \ \setlength\topsep{0pt}\textbf{\foreignlanguage{arabic}{اِتْفَصْحَن}}\ {\color{gray}\texttt{/\sffamily {{\sffamily ʔitfasˤħan}}/}\color{black}}\ [c.]\ \ $\bullet$\ \ \setlength\topsep{0pt}\textbf{\foreignlanguage{arabic}{يِتْفَصْحَن}}\ {\color{gray}\texttt{/\sffamily {{\sffamily jitfasˤħan}}/}\color{black}}\ [i.]\ \color{gray}(msa. \foreignlanguage{arabic}{يتفلسف ويفتي بأمر لا يفهم فيه}~\foreignlanguage{arabic}{\textbf{١.}})\color{black}\  \begin{flushright}\color{gray}\foreignlanguage{arabic}{\textbf{\underline{\foreignlanguage{arabic}{أمثلة}}}: بحب يِتْفَصْحَن بكل شي عامل حاله خبير}\end{flushright}\color{black}} \vspace{2mm}

{\setlength\topsep{0pt}\textbf{\foreignlanguage{arabic}{فَصَاحَة}}\ {\color{gray}\texttt{/\sffamily {{\sffamily fasˤaːħa}}/}\color{black}}\ \textsc{noun}\ [f.]\ \textbf{1.}~eloquence  \textbf{2.}~the state of being worldly-wise\ } \vspace{2mm}

{\setlength\topsep{0pt}\textbf{\foreignlanguage{arabic}{فَصِيح}}\ {\color{gray}\texttt{/\sffamily {{\sffamily fasˤiːħ}}/}\color{black}}\ \textsc{adj}\ [m.]\ \color{gray}(msa. \foreignlanguage{arabic}{فَصِيح}~\foreignlanguage{arabic}{\textbf{١.}})\color{black}\ \textbf{1.}~eloquent\  \begin{flushright}\color{gray}\foreignlanguage{arabic}{\textbf{\underline{\foreignlanguage{arabic}{أمثلة}}}: ما شاء الله عليه وليد فَصِيح اللسان.}\end{flushright}\color{black}} \vspace{2mm}

{\setlength\topsep{0pt}\textbf{\foreignlanguage{arabic}{فَصَّح}}\ {\color{gray}\texttt{/\sffamily {{\sffamily fasˤsˤaħ}}/}\color{black}}\ \textsc{verb}\ [p.]\ \textbf{1.}~make sb worldly-wise.  \textbf{2.}~make sb hard-bitten\ \ $\bullet$\ \ \setlength\topsep{0pt}\textbf{\foreignlanguage{arabic}{فَصِّح}}\ {\color{gray}\texttt{/\sffamily {{\sffamily fasˤsˤiħ}}/}\color{black}}\ [c.]\ \ $\bullet$\ \ \setlength\topsep{0pt}\textbf{\foreignlanguage{arabic}{يفَصِّح}}\ {\color{gray}\texttt{/\sffamily {{\sffamily jfasˤsˤiħ}}/}\color{black}}\ [i.]\  \begin{flushright}\color{gray}\foreignlanguage{arabic}{\textbf{\underline{\foreignlanguage{arabic}{أمثلة}}}: تجوزها وفَصَّحها بمعرفتك}\end{flushright}\color{black}} \vspace{2mm}

{\setlength\topsep{0pt}\textbf{\foreignlanguage{arabic}{فِصِح}}\ {\color{gray}\texttt{/\sffamily {{\sffamily fisˤħa}}/}\color{black}}\ \textsc{adj}\ [f.]\ \color{gray}(msa. \foreignlanguage{arabic}{له خبرة بالحياة}~\foreignlanguage{arabic}{\textbf{١.}})\color{black}\ \textbf{1.}~worldly-wise  \textbf{2.}~hard-bitten\  \begin{flushright}\color{gray}\foreignlanguage{arabic}{\textbf{\underline{\foreignlanguage{arabic}{أمثلة}}}: ما شاء الله القاروطة الصغيرة فِصْحَة}\end{flushright}\color{black}} \vspace{2mm}

\vspace{-3mm}
\markboth{\color{blue}\foreignlanguage{arabic}{ف.ص.ص}\color{blue}{}}{\color{blue}\foreignlanguage{arabic}{ف.ص.ص}\color{blue}{}}\subsection*{\color{blue}\foreignlanguage{arabic}{ف.ص.ص}\color{blue}{}\index{\color{blue}\foreignlanguage{arabic}{ف.ص.ص}\color{blue}{}}} 

{\setlength\topsep{0pt}\textbf{\foreignlanguage{arabic}{تَفْصِيص}}\ {\color{gray}\texttt{/\sffamily {{\sffamily tafsˤiːsˤ}}/}\color{black}}\ \textsc{noun}\ [m.]\ \textbf{1.}~repeated fart.  \textbf{2.}~thorough scrutiny\ } \vspace{2mm}

{\setlength\topsep{0pt}\textbf{\foreignlanguage{arabic}{فَصّ}}\ {\color{gray}\texttt{/\sffamily {{\sffamily fasˤsˤ}}/}\color{black}}\ \textsc{noun}\ [m.]\ \textbf{1.}~fart  \textbf{2.}~clove (of garlic)\ \ $\bullet$\ \ \setlength\topsep{0pt}\textbf{\foreignlanguage{arabic}{فْصُوص}}\ {\color{gray}\texttt{/\sffamily {{\sffamily fsˤuːsˤ}}/}\color{black}}\ [pl.]\  \begin{flushright}\color{gray}\foreignlanguage{arabic}{\textbf{\underline{\foreignlanguage{arabic}{أمثلة}}}: حطي عالطبخة أربعة أو خمسة  فصوص ثوم بالكثير عشان مش زاكي تكثري ثوم}\end{flushright}\color{black}} \vspace{2mm}

{\setlength\topsep{0pt}\textbf{\foreignlanguage{arabic}{فَصَّص}}\ {\color{gray}\texttt{/\sffamily {{\sffamily fasˤsˤasˤ}}/}\color{black}}\ \textsc{verb}\ [p.]\ \textbf{1.}~dissect or analyse (a subject).  \textbf{2.}~fart\ \ $\bullet$\ \ \setlength\topsep{0pt}\textbf{\foreignlanguage{arabic}{فَصِّص}}\ {\color{gray}\texttt{/\sffamily {{\sffamily fasˤsˤisˤ}}/}\color{black}}\ [c.]\ \ $\bullet$\ \ \setlength\topsep{0pt}\textbf{\foreignlanguage{arabic}{يفَصِّص}}\ {\color{gray}\texttt{/\sffamily {{\sffamily jfasˤsˤisˤ}}/}\color{black}}\ [i.]\  \begin{flushright}\color{gray}\foreignlanguage{arabic}{\textbf{\underline{\foreignlanguage{arabic}{أمثلة}}}: تفَصِّصش قدام الناس عيب عليك\ $\bullet$\ \  المحامي مسك القضية فَصَّصها وبعدين اعتذؤ انه يكمل فيها}\end{flushright}\color{black}} \vspace{2mm}

\vspace{-3mm}
\markboth{\color{blue}\foreignlanguage{arabic}{ف.ص.ع}\color{blue}{}}{\color{blue}\foreignlanguage{arabic}{ف.ص.ع}\color{blue}{}}\subsection*{\color{blue}\foreignlanguage{arabic}{ف.ص.ع}\color{blue}{}\index{\color{blue}\foreignlanguage{arabic}{ف.ص.ع}\color{blue}{}}} 

{\setlength\topsep{0pt}\textbf{\foreignlanguage{arabic}{أَفْصَع}}\ {\color{gray}\texttt{/\sffamily {{\sffamily ʔafsˤaʕ}}/}\color{black}}\ \textsc{adj}\ [m.]\ \color{gray}(msa. \foreignlanguage{arabic}{أعْرَج}~\foreignlanguage{arabic}{\textbf{١.}})\color{black}\ \textbf{1.}~sb who limps\ \ $\bullet$\ \ \setlength\topsep{0pt}\textbf{\foreignlanguage{arabic}{فَصْعَا}}\ {\color{gray}\texttt{/\sffamily {{\sffamily fasˤʕa}}/}\color{black}}\ [f.]\ \ $\bullet$\ \ \setlength\topsep{0pt}\textbf{\foreignlanguage{arabic}{فُصُع}}\ {\color{gray}\texttt{/\sffamily {{\sffamily fusˤuʕ}}/}\color{black}}\ [pl.]\  \begin{flushright}\color{gray}\foreignlanguage{arabic}{\textbf{\underline{\foreignlanguage{arabic}{أمثلة}}}: يعني تصوم تصوم وبالأخير تروح تخطب وحدة فَصْعا وحولا؟}\end{flushright}\color{black}} \vspace{2mm}

{\setlength\topsep{0pt}\textbf{\foreignlanguage{arabic}{إِفْصَع}}\ {\color{gray}\texttt{/\sffamily {{\sffamily ʔifsˤaʕ}}/}\color{black}}\ \textsc{adj}\ [m.]\ \color{gray}(msa. \foreignlanguage{arabic}{أعْرَج}~\foreignlanguage{arabic}{\textbf{١.}})\color{black}\ \textbf{1.}~sb who limps\  \begin{flushright}\color{gray}\foreignlanguage{arabic}{\textbf{\underline{\foreignlanguage{arabic}{أمثلة}}}: كنه إِبنك إِفْصَع يا فاتن!}\end{flushright}\color{black}} \vspace{2mm}

{\setlength\topsep{0pt}\textbf{\foreignlanguage{arabic}{فَصْعَة}}\ {\color{gray}\texttt{/\sffamily {{\sffamily fasˤʕa}}/}\color{black}}\ \textsc{noun}\ [f.]\ \textbf{1.}~limping\  \begin{flushright}\color{gray}\foreignlanguage{arabic}{\textbf{\underline{\foreignlanguage{arabic}{أمثلة}}}: عندها بمشيتها فَصْعَة خفيفة بتبين بس تنزل من الدرج أو تطلعه}\end{flushright}\color{black}} \vspace{2mm}

{\setlength\topsep{0pt}\textbf{\foreignlanguage{arabic}{فَصْعُون}}\ {\color{gray}\texttt{/\sffamily {{\sffamily fasˤʕuːn}}/}\color{black}}\ \textsc{adj}\ [m.]\ \color{gray}(msa. \foreignlanguage{arabic}{أطفال يتصرفون ويتحدثون كالبالغين}~\foreignlanguage{arabic}{\textbf{١.}})\color{black}\ \textbf{1.}~adult-like kids who talk and behave like grown-up people\ \ $\bullet$\ \ \setlength\topsep{0pt}\textbf{\foreignlanguage{arabic}{فَصَاعِين}}\ {\color{gray}\texttt{/\sffamily {{\sffamily fasˤaːʕiːn}}/}\color{black}}\ [pl.]\  \begin{flushright}\color{gray}\foreignlanguage{arabic}{\textbf{\underline{\foreignlanguage{arabic}{أمثلة}}}: بس يناموا الفَصاعِين باجيك زيارة عرواق\ $\bullet$\ \  يا عمي فَصْعُونات وبدهن يعملن مثل الكبار}\end{flushright}\color{black}} \vspace{2mm}

\vspace{-3mm}
\markboth{\color{blue}\foreignlanguage{arabic}{ف.ص.ف.ص}\color{blue}{}}{\color{blue}\foreignlanguage{arabic}{ف.ص.ف.ص}\color{blue}{}}\subsection*{\color{blue}\foreignlanguage{arabic}{ف.ص.ف.ص}\color{blue}{}\index{\color{blue}\foreignlanguage{arabic}{ف.ص.ف.ص}\color{blue}{}}} 

{\setlength\topsep{0pt}\textbf{\foreignlanguage{arabic}{فَصْفَص}}\ {\color{gray}\texttt{/\sffamily {{\sffamily fasˤfasˤ}}/}\color{black}}\ \textsc{verb}\ [p.]\ \textbf{1.}~bite down on seeds (crack them between the front teeth).  \textbf{2.}~dissect or nitpick\ \ $\bullet$\ \ \setlength\topsep{0pt}\textbf{\foreignlanguage{arabic}{فَصْفِص}}\ {\color{gray}\texttt{/\sffamily {{\sffamily fasˤfisˤ}}/}\color{black}}\ [c.]\ \ $\bullet$\ \ \setlength\topsep{0pt}\textbf{\foreignlanguage{arabic}{يفَصْفِص}}\ {\color{gray}\texttt{/\sffamily {{\sffamily jfasˤfisˤ}}/}\color{black}}\ [i.]\ \color{gray}(msa. \foreignlanguage{arabic}{يدقق بالتفاصيل الصغيرة ويبحث عن أخطاء}~\foreignlanguage{arabic}{\textbf{٢.}}  .\foreignlanguage{arabic}{يقشِّر البذور بأسنانه}~\foreignlanguage{arabic}{\textbf{١.}})\color{black}\  \begin{flushright}\color{gray}\foreignlanguage{arabic}{\textbf{\underline{\foreignlanguage{arabic}{أمثلة}}}: بتدخل عنده البنت بيفَصْفِصْها فَصْفَصة من ساساها لراسها\ $\bullet$\ \  أنت ضلك قاعد بالدار وفَصْفِص بزر مثل النسوان}\end{flushright}\color{black}} \vspace{2mm}

{\setlength\topsep{0pt}\textbf{\foreignlanguage{arabic}{فَصْفَصِة}}\ {\color{gray}\texttt{/\sffamily {{\sffamily fasˤfasˤe}}/}\color{black}}\ \textsc{noun}\ [f.]\ \textbf{1.}~biting down on seeds.  \textbf{2.}~dissection or nitpicking\ } \vspace{2mm}

\vspace{-3mm}
\markboth{\color{blue}\foreignlanguage{arabic}{ف.ص.ل}\color{blue}{}}{\color{blue}\foreignlanguage{arabic}{ف.ص.ل}\color{blue}{}}\subsection*{\color{blue}\foreignlanguage{arabic}{ف.ص.ل}\color{blue}{}\index{\color{blue}\foreignlanguage{arabic}{ف.ص.ل}\color{blue}{}}} 

{\setlength\topsep{0pt}\textbf{\foreignlanguage{arabic}{اِنْفَصَل}}\ {\color{gray}\texttt{/\sffamily {{\sffamily ʔinfasˤal}}/}\color{black}}\ \textsc{verb}\ [p.]\ \textbf{1.}~be separated.  \textbf{2.}~be expelled\ \ $\bullet$\ \ \setlength\topsep{0pt}\textbf{\foreignlanguage{arabic}{اِنْفِصِل}}\ {\color{gray}\texttt{/\sffamily {{\sffamily ʔinfisˤil}}/}\color{black}}\ [c.]\ \ $\bullet$\ \ \setlength\topsep{0pt}\textbf{\foreignlanguage{arabic}{يِنْفِصِل}}\ {\color{gray}\texttt{/\sffamily {{\sffamily jinfisˤil}}/}\color{black}}\ [i.]\ \color{gray}(msa. \foreignlanguage{arabic}{يَنْفَصِل من شخص أو عمل}~\foreignlanguage{arabic}{\textbf{١.}})\color{black}\  \begin{flushright}\color{gray}\foreignlanguage{arabic}{\textbf{\underline{\foreignlanguage{arabic}{أمثلة}}}: نصيحتي يا ابني اِنْفِصِل عنها لأنه اذا بتضلها عذمتك رح تعملك وجع راس\ $\bullet$\ \  اليوم اِنْفَصَلت من شغلي عشغلة تافهة حسبي الله ونعم الوكيل باللي كان السبب}\end{flushright}\color{black}} \vspace{2mm}

{\setlength\topsep{0pt}\textbf{\foreignlanguage{arabic}{اِنْفِصَال}}\ {\color{gray}\texttt{/\sffamily {{\sffamily ʔinfisˤaːl}}/}\color{black}}\ \textsc{noun}\ [m.]\ \color{gray}(msa. \foreignlanguage{arabic}{اِنفِصال}~\foreignlanguage{arabic}{\textbf{١.}})\color{black}\ \textbf{1.}~separation\  \begin{flushright}\color{gray}\foreignlanguage{arabic}{\textbf{\underline{\foreignlanguage{arabic}{أمثلة}}}: بس اللي فهمته من نور إِنك تعبت كثير بعد الاِنفِصال}\end{flushright}\color{black}} \vspace{2mm}

{\setlength\topsep{0pt}\textbf{\foreignlanguage{arabic}{تَفْصِيل}}\ {\color{gray}\texttt{/\sffamily {{\sffamily tafsˤiːl}}/}\color{black}}\ \textsc{noun}\ [m.]\ \textbf{1.}~tailoring\ } \vspace{2mm}

{\setlength\topsep{0pt}\textbf{\foreignlanguage{arabic}{تَفْصِيلِة}}\ {\color{gray}\texttt{/\sffamily {{\sffamily tafsˤiːle}}/}\color{black}}\ \textsc{noun}\ [f.]\ \textbf{1.}~style or cut (clothes)\ \ $\smblkdiamond$\ \ \setlength\topsep{0pt}\textbf{\foreignlanguage{arabic}{تَفْصِيلِة}}\ \color{gray}(msa. \foreignlanguage{arabic}{تَفْصيلَة}~\foreignlanguage{arabic}{\textbf{١.}})\color{black}\ \textbf{1.}~detail\ \ $\bullet$\ \ \setlength\topsep{0pt}\textbf{\foreignlanguage{arabic}{تَفَاصِيل}}\ {\color{gray}\texttt{/\sffamily {{\sffamily tafaːsˤiːl}}/}\color{black}}\ [pl.]\ \textbf{1.}~details\  \begin{flushright}\color{gray}\foreignlanguage{arabic}{\textbf{\underline{\foreignlanguage{arabic}{أمثلة}}}: ماتدقِّش كثير بالتَّفاصِيل. أهم شي بشكل عام كيف شايفه؟\ $\bullet$\ \  اسألي عن كل تَفْصيلِة صغيرة\ $\bullet$\ \  تَفْصيلِة الفستان مش مضبوطة}\end{flushright}\color{black}} \vspace{2mm}

{\setlength\topsep{0pt}\textbf{\foreignlanguage{arabic}{تْفَصَّل}}\ {\color{gray}\texttt{/\sffamily {{\sffamily tfasˤsˤal}}/}\color{black}}\ \textsc{verb}\ [p.]\ \textbf{1.}~be tailored.  \textbf{2.}~be explained sth in detail\ \ $\bullet$\ \ \setlength\topsep{0pt}\textbf{\foreignlanguage{arabic}{اِتْفَصَّل}}\ {\color{gray}\texttt{/\sffamily {{\sffamily ʔitfasˤsˤal}}/}\color{black}}\ [c.]\ \ $\bullet$\ \ \setlength\topsep{0pt}\textbf{\foreignlanguage{arabic}{يِتْفَصَّل}}\ {\color{gray}\texttt{/\sffamily {{\sffamily jitfasˤsˤal}}/}\color{black}}\ [i.]\  \begin{flushright}\color{gray}\foreignlanguage{arabic}{\textbf{\underline{\foreignlanguage{arabic}{أمثلة}}}: لازم إِجابتك بالامتحان تِتفَصَّل ولا بتنقص علامات\ $\bullet$\ \  بس الثوب يِتفَصَّل أنا ببعث وراكِ}\end{flushright}\color{black}} \vspace{2mm}

{\setlength\topsep{0pt}\textbf{\foreignlanguage{arabic}{فَاصَل}}\ {\color{gray}\texttt{/\sffamily {{\sffamily faːsˤal}}/}\color{black}}\ \textsc{verb}\ [p.]\ \textbf{1.}~bargain  \textbf{2.}~haggle\ \ $\bullet$\ \ \setlength\topsep{0pt}\textbf{\foreignlanguage{arabic}{فَاصِل}}\ {\color{gray}\texttt{/\sffamily {{\sffamily faːsˤil}}/}\color{black}}\ [c.]\ \ $\bullet$\ \ \setlength\topsep{0pt}\textbf{\foreignlanguage{arabic}{يفَاصِل}}\ {\color{gray}\texttt{/\sffamily {{\sffamily jfaːsˤil}}/}\color{black}}\ [i.]\ \color{gray}(msa. \foreignlanguage{arabic}{يُفاصِل}~\foreignlanguage{arabic}{\textbf{١.}})\color{black}\  \begin{flushright}\color{gray}\foreignlanguage{arabic}{\textbf{\underline{\foreignlanguage{arabic}{أمثلة}}}: بالأول اتفقنا ع20 دينار بعدين صار يفاصِل فيني بالسعر بده 15}\end{flushright}\color{black}} \vspace{2mm}

{\setlength\topsep{0pt}\textbf{\foreignlanguage{arabic}{فَاصْلِة}}\ {\color{gray}\texttt{/\sffamily {{\sffamily faːsˤle}}/}\color{black}}\ \textsc{noun}\ [f.]\ \color{gray}(msa. \foreignlanguage{arabic}{فاصِلَة}~\foreignlanguage{arabic}{\textbf{١.}})\color{black}\ \textbf{1.}~comma\ \ $\bullet$\ \ \setlength\topsep{0pt}\textbf{\foreignlanguage{arabic}{فَوَاصِل}}\ {\color{gray}\texttt{/\sffamily {{\sffamily fawaːsˤil}}/}\color{black}}\ [pl.]\  \begin{flushright}\color{gray}\foreignlanguage{arabic}{\textbf{\underline{\foreignlanguage{arabic}{أمثلة}}}: لما تكتب، اكتُب بالحركات والفَواصِل}\end{flushright}\color{black}} \vspace{2mm}

{\setlength\topsep{0pt}\textbf{\foreignlanguage{arabic}{فَصَل}}\ {\color{gray}\texttt{/\sffamily {{\sffamily fasˤal}}/}\color{black}}\ \textsc{verb}\ [p.]\ \textbf{1.}~separate  \textbf{2.}~detach  \textbf{3.}~sack  \textbf{4.}~expell\ \ $\bullet$\ \ \setlength\topsep{0pt}\textbf{\foreignlanguage{arabic}{اِفْصِل}}\ {\color{gray}\texttt{/\sffamily {{\sffamily ʔifsˤil}}/}\color{black}}\ [c.]\ \ $\bullet$\ \ \setlength\topsep{0pt}\textbf{\foreignlanguage{arabic}{يِفْصِل}}\ {\color{gray}\texttt{/\sffamily {{\sffamily jifsˤil}}/}\color{black}}\ [i.]\ \color{gray}(msa. \foreignlanguage{arabic}{يَفْصِل (شيء أو شخص)}~\foreignlanguage{arabic}{\textbf{١.}})\color{black}\  \begin{flushright}\color{gray}\foreignlanguage{arabic}{\textbf{\underline{\foreignlanguage{arabic}{أمثلة}}}: اِفْصِل بين شغلك وحياتك بالبيت عشان تعرف تركز بكل واحد فيهم\ $\bullet$\ \  المدير فَصَل خمس موظفين وقطع أرزاقهم الله لايوفقه}\end{flushright}\color{black}} \vspace{2mm}

{\setlength\topsep{0pt}\textbf{\foreignlanguage{arabic}{فَصِل}}\ {\color{gray}\texttt{/\sffamily {{\sffamily fasˤil}}/}\color{black}}\ \textsc{noun}\ [m.]\ \color{gray}(msa. \foreignlanguage{arabic}{فَصْل (بمدرسة، دراسي، من الفصول الأرعة)}~\foreignlanguage{arabic}{\textbf{١.}})\color{black}\ \textbf{1.}~classroom  \textbf{2.}~semester  \textbf{3.}~season\ \ $\bullet$\ \ \setlength\topsep{0pt}\textbf{\foreignlanguage{arabic}{فْصُول}}\ {\color{gray}\texttt{/\sffamily {{\sffamily fsˤuːl}}/}\color{black}}\ [pl.]\ \ $\bullet$\ \ \textsc{ph.} \color{gray} \foreignlanguage{arabic}{فصولهَا النَاقصة}\color{black}\ {\color{gray}\texttt{/{\sffamily fsˤuːlha ʔinnaː(q)sˤa}/}\color{black}}\ \textbf{1.}~mean actions\  \begin{flushright}\color{gray}\foreignlanguage{arabic}{\textbf{\underline{\foreignlanguage{arabic}{أمثلة}}}: مش قادر أستحمل فصولها الناقصة\ $\bullet$\ \  الفَصِل مرتب  ونظيف كان عشان هيك كرَّمهم المدير}\end{flushright}\color{black}} \vspace{2mm}

{\setlength\topsep{0pt}\textbf{\foreignlanguage{arabic}{فَصِيل}}\ {\color{gray}\texttt{/\sffamily {{\sffamily fasˤiːl}}/}\color{black}}\ \textsc{noun}\ [m.]\ \textbf{1.}~faction  \textbf{2.}~group\ \ $\bullet$\ \ \setlength\topsep{0pt}\textbf{\foreignlanguage{arabic}{فَصَائِل}}\ {\color{gray}\texttt{/\sffamily {{\sffamily fasˤaːʔil}}/}\color{black}}\ [pl.]\ \ $\bullet$\ \ \textsc{ph.} \color{gray} \foreignlanguage{arabic}{فَصِيل سِيَاسِي}\color{black}\ {\color{gray}\texttt{/{\sffamily fasˤiːl sijaːsi}/}\color{black}}\ \color{gray} (msa. \foreignlanguage{arabic}{حِزب سياسِي}~\foreignlanguage{arabic}{\textbf{١.}})\color{black}\ \textbf{1.}~political faction\  \begin{flushright}\color{gray}\foreignlanguage{arabic}{\textbf{\underline{\foreignlanguage{arabic}{أمثلة}}}: أنا مش تابع لأي فَصِيل سِياسِي}\end{flushright}\color{black}} \vspace{2mm}

{\setlength\topsep{0pt}\textbf{\foreignlanguage{arabic}{فَصِيلِة}}\ {\color{gray}\texttt{/\sffamily {{\sffamily fasˤiːle}}/}\color{black}}\ \textsc{noun}\ [f.]\ \color{gray}(msa. \foreignlanguage{arabic}{نوع}~\foreignlanguage{arabic}{\textbf{١.}})\color{black}\ \textbf{1.}~type\ \ $\bullet$\ \ \setlength\topsep{0pt}\textbf{\foreignlanguage{arabic}{فَصَائِل}}\ {\color{gray}\texttt{/\sffamily {{\sffamily fasˤaːʔil}}/}\color{black}}\ [pl.]\ \ $\bullet$\ \ \textsc{ph.} \color{gray} \foreignlanguage{arabic}{فَصِيلِة دم}\color{black}\ {\color{gray}\texttt{/{\sffamily fasˤiːlit damm}/}\color{black}}\ \color{gray} (msa. \foreignlanguage{arabic}{فَصِيلَة دم}~\foreignlanguage{arabic}{\textbf{١.}})\color{black}\ \textbf{1.}~blood type\  \begin{flushright}\color{gray}\foreignlanguage{arabic}{\textbf{\underline{\foreignlanguage{arabic}{أمثلة}}}: شو فَصِيلِة دم أبوك؟}\end{flushright}\color{black}} \vspace{2mm}

{\setlength\topsep{0pt}\textbf{\foreignlanguage{arabic}{فَصَّل}}\ {\color{gray}\texttt{/\sffamily {{\sffamily fasˤsˤal}}/}\color{black}}\ \textsc{verb}\ [p.]\ \textbf{1.}~tailor  \textbf{2.}~explain sth in detail\ \ $\bullet$\ \ \setlength\topsep{0pt}\textbf{\foreignlanguage{arabic}{فَصِّل}}\ {\color{gray}\texttt{/\sffamily {{\sffamily fasˤsˤil}}/}\color{black}}\ [c.]\ \ $\bullet$\ \ \setlength\topsep{0pt}\textbf{\foreignlanguage{arabic}{يفَصِّل}}\ {\color{gray}\texttt{/\sffamily {{\sffamily jfasˤsˤil}}/}\color{black}}\ [i.]\  \begin{flushright}\color{gray}\foreignlanguage{arabic}{\textbf{\underline{\foreignlanguage{arabic}{أمثلة}}}: شو يعني نفَصِّلِّك زلمة عكيفك عشان توافقي تتجوزي\ $\bullet$\ \  لما القاضي يطلب منك أسباب الخلع فَصلي أكثر فيها عشان يقتنع}\end{flushright}\color{black}} \vspace{2mm}

{\setlength\topsep{0pt}\textbf{\foreignlanguage{arabic}{فُصَّاليِّة}}\ {\color{gray}\texttt{/\sffamily {{\sffamily fusˤsˤaːlijje}}/}\color{black}}\ \textsc{noun}\ [f.]\ \color{gray}(msa. \foreignlanguage{arabic}{مِفْصَل الباب}~\foreignlanguage{arabic}{\textbf{١.}})\color{black}\ \textbf{1.}~door hinge\  \begin{flushright}\color{gray}\foreignlanguage{arabic}{\textbf{\underline{\foreignlanguage{arabic}{أمثلة}}}: زيِّت فُصّاليِّة الباب عشان تضلهاش تطلع صوت مزعج هيك؟}\end{flushright}\color{black}} \vspace{2mm}

{\setlength\topsep{0pt}\textbf{\foreignlanguage{arabic}{مَفْصُول}}\ {\color{gray}\texttt{/\sffamily {{\sffamily mafsˤuːl}}/}\color{black}}\ \textsc{noun\textunderscore pass}\ \textbf{1.}~detached  \textbf{2.}~separated  \textbf{3.}~excluded  \textbf{4.}~dismissed  \textbf{5.}~fired\  \begin{flushright}\color{gray}\foreignlanguage{arabic}{\textbf{\underline{\foreignlanguage{arabic}{أمثلة}}}: عبدالله مَفْصُول عن أهله وأصلاً من خمس سنين هو مش ساكن معهم بنفس الدار}\end{flushright}\color{black}} \vspace{2mm}

{\setlength\topsep{0pt}\textbf{\foreignlanguage{arabic}{مُنْفَصِل}}\ {\color{gray}\texttt{/\sffamily {{\sffamily munfasˤil}}/}\color{black}}\ \textsc{adj}\ [m.]\ \textbf{1.}~separated  \textbf{2.}~divorced\  \begin{flushright}\color{gray}\foreignlanguage{arabic}{\textbf{\underline{\foreignlanguage{arabic}{أمثلة}}}: أنا مُنْفَصِل من ست أشهر}\end{flushright}\color{black}} \vspace{2mm}

{\setlength\topsep{0pt}\textbf{\foreignlanguage{arabic}{مْفَاصَلِة}}\ {\color{gray}\texttt{/\sffamily {{\sffamily mfaːsˤale}}/}\color{black}}\ \textsc{noun}\ [f.]\ \textbf{1.}~bargain  \textbf{2.}~haggle\  \begin{flushright}\color{gray}\foreignlanguage{arabic}{\textbf{\underline{\foreignlanguage{arabic}{أمثلة}}}: الشغلة مش مْفاصَلِة يا خالتي والله راعيتك باللي بقدر عليه}\end{flushright}\color{black}} \vspace{2mm}

\vspace{-3mm}
\markboth{\color{blue}\foreignlanguage{arabic}{ف.ص.م}\color{blue}{}}{\color{blue}\foreignlanguage{arabic}{ف.ص.م}\color{blue}{}}\subsection*{\color{blue}\foreignlanguage{arabic}{ف.ص.م}\color{blue}{}\index{\color{blue}\foreignlanguage{arabic}{ف.ص.م}\color{blue}{}}} 

{\setlength\topsep{0pt}\textbf{\foreignlanguage{arabic}{اِنْفَصَم}}\ {\color{gray}\texttt{/\sffamily {{\sffamily ʔinfasˤam}}/}\color{black}}\ \textsc{verb}\ [p.]\ \textbf{1.}~be schizophrenic.  \textbf{2.}~be confused.  \textbf{3.}~be baffled\ \ $\bullet$\ \ \setlength\topsep{0pt}\textbf{\foreignlanguage{arabic}{اِنْفِصِم}}\ {\color{gray}\texttt{/\sffamily {{\sffamily ʔinfisˤim}}/}\color{black}}\ [c.]\ \ $\bullet$\ \ \setlength\topsep{0pt}\textbf{\foreignlanguage{arabic}{يِنْفِصِم}}\ {\color{gray}\texttt{/\sffamily {{\sffamily jinfisˤim}}/}\color{black}}\ [i.]\  \begin{flushright}\color{gray}\foreignlanguage{arabic}{\textbf{\underline{\foreignlanguage{arabic}{أمثلة}}}: طب أنا هلا اِنْفَصَمِت شو أعمل}\end{flushright}\color{black}} \vspace{2mm}

{\setlength\topsep{0pt}\textbf{\foreignlanguage{arabic}{فَصَم}}\ {\color{gray}\texttt{/\sffamily {{\sffamily fasˤam}}/}\color{black}}\ \textsc{verb}\ [p.]\ \textbf{1.}~make sb feel schizophrenic in a way that baffles/confuses sb\ \ $\bullet$\ \ \setlength\topsep{0pt}\textbf{\foreignlanguage{arabic}{اِفْصِم}}\ {\color{gray}\texttt{/\sffamily {{\sffamily ʔifsˤim}}/}\color{black}}\ [c.]\ \ $\bullet$\ \ \setlength\topsep{0pt}\textbf{\foreignlanguage{arabic}{اُفْصُم}}\ {\color{gray}\texttt{/\sffamily {{\sffamily ʔufsˤum}}/}\color{black}}\ [c.]\ \ $\bullet$\ \ \setlength\topsep{0pt}\textbf{\foreignlanguage{arabic}{يِفْصِم}}\ {\color{gray}\texttt{/\sffamily {{\sffamily jifsˤim}}/}\color{black}}\ [i.]\ \ $\bullet$\ \ \setlength\topsep{0pt}\textbf{\foreignlanguage{arabic}{يُفْصُم}}\ {\color{gray}\texttt{/\sffamily {{\sffamily jufsˤum}}/}\color{black}}\ [i.]\  \begin{flushright}\color{gray}\foreignlanguage{arabic}{\textbf{\underline{\foreignlanguage{arabic}{أمثلة}}}: اُفْصُميه! اطلبي منه يجيبلك كاسة مي وبس يجيبلك اياها بهدليه وقوليله إِنك طلبتي كاسة عصير مش مي\ $\bullet$\ \  يازلمة فَصَمِتني يعني أنت هلا بدك اياها ولا لا}\end{flushright}\color{black}} \vspace{2mm}

{\setlength\topsep{0pt}\textbf{\foreignlanguage{arabic}{فَصْمِة}}\ {\color{gray}\texttt{/\sffamily {{\sffamily fasˤme}}/}\color{black}}\ \textsc{noun}\ [f.]\ \textbf{1.}~a situation where sb contradicts himself\  \begin{flushright}\color{gray}\foreignlanguage{arabic}{\textbf{\underline{\foreignlanguage{arabic}{أمثلة}}}: ايش هالفَصْمِة اللي أنت فيها. يعن أنت هلا بدك ولا لا؟}\end{flushright}\color{black}} \vspace{2mm}

{\setlength\topsep{0pt}\textbf{\foreignlanguage{arabic}{فُصَام}}\ {\color{gray}\texttt{/\sffamily {{\sffamily fusˤaːm}}/}\color{black}}\ \textsc{noun}\ [m.]\ \textbf{1.}~schizophrenia  \textbf{2.}~splitting\  \begin{flushright}\color{gray}\foreignlanguage{arabic}{\textbf{\underline{\foreignlanguage{arabic}{أمثلة}}}: حدا قالك انه عندك فُصام}\end{flushright}\color{black}} \vspace{2mm}

{\setlength\topsep{0pt}\textbf{\foreignlanguage{arabic}{مَفْصُوم}}\ {\color{gray}\texttt{/\sffamily {{\sffamily mafsˤuːm}}/}\color{black}}\ \textsc{adj}\ [m.]\ \textbf{1.}~schizophrenic\ \ $\bullet$\ \ \setlength\topsep{0pt}\textbf{\foreignlanguage{arabic}{مَفَاصِيم}}\ {\color{gray}\texttt{/\sffamily {{\sffamily mafaːsˤiːm}}/}\color{black}}\ [pl.]\  \begin{flushright}\color{gray}\foreignlanguage{arabic}{\textbf{\underline{\foreignlanguage{arabic}{أمثلة}}}: إِخوانك المَفاصِيم شو رأيهم بهيك حالة؟ تحكيش انهم موافقين}\end{flushright}\color{black}} \vspace{2mm}

\vspace{-3mm}
\markboth{\color{blue}\foreignlanguage{arabic}{ف.ض.ء}\color{blue}{}}{\color{blue}\foreignlanguage{arabic}{ف.ض.ء}\color{blue}{}}\subsection*{\color{blue}\foreignlanguage{arabic}{ف.ض.ء}\color{blue}{}\index{\color{blue}\foreignlanguage{arabic}{ف.ض.ء}\color{blue}{}}} 

{\setlength\topsep{0pt}\textbf{\foreignlanguage{arabic}{فَضَا}}\ {\color{gray}\texttt{/\sffamily {{\sffamily fa(dˤ)a}}/}\color{black}}\ \textsc{noun}\ [m.]\ \color{gray}(msa. \foreignlanguage{arabic}{فَضاء}~\foreignlanguage{arabic}{\textbf{١.}})\color{black}\ \textbf{1.}~space\ } \vspace{2mm}

{\setlength\topsep{0pt}\textbf{\foreignlanguage{arabic}{فَضَاء}}\ {\color{gray}\texttt{/\sffamily {{\sffamily fa(dˤ)aːʔ}}/}\color{black}}\ \textsc{noun}\ [m.]\ \color{gray}(msa. \foreignlanguage{arabic}{فَضاء}~\foreignlanguage{arabic}{\textbf{١.}})\color{black}\ \textbf{1.}~space\ } \vspace{2mm}

{\setlength\topsep{0pt}\textbf{\foreignlanguage{arabic}{فَضَائِي}}\ {\color{gray}\texttt{/\sffamily {{\sffamily fa(dˤ)aːʔi}}/}\color{black}}\ \textsc{adj}\ [m.]\ \textbf{1.}~relating to space\  \begin{flushright}\color{gray}\foreignlanguage{arabic}{\textbf{\underline{\foreignlanguage{arabic}{أمثلة}}}: لابسة لبس مخليشها شبه المخلوقات الفَضائِية}\end{flushright}\color{black}} \vspace{2mm}

\vspace{-3mm}
\markboth{\color{blue}\foreignlanguage{arabic}{ف.ض.ح}\color{blue}{}}{\color{blue}\foreignlanguage{arabic}{ف.ض.ح}\color{blue}{}}\subsection*{\color{blue}\foreignlanguage{arabic}{ف.ض.ح}\color{blue}{}\index{\color{blue}\foreignlanguage{arabic}{ف.ض.ح}\color{blue}{}}} 

{\setlength\topsep{0pt}\textbf{\foreignlanguage{arabic}{اِنْفَضَح}}\ {\color{gray}\texttt{/\sffamily {{\sffamily ʔinfa(dˤ)aħ}}/}\color{black}}\ \textsc{verb}\ [p.]\ \textbf{1.}~be exposed.  \textbf{2.}~be scandalized\ \ $\bullet$\ \ \setlength\topsep{0pt}\textbf{\foreignlanguage{arabic}{اِنْفِضِح}}\ {\color{gray}\texttt{/\sffamily {{\sffamily ʔinfi(dˤ)iħ}}/}\color{black}}\ [c.]\ \ $\bullet$\ \ \setlength\topsep{0pt}\textbf{\foreignlanguage{arabic}{يِنْفِضِح}}\ {\color{gray}\texttt{/\sffamily {{\sffamily jinfi(dˤ)iħ}}/}\color{black}}\ [i.]\  \begin{flushright}\color{gray}\foreignlanguage{arabic}{\textbf{\underline{\foreignlanguage{arabic}{أمثلة}}}: ضلك ورا النسوان والوساخة تبعتك واِنْفِضِح الله لا يردك}\end{flushright}\color{black}} \vspace{2mm}

{\setlength\topsep{0pt}\textbf{\foreignlanguage{arabic}{فَضَح}}\ {\color{gray}\texttt{/\sffamily {{\sffamily fa(dˤ)aħ}}/}\color{black}}\ \textsc{verb}\ [p.]\ \textbf{1.}~expose  \textbf{2.}~scandalize\ \ $\bullet$\ \ \setlength\topsep{0pt}\textbf{\foreignlanguage{arabic}{اِفْضَح}}\ {\color{gray}\texttt{/\sffamily {{\sffamily ʔif(dˤ)aħ}}/}\color{black}}\ [c.]\ \ $\bullet$\ \ \setlength\topsep{0pt}\textbf{\foreignlanguage{arabic}{يِفْضَح}}\ {\color{gray}\texttt{/\sffamily {{\sffamily jif(dˤ)aħ}}/}\color{black}}\ [i.]\ \color{gray}(msa. \foreignlanguage{arabic}{يَفْضَح}~\foreignlanguage{arabic}{\textbf{١.}})\color{black}\  \begin{flushright}\color{gray}\foreignlanguage{arabic}{\textbf{\underline{\foreignlanguage{arabic}{أمثلة}}}: المرة بطَّلت هالشغلانة حرام عليك تِفْضَحها\ $\bullet$\ \  اِنْفَضَحت بالسوق وهو بيحسس بظهري}\end{flushright}\color{black}} \vspace{2mm}

{\setlength\topsep{0pt}\textbf{\foreignlanguage{arabic}{فَضَّح}}\ {\color{gray}\texttt{/\sffamily {{\sffamily fa(dˤ)(dˤ)aħ}}/}\color{black}}\ \textsc{verb}\ [p.]\ \textbf{1.}~expose  \textbf{2.}~scandalize sb (with details) to more people (a larger segment of people).\ \ $\bullet$\ \ \setlength\topsep{0pt}\textbf{\foreignlanguage{arabic}{فَضِّح}}\ {\color{gray}\texttt{/\sffamily {{\sffamily fa(dˤ)(dˤ)iħ}}/}\color{black}}\ [c.]\ \ $\bullet$\ \ \setlength\topsep{0pt}\textbf{\foreignlanguage{arabic}{يفَضِّح}}\ {\color{gray}\texttt{/\sffamily {{\sffamily jfa(dˤ)(dˤ)iħ}}/}\color{black}}\ [i.]\  \begin{flushright}\color{gray}\foreignlanguage{arabic}{\textbf{\underline{\foreignlanguage{arabic}{أمثلة}}}: ياما فَضَّحت فينا وبدار حماها واحنا ساكتين ومستجملين عشان ولادها}\end{flushright}\color{black}} \vspace{2mm}

{\setlength\topsep{0pt}\textbf{\foreignlanguage{arabic}{فَضِيحَة}}\ {\color{gray}\texttt{/\sffamily {{\sffamily fa(dˤ)iːħa}}/}\color{black}}\ \textsc{noun}\ [f.]\ \color{gray}(msa. \foreignlanguage{arabic}{فضيحَة}~\foreignlanguage{arabic}{\textbf{١.}})\color{black}\ \textbf{1.}~scandal\ \ $\bullet$\ \ \setlength\topsep{0pt}\textbf{\foreignlanguage{arabic}{فَضَايِح}}\ {\color{gray}\texttt{/\sffamily {{\sffamily fa(dˤ)aːjiħ}}/}\color{black}}\ [pl.]\ \ $\bullet$\ \ \textsc{ph.} \color{gray} \foreignlanguage{arabic}{فضيحة بجلَاجل}\color{black}\ {\color{gray}\texttt{/{\sffamily f(dˤ)iːħa b(dʒ)alaː(dʒ)il}/}\color{black}}\ \color{gray} (msa. \foreignlanguage{arabic}{فضيحة كبيرة}~\foreignlanguage{arabic}{\textbf{١.}})\color{black}\ \textbf{1.}~a big scandal\ \ $\bullet$\ \ \textsc{ph.} \color{gray} \foreignlanguage{arabic}{فَضِيحَة العنزة السودَا}\color{black}\ {\color{gray}\texttt{/{\sffamily fa(dˤ)iːħit ʔilʕanze ʔissoːda}/}\color{black}}\ \color{gray} (msa. \foreignlanguage{arabic}{فضيحة كبرى}~\foreignlanguage{arabic}{\textbf{١.}})\color{black}\ \textbf{1.}~a big scandal\  \begin{flushright}\color{gray}\foreignlanguage{arabic}{\textbf{\underline{\foreignlanguage{arabic}{أمثلة}}}: الله يفضحك فَضِيحَة العَنْزِة السُّودا\ $\bullet$\ \  اللي صار اليوم فْضيحَة بجَلاجِل\ $\bullet$\ \  شوف الفَضايِح على أصولها هون\ $\bullet$\ \  صارت فضيحَة وقتها وقاموا عمامها تبروا منها}\end{flushright}\color{black}} \vspace{2mm}

\vspace{-3mm}
\markboth{\color{blue}\foreignlanguage{arabic}{ف.ض.ض}\color{blue}{}}{\color{blue}\foreignlanguage{arabic}{ف.ض.ض}\color{blue}{}}\subsection*{\color{blue}\foreignlanguage{arabic}{ف.ض.ض}\color{blue}{}\index{\color{blue}\foreignlanguage{arabic}{ف.ض.ض}\color{blue}{}}} 

{\setlength\topsep{0pt}\textbf{\foreignlanguage{arabic}{فَضّ}}\ {\color{gray}\texttt{/\sffamily {{\sffamily fadˤdˤ}}/}\color{black}}\ \textsc{verb}\ [p.]\ \textbf{1.}~put an end to sth.  \textbf{2.}~deflower\ \ $\bullet$\ \ \setlength\topsep{0pt}\textbf{\foreignlanguage{arabic}{فُضّ}}\ {\color{gray}\texttt{/\sffamily {{\sffamily fudˤdˤ}}/}\color{black}}\ [c.]\ \ $\bullet$\ \ \setlength\topsep{0pt}\textbf{\foreignlanguage{arabic}{يفُضّ}}\ {\color{gray}\texttt{/\sffamily {{\sffamily jfudˤdˤ}}/}\color{black}}\ [i.]\  \begin{flushright}\color{gray}\foreignlanguage{arabic}{\textbf{\underline{\foreignlanguage{arabic}{أمثلة}}}: قامت طوشة بالجامعة اجت الشرطة فَضّت الاشتباكات}\end{flushright}\color{black}} \vspace{2mm}

{\setlength\topsep{0pt}\textbf{\foreignlanguage{arabic}{فِضَّة}}\ {\color{gray}\texttt{/\sffamily {{\sffamily fi(dˤ)(dˤ)a}}/}\color{black}}\ \textsc{noun}\ [f.]\ \color{gray}(msa. \foreignlanguage{arabic}{فِضَّة}~\foreignlanguage{arabic}{\textbf{١.}})\color{black}\ \textbf{1.}~silver\  \begin{flushright}\color{gray}\foreignlanguage{arabic}{\textbf{\underline{\foreignlanguage{arabic}{أمثلة}}}: جابلي سلسال فِضَّة}\end{flushright}\color{black}} \vspace{2mm}

{\setlength\topsep{0pt}\textbf{\foreignlanguage{arabic}{فِضِّي}}\ {\color{gray}\texttt{/\sffamily {{\sffamily fi(dˤ)(dˤ)i}}/}\color{black}}\ \textsc{adj}\ [m.]\ \textbf{1.}~relating to silver\  \begin{flushright}\color{gray}\foreignlanguage{arabic}{\textbf{\underline{\foreignlanguage{arabic}{أمثلة}}}: الكاسة الفِضِّية مش كثير حلوة. حطي الذهبية أحلى وأرتب.}\end{flushright}\color{black}} \vspace{2mm}

\vspace{-3mm}
\markboth{\color{blue}\foreignlanguage{arabic}{ف.ض.ف.ض}\color{blue}{}}{\color{blue}\foreignlanguage{arabic}{ف.ض.ف.ض}\color{blue}{}}\subsection*{\color{blue}\foreignlanguage{arabic}{ف.ض.ف.ض}\color{blue}{}\index{\color{blue}\foreignlanguage{arabic}{ف.ض.ف.ض}\color{blue}{}}} 

{\setlength\topsep{0pt}\textbf{\foreignlanguage{arabic}{فَضْفَاض}}\ {\color{gray}\texttt{/\sffamily {{\sffamily fa(dˤ)faː(dˤ)}}/}\color{black}}\ \textsc{adj}\ [m.]\ \textbf{1.}~loose  \textbf{2.}~shabby\  \begin{flushright}\color{gray}\foreignlanguage{arabic}{\textbf{\underline{\foreignlanguage{arabic}{أمثلة}}}: مالها عبايتها؟ مش مكسمة أبداََ، بالعكس كثير فَضْفاضة}\end{flushright}\color{black}} \vspace{2mm}

{\setlength\topsep{0pt}\textbf{\foreignlanguage{arabic}{فَضْفَض}}\ {\color{gray}\texttt{/\sffamily {{\sffamily fa(dˤ)fa(dˤ)}}/}\color{black}}\ \textsc{verb}\ [p.]\ \textbf{1.}~loosen the clothes.  \textbf{2.}~remove soap and foam from.  \textbf{3.}~pour sb's heart out\ \ $\bullet$\ \ \setlength\topsep{0pt}\textbf{\foreignlanguage{arabic}{فَضْفِض}}\ {\color{gray}\texttt{/\sffamily {{\sffamily fa(dˤ)fi(dˤ)}}/}\color{black}}\ [c.]\ (src. \color{gray}\foreignlanguage{arabic}{جلزون}\color{black})\ \ $\bullet$\ \ \setlength\topsep{0pt}\textbf{\foreignlanguage{arabic}{يفَضْفِض}}\ {\color{gray}\texttt{/\sffamily {{\sffamily jfa(dˤ)fi(dˤ)}}/}\color{black}}\ [i.]\ \color{gray}(msa. \foreignlanguage{arabic}{يفضي إِلى شخص ويخبره ما يسوءه ويعكر مزاجه لشخص}~\foreignlanguage{arabic}{\textbf{٣.}}  .\foreignlanguage{arabic}{يزيل الصابون والرغوة}~\foreignlanguage{arabic}{\textbf{٢.}}  .\foreignlanguage{arabic}{يوسِّع الثوب}~\foreignlanguage{arabic}{\textbf{١.}})\color{black}\  \begin{flushright}\color{gray}\foreignlanguage{arabic}{\textbf{\underline{\foreignlanguage{arabic}{أمثلة}}}: طالعة روحي بدي أَفَضْفِض لأي حدا!\ $\bullet$\ \  أعطيني دقيقتين أَفَضْفِض هالصحون والملاعق وأنشفهن وبعدها بمرق عندك أشرب فنجان قهوة\ $\bullet$\ \  لازم أَفَضْفِض الثوب بالَك؟}\end{flushright}\color{black}} \vspace{2mm}

{\setlength\topsep{0pt}\textbf{\foreignlanguage{arabic}{فَضْفَضَة}}\ {\color{gray}\texttt{/\sffamily {{\sffamily fa(dˤ)fa(dˤ)a}}/}\color{black}}\ \textsc{noun}\ [f.]\ \textbf{1.}~loosening sth.  \textbf{2.}~removing soap and foam from.  \textbf{3.}~pouring sb's heart out\  \begin{flushright}\color{gray}\foreignlanguage{arabic}{\textbf{\underline{\foreignlanguage{arabic}{أمثلة}}}: حتى الفَضْفَضَة ممنوعة يا الله؟}\end{flushright}\color{black}} \vspace{2mm}

\vspace{-3mm}
\markboth{\color{blue}\foreignlanguage{arabic}{ف.ض.ل}\color{blue}{}}{\color{blue}\foreignlanguage{arabic}{ف.ض.ل}\color{blue}{}}\subsection*{\color{blue}\foreignlanguage{arabic}{ف.ض.ل}\color{blue}{}\index{\color{blue}\foreignlanguage{arabic}{ف.ض.ل}\color{blue}{}}} 

{\setlength\topsep{0pt}\textbf{\foreignlanguage{arabic}{أَفْضَل}}\ {\color{gray}\texttt{/\sffamily {{\sffamily ʔafdˤal}}/}\color{black}}\ \textsc{adj\textunderscore comp}\ \textbf{1.}~better  \textbf{2.}~best\  \begin{flushright}\color{gray}\foreignlanguage{arabic}{\textbf{\underline{\foreignlanguage{arabic}{أمثلة}}}: كريمة أَفْضَل المعلمات اللي عنا حالياً}\end{flushright}\color{black}} \vspace{2mm}

{\setlength\topsep{0pt}\textbf{\foreignlanguage{arabic}{تْفَضَّل}}\ {\color{gray}\texttt{/\sffamily {{\sffamily tfa(dˤ)(dˤ)al}}/}\color{black}}\ \textsc{verb}\ [p.]\ \textbf{1.}~be kind to sb.  \textbf{2.}~do sb a favour.  \textbf{3.}~le sb come in (in a polite way)\ \ $\bullet$\ \ \setlength\topsep{0pt}\textbf{\foreignlanguage{arabic}{اِتْفَضَّل}}\ {\color{gray}\texttt{/\sffamily {{\sffamily ʔitfa(dˤ)(dˤ)al}}/}\color{black}}\ [c.]\ \ $\bullet$\ \ \setlength\topsep{0pt}\textbf{\foreignlanguage{arabic}{يِتْفَضَّل}}\ {\color{gray}\texttt{/\sffamily {{\sffamily jitfa(dˤ)(dˤ)al}}/}\color{black}}\ [i.]\  \begin{flushright}\color{gray}\foreignlanguage{arabic}{\textbf{\underline{\foreignlanguage{arabic}{أمثلة}}}: أستاذ عمر برة خليه يِتْفَضَّل\ $\bullet$\ \  مصطفى تْفَضَّل علي وعأهلي وساعدنا كثير}\end{flushright}\color{black}} \vspace{2mm}

{\setlength\topsep{0pt}\textbf{\foreignlanguage{arabic}{فَضِل}}\ {\color{gray}\texttt{/\sffamily {{\sffamily fa(dˤ)il}}/}\color{black}}\ \textsc{noun}\ [m.]\ \color{gray}(msa. \foreignlanguage{arabic}{مَعْروف}~\foreignlanguage{arabic}{\textbf{١.}})\color{black}\ \textbf{1.}~favour  \textbf{2.}~good action\ \ $\bullet$\ \ \setlength\topsep{0pt}\textbf{\foreignlanguage{arabic}{أَفْضَال}}\ {\color{gray}\texttt{/\sffamily {{\sffamily ʔaf(dˤ)aːl}}/}\color{black}}\ [pl.]\ \ $\bullet$\ \ \textsc{ph.} \color{gray} \foreignlanguage{arabic}{صَاحِب فَضِل}\color{black}\ {\color{gray}\texttt{/{\sffamily sˤaːħib tfa(dˤ)il}/}\color{black}}\ \textbf{1.}~sb who generously gives to others\ \ $\bullet$\ \ \textsc{ph.} \color{gray} \foreignlanguage{arabic}{غَرَّقني بفضْلُه}\color{black}\ {\color{gray}\texttt{/{\sffamily ɣarra(q)ni bfa(dˤ)lo}/}\color{black}}\ \textbf{1.}~be deeply indebted to sb for his favours\  \begin{flushright}\color{gray}\foreignlanguage{arabic}{\textbf{\underline{\foreignlanguage{arabic}{أمثلة}}}: طول عمره صاحِب فَضِل الله يطول بعمره\ $\bullet$\ \  أفضالك مغرقيتني ومغرقة عيلتي ليوم الدين}\end{flushright}\color{black}} \vspace{2mm}

{\setlength\topsep{0pt}\textbf{\foreignlanguage{arabic}{فَضَّل}}\ {\color{gray}\texttt{/\sffamily {{\sffamily fa(dˤ)(dˤ)al}}/}\color{black}}\ \textsc{verb}\ [p.]\ \textbf{1.}~prefer  \textbf{2.}~do sb a favour.  \textbf{3.}~serve food to sb\ \ $\bullet$\ \ \setlength\topsep{0pt}\textbf{\foreignlanguage{arabic}{فَضِّل}}\ {\color{gray}\texttt{/\sffamily {{\sffamily fa(dˤ)(dˤ)il}}/}\color{black}}\ [c.]\ \ $\bullet$\ \ \setlength\topsep{0pt}\textbf{\foreignlanguage{arabic}{يفَضِّل}}\ {\color{gray}\texttt{/\sffamily {{\sffamily jfa(dˤ)(dˤ)il}}/}\color{black}}\ [i.]\  \begin{flushright}\color{gray}\foreignlanguage{arabic}{\textbf{\underline{\foreignlanguage{arabic}{أمثلة}}}: ما بفَضِّل أنت اللي تحكي مع الزلام خلي حدا من إِخوانك يحكي\ $\bullet$\ \  فَضِّلهم كنافة بدل الفواكه أزكى وأبهج}\end{flushright}\color{black}} \vspace{2mm}

{\setlength\topsep{0pt}\textbf{\foreignlanguage{arabic}{مْفَضِّل}}\ {\color{gray}\texttt{/\sffamily {{\sffamily mfa(dˤ)(dˤ)il}}/}\color{black}}\ \textsc{noun\textunderscore act}\ [m.]\ \textbf{1.}~doing sb a favour in a way that makes him indebted to that person\ \ $\bullet$\ \ \textsc{ph.} \color{gray} \foreignlanguage{arabic}{مْفَضِّل عرَاسي من فوق}\color{black}\ {\color{gray}\texttt{/{\sffamily mfa(dˤ)(dˤ)il ʕaraːsi min foː(q)}/}\color{black}}\ \textbf{1.}~doing sb a great favour\  \begin{flushright}\color{gray}\foreignlanguage{arabic}{\textbf{\underline{\foreignlanguage{arabic}{أمثلة}}}: أبوك الله يطول عمره مْفَضِّل عراسي من فوق\ $\bullet$\ \  أنا مْفَضِّل عليك من زمان}\end{flushright}\color{black}} \vspace{2mm}

\vspace{-3mm}
\markboth{\color{blue}\foreignlanguage{arabic}{ف.ض.ي}\color{blue}{}}{\color{blue}\foreignlanguage{arabic}{ف.ض.ي}\color{blue}{}}\subsection*{\color{blue}\foreignlanguage{arabic}{ف.ض.ي}\color{blue}{}\index{\color{blue}\foreignlanguage{arabic}{ف.ض.ي}\color{blue}{}}} 

{\setlength\topsep{0pt}\textbf{\foreignlanguage{arabic}{تْفَضَّى}}\ {\color{gray}\texttt{/\sffamily {{\sffamily tfa(dˤ)(dˤ)a}}/}\color{black}}\ \textsc{verb}\ [p.]\ \textbf{1.}~free onself up to do sth\ \ $\bullet$\ \ \setlength\topsep{0pt}\textbf{\foreignlanguage{arabic}{اِتْفَضَّى}}\ {\color{gray}\texttt{/\sffamily {{\sffamily ʔitfa(dˤ)(dˤ)a}}/}\color{black}}\ [c.]\ \ $\bullet$\ \ \setlength\topsep{0pt}\textbf{\foreignlanguage{arabic}{يِتْفَضَّى}}\ {\color{gray}\texttt{/\sffamily {{\sffamily jitfa(dˤ)(dˤ)a}}/}\color{black}}\ [i.]\  \begin{flushright}\color{gray}\foreignlanguage{arabic}{\textbf{\underline{\foreignlanguage{arabic}{أمثلة}}}: خليني أتْفَضّالك وساعيتها بنشوف}\end{flushright}\color{black}} \vspace{2mm}

{\setlength\topsep{0pt}\textbf{\foreignlanguage{arabic}{فَاضِي}}\ {\color{gray}\texttt{/\sffamily {{\sffamily faː(dˤ)i}}/}\color{black}}\ \textsc{adj}\ [m.]\ \color{gray}(msa. \foreignlanguage{arabic}{لديه وقت فَراغ}~\foreignlanguage{arabic}{\textbf{١.}})\color{black}\ \textbf{1.}~free\ \ $\bullet$\ \ \textsc{ph.} \color{gray} \foreignlanguage{arabic}{فَاضي البَال}\color{black}\ {\color{gray}\texttt{/{\sffamily faːdˤi ʔilbaːl}/}\color{black}}\ \textbf{1.}~to be at peace with the world\ \ $\bullet$\ \ \textsc{ph.} \color{gray} \foreignlanguage{arabic}{فستق فَاضي}\color{black}\ {\color{gray}\texttt{/{\sffamily fustu(q) faː(dˤ)i}/}\color{black}}\ \color{gray} (msa. \foreignlanguage{arabic}{مبالغ فيه}~\foreignlanguage{arabic}{\textbf{١.}})\color{black}\ \textbf{1.}~overrated\  \begin{flushright}\color{gray}\foreignlanguage{arabic}{\textbf{\underline{\foreignlanguage{arabic}{أمثلة}}}: أصلا معروف إِنُّه هاي الجامعة فُسْتُق فاضِي\ $\bullet$\ \  هذا الشُّغُل بده حدا فاضِي البال\ $\bullet$\ \  فاضِي تطلع معي هالمشوار؟}\end{flushright}\color{black}} \vspace{2mm}

{\setlength\topsep{0pt}\textbf{\foreignlanguage{arabic}{فَضَاوِة}}\ {\color{gray}\texttt{/\sffamily {{\sffamily fa(dˤ)aːwe}}/}\color{black}}\ \textsc{noun}\ [f.]\ \color{gray}(msa. \foreignlanguage{arabic}{وقت فَراغ}~\foreignlanguage{arabic}{\textbf{١.}})\color{black}\ \textbf{1.}~free time\  \begin{flushright}\color{gray}\foreignlanguage{arabic}{\textbf{\underline{\foreignlanguage{arabic}{أمثلة}}}: اعملي اياها عفَضاوتَك زي مابدك حبيبي}\end{flushright}\color{black}} \vspace{2mm}

{\setlength\topsep{0pt}\textbf{\foreignlanguage{arabic}{فَضَّى}}\ {\color{gray}\texttt{/\sffamily {{\sffamily fa(dˤ)(dˤ)a}}/}\color{black}}\ \textsc{verb}\ [p.]\ \textbf{1.}~free sb up.  \textbf{2.}~find time to do sth.  \textbf{3.}~free space\ \ $\bullet$\ \ \setlength\topsep{0pt}\textbf{\foreignlanguage{arabic}{فَضِّي}}\ {\color{gray}\texttt{/\sffamily {{\sffamily fa(dˤ)(dˤ)i}}/}\color{black}}\ [c.]\ \ $\bullet$\ \ \setlength\topsep{0pt}\textbf{\foreignlanguage{arabic}{يفَضِّي}}\ {\color{gray}\texttt{/\sffamily {{\sffamily jfa(dˤ)(dˤ)i}}/}\color{black}}\ [i.]\ \color{gray}(msa. \foreignlanguage{arabic}{يحاول أن يجد مكان فارغ}~\foreignlanguage{arabic}{\textbf{٢.}}  .\foreignlanguage{arabic}{يحاول أن يجد وقت فَراغ}~\foreignlanguage{arabic}{\textbf{١.}})\color{black}\  \begin{flushright}\color{gray}\foreignlanguage{arabic}{\textbf{\underline{\foreignlanguage{arabic}{أمثلة}}}: بدي اياك تفَضِّيلي حالك آخر الأسبوع الجاي نروح عالدكتور\ $\bullet$\ \  تعي فَضِّيلي خزانة المطبخ بدي أعزل}\end{flushright}\color{black}} \vspace{2mm}

{\setlength\topsep{0pt}\textbf{\foreignlanguage{arabic}{فِضِي}}\ {\color{gray}\texttt{/\sffamily {{\sffamily fi(dˤ)i}}/}\color{black}}\ \textsc{verb}\ [p.]\ \textbf{1.}~become free\ \ $\bullet$\ \ \setlength\topsep{0pt}\textbf{\foreignlanguage{arabic}{اِفْضَى}}\ {\color{gray}\texttt{/\sffamily {{\sffamily ʔif(dˤ)a}}/}\color{black}}\ [c.]\ \ $\bullet$\ \ \setlength\topsep{0pt}\textbf{\foreignlanguage{arabic}{يِفْضَى}}\ {\color{gray}\texttt{/\sffamily {{\sffamily jif(dˤ)a}}/}\color{black}}\ [i.]\ \color{gray}(msa. \foreignlanguage{arabic}{يُصْبِح لديه وقت فَراغ}~\foreignlanguage{arabic}{\textbf{١.}})\color{black}\  \begin{flushright}\color{gray}\foreignlanguage{arabic}{\textbf{\underline{\foreignlanguage{arabic}{أمثلة}}}: استنى عليه شوي بس يِفْضَى\ $\bullet$\ \  اِفْضالي شوي من شان الله بدي أحكي معك ضروري.\ $\bullet$\ \  فِضِي المكان اللي جنبك؟}\end{flushright}\color{black}} \vspace{2mm}

\vspace{-3mm}
\markboth{\color{blue}\foreignlanguage{arabic}{ف.ط.ح.ل}\color{blue}{}}{\color{blue}\foreignlanguage{arabic}{ف.ط.ح.ل}\color{blue}{}}\subsection*{\color{blue}\foreignlanguage{arabic}{ف.ط.ح.ل}\color{blue}{}\index{\color{blue}\foreignlanguage{arabic}{ف.ط.ح.ل}\color{blue}{}}} 

{\setlength\topsep{0pt}\textbf{\foreignlanguage{arabic}{فَطْحَل}}\ {\color{gray}\texttt{/\sffamily {{\sffamily fatˤħal}}/}\color{black}}\ \textsc{noun}\ [m.]\ \color{gray}(msa. \foreignlanguage{arabic}{خبير بأمور الحياة وذكي}~\foreignlanguage{arabic}{\textbf{١.}})\color{black}\ \textbf{1.}~worldy-wise/sharp-witted\  \begin{flushright}\color{gray}\foreignlanguage{arabic}{\textbf{\underline{\foreignlanguage{arabic}{أمثلة}}}: عاملِّي حاله فَطْحَل زمانه كانه فش غيره بفهم}\end{flushright}\color{black}} \vspace{2mm}

\vspace{-3mm}
\markboth{\color{blue}\foreignlanguage{arabic}{ف.ط.ر}\color{blue}{}}{\color{blue}\foreignlanguage{arabic}{ف.ط.ر}\color{blue}{}}\subsection*{\color{blue}\foreignlanguage{arabic}{ف.ط.ر}\color{blue}{}\index{\color{blue}\foreignlanguage{arabic}{ف.ط.ر}\color{blue}{}}} 

{\setlength\topsep{0pt}\textbf{\foreignlanguage{arabic}{أَفْطَر}}\ {\color{gray}\texttt{/\sffamily {{\sffamily ʔaftˤar}}/}\color{black}}\ \textsc{verb}\ [p.]\ \textbf{1.}~have breakfast.  \textbf{2.}~break sb's fast\ \ $\bullet$\ \ \setlength\topsep{0pt}\textbf{\foreignlanguage{arabic}{اِفْطِر}}\ {\color{gray}\texttt{/\sffamily {{\sffamily ʔaftˤir}}/}\color{black}}\ [c.]\ \ $\bullet$\ \ \setlength\topsep{0pt}\textbf{\foreignlanguage{arabic}{يِفْطِر}}\ {\color{gray}\texttt{/\sffamily {{\sffamily jiftˤir}}/}\color{black}}\ [i.]\ \color{gray}(msa. \foreignlanguage{arabic}{يُفْطِر في رمضان}~\foreignlanguage{arabic}{\textbf{٢.}}  .\foreignlanguage{arabic}{يُفْطِر في الصباح}~\foreignlanguage{arabic}{\textbf{١.}})\color{black}\  \begin{flushright}\color{gray}\foreignlanguage{arabic}{\textbf{\underline{\foreignlanguage{arabic}{أمثلة}}}: بما إِنك حامل بشهرك الأخير بتقدري تِفْطَري وتقضيهم بعدين عفكرة\ $\bullet$\ \  أفْطَرِت ولا لسة؟ اذا ما أفْطَرت تعال اِفْطَر معنا جايبين ترويقة نابلسية}\end{flushright}\color{black}} \vspace{2mm}

{\setlength\topsep{0pt}\textbf{\foreignlanguage{arabic}{اِفْطَار}}\ {\color{gray}\texttt{/\sffamily {{\sffamily ʔiftˤaːr}}/}\color{black}}\ \textsc{noun}\ [m.]\ \color{gray}(msa. \foreignlanguage{arabic}{اِفْطار رمَضان}~\foreignlanguage{arabic}{\textbf{١.}})\color{black}\ \textbf{1.}~the breaking of the (Ramadan) fast\  \begin{flushright}\color{gray}\foreignlanguage{arabic}{\textbf{\underline{\foreignlanguage{arabic}{أمثلة}}}: جهز الاِفْطار ولا لسة؟ أخرى شوي بأذن المغرب يالله}\end{flushright}\color{black}} \vspace{2mm}

{\setlength\topsep{0pt}\textbf{\foreignlanguage{arabic}{فَطِيرَة}}\ {\color{gray}\texttt{/\sffamily {{\sffamily fatˤiːra}}/}\color{black}}\ \textsc{noun}\ [f.]\ \color{gray}(msa. \foreignlanguage{arabic}{فَطِيرة}~\foreignlanguage{arabic}{\textbf{١.}})\color{black}\ \textbf{1.}~pastry  \textbf{2.}~pie\ \ $\bullet$\ \ \setlength\topsep{0pt}\textbf{\foreignlanguage{arabic}{فَطَايِر}}\ {\color{gray}\texttt{/\sffamily {{\sffamily fatˤaːjir}}/}\color{black}}\ [pl.]\  \begin{flushright}\color{gray}\foreignlanguage{arabic}{\textbf{\underline{\foreignlanguage{arabic}{أمثلة}}}: بدي أخبز الفَطايِر لحالي}\end{flushright}\color{black}} \vspace{2mm}

{\setlength\topsep{0pt}\textbf{\foreignlanguage{arabic}{فَطَّر}}\ {\color{gray}\texttt{/\sffamily {{\sffamily fatˤtˤar}}/}\color{black}}\ \textsc{verb}\ [p.]\ \textbf{1.}~serve breakfast to sb (causative).  \textbf{2.}~make sb break his fast (causative)\ \ $\bullet$\ \ \setlength\topsep{0pt}\textbf{\foreignlanguage{arabic}{فَطِّر}}\ {\color{gray}\texttt{/\sffamily {{\sffamily fatˤtˤir}}/}\color{black}}\ [c.]\ \ $\bullet$\ \ \setlength\topsep{0pt}\textbf{\foreignlanguage{arabic}{يفَطِّر}}\ {\color{gray}\texttt{/\sffamily {{\sffamily jfatˤtˤir}}/}\color{black}}\ [i.]\  \begin{flushright}\color{gray}\foreignlanguage{arabic}{\textbf{\underline{\foreignlanguage{arabic}{أمثلة}}}: أنو اللي قال انه العطر بيفَطِّر برمضان\ $\bullet$\ \  فَطِّر العمال حرام عليك من الطبح علحم بطنهم}\end{flushright}\color{black}} \vspace{2mm}

{\setlength\topsep{0pt}\textbf{\foreignlanguage{arabic}{فِطِر}}\ {\color{gray}\texttt{/\sffamily {{\sffamily fitˤir}}/}\color{black}}\ \textsc{verb}\ [p.]\ \textbf{1.}~have breakfast.  \textbf{2.}~break sb's fast\ \ $\bullet$\ \ \setlength\topsep{0pt}\textbf{\foreignlanguage{arabic}{اِفْطَر}}\ {\color{gray}\texttt{/\sffamily {{\sffamily ʔiftˤar}}/}\color{black}}\ [c.]\ \ $\bullet$\ \ \setlength\topsep{0pt}\textbf{\foreignlanguage{arabic}{يِفْطَر}}\ {\color{gray}\texttt{/\sffamily {{\sffamily jiftˤar}}/}\color{black}}\ [i.]\ \color{gray}(msa. \foreignlanguage{arabic}{يُفْطِر في رمضان}~\foreignlanguage{arabic}{\textbf{٢.}}  .\foreignlanguage{arabic}{يُفْطِر في الصباح}~\foreignlanguage{arabic}{\textbf{١.}})\color{black}\  \begin{flushright}\color{gray}\foreignlanguage{arabic}{\textbf{\underline{\foreignlanguage{arabic}{أمثلة}}}: اِفْطَر بسرعة وتعالي الحقني\ $\bullet$\ \  ما قدرت أضلني صايمة ففْطَرت يارب سامحنس}\end{flushright}\color{black}} \vspace{2mm}

{\setlength\topsep{0pt}\textbf{\foreignlanguage{arabic}{فْطُور}}\ {\color{gray}\texttt{/\sffamily {{\sffamily ftˤuːr}}/}\color{black}}\ \textsc{noun}\ [m.]\ \color{gray}(msa. \foreignlanguage{arabic}{اِفْطار الصباح}~\foreignlanguage{arabic}{\textbf{١.}})\color{black}\ \textbf{1.}~breakfast\ } \vspace{2mm}

{\setlength\topsep{0pt}\textbf{\foreignlanguage{arabic}{فْطِير}}\ {\color{gray}\texttt{/\sffamily {{\sffamily ftˤiːr}}/}\color{black}}\ \textsc{noun}\ [m.]\ \textbf{1.}~a type of pie\ \ $\bullet$\ \ \textsc{ph.} \color{gray} \foreignlanguage{arabic}{الخمير وَالفطير}\color{black}\ {\color{gray}\texttt{/{\sffamily ʔilxamiːr wiliftˤiːr}/}\color{black}}\ \color{gray} (msa. \foreignlanguage{arabic}{يوسع شخص ضربا مبرحا}~\foreignlanguage{arabic}{\textbf{١.}})\color{black}\ \textbf{1.}~It is an idiomatic expression that is sarcastically used to refer to sb who beats the hell out of someone else\  \begin{flushright}\color{gray}\foreignlanguage{arabic}{\textbf{\underline{\foreignlanguage{arabic}{أمثلة}}}: قدحه قتلة طالع منُّه الخَمِير والفْطِير}\end{flushright}\color{black}} \vspace{2mm}

\vspace{-3mm}
\markboth{\color{blue}\foreignlanguage{arabic}{ف.ط.س}\color{blue}{}}{\color{blue}\foreignlanguage{arabic}{ف.ط.س}\color{blue}{}}\subsection*{\color{blue}\foreignlanguage{arabic}{ف.ط.س}\color{blue}{}\index{\color{blue}\foreignlanguage{arabic}{ف.ط.س}\color{blue}{}}} 

{\setlength\topsep{0pt}\textbf{\foreignlanguage{arabic}{فَاطِس}}\ {\color{gray}\texttt{/\sffamily {{\sffamily faːtˤis}}/}\color{black}}\ \textsc{adj}\ [m.]\ \textbf{1.}~an impolite way of referring to the deceased person\ \ $\bullet$\ \ \textsc{ph.} \color{gray} \foreignlanguage{arabic}{لعن فَاطسه}\color{black}\ {\color{gray}\texttt{/{\sffamily laʕan faːtˤso}/}\color{black}}\ \textbf{1.}~beat the hell out of sb\  \begin{flushright}\color{gray}\foreignlanguage{arabic}{\textbf{\underline{\foreignlanguage{arabic}{أمثلة}}}: أحمد مسكه ولعن فاطسه لحد ما عرف ان الله حق}\end{flushright}\color{black}} \vspace{2mm}

{\setlength\topsep{0pt}\textbf{\foreignlanguage{arabic}{فَطَس}}\footnote{Disapproving}\ \ {\color{gray}\texttt{/\sffamily {{\sffamily fatˤas}}/}\color{black}}\ \textsc{verb}\ [p.]\ \textbf{1.}~die  \textbf{2.}~kick the bucket\ \ $\bullet$\ \ \setlength\topsep{0pt}\textbf{\foreignlanguage{arabic}{اِفْطُس}}\ {\color{gray}\texttt{/\sffamily {{\sffamily ʔuftˤus}}/}\color{black}}\ [c.]\ \ $\bullet$\ \ \setlength\topsep{0pt}\textbf{\foreignlanguage{arabic}{يُفْطُس}}\ {\color{gray}\texttt{/\sffamily {{\sffamily juftˤus}}/}\color{black}}\ [i.]\ \color{gray}(msa. \foreignlanguage{arabic}{يَموت}~\foreignlanguage{arabic}{\textbf{١.}})\color{black}\ \ $\bullet$\ \ \textsc{ph.} \color{gray} \foreignlanguage{arabic}{فطسنَا من الضحك}\color{black}\ {\color{gray}\texttt{/{\sffamily fatˤasna min ʔi(dˤ)(dˤ)eħik}/}\color{black}}\ \color{gray} (msa. \foreignlanguage{arabic}{يضحك بطريقة هستيرية}~\foreignlanguage{arabic}{\textbf{١.}})\color{black}\ \textbf{1.}~to laugh hysterically\  \begin{flushright}\color{gray}\foreignlanguage{arabic}{\textbf{\underline{\foreignlanguage{arabic}{أمثلة}}}: تف على عمه وبعدها كلنا فَطَسْنا من الضِّحِك\ $\bullet$\ \  افْطُس من الحر أحسن\ $\bullet$\ \  كلب و فَطَس والله لا يرحمه}\end{flushright}\color{black}} \vspace{2mm}

{\setlength\topsep{0pt}\textbf{\foreignlanguage{arabic}{فَطَّس}}\ {\color{gray}\texttt{/\sffamily {{\sffamily fatˤtˤas}}/}\color{black}}\ \textsc{verb}\ [p.]\ \textbf{1.}~rot  \textbf{2.}~stink\ \ $\bullet$\ \ \setlength\topsep{0pt}\textbf{\foreignlanguage{arabic}{فَطِّس}}\ {\color{gray}\texttt{/\sffamily {{\sffamily fatˤtˤis}}/}\color{black}}\ [c.]\ \ $\bullet$\ \ \setlength\topsep{0pt}\textbf{\foreignlanguage{arabic}{يفَطِّس}}\ {\color{gray}\texttt{/\sffamily {{\sffamily jfatˤtˤis}}/}\color{black}}\ [i.]\  \begin{flushright}\color{gray}\foreignlanguage{arabic}{\textbf{\underline{\foreignlanguage{arabic}{أمثلة}}}: ضل الرز برة أسبوع لفَطَّس يعني أوسخ وأقرف من هيك مرة بحياتي ما أريت}\end{flushright}\color{black}} \vspace{2mm}

{\setlength\topsep{0pt}\textbf{\foreignlanguage{arabic}{فْطِيسِة}}\ {\color{gray}\texttt{/\sffamily {{\sffamily ftˤiːse}}/}\color{black}}\ \textsc{noun}\ [f.]\ \color{gray}(msa. \foreignlanguage{arabic}{جيفة متفسخة}~\foreignlanguage{arabic}{\textbf{١.}})\color{black}\ \textbf{1.}~carrion  \textbf{2.}~any stinky or dirty thing\ \ $\bullet$\ \ \setlength\topsep{0pt}\textbf{\foreignlanguage{arabic}{فَطَايِس}}\ {\color{gray}\texttt{/\sffamily {{\sffamily fatˤaːjis}}/}\color{black}}\ [pl.]\  \begin{flushright}\color{gray}\foreignlanguage{arabic}{\textbf{\underline{\foreignlanguage{arabic}{أمثلة}}}: تعالوا لموا فَطايِسكم!\ $\bullet$\ \  شكلها فْطِيسِة متروكة ريحة الغرفة بتقتل}\end{flushright}\color{black}} \vspace{2mm}

{\setlength\topsep{0pt}\textbf{\foreignlanguage{arabic}{مْفَطِّس}}\ {\color{gray}\texttt{/\sffamily {{\sffamily mfatˤtˤis}}/}\color{black}}\ \textsc{adj}\ [m.]\ \textbf{1.}~rotten  \textbf{2.}~stinking\  \begin{flushright}\color{gray}\foreignlanguage{arabic}{\textbf{\underline{\foreignlanguage{arabic}{أمثلة}}}: يرحم جدك يامْفَطِّس! هلا بطلت الخُبِّيزة من مستواك!}\end{flushright}\color{black}} \vspace{2mm}

\vspace{-3mm}
\markboth{\color{blue}\foreignlanguage{arabic}{ف.ط.ع}\color{blue}{}}{\color{blue}\foreignlanguage{arabic}{ف.ط.ع}\color{blue}{}}\subsection*{\color{blue}\foreignlanguage{arabic}{ف.ط.ع}\color{blue}{}\index{\color{blue}\foreignlanguage{arabic}{ف.ط.ع}\color{blue}{}}} 

{\setlength\topsep{0pt}\textbf{\foreignlanguage{arabic}{فُطْعَة}}\ {\color{gray}\texttt{/\sffamily {{\sffamily futˤʕa}}/}\color{black}}\ \textsc{adj/noun}\ (src. \color{gray}\foreignlanguage{arabic}{الضفة الغربية}\color{black})\ \color{gray}(msa. \foreignlanguage{arabic}{قصيرة}~\foreignlanguage{arabic}{\textbf{١.}})\color{black}\ \textbf{1.}~short\  \begin{flushright}\color{gray}\foreignlanguage{arabic}{\textbf{\underline{\foreignlanguage{arabic}{أمثلة}}}: شوف هالفطعة ما اقصرها}\end{flushright}\color{black}} \vspace{2mm}

\vspace{-3mm}
\markboth{\color{blue}\foreignlanguage{arabic}{ف.ط.ف.ط}\color{blue}{}}{\color{blue}\foreignlanguage{arabic}{ف.ط.ف.ط}\color{blue}{}}\subsection*{\color{blue}\foreignlanguage{arabic}{ف.ط.ف.ط}\color{blue}{}\index{\color{blue}\foreignlanguage{arabic}{ف.ط.ف.ط}\color{blue}{}}} 

{\setlength\topsep{0pt}\textbf{\foreignlanguage{arabic}{فَطْفَط}}\ {\color{gray}\texttt{/\sffamily {{\sffamily fatˤfatˤ}}/}\color{black}}\ \textsc{verb}\ [p.]\ \textbf{1.}~react towards sth and jump up and down hesterically\ \ $\bullet$\ \ \setlength\topsep{0pt}\textbf{\foreignlanguage{arabic}{فَطْفِط}}\ {\color{gray}\texttt{/\sffamily {{\sffamily fatˤfitˤ}}/}\color{black}}\ [c.]\ \ $\bullet$\ \ \setlength\topsep{0pt}\textbf{\foreignlanguage{arabic}{يفَطْفِط}}\ {\color{gray}\texttt{/\sffamily {{\sffamily jfatˤfitˤ}}/}\color{black}}\ [i.]\  \begin{flushright}\color{gray}\foreignlanguage{arabic}{\textbf{\underline{\foreignlanguage{arabic}{أمثلة}}}: لما حدا يجيله سيرة الصلاة بنجن وبنحن وبصير يفَطْفِط}\end{flushright}\color{black}} \vspace{2mm}

{\setlength\topsep{0pt}\textbf{\foreignlanguage{arabic}{فَطْفَطَة}}\ {\color{gray}\texttt{/\sffamily {{\sffamily fatˤfatˤa}}/}\color{black}}\ \textsc{noun}\ [f.]\ \textbf{1.}~reacting towards sth and jumping up and down hesterically\  \begin{flushright}\color{gray}\foreignlanguage{arabic}{\textbf{\underline{\foreignlanguage{arabic}{أمثلة}}}: بطلي إِياها الجنون والفَطْفَطَة تبعك. كل ماحدا يحكيلك شي بتنجن وبتصير تفَطْفِط}\end{flushright}\color{black}} \vspace{2mm}

\vspace{-3mm}
\markboth{\color{blue}\foreignlanguage{arabic}{ف.ط.م}\color{blue}{}}{\color{blue}\foreignlanguage{arabic}{ف.ط.م}\color{blue}{}}\subsection*{\color{blue}\foreignlanguage{arabic}{ف.ط.م}\color{blue}{}\index{\color{blue}\foreignlanguage{arabic}{ف.ط.م}\color{blue}{}}} 

{\setlength\topsep{0pt}\textbf{\foreignlanguage{arabic}{اِنْفَطَم}}\ {\color{gray}\texttt{/\sffamily {{\sffamily ʔinfatˤam}}/}\color{black}}\ \textsc{verb}\ [p.]\ \textbf{1.}~be weaned\ \ $\bullet$\ \ \setlength\topsep{0pt}\textbf{\foreignlanguage{arabic}{اِنْفِطِم}}\ {\color{gray}\texttt{/\sffamily {{\sffamily ʔinfitˤim}}/}\color{black}}\ [c.]\ \ $\bullet$\ \ \setlength\topsep{0pt}\textbf{\foreignlanguage{arabic}{يِنْفِطِم}}\ {\color{gray}\texttt{/\sffamily {{\sffamily jinfitˤim}}/}\color{black}}\ [i.]\ \color{gray}(msa. \foreignlanguage{arabic}{يُفْطُم}~\foreignlanguage{arabic}{\textbf{١.}})\color{black}\  \begin{flushright}\color{gray}\foreignlanguage{arabic}{\textbf{\underline{\foreignlanguage{arabic}{أمثلة}}}: حرام توقفي الحليب عنه فجأةاستني عليه يِنْفِطِم شوي شوي}\end{flushright}\color{black}} \vspace{2mm}

{\setlength\topsep{0pt}\textbf{\foreignlanguage{arabic}{فَطَم}}\ {\color{gray}\texttt{/\sffamily {{\sffamily fatˤam}}/}\color{black}}\ \textsc{verb}\ [p.]\ \textbf{1.}~wean\ \ $\bullet$\ \ \setlength\topsep{0pt}\textbf{\foreignlanguage{arabic}{اُفْطُم}}\ {\color{gray}\texttt{/\sffamily {{\sffamily ʔiftˤum}}/}\color{black}}\ [c.]\ \ $\bullet$\ \ \setlength\topsep{0pt}\textbf{\foreignlanguage{arabic}{يِفْطُم}}\ {\color{gray}\texttt{/\sffamily {{\sffamily jiftˤum}}/}\color{black}}\ [i.]\ \color{gray}(msa. \foreignlanguage{arabic}{يَفْطُم}~\foreignlanguage{arabic}{\textbf{١.}})\color{black}\  \begin{flushright}\color{gray}\foreignlanguage{arabic}{\textbf{\underline{\foreignlanguage{arabic}{أمثلة}}}: اُفْطُميه عالسنة ونص مش قبل هيك حرام}\end{flushright}\color{black}} \vspace{2mm}

{\setlength\topsep{0pt}\textbf{\foreignlanguage{arabic}{مَفْطُوم}}\ {\color{gray}\texttt{/\sffamily {{\sffamily maftˤuːm}}/}\color{black}}\ \textsc{adj}\ [m.]\ \color{gray}(msa. \foreignlanguage{arabic}{مَفْطوم}~\foreignlanguage{arabic}{\textbf{١.}})\color{black}\ \textbf{1.}~weaned\  \begin{flushright}\color{gray}\foreignlanguage{arabic}{\textbf{\underline{\foreignlanguage{arabic}{أمثلة}}}: ابنك مَفْطوم صح؟}\end{flushright}\color{black}} \vspace{2mm}

\vspace{-3mm}
\markboth{\color{blue}\foreignlanguage{arabic}{ف.ط.ن}\color{blue}{}}{\color{blue}\foreignlanguage{arabic}{ف.ط.ن}\color{blue}{}}\subsection*{\color{blue}\foreignlanguage{arabic}{ف.ط.ن}\color{blue}{}\index{\color{blue}\foreignlanguage{arabic}{ف.ط.ن}\color{blue}{}}} 

{\setlength\topsep{0pt}\textbf{\foreignlanguage{arabic}{تْفَطَّن}}\ {\color{gray}\texttt{/\sffamily {{\sffamily tfatˤtˤan}}/}\color{black}}\ \textsc{verb}\ [p.]\ \textbf{1.}~remember (make sn effort to remember)\ \ $\bullet$\ \ \setlength\topsep{0pt}\textbf{\foreignlanguage{arabic}{اِتْفَطَّن}}\ {\color{gray}\texttt{/\sffamily {{\sffamily ʔitfatˤtˤan}}/}\color{black}}\ [c.]\ \ $\bullet$\ \ \setlength\topsep{0pt}\textbf{\foreignlanguage{arabic}{يِتْفَطَّن}}\ {\color{gray}\texttt{/\sffamily {{\sffamily jitfatˤtˤan}}/}\color{black}}\ [i.]\ \color{gray}(msa. \foreignlanguage{arabic}{يتذكَّر}~\foreignlanguage{arabic}{\textbf{١.}})\color{black}\  \begin{flushright}\color{gray}\foreignlanguage{arabic}{\textbf{\underline{\foreignlanguage{arabic}{أمثلة}}}: لِسَّة بدكم اياني أستنَّى عليه يِتْفَطَّن شو كان لابس هذاك اليوم}\end{flushright}\color{black}} \vspace{2mm}

{\setlength\topsep{0pt}\textbf{\foreignlanguage{arabic}{فَاطِن}}\ {\color{gray}\texttt{/\sffamily {{\sffamily faːtˤin}}/}\color{black}}\ \textsc{noun\textunderscore act}\ [m.]\ \textbf{1.}~remembering\  \begin{flushright}\color{gray}\foreignlanguage{arabic}{\textbf{\underline{\foreignlanguage{arabic}{أمثلة}}}: مش فاطِن أي يوم بقوا عنا}\end{flushright}\color{black}} \vspace{2mm}

{\setlength\topsep{0pt}\textbf{\foreignlanguage{arabic}{فَطِين}}\ {\color{gray}\texttt{/\sffamily {{\sffamily fatˤiːn}}/}\color{black}}\ \textsc{adj}\ [m.]\ \color{gray}(msa. \foreignlanguage{arabic}{حَذِق}~\foreignlanguage{arabic}{\textbf{٢.}}  \foreignlanguage{arabic}{ذكي}~\foreignlanguage{arabic}{\textbf{١.}})\color{black}\ \textbf{1.}~smart  \textbf{2.}~astute\ } \vspace{2mm}

{\setlength\topsep{0pt}\textbf{\foreignlanguage{arabic}{فَطَّن}}\ {\color{gray}\texttt{/\sffamily {{\sffamily fatˤtˤan}}/}\color{black}}\ \textsc{verb}\ [p.]\ \textbf{1.}~remind\ \ $\bullet$\ \ \setlength\topsep{0pt}\textbf{\foreignlanguage{arabic}{فَطِّن}}\ {\color{gray}\texttt{/\sffamily {{\sffamily fatˤtˤin}}/}\color{black}}\ [c.]\ \ $\bullet$\ \ \setlength\topsep{0pt}\textbf{\foreignlanguage{arabic}{يفَطِّن}}\ {\color{gray}\texttt{/\sffamily {{\sffamily jfatˤtˤin}}/}\color{black}}\ [i.]\ \color{gray}(msa. \foreignlanguage{arabic}{يُذَكِّر}~\foreignlanguage{arabic}{\textbf{١.}})\color{black}\  \begin{flushright}\color{gray}\foreignlanguage{arabic}{\textbf{\underline{\foreignlanguage{arabic}{أمثلة}}}: فَطِّني بكرة عشان موضوع أرض بلعا}\end{flushright}\color{black}} \vspace{2mm}

{\setlength\topsep{0pt}\textbf{\foreignlanguage{arabic}{فِطِن}}\ {\color{gray}\texttt{/\sffamily {{\sffamily fitˤin}}/}\color{black}}\ \textsc{adj}\ [m.]\ \color{gray}(msa. \foreignlanguage{arabic}{حَذِق}~\foreignlanguage{arabic}{\textbf{٢.}}  \foreignlanguage{arabic}{ذكي}~\foreignlanguage{arabic}{\textbf{١.}})\color{black}\ \textbf{1.}~smart  \textbf{2.}~astute\ } \vspace{2mm}

{\setlength\topsep{0pt}\textbf{\foreignlanguage{arabic}{فِطِن}}\ {\color{gray}\texttt{/\sffamily {{\sffamily fitˤin}}/}\color{black}}\ \textsc{verb}\ [p.]\ \textbf{1.}~remember (no effort)\ \ $\bullet$\ \ \setlength\topsep{0pt}\textbf{\foreignlanguage{arabic}{اِفْطَن}}\ {\color{gray}\texttt{/\sffamily {{\sffamily ʔiftˤan}}/}\color{black}}\ [c.]\ \ $\bullet$\ \ \setlength\topsep{0pt}\textbf{\foreignlanguage{arabic}{يِفْطَن}}\ {\color{gray}\texttt{/\sffamily {{\sffamily jiftˤan}}/}\color{black}}\ [i.]\ \color{gray}(msa. \foreignlanguage{arabic}{يتذكَّر}~\foreignlanguage{arabic}{\textbf{١.}})\color{black}\  \begin{flushright}\color{gray}\foreignlanguage{arabic}{\textbf{\underline{\foreignlanguage{arabic}{أمثلة}}}: ما فْطِنتش الا بعد ماروَّحت من عنا}\end{flushright}\color{black}} \vspace{2mm}

\vspace{-3mm}
\markboth{\color{blue}\foreignlanguage{arabic}{ف.ظ.ع}\color{blue}{}}{\color{blue}\foreignlanguage{arabic}{ف.ظ.ع}\color{blue}{}}\subsection*{\color{blue}\foreignlanguage{arabic}{ف.ظ.ع}\color{blue}{}\index{\color{blue}\foreignlanguage{arabic}{ف.ظ.ع}\color{blue}{}}} 

{\setlength\topsep{0pt}\textbf{\foreignlanguage{arabic}{اِسْتَفْظَع}}\ {\color{gray}\texttt{/\sffamily {{\sffamily ʔistafðˤaʕ}}/}\color{black}}\ \textsc{verb}\ [p.]\ \textbf{1.}~consider sth as too horrible an unacceptable\ \ $\bullet$\ \ \setlength\topsep{0pt}\textbf{\foreignlanguage{arabic}{اِسْتَفْظِع}}\ {\color{gray}\texttt{/\sffamily {{\sffamily ʔistafðˤiʕ}}/}\color{black}}\ [c.]\ \ $\bullet$\ \ \setlength\topsep{0pt}\textbf{\foreignlanguage{arabic}{يِسْتَفْظِع}}\ {\color{gray}\texttt{/\sffamily {{\sffamily jistafðˤiʕ}}/}\color{black}}\ [i.]\  \begin{flushright}\color{gray}\foreignlanguage{arabic}{\textbf{\underline{\foreignlanguage{arabic}{أمثلة}}}: للأمانة اِسْتَفْظَعِت المنظر وطلعت عطول يادوب سلمت عام العريس وماقدرت استحمل التشليح والخلاعة اللي كانو هناك بالعرس}\end{flushright}\color{black}} \vspace{2mm}

{\setlength\topsep{0pt}\textbf{\foreignlanguage{arabic}{فَظَاعَة}}\ {\color{gray}\texttt{/\sffamily {{\sffamily fa(ðˤ)aːʕa}}/}\color{black}}\ \textsc{noun}\ [f.]\ \textbf{1.}~the state of being too horrible\  \begin{flushright}\color{gray}\foreignlanguage{arabic}{\textbf{\underline{\foreignlanguage{arabic}{أمثلة}}}: ما استحملتش فَظاعَة المنظر عشان هيك هربت}\end{flushright}\color{black}} \vspace{2mm}

{\setlength\topsep{0pt}\textbf{\foreignlanguage{arabic}{فَظِيع}}\ {\color{gray}\texttt{/\sffamily {{\sffamily faðˤiːʕ}}/}\color{black}}\ \textsc{adj}\ [m.]\ \color{gray}(msa. \foreignlanguage{arabic}{فَظِيع}~\foreignlanguage{arabic}{\textbf{١.}})\color{black}\ \textbf{1.}~horrible  \textbf{2.}~magnificient!  \textbf{3.}~wonderful!  \textbf{4.}~awesome!\ \ $\smblkdiamond$\ \ \setlength\topsep{0pt}\textbf{\foreignlanguage{arabic}{فَظِيع}}\ {\color{gray}\texttt{/fazˤiːʕ/}\color{black}}\ \textbf{1.}~magnificient!  \textbf{2.}~wonderful!  \textbf{3.}~awesome!\  \begin{flushright}\color{gray}\foreignlanguage{arabic}{\textbf{\underline{\foreignlanguage{arabic}{أمثلة}}}: فَظِيعة هالبنت! كل شي شاطرة فيه\ $\bullet$\ \  منظر الجثث فَظِيع جداً}\end{flushright}\color{black}} \vspace{2mm}

{\setlength\topsep{0pt}\textbf{\foreignlanguage{arabic}{فَظَّع}}\ {\color{gray}\texttt{/\sffamily {{\sffamily fa(ðˤ)(ðˤ)aʕ}}/}\color{black}}\ \textsc{verb}\ [p.]\ \textbf{1.}~make atrocities.  \textbf{2.}~be very mean to sb and harm him.  \textbf{3.}~get dressed immodestly\ \ $\bullet$\ \ \setlength\topsep{0pt}\textbf{\foreignlanguage{arabic}{فَظِّع}}\ {\color{gray}\texttt{/\sffamily {{\sffamily fa(ðˤ)(ðˤ)iʕ}}/}\color{black}}\ [c.]\ \ $\bullet$\ \ \setlength\topsep{0pt}\textbf{\foreignlanguage{arabic}{يفَظِّع}}\ {\color{gray}\texttt{/\sffamily {{\sffamily jfa(ðˤ)(ðˤ)iʕ}}/}\color{black}}\ [i.]\  \begin{flushright}\color{gray}\foreignlanguage{arabic}{\textbf{\underline{\foreignlanguage{arabic}{أمثلة}}}: دايما بتفظِّع باللبس تخافيش عليها\ $\bullet$\ \  فَظعي معاه عشان تيجي منه وهو اللي يطلقك ويخلص\ $\bullet$\ \  فَظَّعوا اليهود معنا والله شبحوا عشر شباب من قدامي وصفوهم}\end{flushright}\color{black}} \vspace{2mm}

{\setlength\topsep{0pt}\textbf{\foreignlanguage{arabic}{مِسْتَفْظِع}}\ {\color{gray}\texttt{/\sffamily {{\sffamily mistafðˤiʕ}}/}\color{black}}\ \textsc{noun\textunderscore act}\ [m.]\ \textbf{1.}~considering sth as too horrible an unacceptable\  \begin{flushright}\color{gray}\foreignlanguage{arabic}{\textbf{\underline{\foreignlanguage{arabic}{أمثلة}}}: بقيت بالأول مِسْتَفْظِع منظر الشجر المقطوع هلا ياريت وقف عقطع الشجر. صايرين يحرقوا بهالأحراش هالبناديق}\end{flushright}\color{black}} \vspace{2mm}

{\setlength\topsep{0pt}\textbf{\foreignlanguage{arabic}{مْفَظِّع}}\ {\color{gray}\texttt{/\sffamily {{\sffamily mfa(ðˤ)(ðˤ)iʕ}}/}\color{black}}\ \textsc{noun\textunderscore act}\ [m.]\ \textbf{1.}~making atrocities.  \textbf{2.}~being very mean to sb and harmin him.  \textbf{3.}~getting dressed immodestly\  \begin{flushright}\color{gray}\foreignlanguage{arabic}{\textbf{\underline{\foreignlanguage{arabic}{أمثلة}}}: هذا زيد باقي مْفَظِّع مع ولاد عمه وأصحابه}\end{flushright}\color{black}} \vspace{2mm}

\vspace{-3mm}
\markboth{\color{blue}\foreignlanguage{arabic}{ف.ع.ت.ش}\color{blue}{}}{\color{blue}\foreignlanguage{arabic}{ف.ع.ت.ش}\color{blue}{}}\subsection*{\color{blue}\foreignlanguage{arabic}{ف.ع.ت.ش}\color{blue}{}\index{\color{blue}\foreignlanguage{arabic}{ف.ع.ت.ش}\color{blue}{}}} 

{\setlength\topsep{0pt}\textbf{\foreignlanguage{arabic}{فَعْتَش}}\ {\color{gray}\texttt{/\sffamily {{\sffamily faʕtaʃ}}/}\color{black}}\ \textsc{verb}\ [p.]\ \textbf{1.}~rummage through.  \textbf{2.}~search for sth\ \ $\bullet$\ \ \setlength\topsep{0pt}\textbf{\foreignlanguage{arabic}{فَعْتِش}}\ {\color{gray}\texttt{/\sffamily {{\sffamily faʕtiʃ}}/}\color{black}}\ [c.]\ \ $\bullet$\ \ \setlength\topsep{0pt}\textbf{\foreignlanguage{arabic}{يفَعْتِش}}\ {\color{gray}\texttt{/\sffamily {{\sffamily jfaʕtiʃ}}/}\color{black}}\ [i.]\  \begin{flushright}\color{gray}\foreignlanguage{arabic}{\textbf{\underline{\foreignlanguage{arabic}{أمثلة}}}: فَعْتِش عليها هون ولا هون بلكي بتلاقيها}\end{flushright}\color{black}} \vspace{2mm}

{\setlength\topsep{0pt}\textbf{\foreignlanguage{arabic}{فَعْتَشِة}}\ {\color{gray}\texttt{/\sffamily {{\sffamily faʕtaʃe}}/}\color{black}}\ \textsc{noun}\ [f.]\ \textbf{1.}~rummaging through.  \textbf{2.}~searching for sth\ } \vspace{2mm}

{\setlength\topsep{0pt}\textbf{\foreignlanguage{arabic}{مْفَعْتِش}}\ {\color{gray}\texttt{/\sffamily {{\sffamily mfaʕtiʃ}}/}\color{black}}\ \textsc{noun\textunderscore act}\ [m.]\ \textbf{1.}~rummaging through.  \textbf{2.}~searching for sth\  \begin{flushright}\color{gray}\foreignlanguage{arabic}{\textbf{\underline{\foreignlanguage{arabic}{أمثلة}}}: بقى مْفَعْتِش عليها بكل مكان بس عالفاضي أبصر مين ماخذها}\end{flushright}\color{black}} \vspace{2mm}

\vspace{-3mm}
\markboth{\color{blue}\foreignlanguage{arabic}{ف.ع.ص}\color{blue}{}}{\color{blue}\foreignlanguage{arabic}{ف.ع.ص}\color{blue}{}}\subsection*{\color{blue}\foreignlanguage{arabic}{ف.ع.ص}\color{blue}{}\index{\color{blue}\foreignlanguage{arabic}{ف.ع.ص}\color{blue}{}}} 

{\setlength\topsep{0pt}\textbf{\foreignlanguage{arabic}{اِنْفَعَص}}\ {\color{gray}\texttt{/\sffamily {{\sffamily ʔinfaʕasˤ}}/}\color{black}}\ \textsc{verb}\ [p.]\ \textbf{1.}~be squashed.  \textbf{2.}~be crushed\ \ $\bullet$\ \ \setlength\topsep{0pt}\textbf{\foreignlanguage{arabic}{اِنْفِعِص}}\ {\color{gray}\texttt{/\sffamily {{\sffamily ʔinfiʕisˤ}}/}\color{black}}\ [c.]\ \ $\bullet$\ \ \setlength\topsep{0pt}\textbf{\foreignlanguage{arabic}{اِنِفْعِص}}\ {\color{gray}\texttt{/\sffamily {{\sffamily ʔinifʕisˤ}}/}\color{black}}\ [c.]\ \ $\bullet$\ \ \setlength\topsep{0pt}\textbf{\foreignlanguage{arabic}{يِنْفِعِص}}\ {\color{gray}\texttt{/\sffamily {{\sffamily jinfiʕisˤ}}/}\color{black}}\ [i.]\ \ $\bullet$\ \ \setlength\topsep{0pt}\textbf{\foreignlanguage{arabic}{يِنِفْعِص}}\ {\color{gray}\texttt{/\sffamily {{\sffamily jinifʕisˤ}}/}\color{black}}\ [i.]\  \begin{flushright}\color{gray}\foreignlanguage{arabic}{\textbf{\underline{\foreignlanguage{arabic}{أمثلة}}}: دير بالك ما تِنْفِعِص بوكسة البندورة}\end{flushright}\color{black}} \vspace{2mm}

{\setlength\topsep{0pt}\textbf{\foreignlanguage{arabic}{تْفَعَّص}}\ {\color{gray}\texttt{/\sffamily {{\sffamily tfaʕʕasˤ}}/}\color{black}}\ \textsc{verb}\ [p.]\ \textbf{1.}~be squashed.  \textbf{2.}~be crushed (repeatedly and with force)\ \ $\bullet$\ \ \setlength\topsep{0pt}\textbf{\foreignlanguage{arabic}{اِتْفَعَّص}}\ {\color{gray}\texttt{/\sffamily {{\sffamily ʔitfaʕʕasˤ}}/}\color{black}}\ [c.]\ \ $\bullet$\ \ \setlength\topsep{0pt}\textbf{\foreignlanguage{arabic}{يِتْفَعَّص}}\ {\color{gray}\texttt{/\sffamily {{\sffamily jitfaʕʕasˤ}}/}\color{black}}\ [i.]\  \begin{flushright}\color{gray}\foreignlanguage{arabic}{\textbf{\underline{\foreignlanguage{arabic}{أمثلة}}}: طلعت الموز لفوق عشان ما يِتْفَعَّص}\end{flushright}\color{black}} \vspace{2mm}

{\setlength\topsep{0pt}\textbf{\foreignlanguage{arabic}{تْفَعْوَص}}\ {\color{gray}\texttt{/\sffamily {{\sffamily tfaʕwasˤ}}/}\color{black}}\ \textsc{verb}\ [p.]\ \textbf{1.}~be squashed.  \textbf{2.}~be crushed (repeatedly)\ \ $\bullet$\ \ \setlength\topsep{0pt}\textbf{\foreignlanguage{arabic}{اِتْفَعْوَص}}\ {\color{gray}\texttt{/\sffamily {{\sffamily ʔitfaʕwasˤ}}/}\color{black}}\ [c.]\ \ $\bullet$\ \ \setlength\topsep{0pt}\textbf{\foreignlanguage{arabic}{يِتْفَعْوَص}}\ {\color{gray}\texttt{/\sffamily {{\sffamily jitfaʕwasˤ}}/}\color{black}}\ [i.]\  \begin{flushright}\color{gray}\foreignlanguage{arabic}{\textbf{\underline{\foreignlanguage{arabic}{أمثلة}}}: روح معهم واِتْفَعْوَص بالسيارة الله لايردك\ $\bullet$\ \  كل الفواكه اللي حطيتها بالكيس تْفَعْوَصت}\end{flushright}\color{black}} \vspace{2mm}

{\setlength\topsep{0pt}\textbf{\foreignlanguage{arabic}{فَعَص}}\ {\color{gray}\texttt{/\sffamily {{\sffamily faʕasˤ}}/}\color{black}}\ \textsc{verb}\ [p.]\ \textbf{1.}~squash  \textbf{2.}~crush\ \ $\bullet$\ \ \setlength\topsep{0pt}\textbf{\foreignlanguage{arabic}{اِفْعَص}}\ {\color{gray}\texttt{/\sffamily {{\sffamily ʔifʕasˤ}}/}\color{black}}\ [c.]\ \ $\bullet$\ \ \setlength\topsep{0pt}\textbf{\foreignlanguage{arabic}{يِفْعَص}}\ {\color{gray}\texttt{/\sffamily {{\sffamily jifʕasˤ}}/}\color{black}}\ [i.]\  \begin{flushright}\color{gray}\foreignlanguage{arabic}{\textbf{\underline{\foreignlanguage{arabic}{أمثلة}}}: لا تفعصها بدي أقطعها تقطيع}\end{flushright}\color{black}} \vspace{2mm}

{\setlength\topsep{0pt}\textbf{\foreignlanguage{arabic}{فَعَّص}}\ {\color{gray}\texttt{/\sffamily {{\sffamily faʕʕasˤ}}/}\color{black}}\ \textsc{verb}\ [p.]\ \textbf{1.}~squash  \textbf{2.}~crush (repeatedly and with force)\ \ $\bullet$\ \ \setlength\topsep{0pt}\textbf{\foreignlanguage{arabic}{فَعِّص}}\ {\color{gray}\texttt{/\sffamily {{\sffamily faʕʕisˤ}}/}\color{black}}\ [c.]\ \ $\bullet$\ \ \setlength\topsep{0pt}\textbf{\foreignlanguage{arabic}{يفَعِّص}}\ {\color{gray}\texttt{/\sffamily {{\sffamily jfaʕʕisˤ}}/}\color{black}}\ [i.]\  \begin{flushright}\color{gray}\foreignlanguage{arabic}{\textbf{\underline{\foreignlanguage{arabic}{أمثلة}}}: مسك البيضة وضله يفَعِّص فيها بايديه لحد ما وسخ الأرضيات}\end{flushright}\color{black}} \vspace{2mm}

{\setlength\topsep{0pt}\textbf{\foreignlanguage{arabic}{فَعْوَص}}\ {\color{gray}\texttt{/\sffamily {{\sffamily faʕwasˤ}}/}\color{black}}\ \textsc{verb}\ [p.]\ \textbf{1.}~squash  \textbf{2.}~crush (repeatedly)\ \ $\bullet$\ \ \setlength\topsep{0pt}\textbf{\foreignlanguage{arabic}{فَعْوِص}}\ {\color{gray}\texttt{/\sffamily {{\sffamily faʕwisˤ}}/}\color{black}}\ [c.]\ \ $\bullet$\ \ \setlength\topsep{0pt}\textbf{\foreignlanguage{arabic}{يفَعْوِص}}\ {\color{gray}\texttt{/\sffamily {{\sffamily jfaʕwisˤ}}/}\color{black}}\ [i.]\  \begin{flushright}\color{gray}\foreignlanguage{arabic}{\textbf{\underline{\foreignlanguage{arabic}{أمثلة}}}: امسك هالبندورة وفَعْوِصها عشان أحطها عالباميا}\end{flushright}\color{black}} \vspace{2mm}

{\setlength\topsep{0pt}\textbf{\foreignlanguage{arabic}{مَفْعُوص}}\ {\color{gray}\texttt{/\sffamily {{\sffamily mafʕuːsˤ}}/}\color{black}}\ \textsc{adj}\ [m.]\ \color{gray}(msa. \foreignlanguage{arabic}{طفل يتصرف ويتحدث كالبالغين}~\foreignlanguage{arabic}{\textbf{١.}})\color{black}\ \textbf{1.}~an adult-like kid who talks and behaves like grown-up people\ \ $\bullet$\ \ \setlength\topsep{0pt}\textbf{\foreignlanguage{arabic}{مَفَاعِيص}}\ {\color{gray}\texttt{/\sffamily {{\sffamily mafaːʕiːsˤ}}/}\color{black}}\ [pl.]\  \begin{flushright}\color{gray}\foreignlanguage{arabic}{\textbf{\underline{\foreignlanguage{arabic}{أمثلة}}}: بنتها مَفْعُوصِة لسة ما فقست من البيضة وبدها بلفون}\end{flushright}\color{black}} \vspace{2mm}

{\setlength\topsep{0pt}\textbf{\foreignlanguage{arabic}{مَفْعُوص}}\ {\color{gray}\texttt{/\sffamily {{\sffamily mafʕuːsˤ}}/}\color{black}}\ \textsc{noun\textunderscore pass}\ \textbf{1.}~squashed  \textbf{2.}~smashed  \textbf{3.}~mashed  \textbf{4.}~crushed\  \begin{flushright}\color{gray}\foreignlanguage{arabic}{\textbf{\underline{\foreignlanguage{arabic}{أمثلة}}}: كان المَفْعُوص مفعوص بشنطتي}\end{flushright}\color{black}} \vspace{2mm}

\vspace{-3mm}
\markboth{\color{blue}\foreignlanguage{arabic}{ف.ع.ط}\color{blue}{}}{\color{blue}\foreignlanguage{arabic}{ف.ع.ط}\color{blue}{}}\subsection*{\color{blue}\foreignlanguage{arabic}{ف.ع.ط}\color{blue}{}\index{\color{blue}\foreignlanguage{arabic}{ف.ع.ط}\color{blue}{}}} 

{\setlength\topsep{0pt}\textbf{\foreignlanguage{arabic}{فَاعِط}}\ {\color{gray}\texttt{/\sffamily {{\sffamily faːʕitˤ}}/}\color{black}}\ \textsc{noun\textunderscore act}\ [m.]\ \textbf{1.}~running away.  \textbf{2.}~jumping\  \begin{flushright}\color{gray}\foreignlanguage{arabic}{\textbf{\underline{\foreignlanguage{arabic}{أمثلة}}}: شافلك هالجردون بيهق بحاله وضله فاعِط}\end{flushright}\color{black}} \vspace{2mm}

{\setlength\topsep{0pt}\textbf{\foreignlanguage{arabic}{فَعَط}}\ {\color{gray}\texttt{/\sffamily {{\sffamily faʕatˤ}}/}\color{black}}\ \textsc{verb}\ [p.]\ \color{gray}(msa. \foreignlanguage{arabic}{يَقْفِز}~\foreignlanguage{arabic}{\textbf{٢.}}  \foreignlanguage{arabic}{يَهْرُب}~\foreignlanguage{arabic}{\textbf{١.}})\color{black}\ \textbf{1.}~run away.  \textbf{2.}~jump\ \ $\bullet$\ \ \setlength\topsep{0pt}\textbf{\foreignlanguage{arabic}{اِفْعَط}}\ {\color{gray}\texttt{/\sffamily {{\sffamily ʔifʕatˤ}}/}\color{black}}\ [c.]\ \ $\bullet$\ \ \setlength\topsep{0pt}\textbf{\foreignlanguage{arabic}{يِفْعَط}}\ {\color{gray}\texttt{/\sffamily {{\sffamily jifʕatˤ}}/}\color{black}}\ [i.]\  \begin{flushright}\color{gray}\foreignlanguage{arabic}{\textbf{\underline{\foreignlanguage{arabic}{أمثلة}}}: أول ما شاف جيب الجيش فَعَط بسرعة}\end{flushright}\color{black}} \vspace{2mm}

\vspace{-3mm}
\markboth{\color{blue}\foreignlanguage{arabic}{ف.ع.ع}\color{blue}{}}{\color{blue}\foreignlanguage{arabic}{ف.ع.ع}\color{blue}{}}\subsection*{\color{blue}\foreignlanguage{arabic}{ف.ع.ع}\color{blue}{}\index{\color{blue}\foreignlanguage{arabic}{ف.ع.ع}\color{blue}{}}} 

{\setlength\topsep{0pt}\textbf{\foreignlanguage{arabic}{فَاعِع}}\ {\color{gray}\texttt{/\sffamily {{\sffamily faːʕiʕ}}/}\color{black}}\ \textsc{noun\textunderscore act}\ [m.]\ \textbf{1.}~yelling at sb.  \textbf{2.}~telling sb off\  \begin{flushright}\color{gray}\foreignlanguage{arabic}{\textbf{\underline{\foreignlanguage{arabic}{أمثلة}}}: ليش فاعِع بأختي ولا؟}\end{flushright}\color{black}} \vspace{2mm}

{\setlength\topsep{0pt}\textbf{\foreignlanguage{arabic}{فَعّ}}\ {\color{gray}\texttt{/\sffamily {{\sffamily faʕʕ}}/}\color{black}}\ \textsc{verb}\ [p.]\ \textbf{1.}~yell at sb.  \textbf{2.}~tell sb off\ \ $\bullet$\ \ \setlength\topsep{0pt}\textbf{\foreignlanguage{arabic}{فُعّ}}\ {\color{gray}\texttt{/\sffamily {{\sffamily fuʕʕ}}/}\color{black}}\ [c.]\ \ $\bullet$\ \ \setlength\topsep{0pt}\textbf{\foreignlanguage{arabic}{فِعّ}}\ {\color{gray}\texttt{/\sffamily {{\sffamily fiʕʕ}}/}\color{black}}\ [c.]\ \ $\bullet$\ \ \setlength\topsep{0pt}\textbf{\foreignlanguage{arabic}{يفِعّ}}\ {\color{gray}\texttt{/\sffamily {{\sffamily jfiʕʕ}}/}\color{black}}\ [i.]\ \color{gray}(msa. \foreignlanguage{arabic}{يوبِّخ شخص}~\foreignlanguage{arabic}{\textbf{٢.}}  .\foreignlanguage{arabic}{يصرخ على شخص}~\foreignlanguage{arabic}{\textbf{١.}})\color{black}\ \ $\bullet$\ \ \setlength\topsep{0pt}\textbf{\foreignlanguage{arabic}{يفُعّ}}\ {\color{gray}\texttt{/\sffamily {{\sffamily jfuʕʕ}}/}\color{black}}\ [i.]\ \color{gray}(msa. \foreignlanguage{arabic}{يوبِّخ شخص}~\foreignlanguage{arabic}{\textbf{٢.}}  .\foreignlanguage{arabic}{يصرخ على شخص}~\foreignlanguage{arabic}{\textbf{١.}})\color{black}\  \begin{flushright}\color{gray}\foreignlanguage{arabic}{\textbf{\underline{\foreignlanguage{arabic}{أمثلة}}}: ماعملتله شي. ليش صار يفِع فيني؟\ $\bullet$\ \  فُعِّي بوجهه والله هاد الناقص يتنفتر فيك}\end{flushright}\color{black}} \vspace{2mm}

{\setlength\topsep{0pt}\textbf{\foreignlanguage{arabic}{فَعَّة}}\ {\color{gray}\texttt{/\sffamily {{\sffamily faʕʕa}}/}\color{black}}\ \textsc{noun}\ [f.]\ \textbf{1.}~yelling at sb.  \textbf{2.}~telling sb off.  \textbf{3.}~explosion\ } \vspace{2mm}

\vspace{-3mm}
\markboth{\color{blue}\foreignlanguage{arabic}{ف.ع.ف.ل}\color{blue}{}}{\color{blue}\foreignlanguage{arabic}{ف.ع.ف.ل}\color{blue}{}}\subsection*{\color{blue}\foreignlanguage{arabic}{ف.ع.ف.ل}\color{blue}{}\index{\color{blue}\foreignlanguage{arabic}{ف.ع.ف.ل}\color{blue}{}}} 

{\setlength\topsep{0pt}\textbf{\foreignlanguage{arabic}{تْفَعْفَل}}\ {\color{gray}\texttt{/\sffamily {{\sffamily tfaʕfal}}/}\color{black}}\ \textsc{verb}\ [p.]\ \textbf{1.}~burst out in anger.  \textbf{2.}~have tantrum\ \ $\bullet$\ \ \setlength\topsep{0pt}\textbf{\foreignlanguage{arabic}{تْفَعْفَل}}\ {\color{gray}\texttt{/\sffamily {{\sffamily tfaʕfal}}/}\color{black}}\ [c.]\ \ $\bullet$\ \ \setlength\topsep{0pt}\textbf{\foreignlanguage{arabic}{يِتْفَعْفَل}}\ {\color{gray}\texttt{/\sffamily {{\sffamily jitfaʕfal}}/}\color{black}}\ [i.]\ \color{gray}(msa. \foreignlanguage{arabic}{يَنْفَجِر غضباً}~\foreignlanguage{arabic}{\textbf{١.}})\color{black}\  \begin{flushright}\color{gray}\foreignlanguage{arabic}{\textbf{\underline{\foreignlanguage{arabic}{أمثلة}}}: لما إِمه أخذت منه اللعبة صار يِتْفَعْفَل عالأرض}\end{flushright}\color{black}} \vspace{2mm}

{\setlength\topsep{0pt}\textbf{\foreignlanguage{arabic}{فَعْفَل}}\ {\color{gray}\texttt{/\sffamily {{\sffamily faʕfal}}/}\color{black}}\ \textsc{verb}\ [p.]\ \textbf{1.}~to play with sand\ \ $\bullet$\ \ \setlength\topsep{0pt}\textbf{\foreignlanguage{arabic}{فَعْفِل}}\ {\color{gray}\texttt{/\sffamily {{\sffamily faʕfil}}/}\color{black}}\ [c.]\ \ $\bullet$\ \ \setlength\topsep{0pt}\textbf{\foreignlanguage{arabic}{يفَعْفِل}}\ {\color{gray}\texttt{/\sffamily {{\sffamily faʕfil}}/}\color{black}}\ [i.]\ \color{gray}(msa. \foreignlanguage{arabic}{يلعب بالتراب}~\foreignlanguage{arabic}{\textbf{١.}})\color{black}\  \begin{flushright}\color{gray}\foreignlanguage{arabic}{\textbf{\underline{\foreignlanguage{arabic}{أمثلة}}}: لقيته بعيد عن الأولاد بيفَعْفِل بالتراب}\end{flushright}\color{black}} \vspace{2mm}

\vspace{-3mm}
\markboth{\color{blue}\foreignlanguage{arabic}{ف.ع.ك.ش}\color{blue}{}}{\color{blue}\foreignlanguage{arabic}{ف.ع.ك.ش}\color{blue}{}}\subsection*{\color{blue}\foreignlanguage{arabic}{ف.ع.ك.ش}\color{blue}{}\index{\color{blue}\foreignlanguage{arabic}{ف.ع.ك.ش}\color{blue}{}}} 

{\setlength\topsep{0pt}\textbf{\foreignlanguage{arabic}{تْفَعْكَش}}\ {\color{gray}\texttt{/\sffamily {{\sffamily tfaʕkaʃ}}/}\color{black}}\ \textsc{verb}\ [p.]\ \textbf{1.}~become messy and untidy\ \ $\bullet$\ \ \setlength\topsep{0pt}\textbf{\foreignlanguage{arabic}{اِتْفَعْكَش}}\ {\color{gray}\texttt{/\sffamily {{\sffamily ʔitfaʕkaʃ}}/}\color{black}}\ [c.]\ \ $\bullet$\ \ \setlength\topsep{0pt}\textbf{\foreignlanguage{arabic}{يِتْفَعْكَش}}\ {\color{gray}\texttt{/\sffamily {{\sffamily jitfaʕkaʃ}}/}\color{black}}\ [i.]\  \begin{flushright}\color{gray}\foreignlanguage{arabic}{\textbf{\underline{\foreignlanguage{arabic}{أمثلة}}}: تْفَعْكَشت غرفتي بسببهم هالقرود}\end{flushright}\color{black}} \vspace{2mm}

{\setlength\topsep{0pt}\textbf{\foreignlanguage{arabic}{فَعْكَش}}\ {\color{gray}\texttt{/\sffamily {{\sffamily faʕkaʃ}}/}\color{black}}\ \textsc{verb}\ [p.]\ \textbf{1.}~make a place messy and untidy\ \ $\bullet$\ \ \setlength\topsep{0pt}\textbf{\foreignlanguage{arabic}{فَعْكِش}}\ {\color{gray}\texttt{/\sffamily {{\sffamily faʕkiʃ}}/}\color{black}}\ [c.]\ \ $\bullet$\ \ \setlength\topsep{0pt}\textbf{\foreignlanguage{arabic}{يفَعْكِش}}\ {\color{gray}\texttt{/\sffamily {{\sffamily jfaʕkiʃ}}/}\color{black}}\ [i.]\  \begin{flushright}\color{gray}\foreignlanguage{arabic}{\textbf{\underline{\foreignlanguage{arabic}{أمثلة}}}: أبوي فات المطبخ وفَعْكَشه وطلع}\end{flushright}\color{black}} \vspace{2mm}

{\setlength\topsep{0pt}\textbf{\foreignlanguage{arabic}{فَعْكَشِة}}\ {\color{gray}\texttt{/\sffamily {{\sffamily faʕkaʃe}}/}\color{black}}\ \textsc{noun}\ [f.]\ \textbf{1.}~the state of being messy and untidy\ } \vspace{2mm}

{\setlength\topsep{0pt}\textbf{\foreignlanguage{arabic}{فَعْكُوش}}\ {\color{gray}\texttt{/\sffamily {{\sffamily faʕkuːʃ}}/}\color{black}}\ \textsc{adj}\ [m.]\ \textbf{1.}~messy and untidy\ \ $\bullet$\ \ \setlength\topsep{0pt}\textbf{\foreignlanguage{arabic}{فَعَاكِيش}}\ {\color{gray}\texttt{/\sffamily {{\sffamily faʕaːkiːʃ}}/}\color{black}}\ [pl.]\  \begin{flushright}\color{gray}\foreignlanguage{arabic}{\textbf{\underline{\foreignlanguage{arabic}{أمثلة}}}: غرفتها دايما فَعْكوشة بتترتَّبِش}\end{flushright}\color{black}} \vspace{2mm}

{\setlength\topsep{0pt}\textbf{\foreignlanguage{arabic}{مْفَعْكَش}}\ {\color{gray}\texttt{/\sffamily {{\sffamily mfaʕkaʃ}}/}\color{black}}\ \textsc{adj}\ [m.]\ \textbf{1.}~messy and untidy\  \begin{flushright}\color{gray}\foreignlanguage{arabic}{\textbf{\underline{\foreignlanguage{arabic}{أمثلة}}}: مستحيل أصدق انه في بنت غرفتها مْفَعْكَشة هيك}\end{flushright}\color{black}} \vspace{2mm}

\vspace{-3mm}
\markboth{\color{blue}\foreignlanguage{arabic}{ف.ع.ل}\color{blue}{}}{\color{blue}\foreignlanguage{arabic}{ف.ع.ل}\color{blue}{}}\subsection*{\color{blue}\foreignlanguage{arabic}{ف.ع.ل}\color{blue}{}\index{\color{blue}\foreignlanguage{arabic}{ف.ع.ل}\color{blue}{}}} 

{\setlength\topsep{0pt}\textbf{\foreignlanguage{arabic}{اِفْتَعَل}}\ {\color{gray}\texttt{/\sffamily {{\sffamily ʔiftaʕal}}/}\color{black}}\ \textsc{verb}\ [p.]\ \textbf{1.}~make sth up\ \ $\bullet$\ \ \setlength\topsep{0pt}\textbf{\foreignlanguage{arabic}{اِفْتِعِل}}\ {\color{gray}\texttt{/\sffamily {{\sffamily ʔiftiʕil}}/}\color{black}}\ [c.]\ \ $\bullet$\ \ \setlength\topsep{0pt}\textbf{\foreignlanguage{arabic}{يِفْتِعِل}}\ {\color{gray}\texttt{/\sffamily {{\sffamily jiftiʕil}}/}\color{black}}\ [i.]\  \begin{flushright}\color{gray}\foreignlanguage{arabic}{\textbf{\underline{\foreignlanguage{arabic}{أمثلة}}}: يا الله جوزها ما أنكده بيِفْتِعِل المشاكل اِفْتِعال}\end{flushright}\color{black}} \vspace{2mm}

{\setlength\topsep{0pt}\textbf{\foreignlanguage{arabic}{اِفْتِعَال}}\ {\color{gray}\texttt{/\sffamily {{\sffamily ʔiftiʕaːl}}/}\color{black}}\ \textsc{noun}\ [m.]\ \textbf{1.}~making sth up\ } \vspace{2mm}

{\setlength\topsep{0pt}\textbf{\foreignlanguage{arabic}{اِنْفَعَل}}\ {\color{gray}\texttt{/\sffamily {{\sffamily ʔinfiʕal}}/}\color{black}}\ \textsc{verb}\ [p.]\ \textbf{1.}~overreact  \textbf{2.}~react  \textbf{3.}~be very sensitive\ \ $\bullet$\ \ \setlength\topsep{0pt}\textbf{\foreignlanguage{arabic}{اِنْفِعِل}}\ {\color{gray}\texttt{/\sffamily {{\sffamily ʔinfiʕil}}/}\color{black}}\ [c.]\ \ $\bullet$\ \ \setlength\topsep{0pt}\textbf{\foreignlanguage{arabic}{يِنْفِعِل}}\ {\color{gray}\texttt{/\sffamily {{\sffamily jinfiʕil}}/}\color{black}}\ [i.]\  \begin{flushright}\color{gray}\foreignlanguage{arabic}{\textbf{\underline{\foreignlanguage{arabic}{أمثلة}}}: أنا آسف، اِنْفَعَلت شوي وأنا بحكي}\end{flushright}\color{black}} \vspace{2mm}

{\setlength\topsep{0pt}\textbf{\foreignlanguage{arabic}{اِنْفِعَال}}\ {\color{gray}\texttt{/\sffamily {{\sffamily ʔinfiʕaːl}}/}\color{black}}\ \textsc{noun}\ [m.]\ \textbf{1.}~emotion  \textbf{2.}~excitation\ } \vspace{2mm}

{\setlength\topsep{0pt}\textbf{\foreignlanguage{arabic}{تَفَاعُل}}\ {\color{gray}\texttt{/\sffamily {{\sffamily tafaːʕul}}/}\color{black}}\ \textsc{noun}\ [m.]\ \textbf{1.}~interaction  \textbf{2.}~reaction  \textbf{3.}~reciprocity\  \begin{flushright}\color{gray}\foreignlanguage{arabic}{\textbf{\underline{\foreignlanguage{arabic}{أمثلة}}}: حبيت كثير التفاعل مع قصة البنت اليتيمة}\end{flushright}\color{black}} \vspace{2mm}

{\setlength\topsep{0pt}\textbf{\foreignlanguage{arabic}{تْفَعَّل}}\ {\color{gray}\texttt{/\sffamily {{\sffamily tfaʕʕal}}/}\color{black}}\ \textsc{verb}\ [p.]\ \textbf{1.}~be activated\ \ $\bullet$\ \ \setlength\topsep{0pt}\textbf{\foreignlanguage{arabic}{اِتْفَعَّل}}\ {\color{gray}\texttt{/\sffamily {{\sffamily ʔitfaʕʕal}}/}\color{black}}\ [c.]\ \ $\bullet$\ \ \setlength\topsep{0pt}\textbf{\foreignlanguage{arabic}{يِتْفَعَّل}}\ {\color{gray}\texttt{/\sffamily {{\sffamily jitfaʕʕal}}/}\color{black}}\ [i.]\  \begin{flushright}\color{gray}\foreignlanguage{arabic}{\textbf{\underline{\foreignlanguage{arabic}{أمثلة}}}: هيه الاشتراك تْفَعَّل. بتقدر تستخدمه عادي هلا.}\end{flushright}\color{black}} \vspace{2mm}

{\setlength\topsep{0pt}\textbf{\foreignlanguage{arabic}{فَاعِل}}\ {\color{gray}\texttt{/\sffamily {{\sffamily faːʕil}}/}\color{black}}\ \textsc{noun\textunderscore act}\ [m.]\ \color{gray}(msa. \foreignlanguage{arabic}{فاعِل}~\foreignlanguage{arabic}{\textbf{١.}})\color{black}\ \textbf{1.}~doing  \textbf{2.}~doer\ \ $\bullet$\ \ \textsc{ph.} \color{gray} \foreignlanguage{arabic}{فَاعِل خِير}\color{black}\ \footnote{Approving}\ {\color{gray}\texttt{/{\sffamily faːʕil xeːr}/}\color{black}}\ \color{gray} (msa. \foreignlanguage{arabic}{فاعِل خَيْر}~\foreignlanguage{arabic}{\textbf{١.}})\color{black}\ \textbf{1.}~philanthropist\ \ $\bullet$\ \ \textsc{ph.} \color{gray} \foreignlanguage{arabic}{الفَاعلة التَاركة}\color{black}\ \footnote{Taboo}\ {\color{gray}\texttt{/{\sffamily ʔilfaːʕle ʔittaːrke}/}\color{black}}\ \color{gray} (msa. \foreignlanguage{arabic}{ساقِطَة}~\foreignlanguage{arabic}{\textbf{١.}})\color{black}\ \textbf{1.}~bitch\  \begin{flushright}\color{gray}\foreignlanguage{arabic}{\textbf{\underline{\foreignlanguage{arabic}{أمثلة}}}: شو حكيت يا أخو الفاعْلِة التّارْكِة؟\ $\bullet$\ \  في فاعِل خِير تركلك هالمصاري}\end{flushright}\color{black}} \vspace{2mm}

{\setlength\topsep{0pt}\textbf{\foreignlanguage{arabic}{فَعَالِيِّة}}\ {\color{gray}\texttt{/\sffamily {{\sffamily faʕaːlijje}}/}\color{black}}\ \textsc{noun}\ [f.]\ \color{gray}(msa. \foreignlanguage{arabic}{فَعالِيِّة}~\foreignlanguage{arabic}{\textbf{١.}})\color{black}\ \textbf{1.}~effectiveness\  \begin{flushright}\color{gray}\foreignlanguage{arabic}{\textbf{\underline{\foreignlanguage{arabic}{أمثلة}}}: فَعالِيِّة الدوا بتضلها ليومين}\end{flushright}\color{black}} \vspace{2mm}

{\setlength\topsep{0pt}\textbf{\foreignlanguage{arabic}{فَعَل}}\ {\color{gray}\texttt{/\sffamily {{\sffamily faʕal}}/}\color{black}}\ \textsc{verb}\ [p.]\ \textbf{1.}~do  \textbf{2.}~make\ \ $\bullet$\ \ \setlength\topsep{0pt}\textbf{\foreignlanguage{arabic}{اِفْعَل}}\ {\color{gray}\texttt{/\sffamily {{\sffamily ʔifʕal}}/}\color{black}}\ [c.]\ \ $\bullet$\ \ \setlength\topsep{0pt}\textbf{\foreignlanguage{arabic}{يِفْعَل}}\ {\color{gray}\texttt{/\sffamily {{\sffamily jifʕal}}/}\color{black}}\ [i.]\ \color{gray}(msa. \foreignlanguage{arabic}{يَفْعَل}~\foreignlanguage{arabic}{\textbf{١.}})\color{black}\  \begin{flushright}\color{gray}\foreignlanguage{arabic}{\textbf{\underline{\foreignlanguage{arabic}{أمثلة}}}: شفت سيدنا إِسحاق لما حكا لأبوه بآية يا أبت افْعَل ما تؤمر. هيك الولاد لازم يكونوا مطيعين وبارين بأهاليهم.}\end{flushright}\color{black}} \vspace{2mm}

{\setlength\topsep{0pt}\textbf{\foreignlanguage{arabic}{فَعَّال}}\ {\color{gray}\texttt{/\sffamily {{\sffamily faʕʕaːl}}/}\color{black}}\ \textsc{adj}\ [m.]\ \color{gray}(msa. \foreignlanguage{arabic}{فَعّال}~\foreignlanguage{arabic}{\textbf{١.}})\color{black}\ \textbf{1.}~effective  \textbf{2.}~influential\ } \vspace{2mm}

{\setlength\topsep{0pt}\textbf{\foreignlanguage{arabic}{فَعَّل}}\ {\color{gray}\texttt{/\sffamily {{\sffamily faʕʕal}}/}\color{black}}\ \textsc{verb}\ [p.]\ \textbf{1.}~activate\ \ $\bullet$\ \ \setlength\topsep{0pt}\textbf{\foreignlanguage{arabic}{فَعِّل}}\ {\color{gray}\texttt{/\sffamily {{\sffamily faʕʕil}}/}\color{black}}\ [c.]\ \ $\bullet$\ \ \setlength\topsep{0pt}\textbf{\foreignlanguage{arabic}{يفَعِّل}}\ {\color{gray}\texttt{/\sffamily {{\sffamily jfaʕʕil}}/}\color{black}}\ [i.]\ \color{gray}(msa. \foreignlanguage{arabic}{يُفَعِّل}~\foreignlanguage{arabic}{\textbf{١.}})\color{black}\  \begin{flushright}\color{gray}\foreignlanguage{arabic}{\textbf{\underline{\foreignlanguage{arabic}{أمثلة}}}: فَعِّل حسابك بالبنك}\end{flushright}\color{black}} \vspace{2mm}

{\setlength\topsep{0pt}\textbf{\foreignlanguage{arabic}{فِعِل}}\ {\color{gray}\texttt{/\sffamily {{\sffamily fiʕil}}/}\color{black}}\ \textsc{noun}\ [m.]\ \color{gray}(msa. \foreignlanguage{arabic}{فِعِل}~\foreignlanguage{arabic}{\textbf{١.}})\color{black}\ \textbf{1.}~action  \textbf{2.}~verb\ \ $\bullet$\ \ \setlength\topsep{0pt}\textbf{\foreignlanguage{arabic}{أَفْعَال}}\ {\color{gray}\texttt{/\sffamily {{\sffamily ʔafʕaːl}}/}\color{black}}\ [pl.]\  \begin{flushright}\color{gray}\foreignlanguage{arabic}{\textbf{\underline{\foreignlanguage{arabic}{أمثلة}}}: أَفْعالك كلها غلط}\end{flushright}\color{black}} \vspace{2mm}

{\setlength\topsep{0pt}\textbf{\foreignlanguage{arabic}{مُنْفَعِل}}\ {\color{gray}\texttt{/\sffamily {{\sffamily munfaʕil}}/}\color{black}}\ \textsc{adj}\ [m.]\ \textbf{1.}~agitated  \textbf{2.}~excited\  \begin{flushright}\color{gray}\foreignlanguage{arabic}{\textbf{\underline{\foreignlanguage{arabic}{أمثلة}}}: أنا آسف عشان بقيت مُنْفَعِل شوي وبيجوز رفعت صوتي}\end{flushright}\color{black}} \vspace{2mm}

\vspace{-3mm}
\markboth{\color{blue}\foreignlanguage{arabic}{ف.غ.ر}\color{blue}{}}{\color{blue}\foreignlanguage{arabic}{ف.غ.ر}\color{blue}{}}\subsection*{\color{blue}\foreignlanguage{arabic}{ف.غ.ر}\color{blue}{}\index{\color{blue}\foreignlanguage{arabic}{ف.غ.ر}\color{blue}{}}} 

{\setlength\topsep{0pt}\textbf{\foreignlanguage{arabic}{فَغَر}}\ {\color{gray}\texttt{/\sffamily {{\sffamily faɣar}}/}\color{black}}\ \textsc{verb}\ [p.]\ \textbf{1.}~press on sth with force and make it bulge out.  \textbf{2.}~shout sloudly\ \ $\bullet$\ \ \setlength\topsep{0pt}\textbf{\foreignlanguage{arabic}{اِفْغَر}}\ {\color{gray}\texttt{/\sffamily {{\sffamily ʔifɣar}}/}\color{black}}\ [c.]\ \ $\bullet$\ \ \setlength\topsep{0pt}\textbf{\foreignlanguage{arabic}{يِفْغَر}}\ {\color{gray}\texttt{/\sffamily {{\sffamily jifɣar}}/}\color{black}}\ [i.]\  \begin{flushright}\color{gray}\foreignlanguage{arabic}{\textbf{\underline{\foreignlanguage{arabic}{أمثلة}}}: ضلك عص فيها لحد ما تِفْغَر عيونها\ $\bullet$\ \  اذا طلعلك شي اِفْغَري هالصوت الناس رح تساعدك}\end{flushright}\color{black}} \vspace{2mm}

\vspace{-3mm}
\markboth{\color{blue}\foreignlanguage{arabic}{ف.غ.ص}\color{blue}{}}{\color{blue}\foreignlanguage{arabic}{ف.غ.ص}\color{blue}{}}\subsection*{\color{blue}\foreignlanguage{arabic}{ف.غ.ص}\color{blue}{}\index{\color{blue}\foreignlanguage{arabic}{ف.غ.ص}\color{blue}{}}} 

{\setlength\topsep{0pt}\textbf{\foreignlanguage{arabic}{اِنْفَغَص}}\ {\color{gray}\texttt{/\sffamily {{\sffamily ʔinfaɣasˤ}}/}\color{black}}\ \textsc{verb}\ [p.]\ \textbf{1.}~be mashed up.  \textbf{2.}~be squashed.  \textbf{3.}~not feel comfortable because there is no enough space\ \ $\bullet$\ \ \setlength\topsep{0pt}\textbf{\foreignlanguage{arabic}{اِنْفِغِص}}\ {\color{gray}\texttt{/\sffamily {{\sffamily ʔinfiɣisˤ}}/}\color{black}}\ [c.]\ \ $\bullet$\ \ \setlength\topsep{0pt}\textbf{\foreignlanguage{arabic}{يِنْفِغِص}}\ {\color{gray}\texttt{/\sffamily {{\sffamily jinfiɣisˤ}}/}\color{black}}\ [i.]\  \begin{flushright}\color{gray}\foreignlanguage{arabic}{\textbf{\underline{\foreignlanguage{arabic}{أمثلة}}}: خفت عالمعمول انه يِنفِغِص وهو بالصحن\ $\bullet$\ \  ما شاء الله الثلاثة دبب اِنْفَغَصِت وأنا قاعدة بينهم بالسيارة}\end{flushright}\color{black}} \vspace{2mm}

{\setlength\topsep{0pt}\textbf{\foreignlanguage{arabic}{تْفَغَّص}}\ {\color{gray}\texttt{/\sffamily {{\sffamily tfaɣɣasˤ}}/}\color{black}}\ \textsc{verb}\ [p.]\ \textbf{1.}~be mashed up.  \textbf{2.}~be squashed\ \ $\bullet$\ \ \setlength\topsep{0pt}\textbf{\foreignlanguage{arabic}{اِتْفَغَّص}}\ {\color{gray}\texttt{/\sffamily {{\sffamily ʔitfaɣɣasˤ}}/}\color{black}}\ [c.]\ \ $\bullet$\ \ \setlength\topsep{0pt}\textbf{\foreignlanguage{arabic}{يِتْفَغَّص}}\ {\color{gray}\texttt{/\sffamily {{\sffamily jitfaɣɣasˤ}}/}\color{black}}\ [i.]\  \begin{flushright}\color{gray}\foreignlanguage{arabic}{\textbf{\underline{\foreignlanguage{arabic}{أمثلة}}}: ديري بالك الرز تْفَغَّص}\end{flushright}\color{black}} \vspace{2mm}

{\setlength\topsep{0pt}\textbf{\foreignlanguage{arabic}{فَغَّص}}\ {\color{gray}\texttt{/\sffamily {{\sffamily faɣɣasˤ}}/}\color{black}}\ \textsc{verb}\ [p.]\ \textbf{1.}~mash sth up.  \textbf{2.}~squash sth\ \ $\bullet$\ \ \setlength\topsep{0pt}\textbf{\foreignlanguage{arabic}{فَغِّص}}\ {\color{gray}\texttt{/\sffamily {{\sffamily faɣɣisˤ}}/}\color{black}}\ [c.]\ \ $\bullet$\ \ \setlength\topsep{0pt}\textbf{\foreignlanguage{arabic}{يفَغِّص}}\ {\color{gray}\texttt{/\sffamily {{\sffamily jfaɣɣisˤ}}/}\color{black}}\ [i.]\  \begin{flushright}\color{gray}\foreignlanguage{arabic}{\textbf{\underline{\foreignlanguage{arabic}{أمثلة}}}: يا الله! مسك الرز فَغَّصه بايديه}\end{flushright}\color{black}} \vspace{2mm}

{\setlength\topsep{0pt}\textbf{\foreignlanguage{arabic}{مَفْغُوص}}\ {\color{gray}\texttt{/\sffamily {{\sffamily mafɣuːsˤ}}/}\color{black}}\ \textsc{noun\textunderscore pass}\ \color{gray}(msa. \foreignlanguage{arabic}{مسحوق}~\foreignlanguage{arabic}{\textbf{١.}})\color{black}\ \textbf{1.}~mashed  \textbf{2.}~squashed (partially)\  \begin{flushright}\color{gray}\foreignlanguage{arabic}{\textbf{\underline{\foreignlanguage{arabic}{أمثلة}}}: ناولته تفاحة من شنطتي بس طلعت مفغوصة قد مادعسوا عليها الصغار اليوم}\end{flushright}\color{black}} \vspace{2mm}

{\setlength\topsep{0pt}\textbf{\foreignlanguage{arabic}{مْفَغَّص}}\ {\color{gray}\texttt{/\sffamily {{\sffamily mfaɣɣasˤ}}/}\color{black}}\ \textsc{noun\textunderscore pass}\ \color{gray}(msa. \foreignlanguage{arabic}{مسحوق}~\foreignlanguage{arabic}{\textbf{١.}})\color{black}\ \textbf{1.}~mashed  \textbf{2.}~squashed (entirely)\  \begin{flushright}\color{gray}\foreignlanguage{arabic}{\textbf{\underline{\foreignlanguage{arabic}{أمثلة}}}: الموز مفَغَّص  فش فيه ولاشي سليم}\end{flushright}\color{black}} \vspace{2mm}

\vspace{-3mm}
\markboth{\color{blue}\foreignlanguage{arabic}{ف.غ.م}\color{blue}{}}{\color{blue}\foreignlanguage{arabic}{ف.غ.م}\color{blue}{}}\subsection*{\color{blue}\foreignlanguage{arabic}{ف.غ.م}\color{blue}{}\index{\color{blue}\foreignlanguage{arabic}{ف.غ.م}\color{blue}{}}} 

{\setlength\topsep{0pt}\textbf{\foreignlanguage{arabic}{اِنْفَغَم}}\ {\color{gray}\texttt{/\sffamily {{\sffamily ʔinfaɣam}}/}\color{black}}\ \textsc{verb}\ [p.]\ \textbf{1.}~be bitten\ \ $\bullet$\ \ \setlength\topsep{0pt}\textbf{\foreignlanguage{arabic}{اِنْفِغِم}}\ {\color{gray}\texttt{/\sffamily {{\sffamily ʔinfiɣim}}/}\color{black}}\ [c.]\ \ $\bullet$\ \ \setlength\topsep{0pt}\textbf{\foreignlanguage{arabic}{يِنْفِغِم}}\ {\color{gray}\texttt{/\sffamily {{\sffamily jinfiɣim}}/}\color{black}}\ [i.]\  \begin{flushright}\color{gray}\foreignlanguage{arabic}{\textbf{\underline{\foreignlanguage{arabic}{أمثلة}}}: هاي التفاحة اِنْفَغَمت نصها}\end{flushright}\color{black}} \vspace{2mm}

{\setlength\topsep{0pt}\textbf{\foreignlanguage{arabic}{فَغَم}}\ {\color{gray}\texttt{/\sffamily {{\sffamily faɣam}}/}\color{black}}\ \textsc{verb}\ [p.]\ \textbf{1.}~bite into sth\ \ $\bullet$\ \ \setlength\topsep{0pt}\textbf{\foreignlanguage{arabic}{اِفْغَم}}\ {\color{gray}\texttt{/\sffamily {{\sffamily ʔifɣam}}/}\color{black}}\ [c.]\ \ $\bullet$\ \ \setlength\topsep{0pt}\textbf{\foreignlanguage{arabic}{يِفْغَم}}\ {\color{gray}\texttt{/\sffamily {{\sffamily jifɣam}}/}\color{black}}\ [i.]\ \color{gray}(msa. \foreignlanguage{arabic}{يقضم}~\foreignlanguage{arabic}{\textbf{١.}})\color{black}\  \begin{flushright}\color{gray}\foreignlanguage{arabic}{\textbf{\underline{\foreignlanguage{arabic}{أمثلة}}}: ماهي صغيرة خليه يِفْغَمها فَغِم.\ $\bullet$\ \  اِفْغَمها فَغِم تقدش هسعيات تعملي حالك فيها راقي}\end{flushright}\color{black}} \vspace{2mm}

{\setlength\topsep{0pt}\textbf{\foreignlanguage{arabic}{فَغِم}}\ {\color{gray}\texttt{/\sffamily {{\sffamily faɣim}}/}\color{black}}\ \textsc{noun}\ [m.]\ \color{gray}(msa. \foreignlanguage{arabic}{قَضِم}~\foreignlanguage{arabic}{\textbf{١.}})\color{black}\ \textbf{1.}~biting into sth\ } \vspace{2mm}

\vspace{-3mm}
\markboth{\color{blue}\foreignlanguage{arabic}{ف.ف.ي}\color{blue}{ (ntws)}}{\color{blue}\foreignlanguage{arabic}{ف.ف.ي}\color{blue}{ (ntws)}}\subsection*{\color{blue}\foreignlanguage{arabic}{ف.ف.ي}\color{blue}{ (ntws)}\index{\color{blue}\foreignlanguage{arabic}{ف.ف.ي}\color{blue}{ (ntws)}}} 

{\setlength\topsep{0pt}\textbf{\foreignlanguage{arabic}{فَافِي}}\ {\color{gray}\texttt{/\sffamily {{\sffamily faːfi}}/}\color{black}}\ \textsc{adj}\ [m.]\ \textbf{1.}~effete  \textbf{2.}~sissy\  \begin{flushright}\color{gray}\foreignlanguage{arabic}{\textbf{\underline{\foreignlanguage{arabic}{أمثلة}}}: مش لاقية من الزلام غير هاذ الفافِي؟}\end{flushright}\color{black}} \vspace{2mm}

\vspace{-3mm}
\markboth{\color{blue}\foreignlanguage{arabic}{ف.ق.د}\color{blue}{}}{\color{blue}\foreignlanguage{arabic}{ف.ق.د}\color{blue}{}}\subsection*{\color{blue}\foreignlanguage{arabic}{ف.ق.د}\color{blue}{}\index{\color{blue}\foreignlanguage{arabic}{ف.ق.د}\color{blue}{}}} 

{\setlength\topsep{0pt}\textbf{\foreignlanguage{arabic}{اِسْتَفْقَد}}\ {\color{gray}\texttt{/\sffamily {{\sffamily ʔistafqad}}/}\color{black}}\ \textsc{verb}\ [p.]\ \textbf{1.}~check in on sb and bring him what he needs (especially food)\ \ $\bullet$\ \ \setlength\topsep{0pt}\textbf{\foreignlanguage{arabic}{اِسْتَفْقِد}}\ {\color{gray}\texttt{/\sffamily {{\sffamily ʔistafqid}}/}\color{black}}\ [c.]\ \ $\bullet$\ \ \setlength\topsep{0pt}\textbf{\foreignlanguage{arabic}{يِسْتَفْقِد}}\ {\color{gray}\texttt{/\sffamily {{\sffamily jistafqid}}/}\color{black}}\ [i.]\  \begin{flushright}\color{gray}\foreignlanguage{arabic}{\textbf{\underline{\foreignlanguage{arabic}{أمثلة}}}: أول رمضان اِسْتَفْقَدني بسكبة عدس بس}\end{flushright}\color{black}} \vspace{2mm}

{\setlength\topsep{0pt}\textbf{\foreignlanguage{arabic}{اِفْتَقَد}}\ {\color{gray}\texttt{/\sffamily {{\sffamily ʔiftaqad}}/}\color{black}}\ \textsc{verb}\ [p.]\ \textbf{1.}~miss\ \ $\bullet$\ \ \setlength\topsep{0pt}\textbf{\foreignlanguage{arabic}{اِفْتِقِد}}\ {\color{gray}\texttt{/\sffamily {{\sffamily ʔiftiqid}}/}\color{black}}\ [c.]\ \ $\bullet$\ \ \setlength\topsep{0pt}\textbf{\foreignlanguage{arabic}{اِفْتَقِد}}\ {\color{gray}\texttt{/\sffamily {{\sffamily ʔiftaqid}}/}\color{black}}\ [c.]\ \ $\bullet$\ \ \setlength\topsep{0pt}\textbf{\foreignlanguage{arabic}{يِفْتِقِد}}\ {\color{gray}\texttt{/\sffamily {{\sffamily jiftiqid}}/}\color{black}}\ [i.]\ \color{gray}(msa. \foreignlanguage{arabic}{يَفْتَقِد}~\foreignlanguage{arabic}{\textbf{١.}})\color{black}\ \ $\bullet$\ \ \setlength\topsep{0pt}\textbf{\foreignlanguage{arabic}{يِفْتَقِد}}\ {\color{gray}\texttt{/\sffamily {{\sffamily jiftaqid}}/}\color{black}}\ [i.]\ \color{gray}(msa. \foreignlanguage{arabic}{يَفْتَقِد}~\foreignlanguage{arabic}{\textbf{١.}})\color{black}\  \begin{flushright}\color{gray}\foreignlanguage{arabic}{\textbf{\underline{\foreignlanguage{arabic}{أمثلة}}}: اِفْتَقَدتِك كثير! الله يوفقك ويسهِّل أمرك يا عرين.}\end{flushright}\color{black}} \vspace{2mm}

{\setlength\topsep{0pt}\textbf{\foreignlanguage{arabic}{تْفَقَّد}}\ {\color{gray}\texttt{/\sffamily {{\sffamily tfa(q)(q)ad}}/}\color{black}}\ \textsc{verb}\ [p.]\ \textbf{1.}~check  \textbf{2.}~check in on sb and bring him what he needs (especially food)\ \ $\bullet$\ \ \setlength\topsep{0pt}\textbf{\foreignlanguage{arabic}{اِتْفَقَّد}}\ {\color{gray}\texttt{/\sffamily {{\sffamily ʔitfa(q)(q)ad}}/}\color{black}}\ [c.]\ \ $\bullet$\ \ \setlength\topsep{0pt}\textbf{\foreignlanguage{arabic}{يِتْفَقَّد}}\ {\color{gray}\texttt{/\sffamily {{\sffamily jitfa(q)(q)ad}}/}\color{black}}\ [i.]\  \begin{flushright}\color{gray}\foreignlanguage{arabic}{\textbf{\underline{\foreignlanguage{arabic}{أمثلة}}}: خليته يِتْفَقَّد أغراض غرفتنا وحكى انه كل شي موجود\ $\bullet$\ \  اِتْفَقَّد بنات أختك يمكن بدهم شي أو ناقص عليهم شي}\end{flushright}\color{black}} \vspace{2mm}

{\setlength\topsep{0pt}\textbf{\foreignlanguage{arabic}{فَاقِد}}\ {\color{gray}\texttt{/\sffamily {{\sffamily faːqid}}/}\color{black}}\ \textsc{noun\textunderscore act}\ [m.]\ \textbf{1.}~losing  \textbf{2.}~missing\ \ $\bullet$\ \ \textsc{ph.} \color{gray} \foreignlanguage{arabic}{بدل فَاقِد}\color{black}\ {\color{gray}\texttt{/{\sffamily badal faːqid}/}\color{black}}\ \textbf{1.}~replacement\  \begin{flushright}\color{gray}\foreignlanguage{arabic}{\textbf{\underline{\foreignlanguage{arabic}{أمثلة}}}: بتروح عالأحوال المدنية وبتطلب معاملة بدل فاقِد للهوية وبتطلعها ان شاء الله خلال أيبوع\ $\bullet$\ \  والله يا ثابت إِني فاقدة وجودك معنا برمضان}\end{flushright}\color{black}} \vspace{2mm}

{\setlength\topsep{0pt}\textbf{\foreignlanguage{arabic}{فَقَد}}\ {\color{gray}\texttt{/\sffamily {{\sffamily fa(q)ad}}/}\color{black}}\ \textsc{verb}\ [p.]\ \textbf{1.}~miss  \textbf{2.}~lose  \textbf{3.}~check\ \ $\bullet$\ \ \setlength\topsep{0pt}\textbf{\foreignlanguage{arabic}{اِفْقِد}}\ {\color{gray}\texttt{/\sffamily {{\sffamily ʔif(q)id}}/}\color{black}}\ [c.]\ \ $\bullet$\ \ \setlength\topsep{0pt}\textbf{\foreignlanguage{arabic}{يِفْقِد}}\ {\color{gray}\texttt{/\sffamily {{\sffamily jif(q)id}}/}\color{black}}\ [i.]\  \begin{flushright}\color{gray}\foreignlanguage{arabic}{\textbf{\underline{\foreignlanguage{arabic}{أمثلة}}}: الواحد الا ما يِفْقِد الناس اللي بيحبها\ $\bullet$\ \  اِفْقِد أغراضك مليح بلاش ماتطلع ناسي شي\ $\bullet$\ \  بالحرب فَقَدت اجري}\end{flushright}\color{black}} \vspace{2mm}

{\setlength\topsep{0pt}\textbf{\foreignlanguage{arabic}{فَقِيد}}\ {\color{gray}\texttt{/\sffamily {{\sffamily faqiːd}}/}\color{black}}\ \textsc{adj}\ [m.]\ \textbf{1.}~deceased\  \begin{flushright}\color{gray}\foreignlanguage{arabic}{\textbf{\underline{\foreignlanguage{arabic}{أمثلة}}}: بدنا نعزي أهل الفَقِيد}\end{flushright}\color{black}} \vspace{2mm}

{\setlength\topsep{0pt}\textbf{\foreignlanguage{arabic}{فَقَّد}}\ {\color{gray}\texttt{/\sffamily {{\sffamily fa(q)(q)ad}}/}\color{black}}\ \textsc{verb}\ [p.]\ \textbf{1.}~check\ \ $\bullet$\ \ \setlength\topsep{0pt}\textbf{\foreignlanguage{arabic}{فَقِّد}}\ {\color{gray}\texttt{/\sffamily {{\sffamily fa(q)(q)id}}/}\color{black}}\ [c.]\ \ $\bullet$\ \ \setlength\topsep{0pt}\textbf{\foreignlanguage{arabic}{يفَقِّد}}\ {\color{gray}\texttt{/\sffamily {{\sffamily jfa(q)(q)id}}/}\color{black}}\ [i.]\  \begin{flushright}\color{gray}\foreignlanguage{arabic}{\textbf{\underline{\foreignlanguage{arabic}{أمثلة}}}: طب فَقِّد شنطتك يمكن حدا حطه فيها}\end{flushright}\color{black}} \vspace{2mm}

{\setlength\topsep{0pt}\textbf{\foreignlanguage{arabic}{فُقْدِة}}\ {\color{gray}\texttt{/\sffamily {{\sffamily fuqde}}/}\color{black}}\ \textsc{noun}\ [f.]\ \textbf{1.}~it is a Palestinian tradition where people visit the bride's house and bring her some utensils that she will use in her kitchen. This visit is usually made after one week of the wedding ceremony.\  \begin{flushright}\color{gray}\foreignlanguage{arabic}{\textbf{\underline{\foreignlanguage{arabic}{أمثلة}}}: رايحين عفُقْدِة بنت تحسين شو رأيك تاخذيلها صواني من معرض جنين عليهم عرض ب 50 شيقل}\end{flushright}\color{black}} \vspace{2mm}

{\setlength\topsep{0pt}\textbf{\foreignlanguage{arabic}{مَفْقُود}}\ {\color{gray}\texttt{/\sffamily {{\sffamily mafquːd}}/}\color{black}}\ \textsc{noun\textunderscore pass}\ \textbf{1.}~lost  \textbf{2.}~missing  \textbf{3.}~absent\ } \vspace{2mm}

{\setlength\topsep{0pt}\textbf{\foreignlanguage{arabic}{مِفْتَقِد}}\ {\color{gray}\texttt{/\sffamily {{\sffamily miftaqid}}/}\color{black}}\ \textsc{noun\textunderscore act}\ [m.]\ \textbf{1.}~missing\  \begin{flushright}\color{gray}\foreignlanguage{arabic}{\textbf{\underline{\foreignlanguage{arabic}{أمثلة}}}: يا الله قديش مِفْتَقِد وجودها بيننا رمضان هذا}\end{flushright}\color{black}} \vspace{2mm}

\vspace{-3mm}
\markboth{\color{blue}\foreignlanguage{arabic}{ف.ق.ر}\color{blue}{}}{\color{blue}\foreignlanguage{arabic}{ف.ق.ر}\color{blue}{}}\subsection*{\color{blue}\foreignlanguage{arabic}{ف.ق.ر}\color{blue}{}\index{\color{blue}\foreignlanguage{arabic}{ف.ق.ر}\color{blue}{}}} 

{\setlength\topsep{0pt}\textbf{\foreignlanguage{arabic}{اِفْتَقَر}}\ {\color{gray}\texttt{/\sffamily {{\sffamily ʔiftaqar}}/}\color{black}}\ \textsc{verb}\ [p.]\ \textbf{1.}~lack\ \ $\bullet$\ \ \setlength\topsep{0pt}\textbf{\foreignlanguage{arabic}{اِفْتِقِر}}\ {\color{gray}\texttt{/\sffamily {{\sffamily ʔiftiqir}}/}\color{black}}\ [c.]\ \ $\bullet$\ \ \setlength\topsep{0pt}\textbf{\foreignlanguage{arabic}{اِفْتَقِر}}\ {\color{gray}\texttt{/\sffamily {{\sffamily ʔiftaqir}}/}\color{black}}\ [c.]\ \ $\bullet$\ \ \setlength\topsep{0pt}\textbf{\foreignlanguage{arabic}{يِفْتِقِر}}\ {\color{gray}\texttt{/\sffamily {{\sffamily jiftiqir}}/}\color{black}}\ [i.]\ \color{gray}(msa. \foreignlanguage{arabic}{يَفْتَقِر}~\foreignlanguage{arabic}{\textbf{١.}})\color{black}\ \ $\bullet$\ \ \setlength\topsep{0pt}\textbf{\foreignlanguage{arabic}{يِفْتَقِر}}\ {\color{gray}\texttt{/\sffamily {{\sffamily jiftaqir}}/}\color{black}}\ [i.]\ \color{gray}(msa. \foreignlanguage{arabic}{يَفْتَقِر}~\foreignlanguage{arabic}{\textbf{١.}})\color{black}\  \begin{flushright}\color{gray}\foreignlanguage{arabic}{\textbf{\underline{\foreignlanguage{arabic}{أمثلة}}}: أسلوبه بيِفْتِقِر للباقة والرقي. تحسيه بيدِج الحكي دَج}\end{flushright}\color{black}} \vspace{2mm}

{\setlength\topsep{0pt}\textbf{\foreignlanguage{arabic}{تْفَاقَر}}\ {\color{gray}\texttt{/\sffamily {{\sffamily tfaːqar}}/}\color{black}}\ \textsc{verb}\ [p.]\ \textbf{1.}~pretend to be poor in order to deceive people\ \ $\bullet$\ \ \setlength\topsep{0pt}\textbf{\foreignlanguage{arabic}{اِتْفَاقَر}}\ {\color{gray}\texttt{/\sffamily {{\sffamily ʔitfaːqar}}/}\color{black}}\ [c.]\ \ $\bullet$\ \ \setlength\topsep{0pt}\textbf{\foreignlanguage{arabic}{يِتْفَاقَر}}\ {\color{gray}\texttt{/\sffamily {{\sffamily jitfaːqar}}/}\color{black}}\ [i.]\  \begin{flushright}\color{gray}\foreignlanguage{arabic}{\textbf{\underline{\foreignlanguage{arabic}{أمثلة}}}: اللي بيِتْفاقَر وبيتشكون عالفاضي والملان الله بيسخطه ساعيتها وعنجد بصير فقير}\end{flushright}\color{black}} \vspace{2mm}

{\setlength\topsep{0pt}\textbf{\foreignlanguage{arabic}{فَقَار}}\ {\color{gray}\texttt{/\sffamily {{\sffamily faqaːr}}/}\color{black}}\ \textsc{noun}\ [m.]\ \color{gray}(msa. \foreignlanguage{arabic}{العمود الفَقَري}~\foreignlanguage{arabic}{\textbf{١.}})\color{black}\ \textbf{1.}~spine\ } \vspace{2mm}

{\setlength\topsep{0pt}\textbf{\foreignlanguage{arabic}{فَقَرَة}}\ {\color{gray}\texttt{/\sffamily {{\sffamily faqara}}/}\color{black}}\ \textsc{noun}\ [f.]\ \color{gray}(msa. \foreignlanguage{arabic}{فَقَرَة كتابة}~\foreignlanguage{arabic}{\textbf{٢.}}  .\foreignlanguage{arabic}{فَقَرَة ظهر}~\foreignlanguage{arabic}{\textbf{١.}})\color{black}\ \textbf{1.}~vertebra  \textbf{2.}~paragraph\ } \vspace{2mm}

{\setlength\topsep{0pt}\textbf{\foreignlanguage{arabic}{فَقِر}}\ {\color{gray}\texttt{/\sffamily {{\sffamily fa(q)ir}}/}\color{black}}\ \textsc{noun}\ [m.]\ \color{gray}(msa. \foreignlanguage{arabic}{فَقْر}~\foreignlanguage{arabic}{\textbf{١.}})\color{black}\ \textbf{1.}~paucity\ \ $\bullet$\ \ \textsc{ph.} \color{gray} \foreignlanguage{arabic}{فَقِر دم}\color{black}\ {\color{gray}\texttt{/{\sffamily faqir damm}/}\color{black}}\ \color{gray} (msa. \foreignlanguage{arabic}{فَقْر دم}~\foreignlanguage{arabic}{\textbf{١.}})\color{black}\ \textbf{1.}~anemia\  \begin{flushright}\color{gray}\foreignlanguage{arabic}{\textbf{\underline{\foreignlanguage{arabic}{أمثلة}}}: ليش مصفرنة هيك؟ شكله معك فَقِر دم.}\end{flushright}\color{black}} \vspace{2mm}

{\setlength\topsep{0pt}\textbf{\foreignlanguage{arabic}{فَقِير}}\ {\color{gray}\texttt{/\sffamily {{\sffamily fa(q)iːr}}/}\color{black}}\ \textsc{adj}\ [m.]\ \color{gray}(msa. \foreignlanguage{arabic}{فقِير}~\foreignlanguage{arabic}{\textbf{١.}})\color{black}\ \textbf{1.}~poor\ \ $\bullet$\ \ \setlength\topsep{0pt}\textbf{\foreignlanguage{arabic}{فُقَرَاء}}\ {\color{gray}\texttt{/\sffamily {{\sffamily fuqara}}/}\color{black}}\ [pl.]\  \begin{flushright}\color{gray}\foreignlanguage{arabic}{\textbf{\underline{\foreignlanguage{arabic}{أمثلة}}}: ليش يعني هو احنا كاينين فُقَراء؟\ $\bullet$\ \  أنت رفضتيه عشانه فقِير مش عشان صليتي استخارة}\end{flushright}\color{black}} \vspace{2mm}

{\setlength\topsep{0pt}\textbf{\foreignlanguage{arabic}{فَقَّر}}\ {\color{gray}\texttt{/\sffamily {{\sffamily faqqar}}/}\color{black}}\ \textsc{verb}\ [p.]\ \textbf{1.}~make sb poor.  \textbf{2.}~make paragraphs\ \ $\bullet$\ \ \setlength\topsep{0pt}\textbf{\foreignlanguage{arabic}{فَقِّر}}\ {\color{gray}\texttt{/\sffamily {{\sffamily faqqir}}/}\color{black}}\ [c.]\ \ $\bullet$\ \ \setlength\topsep{0pt}\textbf{\foreignlanguage{arabic}{يفَقِّر}}\ {\color{gray}\texttt{/\sffamily {{\sffamily jfaqqir}}/}\color{black}}\ [i.]\  \begin{flushright}\color{gray}\foreignlanguage{arabic}{\textbf{\underline{\foreignlanguage{arabic}{أمثلة}}}: مارح تفقِّرني هالعشرة شيكل اللي بدفعها كل اسبوع\ $\bullet$\ \  حاول فَقِّرها عشان تبين أرتب للقارئ}\end{flushright}\color{black}} \vspace{2mm}

{\setlength\topsep{0pt}\textbf{\foreignlanguage{arabic}{فُقُر}}\ {\color{gray}\texttt{/\sffamily {{\sffamily fuqur}}/}\color{black}}\ \textsc{noun}\ [m.]\ \color{gray}(msa. \foreignlanguage{arabic}{فَقْر}~\foreignlanguage{arabic}{\textbf{١.}})\color{black}\ \textbf{1.}~paucity\ \ $\bullet$\ \ \textsc{ph.} \color{gray} \foreignlanguage{arabic}{فقر و نقر}\color{black}\ {\color{gray}\texttt{/{\sffamily fuqur wunuqur}/}\color{black}}\ \color{gray} (msa. \foreignlanguage{arabic}{فقر مطقع}~\foreignlanguage{arabic}{\textbf{١.}})\color{black}\ \textbf{1.}~abject poverty\  \begin{flushright}\color{gray}\foreignlanguage{arabic}{\textbf{\underline{\foreignlanguage{arabic}{أمثلة}}}: فُقُر و نُقُرْ! كيف راضيات بهالعيشة وببزرِن\ $\bullet$\ \  مستوى الفُقُر الموجود عنا بالضفة بالذات بالشمال عنجد إِنه مرعب}\end{flushright}\color{black}} \vspace{2mm}

{\setlength\topsep{0pt}\textbf{\foreignlanguage{arabic}{فِقِر}}\ {\color{gray}\texttt{/\sffamily {{\sffamily fi(q)ir}}/}\color{black}}\ \textsc{verb}\ [p.]\ \textbf{1.}~become poor\ \ $\bullet$\ \ \setlength\topsep{0pt}\textbf{\foreignlanguage{arabic}{اِفْقَر}}\ {\color{gray}\texttt{/\sffamily {{\sffamily ʔif(q)ar}}/}\color{black}}\ [c.]\ \ $\bullet$\ \ \setlength\topsep{0pt}\textbf{\foreignlanguage{arabic}{يِفْقَر}}\ {\color{gray}\texttt{/\sffamily {{\sffamily jif(q)ar}}/}\color{black}}\ [i.]\ \color{gray}(msa. \foreignlanguage{arabic}{يُصبِح فقيرا}~\foreignlanguage{arabic}{\textbf{١.}})\color{black}\  \begin{flushright}\color{gray}\foreignlanguage{arabic}{\textbf{\underline{\foreignlanguage{arabic}{أمثلة}}}: مابصير المرة تدشِّر جوزها أول ما يِفْقَر\ $\bullet$\ \  اِفْقَر الله لايردك مية مرة قلتلك تسمش مصارك لمرة}\end{flushright}\color{black}} \vspace{2mm}

{\setlength\topsep{0pt}\textbf{\foreignlanguage{arabic}{مُفْتَقِر}}\ {\color{gray}\texttt{/\sffamily {{\sffamily muftaqir}}/}\color{black}}\ \textsc{noun\textunderscore act}\ [m.]\ \textbf{1.}~lacking\  \begin{flushright}\color{gray}\foreignlanguage{arabic}{\textbf{\underline{\foreignlanguage{arabic}{أمثلة}}}: العيشة بالمخيم مُفْتَقِرة لأدنى أساسيات الحياة}\end{flushright}\color{black}} \vspace{2mm}

\vspace{-3mm}
\markboth{\color{blue}\foreignlanguage{arabic}{ف.ق.ز}\color{blue}{}}{\color{blue}\foreignlanguage{arabic}{ف.ق.ز}\color{blue}{}}\subsection*{\color{blue}\foreignlanguage{arabic}{ف.ق.ز}\color{blue}{}\index{\color{blue}\foreignlanguage{arabic}{ف.ق.ز}\color{blue}{}}} 

{\setlength\topsep{0pt}\textbf{\foreignlanguage{arabic}{فَاقِز}}\ {\color{gray}\texttt{/\sffamily {{\sffamily faːqiz}}/}\color{black}}\ \textsc{adj}\ [m.]\ \textbf{1.}~sprained (twisted/turned)\ } \vspace{2mm}

{\setlength\topsep{0pt}\textbf{\foreignlanguage{arabic}{فَقَز}}\ {\color{gray}\texttt{/\sffamily {{\sffamily faqaz, fakas}}/}\color{black}}\ \textsc{verb}\ [p.]\ \textbf{1.}~sprain (twist/turn) sb's ankle\ \ $\bullet$\ \ \setlength\topsep{0pt}\textbf{\foreignlanguage{arabic}{اُفْقُز}}\ {\color{gray}\texttt{/\sffamily {{\sffamily ʔufquz, ʔufkus}}/}\color{black}}\ [c.]\ \ $\bullet$\ \ \setlength\topsep{0pt}\textbf{\foreignlanguage{arabic}{يُفْقُز}}\ {\color{gray}\texttt{/\sffamily {{\sffamily jufquz, jufkus}}/}\color{black}}\ [i.]\ \color{gray}(msa. \foreignlanguage{arabic}{يَلتوى (كاحل)}~\foreignlanguage{arabic}{\textbf{١.}})\color{black}\  \begin{flushright}\color{gray}\foreignlanguage{arabic}{\textbf{\underline{\foreignlanguage{arabic}{أمثلة}}}: فَقْزَت اجري وأنا بلعب فطبول}\end{flushright}\color{black}} \vspace{2mm}

\vspace{-3mm}
\markboth{\color{blue}\foreignlanguage{arabic}{ف.ق.س}\color{blue}{}}{\color{blue}\foreignlanguage{arabic}{ف.ق.س}\color{blue}{}}\subsection*{\color{blue}\foreignlanguage{arabic}{ف.ق.س}\color{blue}{}\index{\color{blue}\foreignlanguage{arabic}{ف.ق.س}\color{blue}{}}} 

{\setlength\topsep{0pt}\textbf{\foreignlanguage{arabic}{اِنْفَقَس}}\ {\color{gray}\texttt{/\sffamily {{\sffamily ʔinfa(q)as}}/}\color{black}}\ \textsc{verb}\ [p.]\ \textbf{1.}~be embarrassed.  \textbf{2.}~be disappointed\ \ $\bullet$\ \ \setlength\topsep{0pt}\textbf{\foreignlanguage{arabic}{اِنْفِقِس}}\ {\color{gray}\texttt{/\sffamily {{\sffamily ʔinfi(q)is}}/}\color{black}}\ [c.]\ \ $\bullet$\ \ \setlength\topsep{0pt}\textbf{\foreignlanguage{arabic}{اِنِفْقِس}}\ {\color{gray}\texttt{/\sffamily {{\sffamily ʔinif(q)is}}/}\color{black}}\ [c.]\ \ $\bullet$\ \ \setlength\topsep{0pt}\textbf{\foreignlanguage{arabic}{يِنْفِقِس}}\ {\color{gray}\texttt{/\sffamily {{\sffamily jinfi(q)is}}/}\color{black}}\ [i.]\ \ $\bullet$\ \ \setlength\topsep{0pt}\textbf{\foreignlanguage{arabic}{يِنِفْقِس}}\ {\color{gray}\texttt{/\sffamily {{\sffamily jinif(q)is}}/}\color{black}}\ [i.]\  \begin{flushright}\color{gray}\foreignlanguage{arabic}{\textbf{\underline{\foreignlanguage{arabic}{أمثلة}}}: والله اِنْفَقْست مسكينة بس عرفت انه جاي يخطب اختا مش يخطبها هي}\end{flushright}\color{black}} \vspace{2mm}

{\setlength\topsep{0pt}\textbf{\foreignlanguage{arabic}{فَاقِس}}\ {\color{gray}\texttt{/\sffamily {{\sffamily faːɡis}}/}\color{black}}\ \textsc{adj}\ [m.]\ \textbf{1.}~very boring\ } \vspace{2mm}

{\setlength\topsep{0pt}\textbf{\foreignlanguage{arabic}{فَاقِس}}\ {\color{gray}\texttt{/\sffamily {{\sffamily faː(q)is}}/}\color{black}}\ \textsc{noun\textunderscore act}\ \textbf{1.}~hatching  \textbf{2.}~cracking\ \ $\bullet$\ \ \textsc{ph.} \color{gray} \foreignlanguage{arabic}{مش فَاقس من البيضة}\color{black}\ {\color{gray}\texttt{/{\sffamily miʃ faːqis minil beːdˤa}/}\color{black}}\ \textbf{1.}~It is an idiomatic expression that means that sb is inexperienced\ \ $\bullet$\ \ \textsc{ph.} \color{gray} \foreignlanguage{arabic}{بطنهَا فَاقِس}\color{black}\ {\color{gray}\texttt{/{\sffamily batˤinha faːqis}/}\color{black}}\ \textbf{1.}~It is an idiomatic expression that means that a pregnant woman (ninth month) is about to deliver a baby because her belly is too big (the pregnancy bump is narrow and pointed)\  \begin{flushright}\color{gray}\foreignlanguage{arabic}{\textbf{\underline{\foreignlanguage{arabic}{أمثلة}}}: آية بطنها فاقِس شكلها والله العليم رح تولد بهاليومين}\end{flushright}\color{black}} \vspace{2mm}

{\setlength\topsep{0pt}\textbf{\foreignlanguage{arabic}{فَقَس}}\ {\color{gray}\texttt{/\sffamily {{\sffamily fa(q)as}}/}\color{black}}\ \textsc{verb}\ [p.]\ \textbf{1.}~disappoint  \textbf{2.}~embarrass  \textbf{3.}~hatch (eggs).  \textbf{4.}~run away\ \ $\bullet$\ \ \setlength\topsep{0pt}\textbf{\foreignlanguage{arabic}{اِفْقِس}}\ {\color{gray}\texttt{/\sffamily {{\sffamily ʔif(q)is}}/}\color{black}}\ [c.]\ \ $\bullet$\ \ \setlength\topsep{0pt}\textbf{\foreignlanguage{arabic}{اُفْقُس}}\ {\color{gray}\texttt{/\sffamily {{\sffamily ʔuf(q)us}}/}\color{black}}\ [c.]\ \ $\bullet$\ \ \setlength\topsep{0pt}\textbf{\foreignlanguage{arabic}{يِفْقِس}}\ {\color{gray}\texttt{/\sffamily {{\sffamily jif(q)is}}/}\color{black}}\ [i.]\ \color{gray}(msa. \foreignlanguage{arabic}{يهرب}~\foreignlanguage{arabic}{\textbf{٣.}}  \foreignlanguage{arabic}{يَفْقس}~\foreignlanguage{arabic}{\textbf{٢.}}  \foreignlanguage{arabic}{يُحْبِط}~\foreignlanguage{arabic}{\textbf{١.}})\color{black}\ \ $\bullet$\ \ \setlength\topsep{0pt}\textbf{\foreignlanguage{arabic}{يُفْقُس}}\ {\color{gray}\texttt{/\sffamily {{\sffamily juf(q)us}}/}\color{black}}\ [i.]\ \color{gray}(msa. \foreignlanguage{arabic}{يهرب}~\foreignlanguage{arabic}{\textbf{٤.}}  \foreignlanguage{arabic}{يَفْقس}~\foreignlanguage{arabic}{\textbf{٣.}}  \foreignlanguage{arabic}{يُحْرِج}~\foreignlanguage{arabic}{\textbf{٢.}}  \foreignlanguage{arabic}{يُحْبِط}~\foreignlanguage{arabic}{\textbf{١.}})\color{black}\  \begin{flushright}\color{gray}\foreignlanguage{arabic}{\textbf{\underline{\foreignlanguage{arabic}{أمثلة}}}: فَقَس البيض كله وما خلالي ولا بيضة أعمل فيها الكيكس\ $\bullet$\ \  افقس بسرعة وتخبي بين الشجر\ $\bullet$\ \  اجى عنا العريس وأنا متعشمة فيه كتير بعدين فَقَسْنِي ومارجع وراح خطب صاحبتي. الحمدلله كل شي قسمة ونصيب.}\end{flushright}\color{black}} \vspace{2mm}

{\setlength\topsep{0pt}\textbf{\foreignlanguage{arabic}{فَقِس}}\ {\color{gray}\texttt{/\sffamily {{\sffamily fa(q)is}}/}\color{black}}\ \textsc{noun}\ [m.]\ \textbf{1.}~cracking sth (an egg)\ \ $\smblkdiamond$\ \ \setlength\topsep{0pt}\textbf{\foreignlanguage{arabic}{فَقِس}}\ {\color{gray}\texttt{/faqis/}\color{black}}\ \textbf{1.}~chicks  \textbf{2.}~offspring  \textbf{3.}~kids\  \begin{flushright}\color{gray}\foreignlanguage{arabic}{\textbf{\underline{\foreignlanguage{arabic}{أمثلة}}}: هاي العيلة فَقِسهم كثير حلو\ $\bullet$\ \  يعني هلا فَقِس البيض صار اسمه طبيخ؟}\end{flushright}\color{black}} \vspace{2mm}

{\setlength\topsep{0pt}\textbf{\foreignlanguage{arabic}{فَقَّس}}\ {\color{gray}\texttt{/\sffamily {{\sffamily fa(q)(q)as}}/}\color{black}}\ \textsc{verb}\ [p.]\ \textbf{1.}~cracked  \textbf{2.}~hatch  \textbf{3.}~give birth to a baby\ \ $\bullet$\ \ \setlength\topsep{0pt}\textbf{\foreignlanguage{arabic}{فَقِّس}}\ {\color{gray}\texttt{/\sffamily {{\sffamily fa(q)(q)is}}/}\color{black}}\ [c.]\ \ $\bullet$\ \ \setlength\topsep{0pt}\textbf{\foreignlanguage{arabic}{يفَقِّس}}\ {\color{gray}\texttt{/\sffamily {{\sffamily jfa(q)(q)is}}/}\color{black}}\ [i.]\ \color{gray}(msa. \foreignlanguage{arabic}{تَلِد}~\foreignlanguage{arabic}{\textbf{٣.}}  \foreignlanguage{arabic}{يفْقِس}~\foreignlanguage{arabic}{\textbf{٢.}}  \foreignlanguage{arabic}{يُحطَّم}~\foreignlanguage{arabic}{\textbf{١.}})\color{black}\  \begin{flushright}\color{gray}\foreignlanguage{arabic}{\textbf{\underline{\foreignlanguage{arabic}{أمثلة}}}: كل سنة بِتْفَقِّس واحد مش ملحقة عليها مباركات\ $\bullet$\ \  كان عنا جاجة باضت 6 بيضات فَقَّسَت وحدة والباقي قليناهم عالفطور\ $\bullet$\ \  فَقَّس البيض كله وما خلالي ولا بيضة أعمل فيها الكيكس}\end{flushright}\color{black}} \vspace{2mm}

{\setlength\topsep{0pt}\textbf{\foreignlanguage{arabic}{فَقُّوس}}\footnote{Collective noun}\ \ {\color{gray}\texttt{/\sffamily {{\sffamily fa(q)(q)uːs}}/}\color{black}}\ \textsc{noun}\ [m.]\ \textbf{1.}~snake cucumber.  \textbf{2.}~snake melon\ \ $\bullet$\ \ \textsc{ph.} \color{gray} \foreignlanguage{arabic}{خيَار وفقوس}\color{black}\ {\color{gray}\texttt{/{\sffamily xjaːruw fa(q)(q)uːs}/}\color{black}}\ \color{gray} (msa. \foreignlanguage{arabic}{إِخفاق العدالة}~\foreignlanguage{arabic}{\textbf{١.}})\color{black}\ \textbf{1.}~cucumber and snake melon (It is an idiomatic expression that means miscarriages of justice\  \begin{flushright}\color{gray}\foreignlanguage{arabic}{\textbf{\underline{\foreignlanguage{arabic}{أمثلة}}}: القانون عنا خْيار وفَقُّوس}\end{flushright}\color{black}} \vspace{2mm}

{\setlength\topsep{0pt}\textbf{\foreignlanguage{arabic}{فَقُّوسِة}}\footnote{Unit noun}\ \ {\color{gray}\texttt{/\sffamily {{\sffamily faqquːse}}/}\color{black}}\ \textsc{noun}\ [f.]\ \textbf{1.}~one piece of snake cucumber.  \textbf{2.}~snake melon\  \begin{flushright}\color{gray}\foreignlanguage{arabic}{\textbf{\underline{\foreignlanguage{arabic}{أمثلة}}}: تناول هالفَقُّوسِة مني تستحيش}\end{flushright}\color{black}} \vspace{2mm}

{\setlength\topsep{0pt}\textbf{\foreignlanguage{arabic}{فَقْسِة}}\ {\color{gray}\texttt{/\sffamily {{\sffamily faqse}}/}\color{black}}\ \textsc{noun}\ [f.]\ \color{gray}(msa. \foreignlanguage{arabic}{خيبة أمل}~\foreignlanguage{arabic}{\textbf{١.}})\color{black}\ \textbf{1.}~disappointment\  \begin{flushright}\color{gray}\foreignlanguage{arabic}{\textbf{\underline{\foreignlanguage{arabic}{أمثلة}}}: عزموا العالم عالجاهة وياحرام بعدها بطلوا الأهل. الشغلة صارت فَقْسِة للعروس وأهلها.}\end{flushright}\color{black}} \vspace{2mm}

{\setlength\topsep{0pt}\textbf{\foreignlanguage{arabic}{مَفْقَسِة}}\ {\color{gray}\texttt{/\sffamily {{\sffamily mafqase}}/}\color{black}}\ \textsc{noun}\ [f.]\ \textbf{1.}~egg incubator.  \textbf{2.}~the place where people give birth to a lot of babies\ \ $\bullet$\ \ \setlength\topsep{0pt}\textbf{\foreignlanguage{arabic}{مَفَاقِس}}\ {\color{gray}\texttt{/\sffamily {{\sffamily mafaːqis}}/}\color{black}}\ [pl.]\  \begin{flushright}\color{gray}\foreignlanguage{arabic}{\textbf{\underline{\foreignlanguage{arabic}{أمثلة}}}: الله وكيلك المخيم عنا صار مَفْقَسِة}\end{flushright}\color{black}} \vspace{2mm}

\vspace{-3mm}
\markboth{\color{blue}\foreignlanguage{arabic}{ف.ق.ش}\color{blue}{}}{\color{blue}\foreignlanguage{arabic}{ف.ق.ش}\color{blue}{}}\subsection*{\color{blue}\foreignlanguage{arabic}{ف.ق.ش}\color{blue}{}\index{\color{blue}\foreignlanguage{arabic}{ف.ق.ش}\color{blue}{}}} 

{\setlength\topsep{0pt}\textbf{\foreignlanguage{arabic}{فَاقِش}}\ {\color{gray}\texttt{/\sffamily {{\sffamily faː(q)iʃ}}/}\color{black}}\ \textsc{noun\textunderscore act}\ [m.]\ \textbf{1.}~breaking  \textbf{2.}~cracking\ \ $\bullet$\ \ \textsc{ph.} \color{gray} \foreignlanguage{arabic}{وجهه فَاقِش}\color{black}\ {\color{gray}\texttt{/{\sffamily wi(dʒ)ho faː(q)iʃ}/}\color{black}}\ \textbf{1.}~look very tired, pale and sick\  \begin{flushright}\color{gray}\foreignlanguage{arabic}{\textbf{\underline{\foreignlanguage{arabic}{أمثلة}}}: ماله ابنها وجهه فاقِش؟ خير ان شاء الله مايكون صايرله شي\ $\bullet$\ \  أنو اللي فاقِش البيض المحطوط هالمجلى}\end{flushright}\color{black}} \vspace{2mm}

{\setlength\topsep{0pt}\textbf{\foreignlanguage{arabic}{فَقَش}}\ {\color{gray}\texttt{/\sffamily {{\sffamily fa(q)aʃ}}/}\color{black}}\ \textsc{verb}\ [p.]\ \textbf{1.}~break  \textbf{2.}~crack  \textbf{3.}~play finger zills\ \ $\bullet$\ \ \setlength\topsep{0pt}\textbf{\foreignlanguage{arabic}{اِفْقُش}}\ {\color{gray}\texttt{/\sffamily {{\sffamily ʔuf(q)uʃ}}/}\color{black}}\ [c.]\ \ $\bullet$\ \ \setlength\topsep{0pt}\textbf{\foreignlanguage{arabic}{يُفْقُش}}\ {\color{gray}\texttt{/\sffamily {{\sffamily juf(q)uʃ}}/}\color{black}}\ [i.]\ \color{gray}(msa. \foreignlanguage{arabic}{يَكْسِر}~\foreignlanguage{arabic}{\textbf{١.}})\color{black}\  \begin{flushright}\color{gray}\foreignlanguage{arabic}{\textbf{\underline{\foreignlanguage{arabic}{أمثلة}}}: استنى عليه يُفْقُش لحاله وشوف ما احلاهم الصيصان بيكونوا\ $\bullet$\ \  افْقُشيلها وارقصيلها يختي!}\end{flushright}\color{black}} \vspace{2mm}

{\setlength\topsep{0pt}\textbf{\foreignlanguage{arabic}{فَقَّش}}\ {\color{gray}\texttt{/\sffamily {{\sffamily fa(q)(q)aʃ}}/}\color{black}}\ \textsc{verb}\ [p.]\ \textbf{1.}~smash  \textbf{2.}~break (causative)\ \ $\bullet$\ \ \setlength\topsep{0pt}\textbf{\foreignlanguage{arabic}{فَقِّش}}\ {\color{gray}\texttt{/\sffamily {{\sffamily fa(q)(q)iʃ}}/}\color{black}}\ [c.]\ \ $\bullet$\ \ \setlength\topsep{0pt}\textbf{\foreignlanguage{arabic}{يفَقِّش}}\ {\color{gray}\texttt{/\sffamily {{\sffamily jfa(q)(q)iʃ}}/}\color{black}}\ [i.]\ \color{gray}(msa. \foreignlanguage{arabic}{يُحَطِّم}~\foreignlanguage{arabic}{\textbf{١.}})\color{black}\  \begin{flushright}\color{gray}\foreignlanguage{arabic}{\textbf{\underline{\foreignlanguage{arabic}{أمثلة}}}: يعني هو أنا كاينة فاضية ومن الفضاوة بفَقِّش بيض}\end{flushright}\color{black}} \vspace{2mm}

{\setlength\topsep{0pt}\textbf{\foreignlanguage{arabic}{فُقَّاشِيِّة}}\ {\color{gray}\texttt{/\sffamily {{\sffamily fuqqaːʃijje}}/}\color{black}}\ \textsc{noun}\ [f.]\ \color{gray}(msa. \foreignlanguage{arabic}{صْناجَة}~\foreignlanguage{arabic}{\textbf{١.}})\color{black}\ \textbf{1.}~Castanets\  \begin{flushright}\color{gray}\foreignlanguage{arabic}{\textbf{\underline{\foreignlanguage{arabic}{أمثلة}}}: يختي كيف بتستخدمي الفُقّاشِيّات}\end{flushright}\color{black}} \vspace{2mm}

{\setlength\topsep{0pt}\textbf{\foreignlanguage{arabic}{مَفْقُوش}}\ {\color{gray}\texttt{/\sffamily {{\sffamily maf(q)uːʃ}}/}\color{black}}\ \textsc{noun\textunderscore pass}\ \color{gray}(msa. \foreignlanguage{arabic}{مُحَطَّم}~\foreignlanguage{arabic}{\textbf{١.}})\color{black}\ \textbf{1.}~cracked  \textbf{2.}~smashed\  \begin{flushright}\color{gray}\foreignlanguage{arabic}{\textbf{\underline{\foreignlanguage{arabic}{أمثلة}}}: ليشالبيض هيك مَفْقوش؟ أنو اللي فاقْشُه؟}\end{flushright}\color{black}} \vspace{2mm}

\vspace{-3mm}
\markboth{\color{blue}\foreignlanguage{arabic}{ف.ق.ع}\color{blue}{}}{\color{blue}\foreignlanguage{arabic}{ف.ق.ع}\color{blue}{}}\subsection*{\color{blue}\foreignlanguage{arabic}{ف.ق.ع}\color{blue}{}\index{\color{blue}\foreignlanguage{arabic}{ف.ق.ع}\color{blue}{}}} 

{\setlength\topsep{0pt}\textbf{\foreignlanguage{arabic}{اِنْفَقَع}}\ {\color{gray}\texttt{/\sffamily {{\sffamily ʔinfa(q)aʕ}}/}\color{black}}\ \textsc{verb}\ [p.]\ \textbf{1.}~be burst.  \textbf{2.}~be exploded\ \ $\bullet$\ \ \setlength\topsep{0pt}\textbf{\foreignlanguage{arabic}{اِنْفِقِع}}\ {\color{gray}\texttt{/\sffamily {{\sffamily ʔinfi(q)iʕ}}/}\color{black}}\ [c.]\ \ $\bullet$\ \ \setlength\topsep{0pt}\textbf{\foreignlanguage{arabic}{يِنْفِقِع}}\ {\color{gray}\texttt{/\sffamily {{\sffamily jinfi(q)iʕ}}/}\color{black}}\ [i.]\  \begin{flushright}\color{gray}\foreignlanguage{arabic}{\textbf{\underline{\foreignlanguage{arabic}{أمثلة}}}: دير بالك ما يِنْفِقِع البالون}\end{flushright}\color{black}} \vspace{2mm}

{\setlength\topsep{0pt}\textbf{\foreignlanguage{arabic}{فَاقِع}}\ {\color{gray}\texttt{/\sffamily {{\sffamily faːqiʕ}}/}\color{black}}\ \textsc{adj}\ [m.]\ \color{gray}(msa. \foreignlanguage{arabic}{فاقِع}~\foreignlanguage{arabic}{\textbf{١.}})\color{black}\ \textbf{1.}~bright\ \ $\bullet$\ \ \setlength\topsep{0pt}\textbf{\foreignlanguage{arabic}{فَوَاقِع}}\ {\color{gray}\texttt{/\sffamily {{\sffamily fawaːqiʕ}}/}\color{black}}\ [pl.]\  \begin{flushright}\color{gray}\foreignlanguage{arabic}{\textbf{\underline{\foreignlanguage{arabic}{أمثلة}}}: كل الشالات تبعتها ألوانها فَواقِع كانها رايحة عرس\ $\bullet$\ \  حومرتها لونها فاقِع كثير}\end{flushright}\color{black}} \vspace{2mm}

{\setlength\topsep{0pt}\textbf{\foreignlanguage{arabic}{فَقَع}}\ {\color{gray}\texttt{/\sffamily {{\sffamily fa(q)aʕ}}/}\color{black}}\ \textsc{verb}\ [p.]\ \textbf{1.}~explode  \textbf{2.}~burst  \textbf{3.}~be enraged.  \textbf{4.}~get angry\ \ $\bullet$\ \ \setlength\topsep{0pt}\textbf{\foreignlanguage{arabic}{اِفَقَع}}\ {\color{gray}\texttt{/\sffamily {{\sffamily ʔif(q)aʕ}}/}\color{black}}\ [c.]\ \ $\bullet$\ \ \setlength\topsep{0pt}\textbf{\foreignlanguage{arabic}{يِفْقَع}}\ {\color{gray}\texttt{/\sffamily {{\sffamily jif(q)aʕ}}/}\color{black}}\ [i.]\ \ $\bullet$\ \ \textsc{ph.} \color{gray} \foreignlanguage{arabic}{فَقَع ضُحُك}\color{black}\ {\color{gray}\texttt{/{\sffamily fa(q)aʕit (dˤ)uħuk}/}\color{black}}\ \color{gray} (msa. \foreignlanguage{arabic}{يضحك بطريقة هستيرية}~\foreignlanguage{arabic}{\textbf{١.}})\color{black}\ \textbf{1.}~laugh hysterically\ \ $\bullet$\ \ \textsc{ph.} \color{gray} \foreignlanguage{arabic}{فقع مرَارتي}\color{black}\ {\color{gray}\texttt{/{\sffamily fa(q)aʕ maraːrti}/}\color{black}}\ \color{gray} (msa. \foreignlanguage{arabic}{يجعل شخض يفقد أعصابه}~\foreignlanguage{arabic}{\textbf{١.}})\color{black}\ \textbf{1.}~drive sb crazy\ \ $\bullet$\ \ \textsc{ph.} \color{gray} \foreignlanguage{arabic}{فقعت معي}\color{black}\ {\color{gray}\texttt{/{\sffamily faqʕat maʕi}/}\color{black}}\ \color{gray} (msa. \foreignlanguage{arabic}{طفح الكيل}~\foreignlanguage{arabic}{\textbf{١.}})\color{black}\ \textbf{1.}~enough is enough\  \begin{flushright}\color{gray}\foreignlanguage{arabic}{\textbf{\underline{\foreignlanguage{arabic}{أمثلة}}}: خلاص فَقْعَت معي لهون وبس\ $\bullet$\ \  ابنك فَقَع مَرارْتِي\ $\bullet$\ \  لما خرفنا لقصة فَقَعْنا ضُحُك عليه\ $\bullet$\ \  اِفَقَع البلالين بالدبوس\ $\bullet$\ \  فَقَعِْت منه هو بحكي وبتفلسف وانا بغلي وجواتي نار}\end{flushright}\color{black}} \vspace{2mm}

{\setlength\topsep{0pt}\textbf{\foreignlanguage{arabic}{فَقِع}}\footnote{Collective noun}\ \ {\color{gray}\texttt{/\sffamily {{\sffamily faqiʕ}}/}\color{black}}\ \textsc{noun}\ [m.]\ \color{gray}(msa. \foreignlanguage{arabic}{فطر}~\foreignlanguage{arabic}{\textbf{١.}})\color{black}\ \textbf{1.}~mushroom\ } \vspace{2mm}

{\setlength\topsep{0pt}\textbf{\foreignlanguage{arabic}{فَقَّع}}\ {\color{gray}\texttt{/\sffamily {{\sffamily fa(q)(q)aʕ}}/}\color{black}}\ \textsc{verb}\ [p.]\ \textbf{1.}~cause sth to burst.  \textbf{2.}~infuriate sb.  \textbf{3.}~enrage sb\ \ $\bullet$\ \ \setlength\topsep{0pt}\textbf{\foreignlanguage{arabic}{فَقِّع}}\ {\color{gray}\texttt{/\sffamily {{\sffamily fa(q)(q)iʕ}}/}\color{black}}\ [c.]\ \ $\bullet$\ \ \setlength\topsep{0pt}\textbf{\foreignlanguage{arabic}{يفَقِّع}}\ {\color{gray}\texttt{/\sffamily {{\sffamily jfa(q)(q)iʕ}}/}\color{black}}\ [i.]\ \color{gray}(msa. \foreignlanguage{arabic}{يُغْضِب}~\foreignlanguage{arabic}{\textbf{١.}})\color{black}\  \begin{flushright}\color{gray}\foreignlanguage{arabic}{\textbf{\underline{\foreignlanguage{arabic}{أمثلة}}}: فَقِّعله بالُّونته\ $\bullet$\ \  فقَّعْني الله لا يجبره}\end{flushright}\color{black}} \vspace{2mm}

{\setlength\topsep{0pt}\textbf{\foreignlanguage{arabic}{فُقُع}}\ {\color{gray}\texttt{/\sffamily {{\sffamily fuquʕ}}/}\color{black}}\ \textsc{noun}\ [m.]\ \color{gray}(msa. \foreignlanguage{arabic}{رقروق أو الهشيمة أو زهرة الشمس}~\foreignlanguage{arabic}{\textbf{١.}})\color{black}\ \textbf{1.}~Helianthemum\ } \vspace{2mm}

{\setlength\topsep{0pt}\textbf{\foreignlanguage{arabic}{فُقْعَة}}\footnote{Unit noun}\ \ {\color{gray}\texttt{/\sffamily {{\sffamily fuqʕa}}/}\color{black}}\ \textsc{noun}\ [f.]\ \color{gray}(msa. \foreignlanguage{arabic}{حبة فطر}~\foreignlanguage{arabic}{\textbf{١.}})\color{black}\ \textbf{1.}~a mushroom\  \begin{flushright}\color{gray}\foreignlanguage{arabic}{\textbf{\underline{\foreignlanguage{arabic}{أمثلة}}}: واحنا ندور اليوم في السهل لقيت فقعة كبيرة}\end{flushright}\color{black}} \vspace{2mm}

{\setlength\topsep{0pt}\textbf{\foreignlanguage{arabic}{مَفْقُوع}}\ {\color{gray}\texttt{/\sffamily {{\sffamily maf(q)uːʕ}}/}\color{black}}\ \textsc{noun\textunderscore pass}\ \textbf{1.}~bored  \textbf{2.}~be very upset with sb\  \begin{flushright}\color{gray}\foreignlanguage{arabic}{\textbf{\underline{\foreignlanguage{arabic}{أمثلة}}}: أنا مَفْقُوع منك من زمان}\end{flushright}\color{black}} \vspace{2mm}

\vspace{-3mm}
\markboth{\color{blue}\foreignlanguage{arabic}{ف.ق.ف.ق}\color{blue}{}}{\color{blue}\foreignlanguage{arabic}{ف.ق.ف.ق}\color{blue}{}}\subsection*{\color{blue}\foreignlanguage{arabic}{ف.ق.ف.ق}\color{blue}{}\index{\color{blue}\foreignlanguage{arabic}{ف.ق.ف.ق}\color{blue}{}}} 

{\setlength\topsep{0pt}\textbf{\foreignlanguage{arabic}{فَقْفَق}}\ {\color{gray}\texttt{/\sffamily {{\sffamily faqfaq, faʔfaʔ}}/}\color{black}}\ \textsc{verb}\ [p.]\ \textbf{1.}~have blisteres\ \ $\bullet$\ \ \setlength\topsep{0pt}\textbf{\foreignlanguage{arabic}{فَقْفِق}}\ {\color{gray}\texttt{/\sffamily {{\sffamily faqfiq, faʔfiʔ}}/}\color{black}}\ [c.]\ \ $\bullet$\ \ \setlength\topsep{0pt}\textbf{\foreignlanguage{arabic}{يفَقْفِق}}\ {\color{gray}\texttt{/\sffamily {{\sffamily jfaqfiq, jfaʔfiʔ}}/}\color{black}}\ [i.]\ \color{gray}(msa. \foreignlanguage{arabic}{يَتَقَرَّح}~\foreignlanguage{arabic}{\textbf{١.}})\color{black}\  \begin{flushright}\color{gray}\foreignlanguage{arabic}{\textbf{\underline{\foreignlanguage{arabic}{أمثلة}}}: من بعد الشغل بالحراثة ايدي فَقْفَقت}\end{flushright}\color{black}} \vspace{2mm}

{\setlength\topsep{0pt}\textbf{\foreignlanguage{arabic}{مْفَقْفِق}}\ {\color{gray}\texttt{/\sffamily {{\sffamily mfaqfiq, mfaʔfiʔ}}/}\color{black}}\ \textsc{adj}\ [m.]\ \color{gray}(msa. \foreignlanguage{arabic}{مُتَقَرِّح}~\foreignlanguage{arabic}{\textbf{١.}})\color{black}\ \textbf{1.}~blistered\  \begin{flushright}\color{gray}\foreignlanguage{arabic}{\textbf{\underline{\foreignlanguage{arabic}{أمثلة}}}: مالها اجرك مْفَقْفِقة؟\ $\bullet$\ \  شوف دكتور كيف جلدي مْفَقْفِق}\end{flushright}\color{black}} \vspace{2mm}

\vspace{-3mm}
\markboth{\color{blue}\foreignlanguage{arabic}{ف.ق.ق}\color{blue}{}}{\color{blue}\foreignlanguage{arabic}{ف.ق.ق}\color{blue}{}}\subsection*{\color{blue}\foreignlanguage{arabic}{ف.ق.ق}\color{blue}{}\index{\color{blue}\foreignlanguage{arabic}{ف.ق.ق}\color{blue}{}}} 

{\setlength\topsep{0pt}\textbf{\foreignlanguage{arabic}{فَقَّاق}}\ {\color{gray}\texttt{/\sffamily {{\sffamily faqqaːq}}/}\color{black}}\ \textsc{adj}\ [m.]\ \textbf{1.}~foul-mouthed\  \begin{flushright}\color{gray}\foreignlanguage{arabic}{\textbf{\underline{\foreignlanguage{arabic}{أمثلة}}}: مرته فَقّاقة بتسيبش حدا من شرها}\end{flushright}\color{black}} \vspace{2mm}

{\setlength\topsep{0pt}\textbf{\foreignlanguage{arabic}{فَقَّق}}\ {\color{gray}\texttt{/\sffamily {{\sffamily faqqaq}}/}\color{black}}\ \textsc{verb}\ [p.]\ \textbf{1.}~yell at sb and curse at him in an uncivilized way\ \ $\bullet$\ \ \setlength\topsep{0pt}\textbf{\foreignlanguage{arabic}{فَقِّق}}\ {\color{gray}\texttt{/\sffamily {{\sffamily faqqiq}}/}\color{black}}\ [c.]\ \ $\bullet$\ \ \setlength\topsep{0pt}\textbf{\foreignlanguage{arabic}{يفَقِّق}}\ {\color{gray}\texttt{/\sffamily {{\sffamily jfaqqiq}}/}\color{black}}\ [i.]\  \begin{flushright}\color{gray}\foreignlanguage{arabic}{\textbf{\underline{\foreignlanguage{arabic}{أمثلة}}}: بس قلتله جيب كيلو لحمة صار يفَقِّق ويلعن اليوم اللي تجوزني فيه}\end{flushright}\color{black}} \vspace{2mm}

\vspace{-3mm}
\markboth{\color{blue}\foreignlanguage{arabic}{ف.ق.م}\color{blue}{}}{\color{blue}\foreignlanguage{arabic}{ف.ق.م}\color{blue}{}}\subsection*{\color{blue}\foreignlanguage{arabic}{ف.ق.م}\color{blue}{}\index{\color{blue}\foreignlanguage{arabic}{ف.ق.م}\color{blue}{}}} 

{\setlength\topsep{0pt}\textbf{\foreignlanguage{arabic}{اِفْقَم}}\ {\color{gray}\texttt{/\sffamily {{\sffamily ʔifqam}}/}\color{black}}\ \textsc{adj}\ [m.]\ \textbf{1.}~have deformities of the jaws\ \ $\bullet$\ \ \setlength\topsep{0pt}\textbf{\foreignlanguage{arabic}{فَقْمَا}}\ {\color{gray}\texttt{/\sffamily {{\sffamily faqma}}/}\color{black}}\ [f.]\ \ $\bullet$\ \ \setlength\topsep{0pt}\textbf{\foreignlanguage{arabic}{فُقُم}}\ {\color{gray}\texttt{/\sffamily {{\sffamily fuqum}}/}\color{black}}\ [pl.]\  \begin{flushright}\color{gray}\foreignlanguage{arabic}{\textbf{\underline{\foreignlanguage{arabic}{أمثلة}}}: ابن سعيد طويل وشخصية  بس مشكلته اِفْقَم}\end{flushright}\color{black}} \vspace{2mm}

{\setlength\topsep{0pt}\textbf{\foreignlanguage{arabic}{تَفَاقُم}}\ {\color{gray}\texttt{/\sffamily {{\sffamily tafaːqum}}/}\color{black}}\ \textsc{noun}\ [m.]\ \textbf{1.}~aggravation  \textbf{2.}~exacerbation\ } \vspace{2mm}

{\setlength\topsep{0pt}\textbf{\foreignlanguage{arabic}{تْفَاقَم}}\ {\color{gray}\texttt{/\sffamily {{\sffamily tfaːqam}}/}\color{black}}\ \textsc{verb}\ [p.]\ \textbf{1.}~aggravate  \textbf{2.}~exacerbate\ \ $\bullet$\ \ \setlength\topsep{0pt}\textbf{\foreignlanguage{arabic}{اِتْفَاقَم}}\ {\color{gray}\texttt{/\sffamily {{\sffamily ʔitfaːqam}}/}\color{black}}\ [c.]\ \ $\bullet$\ \ \setlength\topsep{0pt}\textbf{\foreignlanguage{arabic}{يِتْفَاقَم}}\ {\color{gray}\texttt{/\sffamily {{\sffamily jitfaːqam}}/}\color{black}}\ [i.]\  \begin{flushright}\color{gray}\foreignlanguage{arabic}{\textbf{\underline{\foreignlanguage{arabic}{أمثلة}}}: لما بلشت خلافاتنا تِتَفاقَم، قررنا ننفصل بهدوء بدون شوشرة عشان الأولاد}\end{flushright}\color{black}} \vspace{2mm}

{\setlength\topsep{0pt}\textbf{\foreignlanguage{arabic}{فَاقَم}}\ {\color{gray}\texttt{/\sffamily {{\sffamily faːqam}}/}\color{black}}\ \textsc{verb}\ [p.]\ \textbf{1.}~make a situation aggravating.  \textbf{2.}~exacerbated\ \ $\bullet$\ \ \setlength\topsep{0pt}\textbf{\foreignlanguage{arabic}{فَاقِم}}\ {\color{gray}\texttt{/\sffamily {{\sffamily faːqim}}/}\color{black}}\ [c.]\ \ $\bullet$\ \ \setlength\topsep{0pt}\textbf{\foreignlanguage{arabic}{يفَاقِم}}\ {\color{gray}\texttt{/\sffamily {{\sffamily jfaːqim}}/}\color{black}}\ [i.]\  \begin{flushright}\color{gray}\foreignlanguage{arabic}{\textbf{\underline{\foreignlanguage{arabic}{أمثلة}}}: تدهنيه بالزيت مش حل عشانه رح يفاقِم المشكلة مش رح يحلها من جذورها}\end{flushright}\color{black}} \vspace{2mm}

{\setlength\topsep{0pt}\textbf{\foreignlanguage{arabic}{فَقْمِة}}\ {\color{gray}\texttt{/\sffamily {{\sffamily faqme}}/}\color{black}}\ \textsc{noun}\ [f.]\ \color{gray}(msa. \foreignlanguage{arabic}{فَقْمَة}~\foreignlanguage{arabic}{\textbf{١.}})\color{black}\ \textbf{1.}~seal\  \begin{flushright}\color{gray}\foreignlanguage{arabic}{\textbf{\underline{\foreignlanguage{arabic}{أمثلة}}}: شكلك باليانس تبع الصلاة مثل الفَقْمِة ههههه}\end{flushright}\color{black}} \vspace{2mm}

{\setlength\topsep{0pt}\textbf{\foreignlanguage{arabic}{مُتَفَاقِم}}\ {\color{gray}\texttt{/\sffamily {{\sffamily mutafaːqim}}/}\color{black}}\ \textsc{adj}\ [m.]\ \textbf{1.}~aggravating  \textbf{2.}~exacerbated\ } \vspace{2mm}

\vspace{-3mm}
\markboth{\color{blue}\foreignlanguage{arabic}{ف.ق.ه}\color{blue}{}}{\color{blue}\foreignlanguage{arabic}{ف.ق.ه}\color{blue}{}}\subsection*{\color{blue}\foreignlanguage{arabic}{ف.ق.ه}\color{blue}{}\index{\color{blue}\foreignlanguage{arabic}{ف.ق.ه}\color{blue}{}}} 

{\setlength\topsep{0pt}\textbf{\foreignlanguage{arabic}{تْفَقَّه}}\ {\color{gray}\texttt{/\sffamily {{\sffamily tfaqqah}}/}\color{black}}\ \textsc{verb}\ [p.]\ \textbf{1.}~seek more knowledge in Fiqh (Islamic jurisprudence)\ \ $\bullet$\ \ \setlength\topsep{0pt}\textbf{\foreignlanguage{arabic}{اِتْفَقَّه}}\ {\color{gray}\texttt{/\sffamily {{\sffamily ʔitfaqqah}}/}\color{black}}\ [c.]\ \ $\bullet$\ \ \setlength\topsep{0pt}\textbf{\foreignlanguage{arabic}{يِتْفَقَّه}}\ {\color{gray}\texttt{/\sffamily {{\sffamily jitfaqqah}}/}\color{black}}\ [i.]\  \begin{flushright}\color{gray}\foreignlanguage{arabic}{\textbf{\underline{\foreignlanguage{arabic}{أمثلة}}}: لازم نتْفَقَّه بأمور ديننا زي الغسل وغيره}\end{flushright}\color{black}} \vspace{2mm}

{\setlength\topsep{0pt}\textbf{\foreignlanguage{arabic}{فَقِيه}}\ {\color{gray}\texttt{/\sffamily {{\sffamily faqiːh}}/}\color{black}}\ \textsc{noun}\ [m.]\ \textbf{1.}~an Islamic scholar in Fiqh.  \textbf{2.}~slamic jurisprudence.\  \begin{flushright}\color{gray}\foreignlanguage{arabic}{\textbf{\underline{\foreignlanguage{arabic}{أمثلة}}}: هلا شيخ المسجد أبو قتادة صار فَقِيه وصرتوا تسألوه عن فتاوي؟}\end{flushright}\color{black}} \vspace{2mm}

{\setlength\topsep{0pt}\textbf{\foreignlanguage{arabic}{فِقِه}}\ {\color{gray}\texttt{/\sffamily {{\sffamily fiqih}}/}\color{black}}\ \textsc{noun}\ [m.]\ \color{gray}(msa. \foreignlanguage{arabic}{فِقْه}~\foreignlanguage{arabic}{\textbf{١.}})\color{black}\ \textbf{1.}~Fiqh  \textbf{2.}~Islamic jurisprudence.\  \begin{flushright}\color{gray}\foreignlanguage{arabic}{\textbf{\underline{\foreignlanguage{arabic}{أمثلة}}}: دارسة فِقِه وشريعة وبتحضِّر للماجستير}\end{flushright}\color{black}} \vspace{2mm}

{\setlength\topsep{0pt}\textbf{\foreignlanguage{arabic}{فِقِه}}\ {\color{gray}\texttt{/\sffamily {{\sffamily fiqih}}/}\color{black}}\ \textsc{verb}\ [p.]\ \textbf{1.}~know\ \ $\bullet$\ \ \setlength\topsep{0pt}\textbf{\foreignlanguage{arabic}{اِفْقَه}}\ {\color{gray}\texttt{/\sffamily {{\sffamily ʔifqah}}/}\color{black}}\ [c.]\ \ $\bullet$\ \ \setlength\topsep{0pt}\textbf{\foreignlanguage{arabic}{يِفْقَه}}\ {\color{gray}\texttt{/\sffamily {{\sffamily jifqah}}/}\color{black}}\ [i.]\ \color{gray}(msa. \foreignlanguage{arabic}{يَعْلَم}~\foreignlanguage{arabic}{\textbf{١.}})\color{black}\ \ $\bullet$\ \ \textsc{ph.} \color{gray} \foreignlanguage{arabic}{لَا يفْقَه شي}\color{black}\ {\color{gray}\texttt{/{\sffamily laː jifqah ʃiː}/}\color{black}}\ \textbf{1.}~have know knowledge.  \textbf{2.}~ignorant\  \begin{flushright}\color{gray}\foreignlanguage{arabic}{\textbf{\underline{\foreignlanguage{arabic}{أمثلة}}}: سألته شوي عن الرياضيات وطلع لا يفْقَه شي}\end{flushright}\color{black}} \vspace{2mm}

\vspace{-3mm}
\markboth{\color{blue}\foreignlanguage{arabic}{ف.ق.ي}\color{blue}{}}{\color{blue}\foreignlanguage{arabic}{ف.ق.ي}\color{blue}{}}\subsection*{\color{blue}\foreignlanguage{arabic}{ف.ق.ي}\color{blue}{}\index{\color{blue}\foreignlanguage{arabic}{ف.ق.ي}\color{blue}{}}} 

{\setlength\topsep{0pt}\textbf{\foreignlanguage{arabic}{فَاقَى}}\ {\color{gray}\texttt{/\sffamily {{\sffamily faaqa, faaka}}/}\color{black}}\ \textsc{verb}\ [p.]\ \textbf{1.}~cry intermittently and make noise\ \ $\bullet$\ \ \setlength\topsep{0pt}\textbf{\foreignlanguage{arabic}{فَاقِي}}\ {\color{gray}\texttt{/\sffamily {{\sffamily faaqi, faaki}}/}\color{black}}\ [c.]\ \ $\bullet$\ \ \setlength\topsep{0pt}\textbf{\foreignlanguage{arabic}{يفَاقِي}}\ {\color{gray}\texttt{/\sffamily {{\sffamily jfaaqi, jfaaki}}/}\color{black}}\ [i.]\ \color{gray}(msa. \foreignlanguage{arabic}{يبكي بشكل متقطِّع}~\foreignlanguage{arabic}{\textbf{١.}})\color{black}\  \begin{flushright}\color{gray}\foreignlanguage{arabic}{\textbf{\underline{\foreignlanguage{arabic}{أمثلة}}}: طول الليل ضل يفاقِي لا نام ولا نيَّم حدا\ $\bullet$\ \  أنت فاقِيله هيك وصدقني هو رح يزهق ويرضى يعيدلك الامتحان}\end{flushright}\color{black}} \vspace{2mm}

{\setlength\topsep{0pt}\textbf{\foreignlanguage{arabic}{فَقَى}}\ {\color{gray}\texttt{/\sffamily {{\sffamily faqa}}/}\color{black}}\ \textsc{verb}\ [p.]\ \textbf{1.}~burst (abscess)\ \ $\bullet$\ \ \setlength\topsep{0pt}\textbf{\foreignlanguage{arabic}{اِفْقِي}}\ {\color{gray}\texttt{/\sffamily {{\sffamily ʔifqi}}/}\color{black}}\ [c.]\ \ $\bullet$\ \ \setlength\topsep{0pt}\textbf{\foreignlanguage{arabic}{يِفْقِي}}\ {\color{gray}\texttt{/\sffamily {{\sffamily jifqi}}/}\color{black}}\ [i.]\ \color{gray}(msa. \foreignlanguage{arabic}{يَفْقِي دمل}~\foreignlanguage{arabic}{\textbf{١.}})\color{black}\ \ $\bullet$\ \ \textsc{ph.} \color{gray} \foreignlanguage{arabic}{فقى الدملة}\color{black}\ {\color{gray}\texttt{/{\sffamily faqa ʔiddumale}/}\color{black}}\ \color{gray} (msa. \foreignlanguage{arabic}{يتَخَلَّص من مشكلة تؤرِّقه}~\foreignlanguage{arabic}{\textbf{١.}})\color{black}\ \textbf{1.}~throw the baby out with the bathwater (get rid of a thorny problem)\  \begin{flushright}\color{gray}\foreignlanguage{arabic}{\textbf{\underline{\foreignlanguage{arabic}{أمثلة}}}: هو فَقَى الدُّمَّلِة وخلاص. ما حدا يراجعه بالموضوع.\ $\bullet$\ \  بعرف أفقيها لحالي بس بخاف}\end{flushright}\color{black}} \vspace{2mm}

{\setlength\topsep{0pt}\textbf{\foreignlanguage{arabic}{مْفَاقَاة}}\ {\color{gray}\texttt{/\sffamily {{\sffamily mfaaqaat, mfaakaat}}/}\color{black}}\ \textsc{noun}\ [f.]\ \color{gray}(msa. \foreignlanguage{arabic}{البكاء بشكل متقطع}~\foreignlanguage{arabic}{\textbf{١.}})\color{black}\ \textbf{1.}~intermittent crying with noise\ } \vspace{2mm}

\vspace{-3mm}
\markboth{\color{blue}\foreignlanguage{arabic}{ف.ك.ح}\color{blue}{}}{\color{blue}\foreignlanguage{arabic}{ف.ك.ح}\color{blue}{}}\subsection*{\color{blue}\foreignlanguage{arabic}{ف.ك.ح}\color{blue}{}\index{\color{blue}\foreignlanguage{arabic}{ف.ك.ح}\color{blue}{}}} 

{\setlength\topsep{0pt}\textbf{\foreignlanguage{arabic}{أَفْكَح}}\ {\color{gray}\texttt{/\sffamily {{\sffamily ʔaf(k)aħ}}/}\color{black}}\ \textsc{adj}\ [m.]\ \textbf{1.}~sb who does not know how to walk with balance\ \ $\bullet$\ \ \setlength\topsep{0pt}\textbf{\foreignlanguage{arabic}{فُكُح}}\ {\color{gray}\texttt{/\sffamily {{\sffamily fu(k)uħ}}/}\color{black}}\ [pl.]\  \begin{flushright}\color{gray}\foreignlanguage{arabic}{\textbf{\underline{\foreignlanguage{arabic}{أمثلة}}}: اخوتها كلهم فُكُح ولا واحد فيهم بيعرف يمشي زي الناس}\end{flushright}\color{black}} \vspace{2mm}

{\setlength\topsep{0pt}\textbf{\foreignlanguage{arabic}{اِفْكَح}}\ {\color{gray}\texttt{/\sffamily {{\sffamily ʔif(k)aħ}}/}\color{black}}\ \textsc{adj}\ [m.]\ \textbf{1.}~sb who does not know how to walk with balance\ \ $\bullet$\ \ \setlength\topsep{0pt}\textbf{\foreignlanguage{arabic}{فَكْحَا}}\ {\color{gray}\texttt{/\sffamily {{\sffamily fa(k)ħa}}/}\color{black}}\ [f.]\ } \vspace{2mm}

{\setlength\topsep{0pt}\textbf{\foreignlanguage{arabic}{تَفْكِيح}}\ {\color{gray}\texttt{/\sffamily {{\sffamily taf(k)iːħ}}/}\color{black}}\ \textsc{noun}\ [m.]\ \textbf{1.}~sitting in W-position.  \textbf{2.}~spreading sb's legs apart while sitting\ } \vspace{2mm}

{\setlength\topsep{0pt}\textbf{\foreignlanguage{arabic}{فَكَح}}\ {\color{gray}\texttt{/\sffamily {{\sffamily fa(k)aħ}}/}\color{black}}\ \textsc{verb}\ [p.]\ \textbf{1.}~go quickly.  \textbf{2.}~run away\ \ $\bullet$\ \ \setlength\topsep{0pt}\textbf{\foreignlanguage{arabic}{اِفْكَح}}\ {\color{gray}\texttt{/\sffamily {{\sffamily ʔif(k)aħ}}/}\color{black}}\ [c.]\ \ $\bullet$\ \ \setlength\topsep{0pt}\textbf{\foreignlanguage{arabic}{يِفْكَح}}\ {\color{gray}\texttt{/\sffamily {{\sffamily jif(k)aħ}}/}\color{black}}\ [i.]\ \color{gray}(msa. \foreignlanguage{arabic}{اذهب أو اهرب بسرعة}~\foreignlanguage{arabic}{\textbf{١.}})\color{black}\ \ $\bullet$\ \ \textsc{ph.} \color{gray} \foreignlanguage{arabic}{الله يفكَحَك}\color{black}\ {\color{gray}\texttt{/{\sffamily ʔalˤlˤa jif(k)aħa(k)}/}\color{black}}\ \textbf{1.}~It is an expression that means that the speaker hopes the hearer to have an accident where his legs are spread apart\  \begin{flushright}\color{gray}\foreignlanguage{arabic}{\textbf{\underline{\foreignlanguage{arabic}{أمثلة}}}: ولك افكَح بسرعة هياته جاي لعندك}\end{flushright}\color{black}} \vspace{2mm}

{\setlength\topsep{0pt}\textbf{\foreignlanguage{arabic}{فَكَّح}}\ {\color{gray}\texttt{/\sffamily {{\sffamily fa(k)(k)aħ}}/}\color{black}}\ \textsc{verb}\ [p.]\ \textbf{1.}~sit in W-position.  \textbf{2.}~spread sb's legs apart while sitting\ \ $\bullet$\ \ \setlength\topsep{0pt}\textbf{\foreignlanguage{arabic}{فَكِّح}}\ {\color{gray}\texttt{/\sffamily {{\sffamily fa(k)(k)iħ}}/}\color{black}}\ [c.]\ \ $\bullet$\ \ \setlength\topsep{0pt}\textbf{\foreignlanguage{arabic}{يفَكِّح}}\ {\color{gray}\texttt{/\sffamily {{\sffamily jfa(k)(k)iħ}}/}\color{black}}\ [i.]\  \begin{flushright}\color{gray}\foreignlanguage{arabic}{\textbf{\underline{\foreignlanguage{arabic}{أمثلة}}}: هيها فَكَّحت اجريها عطول}\end{flushright}\color{black}} \vspace{2mm}

{\setlength\topsep{0pt}\textbf{\foreignlanguage{arabic}{مْفَكِّح}}\ {\color{gray}\texttt{/\sffamily {{\sffamily mfa(k)(k)iħ}}/}\color{black}}\ \textsc{noun\textunderscore act}\ [m.]\ \textbf{1.}~sitting in W-position.  \textbf{2.}~spreading sb's legs apart while sitting\  \begin{flushright}\color{gray}\foreignlanguage{arabic}{\textbf{\underline{\foreignlanguage{arabic}{أمثلة}}}: فتت عليه عالغرفة لقيته مْفَكِّح اجريه وقاعد بيرسم فقدت عقلي وصرت أصوِّت}\end{flushright}\color{black}} \vspace{2mm}

\vspace{-3mm}
\markboth{\color{blue}\foreignlanguage{arabic}{ف.ك.ر}\color{blue}{}}{\color{blue}\foreignlanguage{arabic}{ف.ك.ر}\color{blue}{}}\subsection*{\color{blue}\foreignlanguage{arabic}{ف.ك.ر}\color{blue}{}\index{\color{blue}\foreignlanguage{arabic}{ف.ك.ر}\color{blue}{}}} 

{\setlength\topsep{0pt}\textbf{\foreignlanguage{arabic}{تَفَكُّر}}\ {\color{gray}\texttt{/\sffamily {{\sffamily tafakkur}}/}\color{black}}\ \textsc{noun}\ [m.]\ \textbf{1.}~contemplation  \textbf{2.}~reflection on sth\ } \vspace{2mm}

{\setlength\topsep{0pt}\textbf{\foreignlanguage{arabic}{تَفْكِير}}\ {\color{gray}\texttt{/\sffamily {{\sffamily tafkiːr}}/}\color{black}}\ \textsc{noun}\ [m.]\ \color{gray}(msa. \foreignlanguage{arabic}{تَفْكِير}~\foreignlanguage{arabic}{\textbf{١.}})\color{black}\ \textbf{1.}~thinking\  \begin{flushright}\color{gray}\foreignlanguage{arabic}{\textbf{\underline{\foreignlanguage{arabic}{أمثلة}}}: تَفْكِيرك أعوج زي خلقتك العوجا}\end{flushright}\color{black}} \vspace{2mm}

{\setlength\topsep{0pt}\textbf{\foreignlanguage{arabic}{تْفَكَّر}}\ {\color{gray}\texttt{/\sffamily {{\sffamily tfakkar}}/}\color{black}}\ \textsc{verb}\ [p.]\ \textbf{1.}~contemplate  \textbf{2.}~reflect on sth\ \ $\bullet$\ \ \setlength\topsep{0pt}\textbf{\foreignlanguage{arabic}{اِتْفَكَّر}}\ {\color{gray}\texttt{/\sffamily {{\sffamily ʔitfakkar}}/}\color{black}}\ [c.]\ \ $\bullet$\ \ \setlength\topsep{0pt}\textbf{\foreignlanguage{arabic}{يِتْفَكَّر}}\ {\color{gray}\texttt{/\sffamily {{\sffamily jitfakkar}}/}\color{black}}\ [i.]\  \begin{flushright}\color{gray}\foreignlanguage{arabic}{\textbf{\underline{\foreignlanguage{arabic}{أمثلة}}}: كل فترة والثانية بصفن بحالي وبتْفَكَّر بالكون}\end{flushright}\color{black}} \vspace{2mm}

{\setlength\topsep{0pt}\textbf{\foreignlanguage{arabic}{فَكَّر}}\ {\color{gray}\texttt{/\sffamily {{\sffamily fa(k)(k)ar}}/}\color{black}}\ \textsc{verb}\ [p.]\ \textbf{1.}~think\ \ $\bullet$\ \ \setlength\topsep{0pt}\textbf{\foreignlanguage{arabic}{فَكِّر}}\ {\color{gray}\texttt{/\sffamily {{\sffamily fa(k)(k)ir}}/}\color{black}}\ [c.]\ \ $\bullet$\ \ \setlength\topsep{0pt}\textbf{\foreignlanguage{arabic}{يفَكِّر}}\ {\color{gray}\texttt{/\sffamily {{\sffamily jfa(k)(k)ir}}/}\color{black}}\ [i.]\ \color{gray}(msa. \foreignlanguage{arabic}{يُفَكِّر}~\foreignlanguage{arabic}{\textbf{١.}})\color{black}\  \begin{flushright}\color{gray}\foreignlanguage{arabic}{\textbf{\underline{\foreignlanguage{arabic}{أمثلة}}}: فَكِّر بالعرض اللي عطيتك اياه مش رح تلاقي مثله هالأيام}\end{flushright}\color{black}} \vspace{2mm}

{\setlength\topsep{0pt}\textbf{\foreignlanguage{arabic}{فِكِر}}\ {\color{gray}\texttt{/\sffamily {{\sffamily fikir}}/}\color{black}}\ \textsc{noun}\ [m.]\ \textbf{1.}~mentality  \textbf{2.}~thinking  \textbf{3.}~intellect\ } \vspace{2mm}

{\setlength\topsep{0pt}\textbf{\foreignlanguage{arabic}{فِكْرَة}}\ {\color{gray}\texttt{/\sffamily {{\sffamily fikra}}/}\color{black}}\ \textsc{noun}\ [f.]\ \textbf{1.}~idea  \textbf{2.}~notion  \textbf{3.}~thought\ \ $\bullet$\ \ \setlength\topsep{0pt}\textbf{\foreignlanguage{arabic}{أَفْكَار}}\ {\color{gray}\texttt{/\sffamily {{\sffamily ʔafkaːr}}/}\color{black}}\ [pl.]\ \ $\bullet$\ \ \textsc{ph.} \color{gray} \foreignlanguage{arabic}{عَفِكْرَة}\color{black}\ {\color{gray}\texttt{/{\sffamily ʕafikra}/}\color{black}}\ \textbf{1.}~by the way!\  \begin{flushright}\color{gray}\foreignlanguage{arabic}{\textbf{\underline{\foreignlanguage{arabic}{أمثلة}}}: عفِكْرة مش أنا اللي حكيت معها بخصوص الجمعية. هي اللي إِجت لحالها عرضت خدماتها.\ $\bullet$\ \  عندي شوية أفْكار غلط مع الوقت رح تتصلَّح}\end{flushright}\color{black}} \vspace{2mm}

{\setlength\topsep{0pt}\textbf{\foreignlanguage{arabic}{مُفَكِّر}}\ {\color{gray}\texttt{/\sffamily {{\sffamily mufakkir}}/}\color{black}}\ \textsc{noun}\ [m.]\ \color{gray}(msa. \foreignlanguage{arabic}{مُفَكِّر}~\foreignlanguage{arabic}{\textbf{١.}})\color{black}\ \textbf{1.}~intellectual\  \begin{flushright}\color{gray}\foreignlanguage{arabic}{\textbf{\underline{\foreignlanguage{arabic}{أمثلة}}}: جوزيف مسعد مُفَكِّر كبير وكلامه بيتثقَّل بالذهب}\end{flushright}\color{black}} \vspace{2mm}

{\setlength\topsep{0pt}\textbf{\foreignlanguage{arabic}{مِفْكِر}}\ {\color{gray}\texttt{/\sffamily {{\sffamily mif(k)ir}}/}\color{black}}\ \textsc{adj}\ [m.]\ (src. \color{gray}\foreignlanguage{arabic}{رام الله}\color{black})\ \color{gray}(msa. \foreignlanguage{arabic}{مهموم}~\foreignlanguage{arabic}{\textbf{١.}})\color{black}\ \textbf{1.}~careworn\  \begin{flushright}\color{gray}\foreignlanguage{arabic}{\textbf{\underline{\foreignlanguage{arabic}{أمثلة}}}: يا زلمة وين سارح والله شكلك مفكر}\end{flushright}\color{black}} \vspace{2mm}

{\setlength\topsep{0pt}\textbf{\foreignlanguage{arabic}{مْفَكِّر}}\ {\color{gray}\texttt{/\sffamily {{\sffamily mfakkir}}/}\color{black}}\ \textsc{noun\textunderscore act}\ [m.]\ \textbf{1.}~thinking  \textbf{2.}~assuming\  \begin{flushright}\color{gray}\foreignlanguage{arabic}{\textbf{\underline{\foreignlanguage{arabic}{أمثلة}}}: طول الوقت أنا مْفَكِّر إِنك بدكاش تدخل معنا شراكة}\end{flushright}\color{black}} \vspace{2mm}

\vspace{-3mm}
\markboth{\color{blue}\foreignlanguage{arabic}{ف.ك.ك}\color{blue}{}}{\color{blue}\foreignlanguage{arabic}{ف.ك.ك}\color{blue}{}}\subsection*{\color{blue}\foreignlanguage{arabic}{ف.ك.ك}\color{blue}{}\index{\color{blue}\foreignlanguage{arabic}{ف.ك.ك}\color{blue}{}}} 

{\setlength\topsep{0pt}\textbf{\foreignlanguage{arabic}{اِنْفَكّ}}\ {\color{gray}\texttt{/\sffamily {{\sffamily ʔinfakk}}/}\color{black}}\ \textsc{verb}\ [p.]\ \textbf{1.}~be undone.  \textbf{2.}~be disintegrated\ \ $\bullet$\ \ \setlength\topsep{0pt}\textbf{\foreignlanguage{arabic}{اِنْفَكّ}}\ {\color{gray}\texttt{/\sffamily {{\sffamily ʔinfakk}}/}\color{black}}\ [c.]\ \ $\bullet$\ \ \setlength\topsep{0pt}\textbf{\foreignlanguage{arabic}{يِنْفَكّ}}\ {\color{gray}\texttt{/\sffamily {{\sffamily jinfakk}}/}\color{black}}\ [i.]\  \begin{flushright}\color{gray}\foreignlanguage{arabic}{\textbf{\underline{\foreignlanguage{arabic}{أمثلة}}}: بس دهنته بزيت زيتون اِنْفَك بسهولة}\end{flushright}\color{black}} \vspace{2mm}

{\setlength\topsep{0pt}\textbf{\foreignlanguage{arabic}{تَفْكِيك}}\ {\color{gray}\texttt{/\sffamily {{\sffamily tafkiːk}}/}\color{black}}\ \textsc{noun}\ [m.]\ \textbf{1.}~dismantling  \textbf{2.}~dismemberment  \textbf{3.}~fragmentation\  \begin{flushright}\color{gray}\foreignlanguage{arabic}{\textbf{\underline{\foreignlanguage{arabic}{أمثلة}}}: السحر والحجابات اللهم عافينا سبب تَفْكِيك الأُسَر}\end{flushright}\color{black}} \vspace{2mm}

{\setlength\topsep{0pt}\textbf{\foreignlanguage{arabic}{تْفَكَّك}}\ {\color{gray}\texttt{/\sffamily {{\sffamily tfakkak}}/}\color{black}}\ \textsc{verb}\ [p.]\ \textbf{1.}~be dismantled.  \textbf{2.}~be didintegrated (gradually piece by piece)\ \ $\bullet$\ \ \setlength\topsep{0pt}\textbf{\foreignlanguage{arabic}{اِتْفَكَّك}}\ {\color{gray}\texttt{/\sffamily {{\sffamily ʔitfakkak}}/}\color{black}}\ [c.]\ \ $\bullet$\ \ \setlength\topsep{0pt}\textbf{\foreignlanguage{arabic}{يِتْفَكَّك}}\ {\color{gray}\texttt{/\sffamily {{\sffamily jitfakkak}}/}\color{black}}\ [i.]\  \begin{flushright}\color{gray}\foreignlanguage{arabic}{\textbf{\underline{\foreignlanguage{arabic}{أمثلة}}}: مابدي نتطلَّق والعيلة تِتْفَكَّك}\end{flushright}\color{black}} \vspace{2mm}

{\setlength\topsep{0pt}\textbf{\foreignlanguage{arabic}{فَكّ}}\ {\color{gray}\texttt{/\sffamily {{\sffamily fakk}}/}\color{black}}\ \textsc{noun}\ [m.]\ \color{gray}(msa. \foreignlanguage{arabic}{فَك}~\foreignlanguage{arabic}{\textbf{١.}})\color{black}\ \textbf{1.}~jaw\ \ $\bullet$\ \ \setlength\topsep{0pt}\textbf{\foreignlanguage{arabic}{فْكُوك}}\ {\color{gray}\texttt{/\sffamily {{\sffamily ʔifkuːk}}/}\color{black}}\ [pl.]\  \begin{flushright}\color{gray}\foreignlanguage{arabic}{\textbf{\underline{\foreignlanguage{arabic}{أمثلة}}}: فكوكنا التوحت واحنا بنلوك بهالعلكة\ $\bullet$\ \  وقعت عفَكِّي بيوجعني}\end{flushright}\color{black}} \vspace{2mm}

{\setlength\topsep{0pt}\textbf{\foreignlanguage{arabic}{فَكّ}}\ {\color{gray}\texttt{/\sffamily {{\sffamily fakk}}/}\color{black}}\ \textsc{verb}\ [p.]\ \textbf{1.}~undo  \textbf{2.}~dismantle  \textbf{3.}~change (money)\ \ $\bullet$\ \ \setlength\topsep{0pt}\textbf{\foreignlanguage{arabic}{فِكّ}}\ {\color{gray}\texttt{/\sffamily {{\sffamily fikk}}/}\color{black}}\ [c.]\ \ $\bullet$\ \ \setlength\topsep{0pt}\textbf{\foreignlanguage{arabic}{يفِكّ}}\ {\color{gray}\texttt{/\sffamily {{\sffamily jfikk}}/}\color{black}}\ [i.]\ \ $\bullet$\ \ \textsc{ph.} \color{gray} \foreignlanguage{arabic}{فِكّ العُقِْدة}\color{black}\ {\color{gray}\texttt{/{\sffamily fikk ʔilʕu(q)de}/}\color{black}}\ \textbf{1.}~break into smile\  \begin{flushright}\color{gray}\foreignlanguage{arabic}{\textbf{\underline{\foreignlanguage{arabic}{أمثلة}}}: فِك العُقْدِة خلصنا والله مافي شي مستاهل\ $\bullet$\ \  أنو اللي بده يفِكّ جرة الغاز ويركِّب الجديدة\ $\bullet$\ \  فِكّلي هال50 دينار الله يرضى عليك لعشرات\ $\bullet$\ \  مين فَكّلك الأزرار؟}\end{flushright}\color{black}} \vspace{2mm}

{\setlength\topsep{0pt}\textbf{\foreignlanguage{arabic}{فَكَّك}}\ {\color{gray}\texttt{/\sffamily {{\sffamily fakkak}}/}\color{black}}\ \textsc{verb}\ [p.]\ \textbf{1.}~dismantle (gradually piece by piece)\ \ $\bullet$\ \ \setlength\topsep{0pt}\textbf{\foreignlanguage{arabic}{فَكِّك}}\ {\color{gray}\texttt{/\sffamily {{\sffamily fakkik}}/}\color{black}}\ [c.]\ \ $\bullet$\ \ \setlength\topsep{0pt}\textbf{\foreignlanguage{arabic}{يفَكِّك}}\ {\color{gray}\texttt{/\sffamily {{\sffamily jfakkik}}/}\color{black}}\ [i.]\  \begin{flushright}\color{gray}\foreignlanguage{arabic}{\textbf{\underline{\foreignlanguage{arabic}{أمثلة}}}: حاولت أفَكِّك المكعبات بس ماقدرتش}\end{flushright}\color{black}} \vspace{2mm}

{\setlength\topsep{0pt}\textbf{\foreignlanguage{arabic}{فَكِّة}}\ {\color{gray}\texttt{/\sffamily {{\sffamily fakke}}/}\color{black}}\ \textsc{noun}\ [f.]\ \textbf{1.}~change (money)\  \begin{flushright}\color{gray}\foreignlanguage{arabic}{\textbf{\underline{\foreignlanguage{arabic}{أمثلة}}}: معك فَكِّة تعطيني اياها؟}\end{flushright}\color{black}} \vspace{2mm}

{\setlength\topsep{0pt}\textbf{\foreignlanguage{arabic}{فِكَك}}\ {\color{gray}\texttt{/\sffamily {{\sffamily fitʃatʃ}}/}\color{black}}\ \textsc{noun}\ [m.]\ \color{gray}(msa. \foreignlanguage{arabic}{مسحوق العصير}~\foreignlanguage{arabic}{\textbf{١.}})\color{black}\ \textbf{1.}~juice powder\  \begin{flushright}\color{gray}\foreignlanguage{arabic}{\textbf{\underline{\foreignlanguage{arabic}{أمثلة}}}: عملتلك فِكك بشهي تعال ذوق}\end{flushright}\color{black}} \vspace{2mm}

{\setlength\topsep{0pt}\textbf{\foreignlanguage{arabic}{مَفَكّ}}\ {\color{gray}\texttt{/\sffamily {{\sffamily mafakk}}/}\color{black}}\ \textsc{noun}\ [m.]\ \textbf{1.}~screwdriver\  \begin{flushright}\color{gray}\foreignlanguage{arabic}{\textbf{\underline{\foreignlanguage{arabic}{أمثلة}}}: ناولني المَفَك من فوق الطرابيزة}\end{flushright}\color{black}} \vspace{2mm}

{\setlength\topsep{0pt}\textbf{\foreignlanguage{arabic}{مَفْكُوك}}\ {\color{gray}\texttt{/\sffamily {{\sffamily mafkuːk}}/}\color{black}}\ \textsc{noun\textunderscore pass}\ \textbf{1.}~undone  \textbf{2.}~dismantled\  \begin{flushright}\color{gray}\foreignlanguage{arabic}{\textbf{\underline{\foreignlanguage{arabic}{أمثلة}}}: الجنزير مَفْكوك. أنو اللي فكُّه؟}\end{flushright}\color{black}} \vspace{2mm}

\vspace{-3mm}
\markboth{\color{blue}\foreignlanguage{arabic}{ف.ك.ه}\color{blue}{}}{\color{blue}\foreignlanguage{arabic}{ف.ك.ه}\color{blue}{}}\subsection*{\color{blue}\foreignlanguage{arabic}{ف.ك.ه}\color{blue}{}\index{\color{blue}\foreignlanguage{arabic}{ف.ك.ه}\color{blue}{}}} 

{\setlength\topsep{0pt}\textbf{\foreignlanguage{arabic}{تْفَكْهَن}}\ {\color{gray}\texttt{/\sffamily {{\sffamily tfakhan}}/}\color{black}}\ \textsc{verb}\ [p.]\ \textbf{1.}~go for a picnic.  \textbf{2.}~stroll around for pleasure\ \ $\bullet$\ \ \setlength\topsep{0pt}\textbf{\foreignlanguage{arabic}{اِتْفَكْهَن}}\ {\color{gray}\texttt{/\sffamily {{\sffamily ʔitfakhan}}/}\color{black}}\ [c.]\ \ $\bullet$\ \ \setlength\topsep{0pt}\textbf{\foreignlanguage{arabic}{يِتْفَكْهَن}}\ {\color{gray}\texttt{/\sffamily {{\sffamily jitfakhan}}/}\color{black}}\ [i.]\  \begin{flushright}\color{gray}\foreignlanguage{arabic}{\textbf{\underline{\foreignlanguage{arabic}{أمثلة}}}: خلينا ليوم نطلع نتفَكْهَن بجنين}\end{flushright}\color{black}} \vspace{2mm}

{\setlength\topsep{0pt}\textbf{\foreignlanguage{arabic}{فوَاكِه}}\ {\color{gray}\texttt{/\sffamily {{\sffamily fawaːke}}/}\color{black}}\ \textsc{noun}\ [m.]\ \color{gray}(msa. \foreignlanguage{arabic}{فواكِه}~\foreignlanguage{arabic}{\textbf{١.}})\color{black}\ \textbf{1.}~fruit\ \ $\bullet$\ \ \setlength\topsep{0pt}\textbf{\foreignlanguage{arabic}{فَاكْهَة}}\ {\color{gray}\texttt{/\sffamily {{\sffamily faːkha}}/}\color{black}}\ [f.]\ \color{gray}(msa. \foreignlanguage{arabic}{نوع فاكِهَة}~\foreignlanguage{arabic}{\textbf{١.}})\color{black}\ \textbf{1.}~one type of fruit\  \begin{flushright}\color{gray}\foreignlanguage{arabic}{\textbf{\underline{\foreignlanguage{arabic}{أمثلة}}}: خالي بيجبش فواكِه إِلا نخب أول}\end{flushright}\color{black}} \vspace{2mm}

{\setlength\topsep{0pt}\textbf{\foreignlanguage{arabic}{فَكَهَنْجِي}}\ {\color{gray}\texttt{/\sffamily {{\sffamily fakahan(dʒ)i}}/}\color{black}}\ \textsc{noun}\ [m.]\ \textbf{1.}~fruit vendor\  \begin{flushright}\color{gray}\foreignlanguage{arabic}{\textbf{\underline{\foreignlanguage{arabic}{أمثلة}}}: إِذا مرق الفَكَهَنجي تلانا خبرني بدي أجيب منه كيلو موز عشان الضيوف}\end{flushright}\color{black}} \vspace{2mm}

{\setlength\topsep{0pt}\textbf{\foreignlanguage{arabic}{فَكْهَنِة}}\ {\color{gray}\texttt{/\sffamily {{\sffamily fakhane}}/}\color{black}}\ \textsc{noun}\ [f.]\ \color{gray}(msa. \foreignlanguage{arabic}{رحلَة}~\foreignlanguage{arabic}{\textbf{١.}})\color{black}\ \textbf{1.}~picnic\  \begin{flushright}\color{gray}\foreignlanguage{arabic}{\textbf{\underline{\foreignlanguage{arabic}{أمثلة}}}: طالعين فَكْهَنِة عبتير ايش رأيك تيجي معنا}\end{flushright}\color{black}} \vspace{2mm}

\vspace{-3mm}
\markboth{\color{blue}\foreignlanguage{arabic}{ف.ل.ت}\color{blue}{}}{\color{blue}\foreignlanguage{arabic}{ف.ل.ت}\color{blue}{}}\subsection*{\color{blue}\foreignlanguage{arabic}{ف.ل.ت}\color{blue}{}\index{\color{blue}\foreignlanguage{arabic}{ف.ل.ت}\color{blue}{}}} 

{\setlength\topsep{0pt}\textbf{\foreignlanguage{arabic}{اِنْفَلَت}}\ {\color{gray}\texttt{/\sffamily {{\sffamily ʔinfalat}}/}\color{black}}\ \textsc{verb}\ [p.]\ \textbf{1.}~be released.  \textbf{2.}~be destabilized.  \textbf{3.}~lose control\ \ $\bullet$\ \ \setlength\topsep{0pt}\textbf{\foreignlanguage{arabic}{اِنْفِلِت}}\ {\color{gray}\texttt{/\sffamily {{\sffamily ʔinfilit}}/}\color{black}}\ [c.]\ \ $\bullet$\ \ \setlength\topsep{0pt}\textbf{\foreignlanguage{arabic}{يِنْفِلِت}}\ {\color{gray}\texttt{/\sffamily {{\sffamily jinfilit}}/}\color{black}}\ [i.]\  \begin{flushright}\color{gray}\foreignlanguage{arabic}{\textbf{\underline{\foreignlanguage{arabic}{أمثلة}}}: همي بيخافوا يِنْفِلِت الوضع بالبلد\ $\bullet$\ \  اِنْفَلَت الحبل مني بالغلط}\end{flushright}\color{black}} \vspace{2mm}

{\setlength\topsep{0pt}\textbf{\foreignlanguage{arabic}{تْفَلَّت}}\ {\color{gray}\texttt{/\sffamily {{\sffamily tfallat}}/}\color{black}}\ \textsc{verb}\ [p.]\ \textbf{1.}~deviate morally\ \ $\bullet$\ \ \setlength\topsep{0pt}\textbf{\foreignlanguage{arabic}{اِتْفَلَّت}}\ {\color{gray}\texttt{/\sffamily {{\sffamily ʔitfallat}}/}\color{black}}\ [c.]\ \ $\bullet$\ \ \setlength\topsep{0pt}\textbf{\foreignlanguage{arabic}{يِتْفَلَّت}}\ {\color{gray}\texttt{/\sffamily {{\sffamily jitfallat}}/}\color{black}}\ [i.]\  \begin{flushright}\color{gray}\foreignlanguage{arabic}{\textbf{\underline{\foreignlanguage{arabic}{أمثلة}}}: بده يِتْفَلَّت براحته عشان هيك بدوش حدا فينا يجي معه}\end{flushright}\color{black}} \vspace{2mm}

{\setlength\topsep{0pt}\textbf{\foreignlanguage{arabic}{تْفَلْوَت}}\ {\color{gray}\texttt{/\sffamily {{\sffamily tfalwat}}/}\color{black}}\ \textsc{verb}\ [p.]\ \textbf{1.}~deviate morally\ \ $\bullet$\ \ \setlength\topsep{0pt}\textbf{\foreignlanguage{arabic}{اِتْفَلْوَت}}\ {\color{gray}\texttt{/\sffamily {{\sffamily ʔitfalwat}}/}\color{black}}\ [c.]\ \ $\bullet$\ \ \setlength\topsep{0pt}\textbf{\foreignlanguage{arabic}{يِتْفَلْوَت}}\ {\color{gray}\texttt{/\sffamily {{\sffamily jitfalwat}}/}\color{black}}\ [i.]\  \begin{flushright}\color{gray}\foreignlanguage{arabic}{\textbf{\underline{\foreignlanguage{arabic}{أمثلة}}}: بدها تسكن ببيت لحالها عشان تِتفَلوَت براحتها}\end{flushright}\color{black}} \vspace{2mm}

{\setlength\topsep{0pt}\textbf{\foreignlanguage{arabic}{فَلَتَان}}\ {\color{gray}\texttt{/\sffamily {{\sffamily falataːn}}/}\color{black}}\ \textsc{noun}\ [m.]\ \textbf{1.}~perversion  \textbf{2.}~moral deviation\  \begin{flushright}\color{gray}\foreignlanguage{arabic}{\textbf{\underline{\foreignlanguage{arabic}{أمثلة}}}: وشو الحل مع فَلَتان الشباب والصبايا هالأيام؟}\end{flushright}\color{black}} \vspace{2mm}

{\setlength\topsep{0pt}\textbf{\foreignlanguage{arabic}{فَلَّت}}\ {\color{gray}\texttt{/\sffamily {{\sffamily fallat}}/}\color{black}}\ \textsc{verb}\ [p.]\ \textbf{1.}~let sth go.  \textbf{2.}~fart  \textbf{3.}~break wind\ \ $\bullet$\ \ \setlength\topsep{0pt}\textbf{\foreignlanguage{arabic}{فَلِّت}}\ {\color{gray}\texttt{/\sffamily {{\sffamily fallit}}/}\color{black}}\ [c.]\ \ $\bullet$\ \ \setlength\topsep{0pt}\textbf{\foreignlanguage{arabic}{يفَلِّت}}\ {\color{gray}\texttt{/\sffamily {{\sffamily jfallit}}/}\color{black}}\ [i.]\ \ $\bullet$\ \ \textsc{ph.} \color{gray} \foreignlanguage{arabic}{فَلَّت له الرسن}\color{black}\ {\color{gray}\texttt{/{\sffamily fallat lo ʔirrasan}/}\color{black}}\ \textbf{1.}~let sb behave freely (of his own volition)\  \begin{flushright}\color{gray}\foreignlanguage{arabic}{\textbf{\underline{\foreignlanguage{arabic}{أمثلة}}}: من لمّا أبوه فَلََّت له الرَّسَن وهو بسرح وبمرح عَحَل شَعْرُه بدون لا رَقيب ولا حَسيب\ $\bullet$\ \  فَلِّت شعري يا حيوان\ $\bullet$\ \  في حدا فَلَّت الله يقرفكم}\end{flushright}\color{black}} \vspace{2mm}

{\setlength\topsep{0pt}\textbf{\foreignlanguage{arabic}{فَلُوتِي}}\ {\color{gray}\texttt{/\sffamily {{\sffamily faluːti}}/}\color{black}}\ \textsc{adj}\ [m.]\ \color{gray}(msa. \foreignlanguage{arabic}{يفشي كل ما يعرفه من أسرار ولا يعرف الإِحتفاظ بسر أبدا}~\foreignlanguage{arabic}{\textbf{١.}})\color{black}\ \textbf{1.}~bigmouth  \textbf{2.}~telltale\  \begin{flushright}\color{gray}\foreignlanguage{arabic}{\textbf{\underline{\foreignlanguage{arabic}{أمثلة}}}: معروف بين الناس إِنه لسانه فَلُوتِيوبضل يلعلع كل شي بيعرفه}\end{flushright}\color{black}} \vspace{2mm}

{\setlength\topsep{0pt}\textbf{\foreignlanguage{arabic}{فَلْتَان}}\ {\color{gray}\texttt{/\sffamily {{\sffamily faltaːn}}/}\color{black}}\ \textsc{adj}\ [m.]\ \color{gray}(msa. \foreignlanguage{arabic}{منحَرِف أخلاقياً}~\foreignlanguage{arabic}{\textbf{١.}})\color{black}\ \textbf{1.}~pervert  \textbf{2.}~morally deviant\  \begin{flushright}\color{gray}\foreignlanguage{arabic}{\textbf{\underline{\foreignlanguage{arabic}{أمثلة}}}: بطل عاجبك الجلباب يا فَلْتانة!}\end{flushright}\color{black}} \vspace{2mm}

{\setlength\topsep{0pt}\textbf{\foreignlanguage{arabic}{فِلِت}}\ {\color{gray}\texttt{/\sffamily {{\sffamily filit}}/}\color{black}}\ \textsc{verb}\ [p.]\ \textbf{1.}~run away.  \textbf{2.}~deviate morally\ \ $\bullet$\ \ \setlength\topsep{0pt}\textbf{\foreignlanguage{arabic}{اِفْلَت}}\ {\color{gray}\texttt{/\sffamily {{\sffamily ʔiflat}}/}\color{black}}\ [c.]\ \ $\bullet$\ \ \setlength\topsep{0pt}\textbf{\foreignlanguage{arabic}{اِفْلِت}}\ {\color{gray}\texttt{/\sffamily {{\sffamily ʔiflit}}/}\color{black}}\ [c.]\ \textbf{1.}~leave sth.  \textbf{2.}~leave sth or sb alone.  \textbf{3.}~release  \textbf{4.}~set sth free\ \ $\bullet$\ \ \setlength\topsep{0pt}\textbf{\foreignlanguage{arabic}{يِفْلَت}}\ {\color{gray}\texttt{/\sffamily {{\sffamily jiflat}}/}\color{black}}\ [i.]\ \ $\bullet$\ \ \setlength\topsep{0pt}\textbf{\foreignlanguage{arabic}{يِفْلِت}}\ {\color{gray}\texttt{/\sffamily {{\sffamily jiflit}}/}\color{black}}\ [i.]\ \textbf{1.}~leave sth.  \textbf{2.}~leave sth or sb alone.  \textbf{3.}~release  \textbf{4.}~set sth free\ \ $\bullet$\ \ \textsc{ph.} \color{gray} \foreignlanguage{arabic}{فلت عليه الضحك}\color{black}\ {\color{gray}\texttt{/{\sffamily filit ʕaleː ʔi(dˤ)(dˤ)iħik}/}\color{black}}\ \textbf{1.}~burst into laughter\  \begin{flushright}\color{gray}\foreignlanguage{arabic}{\textbf{\underline{\foreignlanguage{arabic}{أمثلة}}}: كُنّا بعزا وفجأة فِلِت عليه الضِّحِك الله يخزيه خزانا قدام الناس\ $\bullet$\ \  هالمرَّة قدرت تِفْلِت مني المرة الجاي ان شاء الله بسخطك\ $\bullet$\ \  بديش اياه يِفْلَت من إِيدي. ماصدقت عالله أمسكه\ $\bullet$\ \  اِفْلِت شعري ولا!\ $\bullet$\ \  اِفْلَت براحتك مع النسوان. الك رب بحاسبك بس تعملناش مصيبة بحقارتك هاي}\end{flushright}\color{black}} \vspace{2mm}

\vspace{-3mm}
\markboth{\color{blue}\foreignlanguage{arabic}{ف.ل.ج}\color{blue}{}}{\color{blue}\foreignlanguage{arabic}{ف.ل.ج}\color{blue}{}}\subsection*{\color{blue}\foreignlanguage{arabic}{ف.ل.ج}\color{blue}{}\index{\color{blue}\foreignlanguage{arabic}{ف.ل.ج}\color{blue}{}}} 

{\setlength\topsep{0pt}\textbf{\foreignlanguage{arabic}{اِنْفَلَج}}\ {\color{gray}\texttt{/\sffamily {{\sffamily ʔinfala(dʒ)}}/}\color{black}}\ \textsc{verb}\ [p.]\ \textbf{1.}~have a stroke\ \ $\bullet$\ \ \setlength\topsep{0pt}\textbf{\foreignlanguage{arabic}{اِنْفِلِج}}\ {\color{gray}\texttt{/\sffamily {{\sffamily ʔinfili(dʒ)}}/}\color{black}}\ [c.]\ \ $\bullet$\ \ \setlength\topsep{0pt}\textbf{\foreignlanguage{arabic}{يِنْفِلِج}}\ {\color{gray}\texttt{/\sffamily {{\sffamily jinfili(dʒ)}}/}\color{black}}\ [i.]\  \begin{flushright}\color{gray}\foreignlanguage{arabic}{\textbf{\underline{\foreignlanguage{arabic}{أمثلة}}}: اِنْفَلَجت المرة مسكينة من كثر ما شافت تعافت}\end{flushright}\color{black}} \vspace{2mm}

{\setlength\topsep{0pt}\textbf{\foreignlanguage{arabic}{فَالِج}}\ {\color{gray}\texttt{/\sffamily {{\sffamily faːli(dʒ)}}/}\color{black}}\ \textsc{noun}\ [m.]\ \textbf{1.}~hemiplegia  \textbf{2.}~partial paralysis.  \textbf{3.}~stroke\ } \vspace{2mm}

{\setlength\topsep{0pt}\textbf{\foreignlanguage{arabic}{فَلَج}}\ {\color{gray}\texttt{/\sffamily {{\sffamily fala(dʒ)}}/}\color{black}}\ \textsc{verb}\ [p.]\ \textbf{1.}~cause a stroke\ \ $\bullet$\ \ \setlength\topsep{0pt}\textbf{\foreignlanguage{arabic}{اِفْلِج}}\ {\color{gray}\texttt{/\sffamily {{\sffamily ʔifli(dʒ)}}/}\color{black}}\ [c.]\ \ $\bullet$\ \ \setlength\topsep{0pt}\textbf{\foreignlanguage{arabic}{يِفْلِج}}\ {\color{gray}\texttt{/\sffamily {{\sffamily jifli(dʒ)}}/}\color{black}}\ [i.]\ \ $\bullet$\ \ \textsc{ph.} \color{gray} \foreignlanguage{arabic}{بْتِفْلِج}\color{black}\ {\color{gray}\texttt{/{\sffamily btifli(dʒ)}/}\color{black}}\ \textbf{1.}~very magnificent.  \textbf{2.}~very wonderful\ } \vspace{2mm}

{\setlength\topsep{0pt}\textbf{\foreignlanguage{arabic}{مَفْلُوج}}\ {\color{gray}\texttt{/\sffamily {{\sffamily mafluː(dʒ)}}/}\color{black}}\ \textsc{adj}\ [m.]\ \textbf{1.}~have a stroke\ } \vspace{2mm}

\vspace{-3mm}
\markboth{\color{blue}\foreignlanguage{arabic}{ف.ل.ح}\color{blue}{}}{\color{blue}\foreignlanguage{arabic}{ف.ل.ح}\color{blue}{}}\subsection*{\color{blue}\foreignlanguage{arabic}{ف.ل.ح}\color{blue}{}\index{\color{blue}\foreignlanguage{arabic}{ف.ل.ح}\color{blue}{}}} 

{\setlength\topsep{0pt}\textbf{\foreignlanguage{arabic}{فَالِح}}\ {\color{gray}\texttt{/\sffamily {{\sffamily faːliħ}}/}\color{black}}\ \textsc{adj}\ [m.]\ \color{gray}(msa. \foreignlanguage{arabic}{ناجِح}~\foreignlanguage{arabic}{\textbf{١.}})\color{black}\ \textbf{1.}~successful\  \begin{flushright}\color{gray}\foreignlanguage{arabic}{\textbf{\underline{\foreignlanguage{arabic}{أمثلة}}}: ياريتُه فالِح بالدراسة}\end{flushright}\color{black}} \vspace{2mm}

{\setlength\topsep{0pt}\textbf{\foreignlanguage{arabic}{فَلَاحَة}}\ {\color{gray}\texttt{/\sffamily {{\sffamily falaːħa}}/}\color{black}}\ \textsc{noun}\ [f.]\ \textbf{1.}~success  \textbf{2.}~proficiency\  \begin{flushright}\color{gray}\foreignlanguage{arabic}{\textbf{\underline{\foreignlanguage{arabic}{أمثلة}}}: يعني من كثر فَلاحتك أفهم انك بدك تعملها لحالك}\end{flushright}\color{black}} \vspace{2mm}

{\setlength\topsep{0pt}\textbf{\foreignlanguage{arabic}{فَلَح}}\ {\color{gray}\texttt{/\sffamily {{\sffamily falaħ}}/}\color{black}}\ \textsc{verb}\ [p.]\ \textbf{1.}~plant  \textbf{2.}~farm  \textbf{3.}~grow plants\ \ $\bullet$\ \ \setlength\topsep{0pt}\textbf{\foreignlanguage{arabic}{اِفْلَح}}\ {\color{gray}\texttt{/\sffamily {{\sffamily ʔiflaħ}}/}\color{black}}\ [c.]\ \ $\bullet$\ \ \setlength\topsep{0pt}\textbf{\foreignlanguage{arabic}{يِفْلَح}}\ {\color{gray}\texttt{/\sffamily {{\sffamily jiflaħ}}/}\color{black}}\ [i.]\ \color{gray}(msa. \foreignlanguage{arabic}{يَزْرَع}~\foreignlanguage{arabic}{\textbf{١.}})\color{black}\  \begin{flushright}\color{gray}\foreignlanguage{arabic}{\textbf{\underline{\foreignlanguage{arabic}{أمثلة}}}: قال انه بيقدر يِفْلَح الأرض بساعتين بس}\end{flushright}\color{black}} \vspace{2mm}

{\setlength\topsep{0pt}\textbf{\foreignlanguage{arabic}{فَلَّاح}}\ {\color{gray}\texttt{/\sffamily {{\sffamily fallaːħ}}/}\color{black}}\ \textsc{noun}\ [m.]\ \color{gray}(msa. \foreignlanguage{arabic}{مُزارِع}~\foreignlanguage{arabic}{\textbf{١.}})\color{black}\ \textbf{1.}~farmer\ } \vspace{2mm}

{\setlength\topsep{0pt}\textbf{\foreignlanguage{arabic}{فَلَّاحِي}}\ {\color{gray}\texttt{/\sffamily {{\sffamily fallaːħi}}/}\color{black}}\ \textsc{adj}\ [m.]\ \textbf{1.}~relating to farming.  \textbf{2.}~rural\  \begin{flushright}\color{gray}\foreignlanguage{arabic}{\textbf{\underline{\foreignlanguage{arabic}{أمثلة}}}: ليش بتحكي فَلّاحِي زي سيدك؟ احكي مدني زي إِمك وخالاتك}\end{flushright}\color{black}} \vspace{2mm}

{\setlength\topsep{0pt}\textbf{\foreignlanguage{arabic}{فِلِح}}\ {\color{gray}\texttt{/\sffamily {{\sffamily filiħ}}/}\color{black}}\ \textsc{verb}\ [p.]\ \textbf{1.}~succeed\ \ $\bullet$\ \ \setlength\topsep{0pt}\textbf{\foreignlanguage{arabic}{اِفْلَح}}\ {\color{gray}\texttt{/\sffamily {{\sffamily ʔiflaħ}}/}\color{black}}\ [c.]\ \ $\bullet$\ \ \setlength\topsep{0pt}\textbf{\foreignlanguage{arabic}{يِفْلَح}}\ {\color{gray}\texttt{/\sffamily {{\sffamily jiflaħ}}/}\color{black}}\ [i.]\ \color{gray}(msa. \foreignlanguage{arabic}{يَنْجَح}~\foreignlanguage{arabic}{\textbf{١.}})\color{black}\  \begin{flushright}\color{gray}\foreignlanguage{arabic}{\textbf{\underline{\foreignlanguage{arabic}{أمثلة}}}: على الله يِفْلَح هالمرة}\end{flushright}\color{black}} \vspace{2mm}

{\setlength\topsep{0pt}\textbf{\foreignlanguage{arabic}{فْلَاحَة}}\ {\color{gray}\texttt{/\sffamily {{\sffamily flaːħa}}/}\color{black}}\ \textsc{noun}\ [f.]\ \color{gray}(msa. \foreignlanguage{arabic}{زِراعَة}~\foreignlanguage{arabic}{\textbf{١.}})\color{black}\ \textbf{1.}~farming  \textbf{2.}~agriculture\  \begin{flushright}\color{gray}\foreignlanguage{arabic}{\textbf{\underline{\foreignlanguage{arabic}{أمثلة}}}: كار الفْلاحَة بضبطش معي}\end{flushright}\color{black}} \vspace{2mm}

\vspace{-3mm}
\markboth{\color{blue}\foreignlanguage{arabic}{ف.ل.خ}\color{blue}{}}{\color{blue}\foreignlanguage{arabic}{ف.ل.خ}\color{blue}{}}\subsection*{\color{blue}\foreignlanguage{arabic}{ف.ل.خ}\color{blue}{}\index{\color{blue}\foreignlanguage{arabic}{ف.ل.خ}\color{blue}{}}} 

{\setlength\topsep{0pt}\textbf{\foreignlanguage{arabic}{تَفْلِيخ}}\ {\color{gray}\texttt{/\sffamily {{\sffamily tafliːx}}/}\color{black}}\ \textsc{noun}\ [m.]\ \textbf{1.}~dissolving sth.  \textbf{2.}~causing sth to be swollen\ } \vspace{2mm}

{\setlength\topsep{0pt}\textbf{\foreignlanguage{arabic}{تْفَلَّخ}}\ {\color{gray}\texttt{/\sffamily {{\sffamily tfallax}}/}\color{black}}\ \textsc{verb}\ [p.]\ \textbf{1.}~be dissolved.  \textbf{2.}~be ripped off because sth is too heavy\ \ $\bullet$\ \ \setlength\topsep{0pt}\textbf{\foreignlanguage{arabic}{اِتْفَلَّخ}}\ {\color{gray}\texttt{/\sffamily {{\sffamily ʔitfallax}}/}\color{black}}\ [c.]\ \ $\bullet$\ \ \setlength\topsep{0pt}\textbf{\foreignlanguage{arabic}{يِتْفَلَّخ}}\ {\color{gray}\texttt{/\sffamily {{\sffamily jitfallax}}/}\color{black}}\ [i.]\  \begin{flushright}\color{gray}\foreignlanguage{arabic}{\textbf{\underline{\foreignlanguage{arabic}{أمثلة}}}: تْفَلَّخت الشنطة من جنب}\end{flushright}\color{black}} \vspace{2mm}

{\setlength\topsep{0pt}\textbf{\foreignlanguage{arabic}{فَلَخ}}\ {\color{gray}\texttt{/\sffamily {{\sffamily falax}}/}\color{black}}\ \textsc{verb}\ [p.]\ \textbf{1.}~run away quickly\ \ $\bullet$\ \ \setlength\topsep{0pt}\textbf{\foreignlanguage{arabic}{اِفْلَخ}}\ {\color{gray}\texttt{/\sffamily {{\sffamily ʔiflax}}/}\color{black}}\ [c.]\ \ $\bullet$\ \ \setlength\topsep{0pt}\textbf{\foreignlanguage{arabic}{يِفْلَخ}}\ {\color{gray}\texttt{/\sffamily {{\sffamily jiflax}}/}\color{black}}\ [i.]\ \color{gray}(msa. \foreignlanguage{arabic}{يَهْرُب بسرعة}~\foreignlanguage{arabic}{\textbf{١.}})\color{black}\  \begin{flushright}\color{gray}\foreignlanguage{arabic}{\textbf{\underline{\foreignlanguage{arabic}{أمثلة}}}: ما لحق يِفْلَخ الا والشرطة ماسكته}\end{flushright}\color{black}} \vspace{2mm}

{\setlength\topsep{0pt}\textbf{\foreignlanguage{arabic}{فَلَّخ}}\ {\color{gray}\texttt{/\sffamily {{\sffamily fallax}}/}\color{black}}\ \textsc{verb}\ [p.]\ \textbf{1.}~dissolve  \textbf{2.}~cause sth to be swollen\ \ $\bullet$\ \ \setlength\topsep{0pt}\textbf{\foreignlanguage{arabic}{فَلِّخ}}\ {\color{gray}\texttt{/\sffamily {{\sffamily fallix}}/}\color{black}}\ [c.]\ \ $\bullet$\ \ \setlength\topsep{0pt}\textbf{\foreignlanguage{arabic}{يفَلِّخ}}\ {\color{gray}\texttt{/\sffamily {{\sffamily jfallix}}/}\color{black}}\ [i.]\ \color{gray}(msa. \foreignlanguage{arabic}{يورِّم}~\foreignlanguage{arabic}{\textbf{٣.}}  \foreignlanguage{arabic}{يَفسَخ}~\foreignlanguage{arabic}{\textbf{٢.}}  \foreignlanguage{arabic}{يَحِل}~\foreignlanguage{arabic}{\textbf{١.}})\color{black}\  \begin{flushright}\color{gray}\foreignlanguage{arabic}{\textbf{\underline{\foreignlanguage{arabic}{أمثلة}}}: بدك اجري يتفلَّخِن من كثرة الوقوف؟\ $\bullet$\ \  فَلَّخها للشنطة}\end{flushright}\color{black}} \vspace{2mm}

{\setlength\topsep{0pt}\textbf{\foreignlanguage{arabic}{مْفَلِّخ}}\ {\color{gray}\texttt{/\sffamily {{\sffamily mfallix}}/}\color{black}}\ \textsc{noun\textunderscore act}\ [m.]\ \color{gray}(msa. \foreignlanguage{arabic}{هارِباً بسرعة}~\foreignlanguage{arabic}{\textbf{١.}})\color{black}\ \textbf{1.}~run away quickly\  \begin{flushright}\color{gray}\foreignlanguage{arabic}{\textbf{\underline{\foreignlanguage{arabic}{أمثلة}}}: وين مْفَلِِّّخ يا عمو؟}\end{flushright}\color{black}} \vspace{2mm}

\vspace{-3mm}
\markboth{\color{blue}\foreignlanguage{arabic}{ف.ل.س}\color{blue}{}}{\color{blue}\foreignlanguage{arabic}{ف.ل.س}\color{blue}{}}\subsection*{\color{blue}\foreignlanguage{arabic}{ف.ل.س}\color{blue}{}\index{\color{blue}\foreignlanguage{arabic}{ف.ل.س}\color{blue}{}}} 

{\setlength\topsep{0pt}\textbf{\foreignlanguage{arabic}{فَلَّس}}\ {\color{gray}\texttt{/\sffamily {{\sffamily fallas}}/}\color{black}}\ \textsc{verb}\ [p.]\ \textbf{1.}~go bankrupt\ \ $\bullet$\ \ \setlength\topsep{0pt}\textbf{\foreignlanguage{arabic}{فَلِّس}}\ {\color{gray}\texttt{/\sffamily {{\sffamily fallis}}/}\color{black}}\ [c.]\ \ $\bullet$\ \ \setlength\topsep{0pt}\textbf{\foreignlanguage{arabic}{يفَلِّس}}\ {\color{gray}\texttt{/\sffamily {{\sffamily jfallis}}/}\color{black}}\ [i.]\ \color{gray}(msa. \foreignlanguage{arabic}{يُفْلِس}~\foreignlanguage{arabic}{\textbf{١.}})\color{black}\ \ $\bullet$\ \ \textsc{ph.} \color{gray} \foreignlanguage{arabic}{تفَلَّس}\color{black}\ {\color{gray}\texttt{/{\sffamily tafallas}/}\color{black}}\ \color{gray} (msa. \foreignlanguage{arabic}{كثيراً}~\foreignlanguage{arabic}{\textbf{١.}})\color{black}\ \textbf{1.}~a lot\  \begin{flushright}\color{gray}\foreignlanguage{arabic}{\textbf{\underline{\foreignlanguage{arabic}{أمثلة}}}: يا إِنه انضرب تفَلَّس هالمسكين\ $\bullet$\ \  أبوها فَلَّس بعد حرب غزة الأخيرة}\end{flushright}\color{black}} \vspace{2mm}

{\setlength\topsep{0pt}\textbf{\foreignlanguage{arabic}{فِلِس}}\ {\color{gray}\texttt{/\sffamily {{\sffamily filis}}/}\color{black}}\ \textsc{noun}\ [m.]\ \color{gray}(msa. \foreignlanguage{arabic}{فِلْس}~\foreignlanguage{arabic}{\textbf{١.}})\color{black}\ \textbf{1.}~fils\  \begin{flushright}\color{gray}\foreignlanguage{arabic}{\textbf{\underline{\foreignlanguage{arabic}{أمثلة}}}: معيش ولا فِلِس أدفعه للبقالة}\end{flushright}\color{black}} \vspace{2mm}

{\setlength\topsep{0pt}\textbf{\foreignlanguage{arabic}{فْلُوس}}\ {\color{gray}\texttt{/\sffamily {{\sffamily fluːs}}/}\color{black}}\ \textsc{noun}\ [pl.]\ \color{gray}(msa. \foreignlanguage{arabic}{نُقُود}~\foreignlanguage{arabic}{\textbf{١.}})\color{black}\ \textbf{1.}~money\  \begin{flushright}\color{gray}\foreignlanguage{arabic}{\textbf{\underline{\foreignlanguage{arabic}{أمثلة}}}: أذا معك فْلُوس أمورك تمام، أذا معكش الله يكون بعونك}\end{flushright}\color{black}} \vspace{2mm}

{\setlength\topsep{0pt}\textbf{\foreignlanguage{arabic}{مْفَلِّس}}\ {\color{gray}\texttt{/\sffamily {{\sffamily mfallis}}/}\color{black}}\ \textsc{adj}\ [m.]\ \color{gray}(msa. \foreignlanguage{arabic}{مُفْلِس}~\foreignlanguage{arabic}{\textbf{١.}})\color{black}\ \textbf{1.}~penniless  \textbf{2.}~bankrupt  \textbf{3.}~on a shoe string\  \begin{flushright}\color{gray}\foreignlanguage{arabic}{\textbf{\underline{\foreignlanguage{arabic}{أمثلة}}}: كنا مْفَلِّسِين بعد الاجازة}\end{flushright}\color{black}} \vspace{2mm}

\vspace{-3mm}
\markboth{\color{blue}\foreignlanguage{arabic}{ف.ل.س.ط.ن}\color{blue}{ (ntws)}}{\color{blue}\foreignlanguage{arabic}{ف.ل.س.ط.ن}\color{blue}{ (ntws)}}\subsection*{\color{blue}\foreignlanguage{arabic}{ف.ل.س.ط.ن}\color{blue}{ (ntws)}\index{\color{blue}\foreignlanguage{arabic}{ف.ل.س.ط.ن}\color{blue}{ (ntws)}}} 

{\setlength\topsep{0pt}\textbf{\foreignlanguage{arabic}{فَلَسْطِين}}\ {\color{gray}\texttt{/\sffamily {{\sffamily falasˤtˤiːn}}/}\color{black}}\ \textsc{noun\textunderscore prop}\ \textbf{1.}~Palestine\ } \vspace{2mm}

{\setlength\topsep{0pt}\textbf{\foreignlanguage{arabic}{فَلَسْطِينِي}}\ {\color{gray}\texttt{/\sffamily {{\sffamily falasˤtˤiːni}}/}\color{black}}\ \textsc{adj}\ [m.]\ \textbf{1.}~Palestinian\ } \vspace{2mm}

\vspace{-3mm}
\markboth{\color{blue}\foreignlanguage{arabic}{ف.ل.س.ع}\color{blue}{}}{\color{blue}\foreignlanguage{arabic}{ف.ل.س.ع}\color{blue}{}}\subsection*{\color{blue}\foreignlanguage{arabic}{ف.ل.س.ع}\color{blue}{}\index{\color{blue}\foreignlanguage{arabic}{ف.ل.س.ع}\color{blue}{}}} 

{\setlength\topsep{0pt}\textbf{\foreignlanguage{arabic}{فَلْسَع}}\ {\color{gray}\texttt{/\sffamily {{\sffamily falsaʕ}}/}\color{black}}\ \textsc{verb}\ [p.]\ \textbf{1.}~run away\ \ $\bullet$\ \ \setlength\topsep{0pt}\textbf{\foreignlanguage{arabic}{فَلْسِع}}\ {\color{gray}\texttt{/\sffamily {{\sffamily falsiʕ}}/}\color{black}}\ [c.]\ \ $\bullet$\ \ \setlength\topsep{0pt}\textbf{\foreignlanguage{arabic}{يفَلْسِع}}\ {\color{gray}\texttt{/\sffamily {{\sffamily jfalsiʕ}}/}\color{black}}\ [i.]\ \color{gray}(msa. \foreignlanguage{arabic}{يَهْرُب}~\foreignlanguage{arabic}{\textbf{١.}})\color{black}\  \begin{flushright}\color{gray}\foreignlanguage{arabic}{\textbf{\underline{\foreignlanguage{arabic}{أمثلة}}}: منيح فَلْسَعِت ولا كان استلمني ساعة عالأقل}\end{flushright}\color{black}} \vspace{2mm}

{\setlength\topsep{0pt}\textbf{\foreignlanguage{arabic}{فَلْسَعَة}}\ {\color{gray}\texttt{/\sffamily {{\sffamily falsaʕa}}/}\color{black}}\ \textsc{noun}\ [f.]\ \textbf{1.}~running away\ } \vspace{2mm}

{\setlength\topsep{0pt}\textbf{\foreignlanguage{arabic}{مْفَلْسِع}}\ {\color{gray}\texttt{/\sffamily {{\sffamily mfalsiʕ}}/}\color{black}}\ \textsc{noun\textunderscore act}\ [m.]\ \textbf{1.}~running away\  \begin{flushright}\color{gray}\foreignlanguage{arabic}{\textbf{\underline{\foreignlanguage{arabic}{أمثلة}}}: مالفيت وجهي ولا هو مْفَلْسِع}\end{flushright}\color{black}} \vspace{2mm}

\vspace{-3mm}
\markboth{\color{blue}\foreignlanguage{arabic}{ف.ل.س.ف}\color{blue}{}}{\color{blue}\foreignlanguage{arabic}{ف.ل.س.ف}\color{blue}{}}\subsection*{\color{blue}\foreignlanguage{arabic}{ف.ل.س.ف}\color{blue}{}\index{\color{blue}\foreignlanguage{arabic}{ف.ل.س.ف}\color{blue}{}}} 

{\setlength\topsep{0pt}\textbf{\foreignlanguage{arabic}{تْفَلْسَف}}\ {\color{gray}\texttt{/\sffamily {{\sffamily tfalsaf}}/}\color{black}}\ \textsc{verb}\ [p.]\ \textbf{1.}~pontificate about sth in a very idealistic way.  \textbf{2.}~critize sth or sb in an annoying way\ \ $\bullet$\ \ \setlength\topsep{0pt}\textbf{\foreignlanguage{arabic}{اِتْفَلْسَف}}\ {\color{gray}\texttt{/\sffamily {{\sffamily ʔitfalsaf}}/}\color{black}}\ [c.]\ \ $\bullet$\ \ \setlength\topsep{0pt}\textbf{\foreignlanguage{arabic}{يِتْفَلْسَف}}\ {\color{gray}\texttt{/\sffamily {{\sffamily jitfalsaf}}/}\color{black}}\ [i.]\  \begin{flushright}\color{gray}\foreignlanguage{arabic}{\textbf{\underline{\foreignlanguage{arabic}{أمثلة}}}: تقعجش تِتْفَلْسَف والله إِحنا ما ناقصينك}\end{flushright}\color{black}} \vspace{2mm}

{\setlength\topsep{0pt}\textbf{\foreignlanguage{arabic}{تْفِلْسِف}}\ {\color{gray}\texttt{/\sffamily {{\sffamily tfilsif}}/}\color{black}}\ \textsc{noun}\ [m.]\ \textbf{1.}~pontificating about sth in a very idealistic way.  \textbf{2.}~critizing sth or sb in an annoying way\ } \vspace{2mm}

{\setlength\topsep{0pt}\textbf{\foreignlanguage{arabic}{فَلْسَفِة}}\ {\color{gray}\texttt{/\sffamily {{\sffamily falsafe}}/}\color{black}}\ \textsc{noun}\ [f.]\ \textbf{1.}~philosophy  \textbf{2.}~approach  \textbf{3.}~pontificating about sth in a very idealistic way.  \textbf{4.}~critizing sth or sb in an annoying way\  \begin{flushright}\color{gray}\foreignlanguage{arabic}{\textbf{\underline{\foreignlanguage{arabic}{أمثلة}}}: سيبك من شغل الفَلْسَفِة والمثاليات هذا}\end{flushright}\color{black}} \vspace{2mm}

{\setlength\topsep{0pt}\textbf{\foreignlanguage{arabic}{فَيلَسُوف}}\ {\color{gray}\texttt{/\sffamily {{\sffamily fajlasuːf}}/}\color{black}}\ \textsc{noun}\ [m.]\ \textbf{1.}~philosopher  \textbf{2.}~philosophers\ \ $\bullet$\ \ \setlength\topsep{0pt}\textbf{\foreignlanguage{arabic}{فَلَاسِفَة}}\ {\color{gray}\texttt{/\sffamily {{\sffamily falaːsifa}}/}\color{black}}\ [pl.]\ } \vspace{2mm}

{\setlength\topsep{0pt}\textbf{\foreignlanguage{arabic}{مْفَلْسَف}}\ {\color{gray}\texttt{/\sffamily {{\sffamily mfalsaf}}/}\color{black}}\ \textsc{adj}\ [m.]\ \textbf{1.}~self-opinionated  \textbf{2.}~the person who keeps pontificating about sth in a very idealistic way.  \textbf{3.}~the person who keeps critizing people in an annoying way\  \begin{flushright}\color{gray}\foreignlanguage{arabic}{\textbf{\underline{\foreignlanguage{arabic}{أمثلة}}}: عَمَّك كثير مْفَلْسَف!}\end{flushright}\color{black}} \vspace{2mm}

{\setlength\topsep{0pt}\textbf{\foreignlanguage{arabic}{مْفَلْسَفْجِي}}\ {\color{gray}\texttt{/\sffamily {{\sffamily mfalsaf(dʒ)i}}/}\color{black}}\ \textsc{adj}\ [m.]\ \textbf{1.}~self-opinionated  \textbf{2.}~the person who keeps pontificating about sth in a very idealistic way.  \textbf{3.}~the person who keeps critizing people in an annoying way\ } \vspace{2mm}

\vspace{-3mm}
\markboth{\color{blue}\foreignlanguage{arabic}{ف.ل.ص}\color{blue}{}}{\color{blue}\foreignlanguage{arabic}{ف.ل.ص}\color{blue}{}}\subsection*{\color{blue}\foreignlanguage{arabic}{ف.ل.ص}\color{blue}{}\index{\color{blue}\foreignlanguage{arabic}{ف.ل.ص}\color{blue}{}}} 

{\setlength\topsep{0pt}\textbf{\foreignlanguage{arabic}{تَفْلِيص}}\ {\color{gray}\texttt{/\sffamily {{\sffamily tafliːsˤ}}/}\color{black}}\ \textsc{noun}\ [m.]\ \textbf{1.}~the state of being naked\ } \vspace{2mm}

{\setlength\topsep{0pt}\textbf{\foreignlanguage{arabic}{فَلَّص}}\ {\color{gray}\texttt{/\sffamily {{\sffamily fallasˤ}}/}\color{black}}\ \textsc{verb}\ [p.]\ \textbf{1.}~be naked.  \textbf{2.}~have nothing on\ \ $\bullet$\ \ \setlength\topsep{0pt}\textbf{\foreignlanguage{arabic}{فَلِّص}}\ {\color{gray}\texttt{/\sffamily {{\sffamily fallisˤ}}/}\color{black}}\ [c.]\ \ $\bullet$\ \ \setlength\topsep{0pt}\textbf{\foreignlanguage{arabic}{يفَلِّص}}\ {\color{gray}\texttt{/\sffamily {{\sffamily jfallisˤ}}/}\color{black}}\ [i.]\ \color{gray}(msa. \foreignlanguage{arabic}{يتَعرَّى}~\foreignlanguage{arabic}{\textbf{١.}})\color{black}\  \begin{flushright}\color{gray}\foreignlanguage{arabic}{\textbf{\underline{\foreignlanguage{arabic}{أمثلة}}}: راح عنتانيا و فَلَّص عالشط قلعاط يقلعطه}\end{flushright}\color{black}} \vspace{2mm}

{\setlength\topsep{0pt}\textbf{\foreignlanguage{arabic}{مْفَلِّص}}\ {\color{gray}\texttt{/\sffamily {{\sffamily mfallisˤ}}/}\color{black}}\ \textsc{adj}\ [m.]\ \textbf{1.}~being naked.  \textbf{2.}~having nothing on\  \begin{flushright}\color{gray}\foreignlanguage{arabic}{\textbf{\underline{\foreignlanguage{arabic}{أمثلة}}}: أول ما شفته مْفَلِّص هيك عيوني بقوا لبرة}\end{flushright}\color{black}} \vspace{2mm}

\vspace{-3mm}
\markboth{\color{blue}\foreignlanguage{arabic}{ف.ل.ص}\color{blue}{ (ntws)}}{\color{blue}\foreignlanguage{arabic}{ف.ل.ص}\color{blue}{ (ntws)}}\subsection*{\color{blue}\foreignlanguage{arabic}{ف.ل.ص}\color{blue}{ (ntws)}\index{\color{blue}\foreignlanguage{arabic}{ف.ل.ص}\color{blue}{ (ntws)}}} 

{\setlength\topsep{0pt}\textbf{\foreignlanguage{arabic}{فَالْصُو}}\ {\color{gray}\texttt{/\sffamily {{\sffamily faːlsˤu}}/}\color{black}}\ \textsc{noun}\ [m.]\ \textbf{1.}~bogus  \textbf{2.}~false\ } \vspace{2mm}

\vspace{-3mm}
\markboth{\color{blue}\foreignlanguage{arabic}{ف.ل.ط.ح}\color{blue}{}}{\color{blue}\foreignlanguage{arabic}{ف.ل.ط.ح}\color{blue}{}}\subsection*{\color{blue}\foreignlanguage{arabic}{ف.ل.ط.ح}\color{blue}{}\index{\color{blue}\foreignlanguage{arabic}{ف.ل.ط.ح}\color{blue}{}}} 

{\setlength\topsep{0pt}\textbf{\foreignlanguage{arabic}{فَلْطَح}}\ {\color{gray}\texttt{/\sffamily {{\sffamily faltˤaħ}}/}\color{black}}\ \textsc{noun}\ [m.]\ \color{gray}(msa. \foreignlanguage{arabic}{خبير بأمور الحياة وذكي}~\foreignlanguage{arabic}{\textbf{١.}})\color{black}\ \textbf{1.}~worldy-wise/sharp-witted\  \begin{flushright}\color{gray}\foreignlanguage{arabic}{\textbf{\underline{\foreignlanguage{arabic}{أمثلة}}}: شو يا فَلْطَحِة  عصرك بدكاش تبطر خرط عهالناس المسكينة؟}\end{flushright}\color{black}} \vspace{2mm}

\vspace{-3mm}
\markboth{\color{blue}\foreignlanguage{arabic}{ف.ل.ع}\color{blue}{}}{\color{blue}\foreignlanguage{arabic}{ف.ل.ع}\color{blue}{}}\subsection*{\color{blue}\foreignlanguage{arabic}{ف.ل.ع}\color{blue}{}\index{\color{blue}\foreignlanguage{arabic}{ف.ل.ع}\color{blue}{}}} 

{\setlength\topsep{0pt}\textbf{\foreignlanguage{arabic}{اِنْفَلَع}}\ {\color{gray}\texttt{/\sffamily {{\sffamily ʔinfalaʕ}}/}\color{black}}\ \textsc{verb}\ [p.]\ \textbf{1.}~be halved.  \textbf{2.}~be broken into two pieces\ \ $\bullet$\ \ \setlength\topsep{0pt}\textbf{\foreignlanguage{arabic}{اِنْفِلِع}}\ {\color{gray}\texttt{/\sffamily {{\sffamily ʔinfiliʕ}}/}\color{black}}\ [c.]\ \ $\bullet$\ \ \setlength\topsep{0pt}\textbf{\foreignlanguage{arabic}{يِنْفِلِع}}\ {\color{gray}\texttt{/\sffamily {{\sffamily jinfiliʕ}}/}\color{black}}\ [i.]\  \begin{flushright}\color{gray}\foreignlanguage{arabic}{\textbf{\underline{\foreignlanguage{arabic}{أمثلة}}}: خليه يِنْفِلِع الله لا يردُّه}\end{flushright}\color{black}} \vspace{2mm}

{\setlength\topsep{0pt}\textbf{\foreignlanguage{arabic}{فَالِع}}\ {\color{gray}\texttt{/\sffamily {{\sffamily faːliʕ}}/}\color{black}}\ \textsc{adj}\ [m.]\ \textbf{1.}~have cracks.  \textbf{2.}~halved\  \begin{flushright}\color{gray}\foreignlanguage{arabic}{\textbf{\underline{\foreignlanguage{arabic}{أمثلة}}}: ماله الباب فالِع زي هيك؟}\end{flushright}\color{black}} \vspace{2mm}

{\setlength\topsep{0pt}\textbf{\foreignlanguage{arabic}{فَالِع}}\ {\color{gray}\texttt{/\sffamily {{\sffamily faːliʕ}}/}\color{black}}\ \textsc{noun\textunderscore act}\ [m.]\ \textbf{1.}~halving  \textbf{2.}~breaking\  \begin{flushright}\color{gray}\foreignlanguage{arabic}{\textbf{\underline{\foreignlanguage{arabic}{أمثلة}}}: أنو اللي فالِع البطيخة هيك؟}\end{flushright}\color{black}} \vspace{2mm}

{\setlength\topsep{0pt}\textbf{\foreignlanguage{arabic}{فَلَع}}\ {\color{gray}\texttt{/\sffamily {{\sffamily falaʕ}}/}\color{black}}\ \textsc{verb}\ [p.]\ \textbf{1.}~halve sth.  \textbf{2.}~break sth into two pieces\ \ $\bullet$\ \ \setlength\topsep{0pt}\textbf{\foreignlanguage{arabic}{اِفْلَع}}\ {\color{gray}\texttt{/\sffamily {{\sffamily ʔiflaʕ}}/}\color{black}}\ [c.]\ \ $\bullet$\ \ \setlength\topsep{0pt}\textbf{\foreignlanguage{arabic}{يِفْلَع}}\ {\color{gray}\texttt{/\sffamily {{\sffamily jiflaʕ}}/}\color{black}}\ [i.]\  \begin{flushright}\color{gray}\foreignlanguage{arabic}{\textbf{\underline{\foreignlanguage{arabic}{أمثلة}}}: ضربها بالساطور وفَلَعها نصين}\end{flushright}\color{black}} \vspace{2mm}

{\setlength\topsep{0pt}\textbf{\foreignlanguage{arabic}{مَفْلُوع}}\ {\color{gray}\texttt{/\sffamily {{\sffamily mafluːʕ}}/}\color{black}}\ \textsc{noun\textunderscore pass}\ \textbf{1.}~have cracks.  \textbf{2.}~halved\  \begin{flushright}\color{gray}\foreignlanguage{arabic}{\textbf{\underline{\foreignlanguage{arabic}{أمثلة}}}: شوف كيف البندورة مَفْلوعَة ههههه}\end{flushright}\color{black}} \vspace{2mm}

\vspace{-3mm}
\markboth{\color{blue}\foreignlanguage{arabic}{ف.ل.ع.ص}\color{blue}{}}{\color{blue}\foreignlanguage{arabic}{ف.ل.ع.ص}\color{blue}{}}\subsection*{\color{blue}\foreignlanguage{arabic}{ف.ل.ع.ص}\color{blue}{}\index{\color{blue}\foreignlanguage{arabic}{ف.ل.ع.ص}\color{blue}{}}} 

{\setlength\topsep{0pt}\textbf{\foreignlanguage{arabic}{فَلْعُوص}}\ {\color{gray}\texttt{/\sffamily {{\sffamily falʕuːsˤ}}/}\color{black}}\ \textsc{adj}\ [m.]\ \color{gray}(msa. \foreignlanguage{arabic}{طفل وقح يتطاول على الكبار}~\foreignlanguage{arabic}{\textbf{١.}})\color{black}\ \textbf{1.}~an ill-bred child who behaves badly / disrespectfully towards old people\ \ $\bullet$\ \ \setlength\topsep{0pt}\textbf{\foreignlanguage{arabic}{فَلَاعِيص}}\ {\color{gray}\texttt{/\sffamily {{\sffamily falaːʕiːsˤ}}/}\color{black}}\ [pl.]\  \begin{flushright}\color{gray}\foreignlanguage{arabic}{\textbf{\underline{\foreignlanguage{arabic}{أمثلة}}}: خدلك عهالفَلْعُوص كأنه ما عنده أهل يربوه}\end{flushright}\color{black}} \vspace{2mm}

\vspace{-3mm}
\markboth{\color{blue}\foreignlanguage{arabic}{ف.ل.ف.ل}\color{blue}{}}{\color{blue}\foreignlanguage{arabic}{ف.ل.ف.ل}\color{blue}{}}\subsection*{\color{blue}\foreignlanguage{arabic}{ف.ل.ف.ل}\color{blue}{}\index{\color{blue}\foreignlanguage{arabic}{ف.ل.ف.ل}\color{blue}{}}} 

{\setlength\topsep{0pt}\textbf{\foreignlanguage{arabic}{تْفَلْفَل}}\ {\color{gray}\texttt{/\sffamily {{\sffamily tfalfal}}/}\color{black}}\ \textsc{verb}\ [p.]\ \textbf{1.}~be cooked (rice)\ \ $\bullet$\ \ \setlength\topsep{0pt}\textbf{\foreignlanguage{arabic}{اِتْفَلْفَل}}\ {\color{gray}\texttt{/\sffamily {{\sffamily ʔitfalfal}}/}\color{black}}\ [c.]\ \ $\bullet$\ \ \setlength\topsep{0pt}\textbf{\foreignlanguage{arabic}{يِتْفَلْفَل}}\ {\color{gray}\texttt{/\sffamily {{\sffamily jitfalfal}}/}\color{black}}\ [i.]\  \begin{flushright}\color{gray}\foreignlanguage{arabic}{\textbf{\underline{\foreignlanguage{arabic}{أمثلة}}}: ما أحلاه هالرز بيتْفَلْفَل بسرعة}\end{flushright}\color{black}} \vspace{2mm}

{\setlength\topsep{0pt}\textbf{\foreignlanguage{arabic}{فَلَافِل}}\ {\color{gray}\texttt{/\sffamily {{\sffamily falaːfil}}/}\color{black}}\ \textsc{noun}\ [m.]\ \color{gray}(msa. \foreignlanguage{arabic}{طعام مشهور يصنع من الحمص المنقوع والمجروش المضاف إِليه بهارات خاصة مع الثوم والبصل، ويتم قليه على شكل أقراص، ثم يقدم على شكل سندويشات ساخنة مع سلطات متنوعة ومخللات.}~\foreignlanguage{arabic}{\textbf{١.}})\color{black}\ \textbf{1.}~A famous food made from chopped, drenched hummus, topped with special spices with garlic and onions, and fried in the form of tablets, then served in the form of hot sandwiches with salad and pickles.\  \begin{flushright}\color{gray}\foreignlanguage{arabic}{\textbf{\underline{\foreignlanguage{arabic}{أمثلة}}}: نفسي بسندويشة فلافل مع شطة}\end{flushright}\color{black}} \vspace{2mm}

{\setlength\topsep{0pt}\textbf{\foreignlanguage{arabic}{فَلْفَل}}\ {\color{gray}\texttt{/\sffamily {{\sffamily falfal}}/}\color{black}}\ \textsc{verb}\ [p.]\ \textbf{1.}~cook rice\ \ $\bullet$\ \ \setlength\topsep{0pt}\textbf{\foreignlanguage{arabic}{فَلْفِل}}\ {\color{gray}\texttt{/\sffamily {{\sffamily falfil}}/}\color{black}}\ [c.]\ \ $\bullet$\ \ \setlength\topsep{0pt}\textbf{\foreignlanguage{arabic}{يفَلْفِل}}\ {\color{gray}\texttt{/\sffamily {{\sffamily jfalfil}}/}\color{black}}\ [i.]\ \color{gray}(msa. \foreignlanguage{arabic}{يطهو الأرز}~\foreignlanguage{arabic}{\textbf{١.}})\color{black}\  \begin{flushright}\color{gray}\foreignlanguage{arabic}{\textbf{\underline{\foreignlanguage{arabic}{أمثلة}}}: بدي أَفَلْفِل كاستين رز مع شعيرية عشان الملوخية و كاسة رز أصفر بدون ولا شي عشان اللبن}\end{flushright}\color{black}} \vspace{2mm}

{\setlength\topsep{0pt}\textbf{\foreignlanguage{arabic}{فَلْفَلِة}}\ {\color{gray}\texttt{/\sffamily {{\sffamily falfale}}/}\color{black}}\ \textsc{noun}\ [f.]\ \color{gray}(msa. \foreignlanguage{arabic}{طهو الأرز}~\foreignlanguage{arabic}{\textbf{١.}})\color{black}\ \textbf{1.}~cooking rice\  \begin{flushright}\color{gray}\foreignlanguage{arabic}{\textbf{\underline{\foreignlanguage{arabic}{أمثلة}}}: فَلْفَلِة الرز كثير سهلة وبدهاش غلبة أبد. صدق ما بتاخذ أكثر 10 دقايق}\end{flushright}\color{black}} \vspace{2mm}

{\setlength\topsep{0pt}\textbf{\foreignlanguage{arabic}{فِلْفِل}}\footnote{Collective noun}\ \ {\color{gray}\texttt{/\sffamily {{\sffamily filfil}}/}\color{black}}\ \textsc{noun}\ [m.]\ \textbf{1.}~pepper  \textbf{2.}~Bell pepper\ \ $\bullet$\ \ \textsc{ph.} \color{gray} \foreignlanguage{arabic}{فِلْفِل أسود}\color{black}\ {\color{gray}\texttt{/{\sffamily filfil ʔaswad}/}\color{black}}\ \textbf{1.}~pepper\  \begin{flushright}\color{gray}\foreignlanguage{arabic}{\textbf{\underline{\foreignlanguage{arabic}{أمثلة}}}: دارت فِلْفِل أسود عالطبخة وتشعوطت لسانلاتنا\ $\bullet$\ \  تكثريش فِلْفِل عشان بحبوش}\end{flushright}\color{black}} \vspace{2mm}

{\setlength\topsep{0pt}\textbf{\foreignlanguage{arabic}{فِلْفِلِة}}\footnote{Unit noun}\ \ {\color{gray}\texttt{/\sffamily {{\sffamily filfile}}/}\color{black}}\ \textsc{noun}\ [f.]\ \textbf{1.}~Bell pepper (one piece)\ } \vspace{2mm}

{\setlength\topsep{0pt}\textbf{\foreignlanguage{arabic}{فْلَيفْلِة}}\footnote{Collective noun}\ \ {\color{gray}\texttt{/\sffamily {{\sffamily fleːfle}}/}\color{black}}\ \textsc{noun}\ [f.]\ \textbf{1.}~Bell pepper\  \begin{flushright}\color{gray}\foreignlanguage{arabic}{\textbf{\underline{\foreignlanguage{arabic}{أمثلة}}}: حطي معها نص حبة فْلَيفْلِة وراس ثَوم}\end{flushright}\color{black}} \vspace{2mm}

\vspace{-3mm}
\markboth{\color{blue}\foreignlanguage{arabic}{ف.ل.ق}\color{blue}{}}{\color{blue}\foreignlanguage{arabic}{ف.ل.ق}\color{blue}{}}\subsection*{\color{blue}\foreignlanguage{arabic}{ف.ل.ق}\color{blue}{}\index{\color{blue}\foreignlanguage{arabic}{ف.ل.ق}\color{blue}{}}} 

{\setlength\topsep{0pt}\textbf{\foreignlanguage{arabic}{اِنْفَلَق}}\ {\color{gray}\texttt{/\sffamily {{\sffamily ʔinfala(q)}}/}\color{black}}\ \textsc{verb}\ [p.]\ \textbf{1.}~be splitted or cracked in two.  \textbf{2.}~feel very upset\ \ $\bullet$\ \ \setlength\topsep{0pt}\textbf{\foreignlanguage{arabic}{اِنْفِلِق}}\ {\color{gray}\texttt{/\sffamily {{\sffamily ʔinfili(q)}}/}\color{black}}\ [c.]\ \ $\bullet$\ \ \setlength\topsep{0pt}\textbf{\foreignlanguage{arabic}{يِنْفِلِق}}\ {\color{gray}\texttt{/\sffamily {{\sffamily jinfili(q)}}/}\color{black}}\ [i.]\  \begin{flushright}\color{gray}\foreignlanguage{arabic}{\textbf{\underline{\foreignlanguage{arabic}{أمثلة}}}: روح اِنْفِلِق يا!\ $\bullet$\ \  اِنْفَلَقت منها قد ماهي حيوانة ومستفزة}\end{flushright}\color{black}} \vspace{2mm}

{\setlength\topsep{0pt}\textbf{\foreignlanguage{arabic}{تْفَلَّق}}\ {\color{gray}\texttt{/\sffamily {{\sffamily tfallaq}}/}\color{black}}\ \textsc{verb}\ [p.]\ \textbf{1.}~be ripped off.  \textbf{2.}~be worn out\ \ $\bullet$\ \ \setlength\topsep{0pt}\textbf{\foreignlanguage{arabic}{اِتْفَلَّق}}\ {\color{gray}\texttt{/\sffamily {{\sffamily ʔitfallaq}}/}\color{black}}\ [c.]\ \ $\bullet$\ \ \setlength\topsep{0pt}\textbf{\foreignlanguage{arabic}{يِتْفَلَّق}}\ {\color{gray}\texttt{/\sffamily {{\sffamily jitfallaq}}/}\color{black}}\ [i.]\ \color{gray}(msa. \foreignlanguage{arabic}{يَتَمَزَّق}~\foreignlanguage{arabic}{\textbf{١.}})\color{black}\  \begin{flushright}\color{gray}\foreignlanguage{arabic}{\textbf{\underline{\foreignlanguage{arabic}{أمثلة}}}: تْفَلَّق الكيس قد ماهو محشّا}\end{flushright}\color{black}} \vspace{2mm}

{\setlength\topsep{0pt}\textbf{\foreignlanguage{arabic}{فَلَق}}\ {\color{gray}\texttt{/\sffamily {{\sffamily falla(q)}}/}\color{black}}\ \textsc{verb}\ [p.]\ \textbf{1.}~split or crack sth in two\ \ $\bullet$\ \ \setlength\topsep{0pt}\textbf{\foreignlanguage{arabic}{اُفْلُق}}\ {\color{gray}\texttt{/\sffamily {{\sffamily ʔiflu(q)}}/}\color{black}}\ [c.]\ \ $\bullet$\ \ \setlength\topsep{0pt}\textbf{\foreignlanguage{arabic}{يِفْلُق}}\ {\color{gray}\texttt{/\sffamily {{\sffamily jiflu(q)}}/}\color{black}}\ [i.]\ \ $\bullet$\ \ \textsc{ph.} \color{gray} \foreignlanguage{arabic}{حظُّه بيِفْلُق الصخر}\color{black}\ {\color{gray}\texttt{/{\sffamily ħa(ðˤ)(ðˤ)o bjiflu(q) ʔisˤsˤaxir}/}\color{black}}\ \textbf{1.}~it is an idiomatic expression that means that sb is very lucky\  \begin{flushright}\color{gray}\foreignlanguage{arabic}{\textbf{\underline{\foreignlanguage{arabic}{أمثلة}}}: اُفْلُقها بالسكينة زي هيك}\end{flushright}\color{black}} \vspace{2mm}

{\setlength\topsep{0pt}\textbf{\foreignlanguage{arabic}{فَلَّق}}\ {\color{gray}\texttt{/\sffamily {{\sffamily fallaq}}/}\color{black}}\ \textsc{verb}\ [p.]\ \textbf{1.}~split or crack sth in two because sth was stuffed a lot or full.  \textbf{2.}~wear sth out\ \ $\bullet$\ \ \setlength\topsep{0pt}\textbf{\foreignlanguage{arabic}{فَلِّق}}\ {\color{gray}\texttt{/\sffamily {{\sffamily falliq}}/}\color{black}}\ [c.]\ \ $\bullet$\ \ \setlength\topsep{0pt}\textbf{\foreignlanguage{arabic}{يفَلِّق}}\ {\color{gray}\texttt{/\sffamily {{\sffamily jfalliq}}/}\color{black}}\ [i.]\ \color{gray}(msa. \foreignlanguage{arabic}{مَزَّق}~\foreignlanguage{arabic}{\textbf{١.}})\color{black}\  \begin{flushright}\color{gray}\foreignlanguage{arabic}{\textbf{\underline{\foreignlanguage{arabic}{أمثلة}}}: فَلَّقت الشنطة قد ما حشيتها}\end{flushright}\color{black}} \vspace{2mm}

{\setlength\topsep{0pt}\textbf{\foreignlanguage{arabic}{فَلْقَة}}\ {\color{gray}\texttt{/\sffamily {{\sffamily falqa}}/}\color{black}}\ \textsc{noun}\ [f.]\ \color{gray}(msa. \foreignlanguage{arabic}{ضربة}~\foreignlanguage{arabic}{\textbf{١.}})\color{black}\ \textbf{1.}~hit\  \begin{flushright}\color{gray}\foreignlanguage{arabic}{\textbf{\underline{\foreignlanguage{arabic}{أمثلة}}}: طعميناه فَلْقَة نَهْنَه من العياط}\end{flushright}\color{black}} \vspace{2mm}

{\setlength\topsep{0pt}\textbf{\foreignlanguage{arabic}{فَلْقِة}}\ {\color{gray}\texttt{/\sffamily {{\sffamily fal(q)a}}/}\color{black}}\ \textsc{noun}\ [f.]\ \color{gray}(msa. \foreignlanguage{arabic}{قِطْعَة}~\foreignlanguage{arabic}{\textbf{١.}})\color{black}\ \textbf{1.}~piece\  \begin{flushright}\color{gray}\foreignlanguage{arabic}{\textbf{\underline{\foreignlanguage{arabic}{أمثلة}}}: اعطيني فَلْقِة الصابونة}\end{flushright}\color{black}} \vspace{2mm}

\vspace{-3mm}
\markboth{\color{blue}\foreignlanguage{arabic}{ف.ل.ك}\color{blue}{}}{\color{blue}\foreignlanguage{arabic}{ف.ل.ك}\color{blue}{}}\subsection*{\color{blue}\foreignlanguage{arabic}{ف.ل.ك}\color{blue}{}\index{\color{blue}\foreignlanguage{arabic}{ف.ل.ك}\color{blue}{}}} 

{\setlength\topsep{0pt}\textbf{\foreignlanguage{arabic}{فَلَك}}\ {\color{gray}\texttt{/\sffamily {{\sffamily falak}}/}\color{black}}\ \textsc{noun}\ [m.]\ \textbf{1.}~celestial body.  \textbf{2.}~orbit  \textbf{3.}~celestial bodies.  \textbf{4.}~orbits  \textbf{5.}~astronomy\ } \vspace{2mm}

\vspace{-3mm}
\markboth{\color{blue}\foreignlanguage{arabic}{ف.ل.ل}\color{blue}{}}{\color{blue}\foreignlanguage{arabic}{ف.ل.ل}\color{blue}{}}\subsection*{\color{blue}\foreignlanguage{arabic}{ف.ل.ل}\color{blue}{}\index{\color{blue}\foreignlanguage{arabic}{ف.ل.ل}\color{blue}{}}} 

{\setlength\topsep{0pt}\textbf{\foreignlanguage{arabic}{تَفْلِيل}}\ {\color{gray}\texttt{/\sffamily {{\sffamily tafliːl}}/}\color{black}}\ \textsc{noun}\ [m.]\ \textbf{1.}~the filling of sth\  \begin{flushright}\color{gray}\foreignlanguage{arabic}{\textbf{\underline{\foreignlanguage{arabic}{أمثلة}}}: قديش بيوخذه عتَفْلِيل التنك؟}\end{flushright}\color{black}} \vspace{2mm}

{\setlength\topsep{0pt}\textbf{\foreignlanguage{arabic}{تْفَلَّل}}\ {\color{gray}\texttt{/\sffamily {{\sffamily tfallal}}/}\color{black}}\ \textsc{verb}\ [p.]\ \textbf{1.}~be filled to the max\ \ $\bullet$\ \ \setlength\topsep{0pt}\textbf{\foreignlanguage{arabic}{اِتْفَلَّل}}\ {\color{gray}\texttt{/\sffamily {{\sffamily ʔitfallal}}/}\color{black}}\ [c.]\ \ $\bullet$\ \ \setlength\topsep{0pt}\textbf{\foreignlanguage{arabic}{يِتْفَلَّل}}\footnote{English loanword}\ \ {\color{gray}\texttt{/\sffamily {{\sffamily jitfallal}}/}\color{black}}\ [i.]\  \begin{flushright}\color{gray}\foreignlanguage{arabic}{\textbf{\underline{\foreignlanguage{arabic}{أمثلة}}}: بس تِتْفَلَّل القاعة بنبلش الحفل ان شاء الله\ $\bullet$\ \  فَلِّل تنك البنزين للسيارة لو سمحت}\end{flushright}\color{black}} \vspace{2mm}

{\setlength\topsep{0pt}\textbf{\foreignlanguage{arabic}{فَلَّل}}\ {\color{gray}\texttt{/\sffamily {{\sffamily fallal}}/}\color{black}}\ \textsc{verb}\ [p.]\ \textbf{1.}~fill sth to the max.  \textbf{2.}~be full\ \ $\bullet$\ \ \setlength\topsep{0pt}\textbf{\foreignlanguage{arabic}{فَلِّل}}\ {\color{gray}\texttt{/\sffamily {{\sffamily fallil}}/}\color{black}}\ [c.]\ \ $\bullet$\ \ \setlength\topsep{0pt}\textbf{\foreignlanguage{arabic}{يفَلِّل}}\footnote{English loanword}\ \ {\color{gray}\texttt{/\sffamily {{\sffamily jfallil}}/}\color{black}}\ [i.]\  \begin{flushright}\color{gray}\foreignlanguage{arabic}{\textbf{\underline{\foreignlanguage{arabic}{أمثلة}}}: بصراحة أنا فَلِّلِت عالأخير شوفي أحمد إِذا بده يثنِّي.}\end{flushright}\color{black}} \vspace{2mm}

{\setlength\topsep{0pt}\textbf{\foreignlanguage{arabic}{فُلّ}}\ {\color{gray}\texttt{/\sffamily {{\sffamily full}}/}\color{black}}\ \textsc{noun}\ [m.]\ \textbf{1.}~Jasmine\ } \vspace{2mm}

{\setlength\topsep{0pt}\textbf{\foreignlanguage{arabic}{فُلِّة}}\footnote{Unit noun}\ \ {\color{gray}\texttt{/\sffamily {{\sffamily fulle}}/}\color{black}}\ \textsc{noun}\ [f.]\ \textbf{1.}~Jasmine\  \begin{flushright}\color{gray}\foreignlanguage{arabic}{\textbf{\underline{\foreignlanguage{arabic}{أمثلة}}}: تعي شمي هالفُلِّة ما أحلى ريحتها}\end{flushright}\color{black}} \vspace{2mm}

{\setlength\topsep{0pt}\textbf{\foreignlanguage{arabic}{فِلَّة}}\ {\color{gray}\texttt{/\sffamily {{\sffamily villa}}/}\color{black}}\ \textsc{noun}\ [f.]\ \textbf{1.}~villa\ \ $\bullet$\ \ \setlength\topsep{0pt}\textbf{\foreignlanguage{arabic}{فِلَل}}\ {\color{gray}\texttt{/\sffamily {{\sffamily vilal}}/}\color{black}}\ [pl.]\  \begin{flushright}\color{gray}\foreignlanguage{arabic}{\textbf{\underline{\foreignlanguage{arabic}{أمثلة}}}: هاي المنطقة من الطيرة كلها فِلَل ما شاء الله. تحس إِنه المعهد جاي بالغلط.}\end{flushright}\color{black}} \vspace{2mm}

{\setlength\topsep{0pt}\textbf{\foreignlanguage{arabic}{مْفَلِّل}}\ {\color{gray}\texttt{/\sffamily {{\sffamily mfallal}}/}\color{black}}\ \textsc{adj}\ [m.]\ \color{gray}(msa. \foreignlanguage{arabic}{مُمْتَلِئ}~\foreignlanguage{arabic}{\textbf{١.}})\color{black}\ \textbf{1.}~full  \textbf{2.}~filled\  \begin{flushright}\color{gray}\foreignlanguage{arabic}{\textbf{\underline{\foreignlanguage{arabic}{أمثلة}}}: البنزين عندي مْفَلِّل . ان شاء الله بيكفينا روحة رجعة.}\end{flushright}\color{black}} \vspace{2mm}

\vspace{-3mm}
\markboth{\color{blue}\foreignlanguage{arabic}{ف.ل.ل.ك}\color{blue}{}}{\color{blue}\foreignlanguage{arabic}{ف.ل.ل.ك}\color{blue}{}}\subsection*{\color{blue}\foreignlanguage{arabic}{ف.ل.ل.ك}\color{blue}{}\index{\color{blue}\foreignlanguage{arabic}{ف.ل.ل.ك}\color{blue}{}}} 

{\setlength\topsep{0pt}\textbf{\foreignlanguage{arabic}{فَلَّك}}\ {\color{gray}\texttt{/\sffamily {{\sffamily fallak}}/}\color{black}}\ \textsc{verb}\ [p.]\ (src. \color{gray}\foreignlanguage{arabic}{طولكرم}\color{black})\ \textbf{1.}~be allergic to sth\ \ $\bullet$\ \ \setlength\topsep{0pt}\textbf{\foreignlanguage{arabic}{فَلِّك}}\ {\color{gray}\texttt{/\sffamily {{\sffamily fallik}}/}\color{black}}\ [c.]\ \ $\bullet$\ \ \setlength\topsep{0pt}\textbf{\foreignlanguage{arabic}{يفَلِّك}}\ {\color{gray}\texttt{/\sffamily {{\sffamily jfallik}}/}\color{black}}\ [i.]\ \color{gray}(msa. \foreignlanguage{arabic}{يتحسس من شيء}~\foreignlanguage{arabic}{\textbf{١.}})\color{black}\  \begin{flushright}\color{gray}\foreignlanguage{arabic}{\textbf{\underline{\foreignlanguage{arabic}{أمثلة}}}: بس أكلت موز جسمي فَلَّك كله}\end{flushright}\color{black}} \vspace{2mm}

{\setlength\topsep{0pt}\textbf{\foreignlanguage{arabic}{مْفَلِّك}}\ {\color{gray}\texttt{/\sffamily {{\sffamily mfallik}}/}\color{black}}\ \textsc{adj}\ [m.]\ (src. \color{gray}\foreignlanguage{arabic}{طولكرم}\color{black})\ \color{gray}(msa. \foreignlanguage{arabic}{متحسس من شيء}~\foreignlanguage{arabic}{\textbf{١.}})\color{black}\ \textbf{1.}~be allergic to sth\  \begin{flushright}\color{gray}\foreignlanguage{arabic}{\textbf{\underline{\foreignlanguage{arabic}{أمثلة}}}: ماله جسمك مْفََلِّكْ؟}\end{flushright}\color{black}} \vspace{2mm}

\vspace{-3mm}
\markboth{\color{blue}\foreignlanguage{arabic}{ف.ل.م}\color{blue}{}}{\color{blue}\foreignlanguage{arabic}{ف.ل.م}\color{blue}{}}\subsection*{\color{blue}\foreignlanguage{arabic}{ف.ل.م}\color{blue}{}\index{\color{blue}\foreignlanguage{arabic}{ف.ل.م}\color{blue}{}}} 

{\setlength\topsep{0pt}\textbf{\foreignlanguage{arabic}{اِنْفَلَم}}\ {\color{gray}\texttt{/\sffamily {{\sffamily ʔinfalam}}/}\color{black}}\ \textsc{verb}\ [p.]\ \textbf{1.}~be pranked.  \textbf{2.}~be deceived.  \textbf{3.}~depend on sb to do sth then that person does not show up or refrain from doing it\ \ $\bullet$\ \ \setlength\topsep{0pt}\textbf{\foreignlanguage{arabic}{اِنْفِلِم}}\ {\color{gray}\texttt{/\sffamily {{\sffamily ʔinfilim}}/}\color{black}}\ [c.]\ \ $\bullet$\ \ \setlength\topsep{0pt}\textbf{\foreignlanguage{arabic}{يِنْفِلِم}}\ {\color{gray}\texttt{/\sffamily {{\sffamily jinfilim}}/}\color{black}}\ [i.]\  \begin{flushright}\color{gray}\foreignlanguage{arabic}{\textbf{\underline{\foreignlanguage{arabic}{أمثلة}}}: اليوم اِنْفَلَمنا بالشركة اللي بتنظف البلاط}\end{flushright}\color{black}} \vspace{2mm}

{\setlength\topsep{0pt}\textbf{\foreignlanguage{arabic}{فَلَم}}\ {\color{gray}\texttt{/\sffamily {{\sffamily falam}}/}\color{black}}\ \textsc{verb}\ [p.]\ \textbf{1.}~prank sb.  \textbf{2.}~deceive  \textbf{3.}~depend on sb to do sth then that person does not show up or refrain from doing it\ \ $\bullet$\ \ \setlength\topsep{0pt}\textbf{\foreignlanguage{arabic}{اِفْلِم}}\ {\color{gray}\texttt{/\sffamily {{\sffamily ʔiflam}}/}\color{black}}\ [c.]\ \ $\bullet$\ \ \setlength\topsep{0pt}\textbf{\foreignlanguage{arabic}{يِفْلِم}}\ {\color{gray}\texttt{/\sffamily {{\sffamily jiflam}}/}\color{black}}\ [i.]\  \begin{flushright}\color{gray}\foreignlanguage{arabic}{\textbf{\underline{\foreignlanguage{arabic}{أمثلة}}}: أركنت عليها تجيبلي الثومات راحت فَلَمتني}\end{flushright}\color{black}} \vspace{2mm}

{\setlength\topsep{0pt}\textbf{\foreignlanguage{arabic}{فِلِم}}\ {\color{gray}\texttt{/\sffamily {{\sffamily filim}}/}\color{black}}\ \textsc{noun}\ [m.]\ \color{gray}(msa. \foreignlanguage{arabic}{فِلْم}~\foreignlanguage{arabic}{\textbf{١.}})\color{black}\ \textbf{1.}~film  \textbf{2.}~movie\ \ $\bullet$\ \ \setlength\topsep{0pt}\textbf{\foreignlanguage{arabic}{أَفْلَام}}\ {\color{gray}\texttt{/\sffamily {{\sffamily ʔaflaːm}}/}\color{black}}\ [pl.]\ \ $\bullet$\ \ \textsc{ph.} \color{gray} \foreignlanguage{arabic}{سَحب أفْلَام}\color{black}\ {\color{gray}\texttt{/{\sffamily saħab ʔaflaːm}/}\color{black}}\ \color{gray} (msa. \foreignlanguage{arabic}{يَخْدَع أو يَكْذِب على شخص}~\foreignlanguage{arabic}{\textbf{١.}})\color{black}\ \textbf{1.}~deceive sb.  \textbf{2.}~tell lies\  \begin{flushright}\color{gray}\foreignlanguage{arabic}{\textbf{\underline{\foreignlanguage{arabic}{أمثلة}}}: الصبح كله نايم وبالليل مقضِّيها أفْلام ومسلسلات}\end{flushright}\color{black}} \vspace{2mm}

\vspace{-3mm}
\markboth{\color{blue}\foreignlanguage{arabic}{ف.ل.ن}\color{blue}{}}{\color{blue}\foreignlanguage{arabic}{ف.ل.ن}\color{blue}{}}\subsection*{\color{blue}\foreignlanguage{arabic}{ف.ل.ن}\color{blue}{}\index{\color{blue}\foreignlanguage{arabic}{ف.ل.ن}\color{blue}{}}} 

{\setlength\topsep{0pt}\textbf{\foreignlanguage{arabic}{فْلَان}}\ {\color{gray}\texttt{/\sffamily {{\sffamily flaːn}}/}\color{black}}\ \textsc{noun}\ [m.]\ \textbf{1.}~so-and-so  \textbf{2.}~such-and-such  \textbf{3.}~X\ } \vspace{2mm}

\vspace{-3mm}
\markboth{\color{blue}\foreignlanguage{arabic}{ف.ل.ن}\color{blue}{ (ntws)}}{\color{blue}\foreignlanguage{arabic}{ف.ل.ن}\color{blue}{ (ntws)}}\subsection*{\color{blue}\foreignlanguage{arabic}{ف.ل.ن}\color{blue}{ (ntws)}\index{\color{blue}\foreignlanguage{arabic}{ف.ل.ن}\color{blue}{ (ntws)}}} 

{\setlength\topsep{0pt}\textbf{\foreignlanguage{arabic}{فَلِّين}}\ {\color{gray}\texttt{/\sffamily {{\sffamily falliːn}}/}\color{black}}\ \textsc{noun}\ [m.]\ \textbf{1.}~cork\ } \vspace{2mm}

{\setlength\topsep{0pt}\textbf{\foreignlanguage{arabic}{فَلِّينِة}}\ {\color{gray}\texttt{/\sffamily {{\sffamily falliːne}}/}\color{black}}\ \textsc{noun}\ [f.]\ \textbf{1.}~cork board\  \begin{flushright}\color{gray}\foreignlanguage{arabic}{\textbf{\underline{\foreignlanguage{arabic}{أمثلة}}}: جيب فَلِّينِة كبيرة وحط عليها صور وأحرف وأرقام للولاد}\end{flushright}\color{black}} \vspace{2mm}

\vspace{-3mm}
\markboth{\color{blue}\foreignlanguage{arabic}{ف.ل.ي}\color{blue}{}}{\color{blue}\foreignlanguage{arabic}{ف.ل.ي}\color{blue}{}}\subsection*{\color{blue}\foreignlanguage{arabic}{ف.ل.ي}\color{blue}{}\index{\color{blue}\foreignlanguage{arabic}{ف.ل.ي}\color{blue}{}}} 

{\setlength\topsep{0pt}\textbf{\foreignlanguage{arabic}{تِفْلَايِة}}\ {\color{gray}\texttt{/\sffamily {{\sffamily tiflaːje}}/}\color{black}}\ \textsc{noun}\ [f.]\ \textbf{1.}~sifting through sth.  \textbf{2.}~sift sth out\ } \vspace{2mm}

{\setlength\topsep{0pt}\textbf{\foreignlanguage{arabic}{تْفَلَّى}}\ {\color{gray}\texttt{/\sffamily {{\sffamily tfalla}}/}\color{black}}\ \textsc{verb}\ [p.]\ \textbf{1.}~be scraped off (dandruff).  \textbf{2.}~be sifted through.  \textbf{3.}~be sift out\ \ $\bullet$\ \ \setlength\topsep{0pt}\textbf{\foreignlanguage{arabic}{اِتْفَلَّى}}\ {\color{gray}\texttt{/\sffamily {{\sffamily ʔitfalla}}/}\color{black}}\ [c.]\ \ $\bullet$\ \ \setlength\topsep{0pt}\textbf{\foreignlanguage{arabic}{يِتْفَلَّى}}\ {\color{gray}\texttt{/\sffamily {{\sffamily jitfalla}}/}\color{black}}\ [i.]\  \begin{flushright}\color{gray}\foreignlanguage{arabic}{\textbf{\underline{\foreignlanguage{arabic}{أمثلة}}}: الفريكة ملانة صرار لازم تِتْفَلَّى مليح}\end{flushright}\color{black}} \vspace{2mm}

{\setlength\topsep{0pt}\textbf{\foreignlanguage{arabic}{فَلَّى}}\ {\color{gray}\texttt{/\sffamily {{\sffamily falla}}/}\color{black}}\ \textsc{verb}\ [p.]\ \textbf{1.}~scrape off dandruff.  \textbf{2.}~sift through sth.  \textbf{3.}~sift sth out\ \ $\bullet$\ \ \setlength\topsep{0pt}\textbf{\foreignlanguage{arabic}{فَلِّي}}\ {\color{gray}\texttt{/\sffamily {{\sffamily falli}}/}\color{black}}\ [c.]\ \ $\bullet$\ \ \setlength\topsep{0pt}\textbf{\foreignlanguage{arabic}{يفَلِّي}}\ {\color{gray}\texttt{/\sffamily {{\sffamily jfalli}}/}\color{black}}\ [i.]\  \begin{flushright}\color{gray}\foreignlanguage{arabic}{\textbf{\underline{\foreignlanguage{arabic}{أمثلة}}}: تعالي فَلِّيلي شعري والله القشرى أكلت راسي أكل\ $\bullet$\ \  فَلِّيت الفصول كلها وما كنت ألاقيه}\end{flushright}\color{black}} \vspace{2mm}

{\setlength\topsep{0pt}\textbf{\foreignlanguage{arabic}{مْفَلَّى}}\ {\color{gray}\texttt{/\sffamily {{\sffamily mfalla}}/}\color{black}}\ \textsc{adj}\ [m.]\ \textbf{1.}~sifted\  \begin{flushright}\color{gray}\foreignlanguage{arabic}{\textbf{\underline{\foreignlanguage{arabic}{أمثلة}}}: الملف مْفَلَّى تِفْلايِة أتحدّاك تلاقي فيه شي غلط}\end{flushright}\color{black}} \vspace{2mm}

\vspace{-3mm}
\markboth{\color{blue}\foreignlanguage{arabic}{ف.ن.ت.ز}\color{blue}{}}{\color{blue}\foreignlanguage{arabic}{ف.ن.ت.ز}\color{blue}{}}\subsection*{\color{blue}\foreignlanguage{arabic}{ف.ن.ت.ز}\color{blue}{}\index{\color{blue}\foreignlanguage{arabic}{ف.ن.ت.ز}\color{blue}{}}} 

{\setlength\topsep{0pt}\textbf{\foreignlanguage{arabic}{تْفَنْتَز}}\ {\color{gray}\texttt{/\sffamily {{\sffamily tfantˤaz}}/}\color{black}}\ \textsc{verb}\ [p.]\ \textbf{1.}~enjoy oneself.  \textbf{2.}~have a good time\ \ $\bullet$\ \ \setlength\topsep{0pt}\textbf{\foreignlanguage{arabic}{اِتْفَنْتَز}}\ {\color{gray}\texttt{/\sffamily {{\sffamily ʔitfantˤaz}}/}\color{black}}\ [c.]\ \ $\bullet$\ \ \setlength\topsep{0pt}\textbf{\foreignlanguage{arabic}{يِتْفَنْتَز}}\ {\color{gray}\texttt{/\sffamily {{\sffamily jitfantˤaz}}/}\color{black}}\ [i.]\  \begin{flushright}\color{gray}\foreignlanguage{arabic}{\textbf{\underline{\foreignlanguage{arabic}{أمثلة}}}: هي ليش فَنْجَرَت عيونها هيك؟}\end{flushright}\color{black}} \vspace{2mm}

{\setlength\topsep{0pt}\textbf{\foreignlanguage{arabic}{فَنْتَز}}\ {\color{gray}\texttt{/\sffamily {{\sffamily fantˤaz}}/}\color{black}}\ \textsc{verb}\ [p.]\ \textbf{1.}~enjoy oneself.  \textbf{2.}~have a good time\ \ $\bullet$\ \ \setlength\topsep{0pt}\textbf{\foreignlanguage{arabic}{فَنْتِز}}\ {\color{gray}\texttt{/\sffamily {{\sffamily fantˤiz}}/}\color{black}}\ [c.]\ \ $\bullet$\ \ \setlength\topsep{0pt}\textbf{\foreignlanguage{arabic}{يفَنْتِز}}\ {\color{gray}\texttt{/\sffamily {{\sffamily jfantˤiz}}/}\color{black}}\ [i.]\  \begin{flushright}\color{gray}\foreignlanguage{arabic}{\textbf{\underline{\foreignlanguage{arabic}{أمثلة}}}: أنا ولينا فَنْتَزنا واحنا برام الله وقت الكريسماس}\end{flushright}\color{black}} \vspace{2mm}

{\setlength\topsep{0pt}\textbf{\foreignlanguage{arabic}{فَنْتَزِيِّة}}\ {\color{gray}\texttt{/\sffamily {{\sffamily fantˤazˤijje}}/}\color{black}}\ \textsc{noun}\ [f.]\ \color{gray}(msa. \foreignlanguage{arabic}{نمط حياة فاره}~\foreignlanguage{arabic}{\textbf{١.}})\color{black}\ \textbf{1.}~a highly prestigious lifestyle\  \begin{flushright}\color{gray}\foreignlanguage{arabic}{\textbf{\underline{\foreignlanguage{arabic}{أمثلة}}}: عفكرة عشنا الفََنْتَزِيِّة احنا ولاد عز, يرحم جدك لما كان يركب عالحمار ويفكره حصان}\end{flushright}\color{black}} \vspace{2mm}

{\setlength\topsep{0pt}\textbf{\foreignlanguage{arabic}{مْفَنْتِز}}\ {\color{gray}\texttt{/\sffamily {{\sffamily mfantˤiz}}/}\color{black}}\ \textsc{adj}\ [m.]\ \textbf{1.}~enjoying oneself.  \textbf{2.}~having a good time\  \begin{flushright}\color{gray}\foreignlanguage{arabic}{\textbf{\underline{\foreignlanguage{arabic}{أمثلة}}}: أما شو كنت مْفَنْتِزة بعمان ماكانش الي نفس أروِّح}\end{flushright}\color{black}} \vspace{2mm}

\vspace{-3mm}
\markboth{\color{blue}\foreignlanguage{arabic}{ف.ن.ج.ر}\color{blue}{}}{\color{blue}\foreignlanguage{arabic}{ف.ن.ج.ر}\color{blue}{}}\subsection*{\color{blue}\foreignlanguage{arabic}{ف.ن.ج.ر}\color{blue}{}\index{\color{blue}\foreignlanguage{arabic}{ف.ن.ج.ر}\color{blue}{}}} 

{\setlength\topsep{0pt}\textbf{\foreignlanguage{arabic}{فَنْجَر}}\ {\color{gray}\texttt{/\sffamily {{\sffamily fan(dʒ)ar}}/}\color{black}}\ \textsc{verb}\ [p.]\ \textbf{1.}~open sb's eyes widely\ \ $\bullet$\ \ \setlength\topsep{0pt}\textbf{\foreignlanguage{arabic}{فَنْجِر}}\ {\color{gray}\texttt{/\sffamily {{\sffamily fan(dʒ)ir}}/}\color{black}}\ [c.]\ \ $\bullet$\ \ \setlength\topsep{0pt}\textbf{\foreignlanguage{arabic}{يفَنْجِر}}\ {\color{gray}\texttt{/\sffamily {{\sffamily jfan(dʒ)ir}}/}\color{black}}\ [i.]\ \color{gray}(msa. \foreignlanguage{arabic}{يفتح عينيه بشكل واسع}~\foreignlanguage{arabic}{\textbf{١.}})\color{black}\ } \vspace{2mm}

{\setlength\topsep{0pt}\textbf{\foreignlanguage{arabic}{مْفَنْجِر}}\ {\color{gray}\texttt{/\sffamily {{\sffamily mfan(dʒ)ir}}/}\color{black}}\ \textsc{noun\textunderscore act}\ [m.]\ \color{gray}(msa. \foreignlanguage{arabic}{يفتح عينيه بشكل واسع}~\foreignlanguage{arabic}{\textbf{١.}})\color{black}\ \textbf{1.}~opening sb's eyes widely\  \begin{flushright}\color{gray}\foreignlanguage{arabic}{\textbf{\underline{\foreignlanguage{arabic}{أمثلة}}}: كان مْفَنْجِر عيونه بس فتحلنا الباب فكلنا خفنا منه}\end{flushright}\color{black}} \vspace{2mm}

\vspace{-3mm}
\markboth{\color{blue}\foreignlanguage{arabic}{ف.ن.ج.ل}\color{blue}{}}{\color{blue}\foreignlanguage{arabic}{ف.ن.ج.ل}\color{blue}{}}\subsection*{\color{blue}\foreignlanguage{arabic}{ف.ن.ج.ل}\color{blue}{}\index{\color{blue}\foreignlanguage{arabic}{ف.ن.ج.ل}\color{blue}{}}} 

{\setlength\topsep{0pt}\textbf{\foreignlanguage{arabic}{فِنْجَال}}\ {\color{gray}\texttt{/\sffamily {{\sffamily fin(dʒ)aːl}}/}\color{black}}\ \textsc{noun}\ [m.]\ \color{gray}(msa. \foreignlanguage{arabic}{فِنْجان}~\foreignlanguage{arabic}{\textbf{١.}})\color{black}\ \textbf{1.}~cup\ \ $\bullet$\ \ \setlength\topsep{0pt}\textbf{\foreignlanguage{arabic}{فَنَاجِيل}}\ {\color{gray}\texttt{/\sffamily {{\sffamily fana(dʒ)iːl}}/}\color{black}}\ [pl.]\  \begin{flushright}\color{gray}\foreignlanguage{arabic}{\textbf{\underline{\foreignlanguage{arabic}{أمثلة}}}: عيري المقادير بالفِنْجال أضمنلك}\end{flushright}\color{black}} \vspace{2mm}

\vspace{-3mm}
\markboth{\color{blue}\foreignlanguage{arabic}{ف.ن.ج.ن}\color{blue}{}}{\color{blue}\foreignlanguage{arabic}{ف.ن.ج.ن}\color{blue}{}}\subsection*{\color{blue}\foreignlanguage{arabic}{ف.ن.ج.ن}\color{blue}{}\index{\color{blue}\foreignlanguage{arabic}{ف.ن.ج.ن}\color{blue}{}}} 

{\setlength\topsep{0pt}\textbf{\foreignlanguage{arabic}{فِنْجَان}}\ {\color{gray}\texttt{/\sffamily {{\sffamily fin(dʒ)aːn}}/}\color{black}}\ \textsc{noun}\ [m.]\ \color{gray}(msa. \foreignlanguage{arabic}{فِنْجان}~\foreignlanguage{arabic}{\textbf{١.}})\color{black}\ \textbf{1.}~cup\ \ $\bullet$\ \ \setlength\topsep{0pt}\textbf{\foreignlanguage{arabic}{فَنَاجِين}}\ {\color{gray}\texttt{/\sffamily {{\sffamily fana(dʒ)iːn}}/}\color{black}}\ [pl.]\ \ $\bullet$\ \ \textsc{ph.} \color{gray} \foreignlanguage{arabic}{فِنْجَان قهوة}\color{black}\ {\color{gray}\texttt{/{\sffamily fin(dʒ)aːn (q)ahwe}/}\color{black}}\ \textbf{1.}~usually talk and have some coffee\ \ $\bullet$\ \ \textsc{ph.} \color{gray} \foreignlanguage{arabic}{فِنْجَان القَاضِي}\color{black}\ {\color{gray}\texttt{/{\sffamily fin(dʒ)aːn ʔil(q)aː(dˤ)i}/}\color{black}}\ \textbf{1.}~Hedge bindweed.  \textbf{2.}~Convolvulus sepium\  \begin{flushright}\color{gray}\foreignlanguage{arabic}{\textbf{\underline{\foreignlanguage{arabic}{أمثلة}}}: بدي أشرب من تحت إِيدك فِنْجان قهوة\ $\bullet$\ \  فَناجِينها فيهم ريحة زفرة}\end{flushright}\color{black}} \vspace{2mm}

\vspace{-3mm}
\markboth{\color{blue}\foreignlanguage{arabic}{ف.ن.د}\color{blue}{}}{\color{blue}\foreignlanguage{arabic}{ف.ن.د}\color{blue}{}}\subsection*{\color{blue}\foreignlanguage{arabic}{ف.ن.د}\color{blue}{}\index{\color{blue}\foreignlanguage{arabic}{ف.ن.د}\color{blue}{}}} 

{\setlength\topsep{0pt}\textbf{\foreignlanguage{arabic}{تَفْنِيد}}\ {\color{gray}\texttt{/\sffamily {{\sffamily tafniːd}}/}\color{black}}\ \textsc{noun}\ [m.]\ \textbf{1.}~refutation\ } \vspace{2mm}

{\setlength\topsep{0pt}\textbf{\foreignlanguage{arabic}{تْفَنَّد}}\ {\color{gray}\texttt{/\sffamily {{\sffamily tfannad}}/}\color{black}}\ \textsc{verb}\ [p.]\ \textbf{1.}~be refuted\ \ $\bullet$\ \ \setlength\topsep{0pt}\textbf{\foreignlanguage{arabic}{اِتْفَنَّد}}\ {\color{gray}\texttt{/\sffamily {{\sffamily ʔitfannad}}/}\color{black}}\ [c.]\ \ $\bullet$\ \ \setlength\topsep{0pt}\textbf{\foreignlanguage{arabic}{يِتْفَنَّد}}\ {\color{gray}\texttt{/\sffamily {{\sffamily jitfannad}}/}\color{black}}\ [i.]\  \begin{flushright}\color{gray}\foreignlanguage{arabic}{\textbf{\underline{\foreignlanguage{arabic}{أمثلة}}}: بس أستاذ، بالكتاب بيقولوا إنه نظرية التطور تْفَنَّدت زمان}\end{flushright}\color{black}} \vspace{2mm}

{\setlength\topsep{0pt}\textbf{\foreignlanguage{arabic}{فَنَّد}}\ {\color{gray}\texttt{/\sffamily {{\sffamily fannad}}/}\color{black}}\ \textsc{verb}\ [p.]\ \textbf{1.}~refute\ \ $\bullet$\ \ \setlength\topsep{0pt}\textbf{\foreignlanguage{arabic}{فَنِّد}}\ {\color{gray}\texttt{/\sffamily {{\sffamily fannid}}/}\color{black}}\ [c.]\ \ $\bullet$\ \ \setlength\topsep{0pt}\textbf{\foreignlanguage{arabic}{يفَنِّد}}\ {\color{gray}\texttt{/\sffamily {{\sffamily jfannid}}/}\color{black}}\ [i.]\ \color{gray}(msa. \foreignlanguage{arabic}{يُفَنِّد}~\foreignlanguage{arabic}{\textbf{١.}})\color{black}\  \begin{flushright}\color{gray}\foreignlanguage{arabic}{\textbf{\underline{\foreignlanguage{arabic}{أمثلة}}}: هاي النظرية فَنَّدوها من زمان وصاروا يعتمدوا نظرية جديدة اسمها النظرية النسبية}\end{flushright}\color{black}} \vspace{2mm}

\vspace{-3mm}
\markboth{\color{blue}\foreignlanguage{arabic}{ف.ن.س}\color{blue}{}}{\color{blue}\foreignlanguage{arabic}{ف.ن.س}\color{blue}{}}\subsection*{\color{blue}\foreignlanguage{arabic}{ف.ن.س}\color{blue}{}\index{\color{blue}\foreignlanguage{arabic}{ف.ن.س}\color{blue}{}}} 

{\setlength\topsep{0pt}\textbf{\foreignlanguage{arabic}{اِنْفَنَس}}\ {\color{gray}\texttt{/\sffamily {{\sffamily ʔinfanas}}/}\color{black}}\ \textsc{verb}\ [p.]\ \textbf{1.}~be embarrassed\ \ $\bullet$\ \ \setlength\topsep{0pt}\textbf{\foreignlanguage{arabic}{اِنْفِنِس}}\ {\color{gray}\texttt{/\sffamily {{\sffamily ʔinfinis}}/}\color{black}}\ [c.]\ \ $\bullet$\ \ \setlength\topsep{0pt}\textbf{\foreignlanguage{arabic}{يِنْفِنِس}}\ {\color{gray}\texttt{/\sffamily {{\sffamily jinfinis}}/}\color{black}}\ [i.]\  \begin{flushright}\color{gray}\foreignlanguage{arabic}{\textbf{\underline{\foreignlanguage{arabic}{أمثلة}}}: ياحرام حسيت شكله اِنْفَنَس مسكين}\end{flushright}\color{black}} \vspace{2mm}

{\setlength\topsep{0pt}\textbf{\foreignlanguage{arabic}{فَانُوس}}\ {\color{gray}\texttt{/\sffamily {{\sffamily faːnuːs}}/}\color{black}}\ \textsc{noun}\ [m.]\ \textbf{1.}~traditional lantern\ \ $\bullet$\ \ \setlength\topsep{0pt}\textbf{\foreignlanguage{arabic}{فَوَانِيس}}\ {\color{gray}\texttt{/\sffamily {{\sffamily fawaːniːs}}/}\color{black}}\ [pl.]\  \begin{flushright}\color{gray}\foreignlanguage{arabic}{\textbf{\underline{\foreignlanguage{arabic}{أمثلة}}}: أحلى شي وقت رمضان الفوانِيس}\end{flushright}\color{black}} \vspace{2mm}

{\setlength\topsep{0pt}\textbf{\foreignlanguage{arabic}{فَنَس}}\ {\color{gray}\texttt{/\sffamily {{\sffamily fanas}}/}\color{black}}\ \textsc{verb}\ [p.]\ \textbf{1.}~embarrass\ \ $\bullet$\ \ \setlength\topsep{0pt}\textbf{\foreignlanguage{arabic}{اِفْنِس}}\ {\color{gray}\texttt{/\sffamily {{\sffamily ʔifnis}}/}\color{black}}\ [c.]\ \ $\bullet$\ \ \setlength\topsep{0pt}\textbf{\foreignlanguage{arabic}{يِفْنِس}}\ {\color{gray}\texttt{/\sffamily {{\sffamily jifnis}}/}\color{black}}\ [i.]\ \color{gray}(msa. \foreignlanguage{arabic}{يُحْرِج}~\foreignlanguage{arabic}{\textbf{١.}})\color{black}\  \begin{flushright}\color{gray}\foreignlanguage{arabic}{\textbf{\underline{\foreignlanguage{arabic}{أمثلة}}}: عقد ما كنت طايرة بالسما فَنَسني الحقير}\end{flushright}\color{black}} \vspace{2mm}

{\setlength\topsep{0pt}\textbf{\foreignlanguage{arabic}{مَفْنُوس}}\ {\color{gray}\texttt{/\sffamily {{\sffamily mafnuːs}}/}\color{black}}\ \textsc{adj}\ [m.]\ \textbf{1.}~embarrassed\  \begin{flushright}\color{gray}\foreignlanguage{arabic}{\textbf{\underline{\foreignlanguage{arabic}{أمثلة}}}: يا حرام مَفْنوسة هلا بتلاقيها}\end{flushright}\color{black}} \vspace{2mm}

\vspace{-3mm}
\markboth{\color{blue}\foreignlanguage{arabic}{ف.ن.ش}\color{blue}{}}{\color{blue}\foreignlanguage{arabic}{ف.ن.ش}\color{blue}{}}\subsection*{\color{blue}\foreignlanguage{arabic}{ف.ن.ش}\color{blue}{}\index{\color{blue}\foreignlanguage{arabic}{ف.ن.ش}\color{blue}{}}} 

{\setlength\topsep{0pt}\textbf{\foreignlanguage{arabic}{تَفْنِيش}}\ {\color{gray}\texttt{/\sffamily {{\sffamily tafniːʃ}}/}\color{black}}\ \textsc{noun}\ [m.]\ \color{gray}(msa. \foreignlanguage{arabic}{تسريح شخص من العمل}~\foreignlanguage{arabic}{\textbf{١.}})\color{black}\ \textbf{1.}~dismissal\  \begin{flushright}\color{gray}\foreignlanguage{arabic}{\textbf{\underline{\foreignlanguage{arabic}{أمثلة}}}: في حملة تَفْنيشات بالشركة والله يستر ما نتفَنَّش احنا}\end{flushright}\color{black}} \vspace{2mm}

{\setlength\topsep{0pt}\textbf{\foreignlanguage{arabic}{تْفَنَّش}}\ {\color{gray}\texttt{/\sffamily {{\sffamily tfannaʃ}}/}\color{black}}\ \textsc{verb}\ [p.]\ \textbf{1.}~be sacked.  \textbf{2.}~be dismissed\ \ $\bullet$\ \ \setlength\topsep{0pt}\textbf{\foreignlanguage{arabic}{اِتْفَنَّش}}\ {\color{gray}\texttt{/\sffamily {{\sffamily ʔitfannaʃ}}/}\color{black}}\ [c.]\ \ $\bullet$\ \ \setlength\topsep{0pt}\textbf{\foreignlanguage{arabic}{يِتْفَنَّش}}\ {\color{gray}\texttt{/\sffamily {{\sffamily jitfannaʃ}}/}\color{black}}\ [i.]\  \begin{flushright}\color{gray}\foreignlanguage{arabic}{\textbf{\underline{\foreignlanguage{arabic}{أمثلة}}}: أبوها المسكين تْفَنَّش من شغله ومش ملاقي شغل ثاني}\end{flushright}\color{black}} \vspace{2mm}

{\setlength\topsep{0pt}\textbf{\foreignlanguage{arabic}{فَنَّش}}\ {\color{gray}\texttt{/\sffamily {{\sffamily fannaʃ}}/}\color{black}}\ \textsc{verb}\ [p.]\ \textbf{1.}~sack  \textbf{2.}~dismiss\ \ $\bullet$\ \ \setlength\topsep{0pt}\textbf{\foreignlanguage{arabic}{فَنِّش}}\footnote{English loanword (finish)}\ \ {\color{gray}\texttt{/\sffamily {{\sffamily fanniʃ}}/}\color{black}}\ [c.]\ \ $\bullet$\ \ \setlength\topsep{0pt}\textbf{\foreignlanguage{arabic}{يفَنِّش}}\ {\color{gray}\texttt{/\sffamily {{\sffamily jfanniʃ}}/}\color{black}}\ [i.]\ \color{gray}(msa. \foreignlanguage{arabic}{يُسَرِّح من العمل}~\foreignlanguage{arabic}{\textbf{١.}})\color{black}\  \begin{flushright}\color{gray}\foreignlanguage{arabic}{\textbf{\underline{\foreignlanguage{arabic}{أمثلة}}}: فَنَّشوه من شغله}\end{flushright}\color{black}} \vspace{2mm}

{\setlength\topsep{0pt}\textbf{\foreignlanguage{arabic}{مْفَنِّش}}\ {\color{gray}\texttt{/\sffamily {{\sffamily mfanniʃ}}/}\color{black}}\ \textsc{adj}\ [m.]\ (src. \color{gray}\foreignlanguage{arabic}{القدس}\color{black})\ \color{gray}(msa. \foreignlanguage{arabic}{الأذن الوطواطية}~\foreignlanguage{arabic}{\textbf{١.}})\color{black}\ \textbf{1.}~protruding ear\  \begin{flushright}\color{gray}\foreignlanguage{arabic}{\textbf{\underline{\foreignlanguage{arabic}{أمثلة}}}: بقينا بدنا نخطبها لأبننا بس انتبهنا انه ودنينها مْفَنْشات}\end{flushright}\color{black}} \vspace{2mm}

\vspace{-3mm}
\markboth{\color{blue}\foreignlanguage{arabic}{ف.ن.ط}\color{blue}{}}{\color{blue}\foreignlanguage{arabic}{ف.ن.ط}\color{blue}{}}\subsection*{\color{blue}\foreignlanguage{arabic}{ف.ن.ط}\color{blue}{}\index{\color{blue}\foreignlanguage{arabic}{ف.ن.ط}\color{blue}{}}} 

{\setlength\topsep{0pt}\textbf{\foreignlanguage{arabic}{تْفَنَّط}}\ {\color{gray}\texttt{/\sffamily {{\sffamily tfannatˤ}}/}\color{black}}\ \textsc{verb}\ [p.]\ \textbf{1.}~be shuffled (the cards)\ \ $\bullet$\ \ \setlength\topsep{0pt}\textbf{\foreignlanguage{arabic}{اِتْفَنَّط}}\ {\color{gray}\texttt{/\sffamily {{\sffamily ʔitfannatˤ}}/}\color{black}}\ [c.]\ \ $\bullet$\ \ \setlength\topsep{0pt}\textbf{\foreignlanguage{arabic}{يِتْفَنَّط}}\ {\color{gray}\texttt{/\sffamily {{\sffamily jitfannatˤ}}/}\color{black}}\ [i.]\ } \vspace{2mm}

{\setlength\topsep{0pt}\textbf{\foreignlanguage{arabic}{فَنَّط}}\ {\color{gray}\texttt{/\sffamily {{\sffamily fannatˤ}}/}\color{black}}\ \textsc{verb}\ [p.]\ \textbf{1.}~shuffle the cards\ \ $\bullet$\ \ \setlength\topsep{0pt}\textbf{\foreignlanguage{arabic}{فَنِّط}}\ {\color{gray}\texttt{/\sffamily {{\sffamily fannitˤ}}/}\color{black}}\ [c.]\ \ $\bullet$\ \ \setlength\topsep{0pt}\textbf{\foreignlanguage{arabic}{يفَنِّط}}\ {\color{gray}\texttt{/\sffamily {{\sffamily jfannitˤ}}/}\color{black}}\ [i.]\  \begin{flushright}\color{gray}\foreignlanguage{arabic}{\textbf{\underline{\foreignlanguage{arabic}{أمثلة}}}: خلوا فيصل هو اللي يفَنِّط الشدة\ $\bullet$\ \  الشدة ما تْفَنَّطتش مليح}\end{flushright}\color{black}} \vspace{2mm}

{\setlength\topsep{0pt}\textbf{\foreignlanguage{arabic}{فُنُط}}\ {\color{gray}\texttt{/\sffamily {{\sffamily funutˤ}}/}\color{black}}\ \textsc{noun}\ [f.]\ \color{gray}(msa. \foreignlanguage{arabic}{فوضى}~\foreignlanguage{arabic}{\textbf{١.}})\color{black}\ \textbf{1.}~choas\  \begin{flushright}\color{gray}\foreignlanguage{arabic}{\textbf{\underline{\foreignlanguage{arabic}{أمثلة}}}: أجوا الصغار وعملوا فنط في الدار}\end{flushright}\color{black}} \vspace{2mm}

\vspace{-3mm}
\markboth{\color{blue}\foreignlanguage{arabic}{ف.ن.ع}\color{blue}{}}{\color{blue}\foreignlanguage{arabic}{ف.ن.ع}\color{blue}{}}\subsection*{\color{blue}\foreignlanguage{arabic}{ف.ن.ع}\color{blue}{}\index{\color{blue}\foreignlanguage{arabic}{ف.ن.ع}\color{blue}{}}} 

{\setlength\topsep{0pt}\textbf{\foreignlanguage{arabic}{فَنَّع}}\ {\color{gray}\texttt{/\sffamily {{\sffamily fannaʕ}}/}\color{black}}\ \textsc{verb}\ [p.]\ \textbf{1.}~make troubles\ \ $\bullet$\ \ \setlength\topsep{0pt}\textbf{\foreignlanguage{arabic}{فَنِّع}}\ {\color{gray}\texttt{/\sffamily {{\sffamily fanniʕ}}/}\color{black}}\ [c.]\ \ $\bullet$\ \ \setlength\topsep{0pt}\textbf{\foreignlanguage{arabic}{يفَنِّع}}\ {\color{gray}\texttt{/\sffamily {{\sffamily jfanniʕ}}/}\color{black}}\ [i.]\  \begin{flushright}\color{gray}\foreignlanguage{arabic}{\textbf{\underline{\foreignlanguage{arabic}{أمثلة}}}: أنت بدكاش تعقل؟ بدك تضلك تفَنِّعلنا هيك!}\end{flushright}\color{black}} \vspace{2mm}

\vspace{-3mm}
\markboth{\color{blue}\foreignlanguage{arabic}{ف.ن.ك.ح}\color{blue}{}}{\color{blue}\foreignlanguage{arabic}{ف.ن.ك.ح}\color{blue}{}}\subsection*{\color{blue}\foreignlanguage{arabic}{ف.ن.ك.ح}\color{blue}{}\index{\color{blue}\foreignlanguage{arabic}{ف.ن.ك.ح}\color{blue}{}}} 

{\setlength\topsep{0pt}\textbf{\foreignlanguage{arabic}{تْفَنْكَح}}\ {\color{gray}\texttt{/\sffamily {{\sffamily tfan(k)aħ}}/}\color{black}}\ \textsc{verb}\ [p.]\ \textbf{1.}~tease  \textbf{2.}~exasperate\ \ $\bullet$\ \ \setlength\topsep{0pt}\textbf{\foreignlanguage{arabic}{اِتْفَنْكَح}}\ {\color{gray}\texttt{/\sffamily {{\sffamily ʔitfan(k)aħ}}/}\color{black}}\ [c.]\ \ $\bullet$\ \ \setlength\topsep{0pt}\textbf{\foreignlanguage{arabic}{يِتْفَنْكَح}}\ {\color{gray}\texttt{/\sffamily {{\sffamily jitfan(k)aħ}}/}\color{black}}\ [i.]\ \color{gray}(msa. \foreignlanguage{arabic}{يغيظ}~\foreignlanguage{arabic}{\textbf{١.}})\color{black}\  \begin{flushright}\color{gray}\foreignlanguage{arabic}{\textbf{\underline{\foreignlanguage{arabic}{أمثلة}}}: أخوي المشكلجي ببتفنكح عليه كل ما يشوفه ساكت}\end{flushright}\color{black}} \vspace{2mm}

{\setlength\topsep{0pt}\textbf{\foreignlanguage{arabic}{فَنْكَحَة}}\ {\color{gray}\texttt{/\sffamily {{\sffamily fan(k)aħa}}/}\color{black}}\ \textsc{noun}\ [f.]\ \textbf{1.}~teasing  \textbf{2.}~exasperation\ } \vspace{2mm}

\vspace{-3mm}
\markboth{\color{blue}\foreignlanguage{arabic}{ف.ن.ل}\color{blue}{ (ntws)}}{\color{blue}\foreignlanguage{arabic}{ف.ن.ل}\color{blue}{ (ntws)}}\subsection*{\color{blue}\foreignlanguage{arabic}{ف.ن.ل}\color{blue}{ (ntws)}\index{\color{blue}\foreignlanguage{arabic}{ف.ن.ل}\color{blue}{ (ntws)}}} 

{\setlength\topsep{0pt}\textbf{\foreignlanguage{arabic}{فَانِيلَّا}}\ {\color{gray}\texttt{/\sffamily {{\sffamily faːnilla}}/}\color{black}}\ \textsc{noun}\ [f.]\ \textbf{1.}~undershirt\ } \vspace{2mm}

\vspace{-3mm}
\markboth{\color{blue}\foreignlanguage{arabic}{ف.ن.ن}\color{blue}{}}{\color{blue}\foreignlanguage{arabic}{ف.ن.ن}\color{blue}{}}\subsection*{\color{blue}\foreignlanguage{arabic}{ف.ن.ن}\color{blue}{}\index{\color{blue}\foreignlanguage{arabic}{ف.ن.ن}\color{blue}{}}} 

{\setlength\topsep{0pt}\textbf{\foreignlanguage{arabic}{تْفَنَّن}}\ {\color{gray}\texttt{/\sffamily {{\sffamily tfannan}}/}\color{black}}\ \textsc{verb}\ [p.]\ \textbf{1.}~be ingenious.  \textbf{2.}~be inventive\ \ $\bullet$\ \ \setlength\topsep{0pt}\textbf{\foreignlanguage{arabic}{اِتْفَنَّن}}\ {\color{gray}\texttt{/\sffamily {{\sffamily ʔitfannan}}/}\color{black}}\ [c.]\ \ $\bullet$\ \ \setlength\topsep{0pt}\textbf{\foreignlanguage{arabic}{يِتْفَنَّن}}\ {\color{gray}\texttt{/\sffamily {{\sffamily jitfannan}}/}\color{black}}\ [i.]\  \begin{flushright}\color{gray}\foreignlanguage{arabic}{\textbf{\underline{\foreignlanguage{arabic}{أمثلة}}}: تْفَنَّنت بالطبيخ وأنا لحالي}\end{flushright}\color{black}} \vspace{2mm}

{\setlength\topsep{0pt}\textbf{\foreignlanguage{arabic}{فَنّ}}\ {\color{gray}\texttt{/\sffamily {{\sffamily fann}}/}\color{black}}\ \textsc{noun}\ [m.]\ \color{gray}(msa. \foreignlanguage{arabic}{فَن}~\foreignlanguage{arabic}{\textbf{١.}})\color{black}\ \textbf{1.}~art\ } \vspace{2mm}

{\setlength\topsep{0pt}\textbf{\foreignlanguage{arabic}{فَنّ}}\ {\color{gray}\texttt{/\sffamily {{\sffamily fann}}/}\color{black}}\ \textsc{verb}\ [p.]\ \textbf{1.}~urinate  \textbf{2.}~pee\ \ $\bullet$\ \ \setlength\topsep{0pt}\textbf{\foreignlanguage{arabic}{فِنّ}}\ {\color{gray}\texttt{/\sffamily {{\sffamily finn}}/}\color{black}}\ [c.]\ \ $\bullet$\ \ \setlength\topsep{0pt}\textbf{\foreignlanguage{arabic}{يفِنّ}}\ {\color{gray}\texttt{/\sffamily {{\sffamily jfinn}}/}\color{black}}\ [i.]\ \color{gray}(msa. \foreignlanguage{arabic}{يَتَبوَّل}~\foreignlanguage{arabic}{\textbf{١.}})\color{black}\  \begin{flushright}\color{gray}\foreignlanguage{arabic}{\textbf{\underline{\foreignlanguage{arabic}{أمثلة}}}: خليه يفِنّ قبل ما ينام بلاش ما يغرِّق الدنيا}\end{flushright}\color{black}} \vspace{2mm}

{\setlength\topsep{0pt}\textbf{\foreignlanguage{arabic}{فَنَّان}}\ {\color{gray}\texttt{/\sffamily {{\sffamily fannaːn}}/}\color{black}}\ \textsc{adj}\ [m.]\ \textbf{1.}~ingenious  \textbf{2.}~inventive\  \begin{flushright}\color{gray}\foreignlanguage{arabic}{\textbf{\underline{\foreignlanguage{arabic}{أمثلة}}}: أنت فَنّان فش زيك}\end{flushright}\color{black}} \vspace{2mm}

{\setlength\topsep{0pt}\textbf{\foreignlanguage{arabic}{فَنَّان}}\ {\color{gray}\texttt{/\sffamily {{\sffamily fannaːn}}/}\color{black}}\ \textsc{noun}\ [m.]\ \color{gray}(msa. \foreignlanguage{arabic}{فَنّان}~\foreignlanguage{arabic}{\textbf{١.}})\color{black}\ \textbf{1.}~artist\ } \vspace{2mm}

{\setlength\topsep{0pt}\textbf{\foreignlanguage{arabic}{فَنَّن}}\ {\color{gray}\texttt{/\sffamily {{\sffamily fannan}}/}\color{black}}\ \textsc{verb}\ [p.]\ \textbf{1.}~make sb urinate.  \textbf{2.}~make sb pee\ \ $\bullet$\ \ \setlength\topsep{0pt}\textbf{\foreignlanguage{arabic}{فَنِّن}}\ {\color{gray}\texttt{/\sffamily {{\sffamily fannin}}/}\color{black}}\ [c.]\ \ $\bullet$\ \ \setlength\topsep{0pt}\textbf{\foreignlanguage{arabic}{يفَنِّن}}\ {\color{gray}\texttt{/\sffamily {{\sffamily jfannin}}/}\color{black}}\ [i.]\  \begin{flushright}\color{gray}\foreignlanguage{arabic}{\textbf{\underline{\foreignlanguage{arabic}{أمثلة}}}: كل 3 ساعات حاولي فَنِّنيه وهيك بيبطِّل يعملها عحاله}\end{flushright}\color{black}} \vspace{2mm}

{\setlength\topsep{0pt}\textbf{\foreignlanguage{arabic}{فَنِّة}}\ {\color{gray}\texttt{/\sffamily {{\sffamily fanne}}/}\color{black}}\ \textsc{noun}\ [f.]\ \color{gray}(msa. \foreignlanguage{arabic}{بَوْل}~\foreignlanguage{arabic}{\textbf{١.}})\color{black}\ \textbf{1.}~urine  \textbf{2.}~pee\ \ $\bullet$\ \ \textsc{ph.} \color{gray} \foreignlanguage{arabic}{فَنِّة جديدة}\color{black}\ {\color{gray}\texttt{/{\sffamily fanne ʔi(dʒ)diːde}/}\color{black}}\ \textbf{1.}~It is an idiomatic expression that means that sth has become a trend\  \begin{flushright}\color{gray}\foreignlanguage{arabic}{\textbf{\underline{\foreignlanguage{arabic}{أمثلة}}}: هاي فَنِّة جديدة يا حبيبتي\ $\bullet$\ \  اجيت أقعد عالكنب لقيتها كلها فَنِّة}\end{flushright}\color{black}} \vspace{2mm}

{\setlength\topsep{0pt}\textbf{\foreignlanguage{arabic}{فَنِّي}}\ {\color{gray}\texttt{/\sffamily {{\sffamily fanni}}/}\color{black}}\ \textsc{adj}\ [m.]\ \textbf{1.}~artistic\  \begin{flushright}\color{gray}\foreignlanguage{arabic}{\textbf{\underline{\foreignlanguage{arabic}{أمثلة}}}: شو هالتحفة الفنيِّة يا وسام!}\end{flushright}\color{black}} \vspace{2mm}

\vspace{-3mm}
\markboth{\color{blue}\foreignlanguage{arabic}{ف.ن.ي}\color{blue}{}}{\color{blue}\foreignlanguage{arabic}{ف.ن.ي}\color{blue}{}}\subsection*{\color{blue}\foreignlanguage{arabic}{ف.ن.ي}\color{blue}{}\index{\color{blue}\foreignlanguage{arabic}{ف.ن.ي}\color{blue}{}}} 

{\setlength\topsep{0pt}\textbf{\foreignlanguage{arabic}{أَفْنَى}}\ {\color{gray}\texttt{/\sffamily {{\sffamily ʔafna}}/}\color{black}}\ \textsc{verb}\ [p.]\ \textbf{1.}~come to nothing.  \textbf{2.}~perish\ \ $\bullet$\ \ \setlength\topsep{0pt}\textbf{\foreignlanguage{arabic}{اِفْنِي}}\ {\color{gray}\texttt{/\sffamily {{\sffamily ʔifni}}/}\color{black}}\ [c.]\ \ $\bullet$\ \ \setlength\topsep{0pt}\textbf{\foreignlanguage{arabic}{يِفْنِي}}\ {\color{gray}\texttt{/\sffamily {{\sffamily jifni}}/}\color{black}}\ [i.]\  \begin{flushright}\color{gray}\foreignlanguage{arabic}{\textbf{\underline{\foreignlanguage{arabic}{أمثلة}}}: أبوي أَفْنَى حياته بمدرسة الوكالة}\end{flushright}\color{black}} \vspace{2mm}

{\setlength\topsep{0pt}\textbf{\foreignlanguage{arabic}{فَانِي}}\ {\color{gray}\texttt{/\sffamily {{\sffamily faːni}}/}\color{black}}\ \textsc{adj}\ [m.]\ \textbf{1.}~transient  \textbf{2.}~ephemeral\  \begin{flushright}\color{gray}\foreignlanguage{arabic}{\textbf{\underline{\foreignlanguage{arabic}{أمثلة}}}: كل شي بهالدنيا فانِي الا وجهه الكريم}\end{flushright}\color{black}} \vspace{2mm}

{\setlength\topsep{0pt}\textbf{\foreignlanguage{arabic}{فَنَاء}}\ {\color{gray}\texttt{/\sffamily {{\sffamily fanaːʔ}}/}\color{black}}\ \textsc{noun}\ [m.]\ \textbf{1.}~annihilation  \textbf{2.}~yard\ } \vspace{2mm}

{\setlength\topsep{0pt}\textbf{\foreignlanguage{arabic}{فَنَى}}\ {\color{gray}\texttt{/\sffamily {{\sffamily fana}}/}\color{black}}\ \textsc{verb}\ [p.]\ \textbf{1.}~come to nothing.  \textbf{2.}~perish\ \ $\bullet$\ \ \setlength\topsep{0pt}\textbf{\foreignlanguage{arabic}{اِفْنَي}}\ {\color{gray}\texttt{/\sffamily {{\sffamily ʔifna}}/}\color{black}}\ [c.]\ \ $\bullet$\ \ \setlength\topsep{0pt}\textbf{\foreignlanguage{arabic}{يِفْنَي}}\ {\color{gray}\texttt{/\sffamily {{\sffamily jifna}}/}\color{black}}\ [i.]\  \begin{flushright}\color{gray}\foreignlanguage{arabic}{\textbf{\underline{\foreignlanguage{arabic}{أمثلة}}}: هاي سنة الحياة يانور. بالأخير كلنا رح نِفنى ونموت.}\end{flushright}\color{black}} \vspace{2mm}

\vspace{-3mm}
\markboth{\color{blue}\foreignlanguage{arabic}{ف.ه.ق}\color{blue}{}}{\color{blue}\foreignlanguage{arabic}{ف.ه.ق}\color{blue}{}}\subsection*{\color{blue}\foreignlanguage{arabic}{ف.ه.ق}\color{blue}{}\index{\color{blue}\foreignlanguage{arabic}{ف.ه.ق}\color{blue}{}}} 

{\setlength\topsep{0pt}\textbf{\foreignlanguage{arabic}{فَاهَق}}\ {\color{gray}\texttt{/\sffamily {{\sffamily faːha(q)}}/}\color{black}}\ \textsc{verb}\ [p.]\ \textbf{1.}~catch one's breath (usually because of crying)\ \ $\bullet$\ \ \setlength\topsep{0pt}\textbf{\foreignlanguage{arabic}{فَاهِق}}\ {\color{gray}\texttt{/\sffamily {{\sffamily faːhi(q)}}/}\color{black}}\ [c.]\ \ $\bullet$\ \ \setlength\topsep{0pt}\textbf{\foreignlanguage{arabic}{يفَاهِق}}\ {\color{gray}\texttt{/\sffamily {{\sffamily jfaːhi(q)}}/}\color{black}}\ [i.]\  \begin{flushright}\color{gray}\foreignlanguage{arabic}{\textbf{\underline{\foreignlanguage{arabic}{أمثلة}}}: ولك مالك بِتفاهِق مْفاهَقَة أنو اللي عمل فيك هيك؟}\end{flushright}\color{black}} \vspace{2mm}

{\setlength\topsep{0pt}\textbf{\foreignlanguage{arabic}{فَهَق}}\ {\color{gray}\texttt{/\sffamily {{\sffamily faha(q)}}/}\color{black}}\ \textsc{verb}\ [p.]\ \textbf{1.}~gasp  \textbf{2.}~catch one's breath\ \ $\bullet$\ \ \setlength\topsep{0pt}\textbf{\foreignlanguage{arabic}{اِفْهَق}}\ {\color{gray}\texttt{/\sffamily {{\sffamily ʔifha(q)}}/}\color{black}}\ [c.]\ \ $\bullet$\ \ \setlength\topsep{0pt}\textbf{\foreignlanguage{arabic}{يِفْهَق}}\ {\color{gray}\texttt{/\sffamily {{\sffamily jifha(q)}}/}\color{black}}\ [i.]\  \begin{flushright}\color{gray}\foreignlanguage{arabic}{\textbf{\underline{\foreignlanguage{arabic}{أمثلة}}}: لما حكالي عن سعرها فَهَقِت فَهْقَة كل الناس سمعتها}\end{flushright}\color{black}} \vspace{2mm}

{\setlength\topsep{0pt}\textbf{\foreignlanguage{arabic}{فَهْقَة}}\ {\color{gray}\texttt{/\sffamily {{\sffamily fah(q)a}}/}\color{black}}\ \textsc{noun}\ [f.]\ \textbf{1.}~gasping  \textbf{2.}~catching one's breath\  \begin{flushright}\color{gray}\foreignlanguage{arabic}{\textbf{\underline{\foreignlanguage{arabic}{أمثلة}}}: شو هالفَهْقَة يغُص بالك!}\end{flushright}\color{black}} \vspace{2mm}

{\setlength\topsep{0pt}\textbf{\foreignlanguage{arabic}{مْفَاهَقَة}}\ {\color{gray}\texttt{/\sffamily {{\sffamily mfaːha(q)a}}/}\color{black}}\ \textsc{noun}\ [f.]\ \textbf{1.}~catching one's breath\ } \vspace{2mm}

\vspace{-3mm}
\markboth{\color{blue}\foreignlanguage{arabic}{ف.ه.ل.ق}\color{blue}{}}{\color{blue}\foreignlanguage{arabic}{ف.ه.ل.ق}\color{blue}{}}\subsection*{\color{blue}\foreignlanguage{arabic}{ف.ه.ل.ق}\color{blue}{}\index{\color{blue}\foreignlanguage{arabic}{ف.ه.ل.ق}\color{blue}{}}} 

{\setlength\topsep{0pt}\textbf{\foreignlanguage{arabic}{فَهْلَق}}\ {\color{gray}\texttt{/\sffamily {{\sffamily fahlaq}}/}\color{black}}\ \textsc{verb}\ [p.]\ \textbf{1.}~giggle\ \ $\bullet$\ \ \setlength\topsep{0pt}\textbf{\foreignlanguage{arabic}{فَهْلِق}}\ {\color{gray}\texttt{/\sffamily {{\sffamily fahliq}}/}\color{black}}\ [c.]\ \ $\bullet$\ \ \setlength\topsep{0pt}\textbf{\foreignlanguage{arabic}{يفَهْلِق}}\ {\color{gray}\texttt{/\sffamily {{\sffamily jfahliq}}/}\color{black}}\ [i.]\ \color{gray}(msa. \foreignlanguage{arabic}{يقهقه}~\foreignlanguage{arabic}{\textbf{١.}})\color{black}\  \begin{flushright}\color{gray}\foreignlanguage{arabic}{\textbf{\underline{\foreignlanguage{arabic}{أمثلة}}}: الله يرحمه لما بقى يفَهْلِقْ، بقى صوتة ضُحْحكه يوصل آخر الدنيا}\end{flushright}\color{black}} \vspace{2mm}

{\setlength\topsep{0pt}\textbf{\foreignlanguage{arabic}{فَهْلَقَة}}\ {\color{gray}\texttt{/\sffamily {{\sffamily fahlaqa}}/}\color{black}}\ \textsc{noun}\ [f.]\ \color{gray}(msa. \foreignlanguage{arabic}{قَهْقَهَة}~\foreignlanguage{arabic}{\textbf{١.}})\color{black}\ \textbf{1.}~giggle\  \begin{flushright}\color{gray}\foreignlanguage{arabic}{\textbf{\underline{\foreignlanguage{arabic}{أمثلة}}}: بكفي فَهْْلَقَة وركز بالعدسات فليهن منيح}\end{flushright}\color{black}} \vspace{2mm}

\vspace{-3mm}
\markboth{\color{blue}\foreignlanguage{arabic}{ف.ه.ل.و.ي}\color{blue}{ (ntws)}}{\color{blue}\foreignlanguage{arabic}{ف.ه.ل.و.ي}\color{blue}{ (ntws)}}\subsection*{\color{blue}\foreignlanguage{arabic}{ف.ه.ل.و.ي}\color{blue}{ (ntws)}\index{\color{blue}\foreignlanguage{arabic}{ف.ه.ل.و.ي}\color{blue}{ (ntws)}}} 

{\setlength\topsep{0pt}\textbf{\foreignlanguage{arabic}{فَهْلَوِي}}\ {\color{gray}\texttt{/\sffamily {{\sffamily fahlawi}}/}\color{black}}\ \textsc{adj}\ [m.]\ \textbf{1.}~smart  \textbf{2.}~clever\  \begin{flushright}\color{gray}\foreignlanguage{arabic}{\textbf{\underline{\foreignlanguage{arabic}{أمثلة}}}: عاملي حاله فَهْلَوِي وفهمان وهو اللي بيدري بيدري}\end{flushright}\color{black}} \vspace{2mm}

{\setlength\topsep{0pt}\textbf{\foreignlanguage{arabic}{فَهْلَوِيِّة}}\ {\color{gray}\texttt{/\sffamily {{\sffamily fahlawijje}}/}\color{black}}\ \textsc{noun}\ [f.]\ \textbf{1.}~smartness\ } \vspace{2mm}

\vspace{-3mm}
\markboth{\color{blue}\foreignlanguage{arabic}{ف.ه.م}\color{blue}{}}{\color{blue}\foreignlanguage{arabic}{ف.ه.م}\color{blue}{}}\subsection*{\color{blue}\foreignlanguage{arabic}{ف.ه.م}\color{blue}{}\index{\color{blue}\foreignlanguage{arabic}{ف.ه.م}\color{blue}{}}} 

{\setlength\topsep{0pt}\textbf{\foreignlanguage{arabic}{اِسْتَفْهَم}}\ {\color{gray}\texttt{/\sffamily {{\sffamily ʔistafham}}/}\color{black}}\ \textsc{verb}\ [p.]\ \textbf{1.}~ask about sth.  \textbf{2.}~inquire about sth\ \ $\bullet$\ \ \setlength\topsep{0pt}\textbf{\foreignlanguage{arabic}{اِسْتَفْهِم}}\ {\color{gray}\texttt{/\sffamily {{\sffamily ʔistafhim}}/}\color{black}}\ [c.]\ \ $\bullet$\ \ \setlength\topsep{0pt}\textbf{\foreignlanguage{arabic}{يِسْتَفْهِم}}\ {\color{gray}\texttt{/\sffamily {{\sffamily jistafhim}}/}\color{black}}\ [i.]\  \begin{flushright}\color{gray}\foreignlanguage{arabic}{\textbf{\underline{\foreignlanguage{arabic}{أمثلة}}}: اِسْتَفْهِم منه عن مكان السكن والسعر وهيك}\end{flushright}\color{black}} \vspace{2mm}

{\setlength\topsep{0pt}\textbf{\foreignlanguage{arabic}{اِسْتِفْهَام}}\ {\color{gray}\texttt{/\sffamily {{\sffamily ʔistifhaːm}}/}\color{black}}\ \textsc{noun}\ [m.]\ \color{gray}(msa. \foreignlanguage{arabic}{اِسْتِفْهام}~\foreignlanguage{arabic}{\textbf{١.}})\color{black}\ \textbf{1.}~inquiry\ \ $\bullet$\ \ \textsc{ph.} \color{gray} \foreignlanguage{arabic}{علَامة اِسْتِفْهَام}\color{black}\ {\color{gray}\texttt{/{\sffamily ʕalaːmit ʔistifhaːm}/}\color{black}}\ \textbf{1.}~a question mark ?\ \ $\bullet$\ \ \textsc{ph.} \color{gray} \foreignlanguage{arabic}{في مية علَامة اِسْتِفْهَام على}\color{black}\ {\color{gray}\texttt{/{\sffamily fiː miːt ʕalaːmit ʔistifhaːm}/}\color{black}}\ \textbf{1.}~sth is questionable\  \begin{flushright}\color{gray}\foreignlanguage{arabic}{\textbf{\underline{\foreignlanguage{arabic}{أمثلة}}}: موضوع تأخر زواجه لل48 في مية علامة اِسْتِفْهام عليه وتكونيش هبلة أحسنلك}\end{flushright}\color{black}} \vspace{2mm}

{\setlength\topsep{0pt}\textbf{\foreignlanguage{arabic}{اِنْفَهَم}}\ {\color{gray}\texttt{/\sffamily {{\sffamily ʔinfaham}}/}\color{black}}\ \textsc{verb}\ [p.]\ \textbf{1.}~be understood\ \ $\bullet$\ \ \setlength\topsep{0pt}\textbf{\foreignlanguage{arabic}{اِنْفِهِم}}\ {\color{gray}\texttt{/\sffamily {{\sffamily ʔinfihim}}/}\color{black}}\ [c.]\ \ $\bullet$\ \ \setlength\topsep{0pt}\textbf{\foreignlanguage{arabic}{يِنْفِهِم}}\ {\color{gray}\texttt{/\sffamily {{\sffamily jinfihim}}/}\color{black}}\ [i.]\ \color{gray}(msa. \foreignlanguage{arabic}{يُفْهَم}~\foreignlanguage{arabic}{\textbf{١.}})\color{black}\  \begin{flushright}\color{gray}\foreignlanguage{arabic}{\textbf{\underline{\foreignlanguage{arabic}{أمثلة}}}: بنت تشتغل وتعيش لحالها برام الله رح تِنْفِهِم غلط}\end{flushright}\color{black}} \vspace{2mm}

{\setlength\topsep{0pt}\textbf{\foreignlanguage{arabic}{تْفَاهَم}}\ {\color{gray}\texttt{/\sffamily {{\sffamily tfaːham}}/}\color{black}}\ \textsc{verb}\ [p.]\ \textbf{1.}~seek a mutual understanding.  \textbf{2.}~agree upon\ \ $\bullet$\ \ \setlength\topsep{0pt}\textbf{\foreignlanguage{arabic}{اِتْفَاهَم}}\ {\color{gray}\texttt{/\sffamily {{\sffamily ʔitfaːham}}/}\color{black}}\ [c.]\ \ $\bullet$\ \ \setlength\topsep{0pt}\textbf{\foreignlanguage{arabic}{يِتْفَاهَم}}\ {\color{gray}\texttt{/\sffamily {{\sffamily jitfaːham}}/}\color{black}}\ [i.]\ \color{gray}(msa. \foreignlanguage{arabic}{يحاول الوصول الى اتفاق}~\foreignlanguage{arabic}{\textbf{١.}})\color{black}\  \begin{flushright}\color{gray}\foreignlanguage{arabic}{\textbf{\underline{\foreignlanguage{arabic}{أمثلة}}}: خلينا نِتْفاهَم عالسعر بالأول بعدين بشوفها}\end{flushright}\color{black}} \vspace{2mm}

{\setlength\topsep{0pt}\textbf{\foreignlanguage{arabic}{تْفَهَّم}}\ {\color{gray}\texttt{/\sffamily {{\sffamily tfahham}}/}\color{black}}\ \textsc{verb}\ [p.]\ \textbf{1.}~totally understand sth\ \ $\bullet$\ \ \setlength\topsep{0pt}\textbf{\foreignlanguage{arabic}{اِتْفَهَّم}}\ {\color{gray}\texttt{/\sffamily {{\sffamily ʔitfahham}}/}\color{black}}\ [c.]\ \ $\bullet$\ \ \setlength\topsep{0pt}\textbf{\foreignlanguage{arabic}{يِتْفَهَّم}}\ {\color{gray}\texttt{/\sffamily {{\sffamily jitfahham}}/}\color{black}}\ [i.]\ \color{gray}(msa. \foreignlanguage{arabic}{يَتَفَهَّم}~\foreignlanguage{arabic}{\textbf{١.}})\color{black}\  \begin{flushright}\color{gray}\foreignlanguage{arabic}{\textbf{\underline{\foreignlanguage{arabic}{أمثلة}}}: أنا بتْفَهَّم وضعك ووضع أهلك وعاذرتك والله بس الناس شو بيعرفها}\end{flushright}\color{black}} \vspace{2mm}

{\setlength\topsep{0pt}\textbf{\foreignlanguage{arabic}{تْفَهْمَن}}\ {\color{gray}\texttt{/\sffamily {{\sffamily tfahman}}/}\color{black}}\ \textsc{verb}\ [p.]\ \textbf{1.}~pretent to have knowledge.  \textbf{2.}~pretent to be Mr. Know-it-All\ \ $\bullet$\ \ \setlength\topsep{0pt}\textbf{\foreignlanguage{arabic}{اِتْفَهْمَن}}\ {\color{gray}\texttt{/\sffamily {{\sffamily ʔitfahman}}/}\color{black}}\ [c.]\ \ $\bullet$\ \ \setlength\topsep{0pt}\textbf{\foreignlanguage{arabic}{يِتْفَهْمَن}}\ {\color{gray}\texttt{/\sffamily {{\sffamily jitfahman}}/}\color{black}}\ [i.]\  \begin{flushright}\color{gray}\foreignlanguage{arabic}{\textbf{\underline{\foreignlanguage{arabic}{أمثلة}}}: أحلى شي صار بده يِتْفَهْمَن علينا ويفرجينا قديش هو فهما وشاطر بالسيارات}\end{flushright}\color{black}} \vspace{2mm}

{\setlength\topsep{0pt}\textbf{\foreignlanguage{arabic}{فَاهِم}}\ {\color{gray}\texttt{/\sffamily {{\sffamily faːhim}}/}\color{black}}\ \textsc{noun\textunderscore act}\ [m.]\ \textbf{1.}~understanding  \textbf{2.}~being aware.  \textbf{3.}~having knowledge\  \begin{flushright}\color{gray}\foreignlanguage{arabic}{\textbf{\underline{\foreignlanguage{arabic}{أمثلة}}}: أنا مش فاهِم أنت عشو عم تحكي\ $\bullet$\ \  مش فاهِم ولا شي بمادة الرياضيات}\end{flushright}\color{black}} \vspace{2mm}

{\setlength\topsep{0pt}\textbf{\foreignlanguage{arabic}{فَهِيم}}\ {\color{gray}\texttt{/\sffamily {{\sffamily fahiːm}}/}\color{black}}\ \textsc{adj}\ [m.]\ \textbf{1.}~discerning  \textbf{2.}~intelligent\  \begin{flushright}\color{gray}\foreignlanguage{arabic}{\textbf{\underline{\foreignlanguage{arabic}{أمثلة}}}: يا فَهيم مية مرة قلتلك ما تكب الزبار بنسقيه للبهايم اللي زيك يا بهيمة}\end{flushright}\color{black}} \vspace{2mm}

{\setlength\topsep{0pt}\textbf{\foreignlanguage{arabic}{فَهَّم}}\ {\color{gray}\texttt{/\sffamily {{\sffamily fahham}}/}\color{black}}\ \textsc{verb}\ [p.]\ \textbf{1.}~explain sth to sb in order to make it understood\ \ $\bullet$\ \ \setlength\topsep{0pt}\textbf{\foreignlanguage{arabic}{فَهِّم}}\ {\color{gray}\texttt{/\sffamily {{\sffamily fahhim}}/}\color{black}}\ [c.]\ \ $\bullet$\ \ \setlength\topsep{0pt}\textbf{\foreignlanguage{arabic}{يفَهِّم}}\ {\color{gray}\texttt{/\sffamily {{\sffamily jfahhim}}/}\color{black}}\ [i.]\ \color{gray}(msa. \foreignlanguage{arabic}{يَشْرَح}~\foreignlanguage{arabic}{\textbf{١.}})\color{black}\  \begin{flushright}\color{gray}\foreignlanguage{arabic}{\textbf{\underline{\foreignlanguage{arabic}{أمثلة}}}: طب فَهِّمني ليش عملت هيك؟}\end{flushright}\color{black}} \vspace{2mm}

{\setlength\topsep{0pt}\textbf{\foreignlanguage{arabic}{فَهْمَان}}\ {\color{gray}\texttt{/\sffamily {{\sffamily fahmaːn}}/}\color{black}}\ \textsc{adj}\ [m.]\ \textbf{1.}~very wise and understanding\  \begin{flushright}\color{gray}\foreignlanguage{arabic}{\textbf{\underline{\foreignlanguage{arabic}{أمثلة}}}: جوزها فَهْمان أحسن منها بمليون مرة}\end{flushright}\color{black}} \vspace{2mm}

{\setlength\topsep{0pt}\textbf{\foreignlanguage{arabic}{فَهْمَان}}\ {\color{gray}\texttt{/\sffamily {{\sffamily fahmaːn}}/}\color{black}}\ \textsc{noun\textunderscore act}\ [m.]\ \textbf{1.}~understanding  \textbf{2.}~being aware.  \textbf{3.}~having knowledge\  \begin{flushright}\color{gray}\foreignlanguage{arabic}{\textbf{\underline{\foreignlanguage{arabic}{أمثلة}}}: لا أنا فَهْمانة عليك ولا أنت فَهْمان علي}\end{flushright}\color{black}} \vspace{2mm}

{\setlength\topsep{0pt}\textbf{\foreignlanguage{arabic}{فِهِم}}\ {\color{gray}\texttt{/\sffamily {{\sffamily fihim}}/}\color{black}}\ \textsc{noun}\ [m.]\ \color{gray}(msa. \foreignlanguage{arabic}{فِهْم}~\foreignlanguage{arabic}{\textbf{١.}})\color{black}\ \textbf{1.}~understanding\ \ $\bullet$\ \ \textsc{ph.} \color{gray} \foreignlanguage{arabic}{قِلِّة فِهِم}\color{black}\ {\color{gray}\texttt{/{\sffamily (q)illit fihim}/}\color{black}}\ \textbf{1.}~sb who does not act in a way that shows respect towards others\ } \vspace{2mm}

{\setlength\topsep{0pt}\textbf{\foreignlanguage{arabic}{فِهِم}}\ {\color{gray}\texttt{/\sffamily {{\sffamily fihim}}/}\color{black}}\ \textsc{verb}\ [p.]\ \textbf{1.}~understand  \textbf{2.}~become aware.  \textbf{3.}~have knowledge\ \ $\bullet$\ \ \setlength\topsep{0pt}\textbf{\foreignlanguage{arabic}{اِفْهَم}}\ {\color{gray}\texttt{/\sffamily {{\sffamily ʔifham}}/}\color{black}}\ [c.]\ \ $\bullet$\ \ \setlength\topsep{0pt}\textbf{\foreignlanguage{arabic}{يِفْهَم}}\ {\color{gray}\texttt{/\sffamily {{\sffamily jifham}}/}\color{black}}\ [i.]\ \color{gray}(msa. \foreignlanguage{arabic}{يَفْهَم}~\foreignlanguage{arabic}{\textbf{١.}})\color{black}\  \begin{flushright}\color{gray}\foreignlanguage{arabic}{\textbf{\underline{\foreignlanguage{arabic}{أمثلة}}}: بس يِفْهَم آخر درسين بقدر أشرحله المادة الجديدة، اما قبل هيك صعب\ $\bullet$\ \  افْهَم علي شو بحكي. أنا بدي مصلحتك}\end{flushright}\color{black}} \vspace{2mm}

{\setlength\topsep{0pt}\textbf{\foreignlanguage{arabic}{مَفْهُوم}}\ {\color{gray}\texttt{/\sffamily {{\sffamily mafhuːm}}/}\color{black}}\ \textsc{interj}\ \textbf{1.}~Understood!\  \begin{flushright}\color{gray}\foreignlanguage{arabic}{\textbf{\underline{\foreignlanguage{arabic}{أمثلة}}}: أنا رح أروح لحالي واذا مارجعت بعد نص ساعة بتلحقني.مَفهوم!}\end{flushright}\color{black}} \vspace{2mm}

{\setlength\topsep{0pt}\textbf{\foreignlanguage{arabic}{مَفْهُوم}}\ {\color{gray}\texttt{/\sffamily {{\sffamily mafhuːm}}/}\color{black}}\ \textsc{noun}\ [m.]\ \color{gray}(msa. \foreignlanguage{arabic}{مَفهوم}~\foreignlanguage{arabic}{\textbf{١.}})\color{black}\ \textbf{1.}~concept\ \ $\bullet$\ \ \setlength\topsep{0pt}\textbf{\foreignlanguage{arabic}{مَفَاهِيم}}\ {\color{gray}\texttt{/\sffamily {{\sffamily mafaːhiːm}}/}\color{black}}\ [pl.]\  \begin{flushright}\color{gray}\foreignlanguage{arabic}{\textbf{\underline{\foreignlanguage{arabic}{أمثلة}}}: في كثير مَفاهِيم وفِيم زرعوها فينا أجدادنا ولازم نضل متمسكين فيها}\end{flushright}\color{black}} \vspace{2mm}

{\setlength\topsep{0pt}\textbf{\foreignlanguage{arabic}{مُتَفَهِّم}}\ {\color{gray}\texttt{/\sffamily {{\sffamily mutafahhim}}/}\color{black}}\ \textsc{adj}\ [m.]\ \color{gray}(msa. \foreignlanguage{arabic}{مُتَفَهِِّم}~\foreignlanguage{arabic}{\textbf{١.}})\color{black}\ \textbf{1.}~understanding\  \begin{flushright}\color{gray}\foreignlanguage{arabic}{\textbf{\underline{\foreignlanguage{arabic}{أمثلة}}}: الأستاذ مُتَفَهِِّم عادي احكيله رح يوافق}\end{flushright}\color{black}} \vspace{2mm}

\vspace{-3mm}
\markboth{\color{blue}\foreignlanguage{arabic}{ف.ه.ي}\color{blue}{}}{\color{blue}\foreignlanguage{arabic}{ف.ه.ي}\color{blue}{}}\subsection*{\color{blue}\foreignlanguage{arabic}{ف.ه.ي}\color{blue}{}\index{\color{blue}\foreignlanguage{arabic}{ف.ه.ي}\color{blue}{}}} 

{\setlength\topsep{0pt}\textbf{\foreignlanguage{arabic}{فَاهِي}}\ {\color{gray}\texttt{/\sffamily {{\sffamily faːhi}}/}\color{black}}\ \textsc{adj}\ [m.]\ \color{gray}(msa. \foreignlanguage{arabic}{ليس له طعم}~\foreignlanguage{arabic}{\textbf{١.}})\color{black}\ \textbf{1.}~tasteless\  \begin{flushright}\color{gray}\foreignlanguage{arabic}{\textbf{\underline{\foreignlanguage{arabic}{أمثلة}}}: الأكل فاهِي بالمرة ماعجبني}\end{flushright}\color{black}} \vspace{2mm}

{\setlength\topsep{0pt}\textbf{\foreignlanguage{arabic}{فَهَّى}}\ {\color{gray}\texttt{/\sffamily {{\sffamily fahha}}/}\color{black}}\ \textsc{verb}\ [p.]\ \textbf{1.}~gape st sth\ \ $\bullet$\ \ \setlength\topsep{0pt}\textbf{\foreignlanguage{arabic}{فَهِّي}}\ {\color{gray}\texttt{/\sffamily {{\sffamily fahhi}}/}\color{black}}\ [c.]\ \ $\bullet$\ \ \setlength\topsep{0pt}\textbf{\foreignlanguage{arabic}{يفَهِّي}}\ {\color{gray}\texttt{/\sffamily {{\sffamily jfahhi}}/}\color{black}}\ [i.]\  \begin{flushright}\color{gray}\foreignlanguage{arabic}{\textbf{\underline{\foreignlanguage{arabic}{أمثلة}}}: قعدت أشرحله عن القسمة والكسور فبس شفته فَهَّى رميت الكتاب بوجهه وقلتله يجعلك ماتعلمت!}\end{flushright}\color{black}} \vspace{2mm}

{\setlength\topsep{0pt}\textbf{\foreignlanguage{arabic}{مْفَهِّي}}\ {\color{gray}\texttt{/\sffamily {{\sffamily mfahhi}}/}\color{black}}\ \textsc{adj}\ [m.]\ \textbf{1.}~gaping at sth\  \begin{flushright}\color{gray}\foreignlanguage{arabic}{\textbf{\underline{\foreignlanguage{arabic}{أمثلة}}}: مالك مْفَهِّي هيك؟ شو الصعبة فيها احكيلي!}\end{flushright}\color{black}} \vspace{2mm}

\vspace{-3mm}
\markboth{\color{blue}\foreignlanguage{arabic}{ف.و.ت}\color{blue}{}}{\color{blue}\foreignlanguage{arabic}{ف.و.ت}\color{blue}{}}\subsection*{\color{blue}\foreignlanguage{arabic}{ف.و.ت}\color{blue}{}\index{\color{blue}\foreignlanguage{arabic}{ف.و.ت}\color{blue}{}}} 

{\setlength\topsep{0pt}\textbf{\foreignlanguage{arabic}{تَفْوِيت}}\ {\color{gray}\texttt{/\sffamily {{\sffamily tafwiːt}}/}\color{black}}\ \textsc{noun}\ [m.]\ \textbf{1.}~ignoring sth that bothers a person (be heedless of sth)\  \begin{flushright}\color{gray}\foreignlanguage{arabic}{\textbf{\underline{\foreignlanguage{arabic}{أمثلة}}}: بدك تتعلم تفويت كثير قصص ما بتعجبك عشان تقدر تعيش}\end{flushright}\color{black}} \vspace{2mm}

{\setlength\topsep{0pt}\textbf{\foreignlanguage{arabic}{فَات}}\ {\color{gray}\texttt{/\sffamily {{\sffamily faːt}}/}\color{black}}\ \textsc{verb}\ [p.]\ \textbf{1.}~enter  \textbf{2.}~miss sth (sth escapes a person)\ \ $\bullet$\ \ \setlength\topsep{0pt}\textbf{\foreignlanguage{arabic}{فُوت}}\ {\color{gray}\texttt{/\sffamily {{\sffamily fuːt}}/}\color{black}}\ [c.]\ \ $\bullet$\ \ \setlength\topsep{0pt}\textbf{\foreignlanguage{arabic}{يفُوت}}\ {\color{gray}\texttt{/\sffamily {{\sffamily jfuːt}}/}\color{black}}\ [i.]\ \color{gray}(msa. \foreignlanguage{arabic}{يَفوت}~\foreignlanguage{arabic}{\textbf{٢.}}  \foreignlanguage{arabic}{يَدْخُل}~\foreignlanguage{arabic}{\textbf{١.}})\color{black}\ \ $\bullet$\ \ \textsc{ph.} \color{gray} \foreignlanguage{arabic}{فتك بَالحكي}\color{black}\ {\color{gray}\texttt{/{\sffamily futtak bilħaki}/}\color{black}}\ \textbf{1.}~I forgot to mention sth about X\  \begin{flushright}\color{gray}\foreignlanguage{arabic}{\textbf{\underline{\foreignlanguage{arabic}{أمثلة}}}: فُتَّك بالحَكِي آه وكان عنده 3 بنات مثل القمر من مرته الأولانية. المهم نرجع للمرة الثانية .............\ $\bullet$\ \  فوت أهلا وسهلا\ $\bullet$\ \  فاتَتني هاي كيف ما أخذت بالي انهم اربعة}\end{flushright}\color{black}} \vspace{2mm}

{\setlength\topsep{0pt}\textbf{\foreignlanguage{arabic}{فَايِت}}\ {\color{gray}\texttt{/\sffamily {{\sffamily faːjit}}/}\color{black}}\ \textsc{noun\textunderscore act}\ \textbf{1.}~entering\  \begin{flushright}\color{gray}\foreignlanguage{arabic}{\textbf{\underline{\foreignlanguage{arabic}{أمثلة}}}: أنا فايِت عدارهم وماكل فيها}\end{flushright}\color{black}} \vspace{2mm}

{\setlength\topsep{0pt}\textbf{\foreignlanguage{arabic}{فَوتِة}}\ {\color{gray}\texttt{/\sffamily {{\sffamily foːte}}/}\color{black}}\ \textsc{noun}\ [f.]\ \textbf{1.}~adventure  \textbf{2.}~entering  \textbf{3.}~situation\  \begin{flushright}\color{gray}\foreignlanguage{arabic}{\textbf{\underline{\foreignlanguage{arabic}{أمثلة}}}: والله هاي فوتتي عالدار لسة ملحقتش أغير أواعِيي\ $\bullet$\ \  أنت مش قد هالفوتِة اسمع مني}\end{flushright}\color{black}} \vspace{2mm}

{\setlength\topsep{0pt}\textbf{\foreignlanguage{arabic}{فَوَّت}}\ {\color{gray}\texttt{/\sffamily {{\sffamily fawwat}}/}\color{black}}\ \textsc{verb}\ [p.]\ \textbf{1.}~make sb enter (causative).  \textbf{2.}~miss sth (intentionally).  \textbf{3.}~ignore sth that bothers a person (be heedless of sth)\ \ $\bullet$\ \ \setlength\topsep{0pt}\textbf{\foreignlanguage{arabic}{فَوِّت}}\ {\color{gray}\texttt{/\sffamily {{\sffamily fawwit}}/}\color{black}}\ [c.]\ \ $\bullet$\ \ \setlength\topsep{0pt}\textbf{\foreignlanguage{arabic}{يفَوِّت}}\ {\color{gray}\texttt{/\sffamily {{\sffamily jfawwit}}/}\color{black}}\ [i.]\  \begin{flushright}\color{gray}\foreignlanguage{arabic}{\textbf{\underline{\foreignlanguage{arabic}{أمثلة}}}: أوعك تفَوِّت الفرصة\ $\bullet$\ \  يازلمة فَوِّت وتدقرش عكل شي ولا بتتعب كثير\ $\bullet$\ \  هو فَوَّتني عداره عشان خاطر أبوي بس}\end{flushright}\color{black}} \vspace{2mm}

\vspace{-3mm}
\markboth{\color{blue}\foreignlanguage{arabic}{ف.و.ج}\color{blue}{}}{\color{blue}\foreignlanguage{arabic}{ف.و.ج}\color{blue}{}}\subsection*{\color{blue}\foreignlanguage{arabic}{ف.و.ج}\color{blue}{}\index{\color{blue}\foreignlanguage{arabic}{ف.و.ج}\color{blue}{}}} 

{\setlength\topsep{0pt}\textbf{\foreignlanguage{arabic}{فَوج}}\ {\color{gray}\texttt{/\sffamily {{\sffamily foː(dʒ)}}/}\color{black}}\ \textsc{noun}\ [m.]\ \color{gray}(msa. \foreignlanguage{arabic}{فَوْج}~\foreignlanguage{arabic}{\textbf{١.}})\color{black}\ \textbf{1.}~battalion  \textbf{2.}~regiment\ \ $\bullet$\ \ \setlength\topsep{0pt}\textbf{\foreignlanguage{arabic}{أَفْوَاج}}\ {\color{gray}\texttt{/\sffamily {{\sffamily ʔafwaː(dʒ)}}/}\color{black}}\ [pl.]\  \begin{flushright}\color{gray}\foreignlanguage{arabic}{\textbf{\underline{\foreignlanguage{arabic}{أمثلة}}}: خرَّجت الجامعة اليوم أول فُوج من تخصص الهندسة الزراعية}\end{flushright}\color{black}} \vspace{2mm}

\vspace{-3mm}
\markboth{\color{blue}\foreignlanguage{arabic}{ف.و.ح}\color{blue}{}}{\color{blue}\foreignlanguage{arabic}{ف.و.ح}\color{blue}{}}\subsection*{\color{blue}\foreignlanguage{arabic}{ف.و.ح}\color{blue}{}\index{\color{blue}\foreignlanguage{arabic}{ف.و.ح}\color{blue}{}}} 

{\setlength\topsep{0pt}\textbf{\foreignlanguage{arabic}{فَاح}}\ {\color{gray}\texttt{/\sffamily {{\sffamily faːħ}}/}\color{black}}\ \textsc{verb}\ [p.]\ \textbf{1.}~spread odour\ \ $\bullet$\ \ \setlength\topsep{0pt}\textbf{\foreignlanguage{arabic}{فُوح}}\ {\color{gray}\texttt{/\sffamily {{\sffamily fuːħ}}/}\color{black}}\ [c.]\ \ $\bullet$\ \ \setlength\topsep{0pt}\textbf{\foreignlanguage{arabic}{يفُوح}}\ {\color{gray}\texttt{/\sffamily {{\sffamily jfuːħ}}/}\color{black}}\ [i.]\ \color{gray}(msa. \foreignlanguage{arabic}{يَفوح}~\foreignlanguage{arabic}{\textbf{١.}})\color{black}\  \begin{flushright}\color{gray}\foreignlanguage{arabic}{\textbf{\underline{\foreignlanguage{arabic}{أمثلة}}}: خايفة تفوح ريحته}\end{flushright}\color{black}} \vspace{2mm}

{\setlength\topsep{0pt}\textbf{\foreignlanguage{arabic}{فَايِح}}\ {\color{gray}\texttt{/\sffamily {{\sffamily faːjiħ}}/}\color{black}}\ \textsc{adj}\ [m.]\ \textbf{1.}~spreading odour\  \begin{flushright}\color{gray}\foreignlanguage{arabic}{\textbf{\underline{\foreignlanguage{arabic}{أمثلة}}}: ريحة عطرها فايِحَة}\end{flushright}\color{black}} \vspace{2mm}

{\setlength\topsep{0pt}\textbf{\foreignlanguage{arabic}{فَوَّاحَة}}\ {\color{gray}\texttt{/\sffamily {{\sffamily fawwaːħa}}/}\color{black}}\ \textsc{noun}\ [f.]\ \color{gray}(msa. \foreignlanguage{arabic}{شباك صغير أعلى الغرفة}~\foreignlanguage{arabic}{\textbf{١.}})\color{black}\ \textbf{1.}~a small window on top of a room\  \begin{flushright}\color{gray}\foreignlanguage{arabic}{\textbf{\underline{\foreignlanguage{arabic}{أمثلة}}}: سكري الفواحة في هوا بارد بدخل}\end{flushright}\color{black}} \vspace{2mm}

{\setlength\topsep{0pt}\textbf{\foreignlanguage{arabic}{فَوَّح}}\ {\color{gray}\texttt{/\sffamily {{\sffamily fawwaħ}}/}\color{black}}\ \textsc{verb}\ [p.]\ \textbf{1.}~spread odour (intensively)\ \ $\bullet$\ \ \setlength\topsep{0pt}\textbf{\foreignlanguage{arabic}{فَوِّح}}\ {\color{gray}\texttt{/\sffamily {{\sffamily fawwiħ}}/}\color{black}}\ [c.]\ \ $\bullet$\ \ \setlength\topsep{0pt}\textbf{\foreignlanguage{arabic}{يفَوِّح}}\ {\color{gray}\texttt{/\sffamily {{\sffamily jfawwiħ}}/}\color{black}}\ [i.]\  \begin{flushright}\color{gray}\foreignlanguage{arabic}{\textbf{\underline{\foreignlanguage{arabic}{أمثلة}}}: حطيت عليها مسك عشان تفَوِّح ريحة حلوة قبل مايجوا الضيوف}\end{flushright}\color{black}} \vspace{2mm}

\vspace{-3mm}
\markboth{\color{blue}\foreignlanguage{arabic}{ف.و.د.س}\color{blue}{}}{\color{blue}\foreignlanguage{arabic}{ف.و.د.س}\color{blue}{}}\subsection*{\color{blue}\foreignlanguage{arabic}{ف.و.د.س}\color{blue}{}\index{\color{blue}\foreignlanguage{arabic}{ف.و.د.س}\color{blue}{}}} 

{\setlength\topsep{0pt}\textbf{\foreignlanguage{arabic}{فَودِس}}\ {\color{gray}\texttt{/\sffamily {{\sffamily foːdis}}/}\color{black}}\ \textsc{verb}\ [c.]\ \textbf{1.}~take a day off.  \textbf{2.}~on break\ \ $\bullet$\ \ \setlength\topsep{0pt}\textbf{\foreignlanguage{arabic}{يفَودِس}}\ {\color{gray}\texttt{/\sffamily {{\sffamily jfoːdis}}/}\color{black}}\ [i.]\ \color{gray}(msa. \foreignlanguage{arabic}{يأخذ إِجازة}~\foreignlanguage{arabic}{\textbf{١.}})\color{black}\  \begin{flushright}\color{gray}\foreignlanguage{arabic}{\textbf{\underline{\foreignlanguage{arabic}{أمثلة}}}: جاي عبالي أفَودِس اليوم وبكرة ويجعل ما حدا اشتغل}\end{flushright}\color{black}} \vspace{2mm}

\vspace{-3mm}
\markboth{\color{blue}\foreignlanguage{arabic}{ف.و.د.س}\color{blue}{ (ntws)}}{\color{blue}\foreignlanguage{arabic}{ف.و.د.س}\color{blue}{ (ntws)}}\subsection*{\color{blue}\foreignlanguage{arabic}{ف.و.د.س}\color{blue}{ (ntws)}\index{\color{blue}\foreignlanguage{arabic}{ف.و.د.س}\color{blue}{ (ntws)}}} 

{\setlength\topsep{0pt}\textbf{\foreignlanguage{arabic}{فَودَاس}}\ {\color{gray}\texttt{/\sffamily {{\sffamily foːdaːs}}/}\color{black}}\ \textsc{noun}\ [m.]\ \color{gray}(msa. \foreignlanguage{arabic}{إِجازة}~\foreignlanguage{arabic}{\textbf{٢.}}  \foreignlanguage{arabic}{عطلة}~\foreignlanguage{arabic}{\textbf{١.}})\color{black}\ \textbf{1.}~holiday  \textbf{2.}~off-duty\  \begin{flushright}\color{gray}\foreignlanguage{arabic}{\textbf{\underline{\foreignlanguage{arabic}{أمثلة}}}: أعطونا فُوداس عشان عيد العمال}\end{flushright}\color{black}} \vspace{2mm}

{\setlength\topsep{0pt}\textbf{\foreignlanguage{arabic}{فَودَس}}\ {\color{gray}\texttt{/\sffamily {{\sffamily foːdas}}/}\color{black}}\ \textsc{verb}\ [p.]\ \textbf{1.}~take a day off.  \textbf{2.}~on break\  \begin{flushright}\color{gray}\foreignlanguage{arabic}{\textbf{\underline{\foreignlanguage{arabic}{أمثلة}}}: فودَسنا امبارح شو الك عندي}\end{flushright}\color{black}} \vspace{2mm}

{\setlength\topsep{0pt}\textbf{\foreignlanguage{arabic}{مْفَودِس}}\ {\color{gray}\texttt{/\sffamily {{\sffamily mfoːdis}}/}\color{black}}\ \textsc{noun\textunderscore act}\ [m.]\ \color{gray}(msa. \foreignlanguage{arabic}{بإِجازة}~\foreignlanguage{arabic}{\textbf{٢.}}  .\foreignlanguage{arabic}{خارج موعد العمل}~\foreignlanguage{arabic}{\textbf{١.}})\color{black}\ \textbf{1.}~off-duty\  \begin{flushright}\color{gray}\foreignlanguage{arabic}{\textbf{\underline{\foreignlanguage{arabic}{أمثلة}}}: أنا مْفودْسِة اليوم ما حدا يحكي معي}\end{flushright}\color{black}} \vspace{2mm}

\vspace{-3mm}
\markboth{\color{blue}\foreignlanguage{arabic}{ف.و.ر}\color{blue}{}}{\color{blue}\foreignlanguage{arabic}{ف.و.ر}\color{blue}{}}\subsection*{\color{blue}\foreignlanguage{arabic}{ف.و.ر}\color{blue}{}\index{\color{blue}\foreignlanguage{arabic}{ف.و.ر}\color{blue}{}}} 

{\setlength\topsep{0pt}\textbf{\foreignlanguage{arabic}{فَار}}\ {\color{gray}\texttt{/\sffamily {{\sffamily faːr}}/}\color{black}}\ \textsc{verb}\ [p.]\ \textbf{1.}~boil (shorter time)\ \ $\bullet$\ \ \setlength\topsep{0pt}\textbf{\foreignlanguage{arabic}{فُور}}\ {\color{gray}\texttt{/\sffamily {{\sffamily fuːr}}/}\color{black}}\ [c.]\ \ $\bullet$\ \ \setlength\topsep{0pt}\textbf{\foreignlanguage{arabic}{يفُور}}\ {\color{gray}\texttt{/\sffamily {{\sffamily jfuːr}}/}\color{black}}\ [i.]\  \begin{flushright}\color{gray}\foreignlanguage{arabic}{\textbf{\underline{\foreignlanguage{arabic}{أمثلة}}}: خليه يسخن بس بديش إِياه يفُور عالنار\ $\bullet$\ \  فارت القهوة ولا بعدها؟}\end{flushright}\color{black}} \vspace{2mm}

{\setlength\topsep{0pt}\textbf{\foreignlanguage{arabic}{فَور}}\ {\color{gray}\texttt{/\sffamily {{\sffamily foːr}}/}\color{black}}\ \textsc{noun}\ [m.]\ \textbf{1.}~immediately  \textbf{2.}~at once\ } \vspace{2mm}

{\setlength\topsep{0pt}\textbf{\foreignlanguage{arabic}{فَورَة}}\ {\color{gray}\texttt{/\sffamily {{\sffamily foːra}}/}\color{black}}\ \textsc{noun}\ [f.]\ \color{gray}(msa. \foreignlanguage{arabic}{نزهة السجناء اليومية خارج زنزاناتهم و داخل أسوار السجن}~\foreignlanguage{arabic}{\textbf{١.}})\color{black}\ \textbf{1.}~outdoor recreational activity for inmates\ \ $\bullet$\ \ \textsc{ph.} \color{gray} \foreignlanguage{arabic}{فَورَة دَمّ}\color{black}\ {\color{gray}\texttt{/{\sffamily foːrat damm}/}\color{black}}\ \color{gray} (msa. \foreignlanguage{arabic}{فورَة غضب}~\foreignlanguage{arabic}{\textbf{١.}})\color{black}\ \textbf{1.}~an upsurge of rage\ \ $\bullet$\ \ \textsc{ph.} \color{gray} \foreignlanguage{arabic}{فَورَة دَمّ}\color{black}\ {\color{gray}\texttt{/{\sffamily foːrat damm}/}\color{black}}\ \textbf{1.}~when the all the family members of the murderer disown their son publically in front of the village dwellers. It usually happens after three days of the killing.\  \begin{flushright}\color{gray}\foreignlanguage{arabic}{\textbf{\underline{\foreignlanguage{arabic}{أمثلة}}}: الموضوع كله عبعضه فُورِة دَم الله يخزيك يا شيطان\ $\bullet$\ \  بقى عنا فُورَة كل يوم مدتها نص ساعة نقضيها نلعب طرنيب مع الشباب}\end{flushright}\color{black}} \vspace{2mm}

{\setlength\topsep{0pt}\textbf{\foreignlanguage{arabic}{فَوَارِي}}\ {\color{gray}\texttt{/\sffamily {{\sffamily fawaːri}}/}\color{black}}\ \textsc{noun}\ [m.]\ \color{gray}(msa. \foreignlanguage{arabic}{فأس صغير لتقطيع الخشب}~\foreignlanguage{arabic}{\textbf{١.}})\color{black}\ \textbf{1.}~a small axe to chop wood\ } \vspace{2mm}

{\setlength\topsep{0pt}\textbf{\foreignlanguage{arabic}{فَوَّر}}\ {\color{gray}\texttt{/\sffamily {{\sffamily fawwar}}/}\color{black}}\ \textsc{verb}\ [p.]\ \textbf{1.}~boil  \textbf{2.}~make sth boil (intensely)\ \ $\bullet$\ \ \setlength\topsep{0pt}\textbf{\foreignlanguage{arabic}{فَوِّر}}\ {\color{gray}\texttt{/\sffamily {{\sffamily fawwir}}/}\color{black}}\ [c.]\ \ $\bullet$\ \ \setlength\topsep{0pt}\textbf{\foreignlanguage{arabic}{يفَوِّر}}\ {\color{gray}\texttt{/\sffamily {{\sffamily jfawwir}}/}\color{black}}\ [i.]\ \ $\bullet$\ \ \textsc{ph.} \color{gray} \foreignlanguage{arabic}{فَوَّر لي دمي}\color{black}\ {\color{gray}\texttt{/{\sffamily fawwar li dammi}/}\color{black}}\ \color{gray} (msa. \foreignlanguage{arabic}{يُغْضِب شَخْص}~\foreignlanguage{arabic}{\textbf{١.}})\color{black}\ \textbf{1.}~enrage sb\  \begin{flushright}\color{gray}\foreignlanguage{arabic}{\textbf{\underline{\foreignlanguage{arabic}{أمثلة}}}: فَوَّر لي دَمِّي هالحيوان\ $\bullet$\ \  استنى لحديت ما القهوة تفوِّر عالنار}\end{flushright}\color{black}} \vspace{2mm}

{\setlength\topsep{0pt}\textbf{\foreignlanguage{arabic}{مْفَوَّر}}\ {\color{gray}\texttt{/\sffamily {{\sffamily mfawwir}}/}\color{black}}\ \textsc{adj}\ [m.]\ \textbf{1.}~boiling\ } \vspace{2mm}

{\setlength\topsep{0pt}\textbf{\foreignlanguage{arabic}{مْفَوَّرَة}}\ {\color{gray}\texttt{/\sffamily {{\sffamily mfawwara}}/}\color{black}}\ \textsc{noun}\ [f.]\ \textbf{1.}~it is a traditional dish that is made from yoghurt, onions and olive oil that are cooked together until the yoghurt boils. It is usually eaten with Yufka.\ } \vspace{2mm}

\vspace{-3mm}
\markboth{\color{blue}\foreignlanguage{arabic}{ف.و.ز}\color{blue}{}}{\color{blue}\foreignlanguage{arabic}{ف.و.ز}\color{blue}{}}\subsection*{\color{blue}\foreignlanguage{arabic}{ف.و.ز}\color{blue}{}\index{\color{blue}\foreignlanguage{arabic}{ف.و.ز}\color{blue}{}}} 

{\setlength\topsep{0pt}\textbf{\foreignlanguage{arabic}{فَاز}}\ {\color{gray}\texttt{/\sffamily {{\sffamily faːz}}/}\color{black}}\ \textsc{verb}\ [p.]\ \textbf{1.}~win\ \ $\bullet$\ \ \setlength\topsep{0pt}\textbf{\foreignlanguage{arabic}{فُوز}}\ {\color{gray}\texttt{/\sffamily {{\sffamily fuːz}}/}\color{black}}\ [c.]\ \ $\bullet$\ \ \setlength\topsep{0pt}\textbf{\foreignlanguage{arabic}{يفُوز}}\ {\color{gray}\texttt{/\sffamily {{\sffamily jfuːz}}/}\color{black}}\ [i.]\ \color{gray}(msa. \foreignlanguage{arabic}{يَفُوز}~\foreignlanguage{arabic}{\textbf{١.}})\color{black}\  \begin{flushright}\color{gray}\foreignlanguage{arabic}{\textbf{\underline{\foreignlanguage{arabic}{أمثلة}}}: إِحنا اللي فُزْنا مش همي}\end{flushright}\color{black}} \vspace{2mm}

{\setlength\topsep{0pt}\textbf{\foreignlanguage{arabic}{فَايِز}}\ {\color{gray}\texttt{/\sffamily {{\sffamily faːjiz}}/}\color{black}}\ \textsc{noun\textunderscore act}\ [m.]\ \textbf{1.}~winning\  \begin{flushright}\color{gray}\foreignlanguage{arabic}{\textbf{\underline{\foreignlanguage{arabic}{أمثلة}}}: إِحنا فايزين عليهم بواحد صفر}\end{flushright}\color{black}} \vspace{2mm}

{\setlength\topsep{0pt}\textbf{\foreignlanguage{arabic}{فَوز}}\ {\color{gray}\texttt{/\sffamily {{\sffamily foːz}}/}\color{black}}\ \textsc{noun}\ [m.]\ \color{gray}(msa. \foreignlanguage{arabic}{فَوْز}~\foreignlanguage{arabic}{\textbf{١.}})\color{black}\ \textbf{1.}~victory\  \begin{flushright}\color{gray}\foreignlanguage{arabic}{\textbf{\underline{\foreignlanguage{arabic}{أمثلة}}}: ألف مبروك فُوزكم على فريق مخيم بلاطة}\end{flushright}\color{black}} \vspace{2mm}

{\setlength\topsep{0pt}\textbf{\foreignlanguage{arabic}{فَوَّز}}\ {\color{gray}\texttt{/\sffamily {{\sffamily fawwaz}}/}\color{black}}\ \textsc{verb}\ [p.]\ \textbf{1.}~make sb win (causative)\ \ $\bullet$\ \ \setlength\topsep{0pt}\textbf{\foreignlanguage{arabic}{فَوِّز}}\ {\color{gray}\texttt{/\sffamily {{\sffamily fawwiz}}/}\color{black}}\ [c.]\ \ $\bullet$\ \ \setlength\topsep{0pt}\textbf{\foreignlanguage{arabic}{يفَوِّز}}\ {\color{gray}\texttt{/\sffamily {{\sffamily jfawwiz}}/}\color{black}}\ [i.]\  \begin{flushright}\color{gray}\foreignlanguage{arabic}{\textbf{\underline{\foreignlanguage{arabic}{أمثلة}}}: من شان الله خليه يفَوِّزني بالمسابقة}\end{flushright}\color{black}} \vspace{2mm}

\vspace{-3mm}
\markboth{\color{blue}\foreignlanguage{arabic}{ف.و.ش}\color{blue}{}}{\color{blue}\foreignlanguage{arabic}{ف.و.ش}\color{blue}{}}\subsection*{\color{blue}\foreignlanguage{arabic}{ف.و.ش}\color{blue}{}\index{\color{blue}\foreignlanguage{arabic}{ف.و.ش}\color{blue}{}}} 

{\setlength\topsep{0pt}\textbf{\foreignlanguage{arabic}{فَاش}}\ {\color{gray}\texttt{/\sffamily {{\sffamily faːʃ}}/}\color{black}}\ \textsc{verb}\ [p.]\ \textbf{1.}~float\ \ $\bullet$\ \ \setlength\topsep{0pt}\textbf{\foreignlanguage{arabic}{فُوش}}\ {\color{gray}\texttt{/\sffamily {{\sffamily fuːʃ}}/}\color{black}}\ [c.]\ \ $\bullet$\ \ \setlength\topsep{0pt}\textbf{\foreignlanguage{arabic}{يفُوش}}\ {\color{gray}\texttt{/\sffamily {{\sffamily jfuːʃ}}/}\color{black}}\ [i.]\ \color{gray}(msa. \foreignlanguage{arabic}{يَطْفو}~\foreignlanguage{arabic}{\textbf{١.}})\color{black}\  \begin{flushright}\color{gray}\foreignlanguage{arabic}{\textbf{\underline{\foreignlanguage{arabic}{أمثلة}}}: ماعرفش يفوش أبداً}\end{flushright}\color{black}} \vspace{2mm}

{\setlength\topsep{0pt}\textbf{\foreignlanguage{arabic}{فَايِش}}\ {\color{gray}\texttt{/\sffamily {{\sffamily faːjiʃ}}/}\color{black}}\ \textsc{adj}\ [m.]\ \textbf{1.}~floating\  \begin{flushright}\color{gray}\foreignlanguage{arabic}{\textbf{\underline{\foreignlanguage{arabic}{أمثلة}}}: اطلع عالبحر في شي فايِش هناك}\end{flushright}\color{black}} \vspace{2mm}

{\setlength\topsep{0pt}\textbf{\foreignlanguage{arabic}{فَوَّاشِة}}\ {\color{gray}\texttt{/\sffamily {{\sffamily fawwaːʃe}}/}\color{black}}\ \textsc{noun}\ [f.]\ \textbf{1.}~float vest\  \begin{flushright}\color{gray}\foreignlanguage{arabic}{\textbf{\underline{\foreignlanguage{arabic}{أمثلة}}}: ييي عاليهود! لابس فَوّاشات زي الولاد الصغار!}\end{flushright}\color{black}} \vspace{2mm}

\vspace{-3mm}
\markboth{\color{blue}\foreignlanguage{arabic}{ف.و.ض}\color{blue}{}}{\color{blue}\foreignlanguage{arabic}{ف.و.ض}\color{blue}{}}\subsection*{\color{blue}\foreignlanguage{arabic}{ف.و.ض}\color{blue}{}\index{\color{blue}\foreignlanguage{arabic}{ف.و.ض}\color{blue}{}}} 

{\setlength\topsep{0pt}\textbf{\foreignlanguage{arabic}{تْفَاوَض}}\ {\color{gray}\texttt{/\sffamily {{\sffamily tfaːwa(dˤ)}}/}\color{black}}\ \textsc{verb}\ [p.]\ \textbf{1.}~negotiate with.  \textbf{2.}~parley with\ \ $\bullet$\ \ \setlength\topsep{0pt}\textbf{\foreignlanguage{arabic}{اِتْفَاوَض}}\ {\color{gray}\texttt{/\sffamily {{\sffamily ʔitfaːwa(dˤ)}}/}\color{black}}\ [c.]\ \ $\bullet$\ \ \setlength\topsep{0pt}\textbf{\foreignlanguage{arabic}{يِتْفَاوَض}}\ {\color{gray}\texttt{/\sffamily {{\sffamily jitfaːwa(dˤ)}}/}\color{black}}\ [i.]\  \begin{flushright}\color{gray}\foreignlanguage{arabic}{\textbf{\underline{\foreignlanguage{arabic}{أمثلة}}}: خلينا نِتفاوَض عالأسعار من هلا}\end{flushright}\color{black}} \vspace{2mm}

{\setlength\topsep{0pt}\textbf{\foreignlanguage{arabic}{فَاوَض}}\ {\color{gray}\texttt{/\sffamily {{\sffamily faːwa(dˤ)}}/}\color{black}}\ \textsc{verb}\ [p.]\ \textbf{1.}~negotiate with.  \textbf{2.}~parley with\ \ $\bullet$\ \ \setlength\topsep{0pt}\textbf{\foreignlanguage{arabic}{فَاوِض}}\ {\color{gray}\texttt{/\sffamily {{\sffamily faːwi(dˤ)}}/}\color{black}}\ [c.]\ \ $\bullet$\ \ \setlength\topsep{0pt}\textbf{\foreignlanguage{arabic}{يفَاوِض}}\ {\color{gray}\texttt{/\sffamily {{\sffamily jfaːwi(dˤ)}}/}\color{black}}\ [i.]\ } \vspace{2mm}

{\setlength\topsep{0pt}\textbf{\foreignlanguage{arabic}{فَوْضَوِي}}\ {\color{gray}\texttt{/\sffamily {{\sffamily faw(dˤ)awi}}/}\color{black}}\ \textsc{adj}\ [m.]\ \textbf{1.}~chaotic\  \begin{flushright}\color{gray}\foreignlanguage{arabic}{\textbf{\underline{\foreignlanguage{arabic}{أمثلة}}}: أنت بني آدم فَوْضَوِي}\end{flushright}\color{black}} \vspace{2mm}

{\setlength\topsep{0pt}\textbf{\foreignlanguage{arabic}{فَوْضَى}}\ {\color{gray}\texttt{/\sffamily {{\sffamily faw(dˤ)a}}/}\color{black}}\ \textsc{noun}\ [f.]\ \textbf{1.}~chaos  \textbf{2.}~anarchy\ } \vspace{2mm}

{\setlength\topsep{0pt}\textbf{\foreignlanguage{arabic}{مُفَاوَضَة}}\ {\color{gray}\texttt{/\sffamily {{\sffamily mufaːwa(dˤ)a}}/}\color{black}}\ \textsc{noun}\ [f.]\ \textbf{1.}~negotiation  \textbf{2.}~discussion  \textbf{3.}~talk\ } \vspace{2mm}

\vspace{-3mm}
\markboth{\color{blue}\foreignlanguage{arabic}{ف.و.ط}\color{blue}{}}{\color{blue}\foreignlanguage{arabic}{ف.و.ط}\color{blue}{}}\subsection*{\color{blue}\foreignlanguage{arabic}{ف.و.ط}\color{blue}{}\index{\color{blue}\foreignlanguage{arabic}{ف.و.ط}\color{blue}{}}} 

{\setlength\topsep{0pt}\textbf{\foreignlanguage{arabic}{تْفَوَّط}}\ {\color{gray}\texttt{/\sffamily {{\sffamily tfawwatˤ}}/}\color{black}}\ \textsc{verb}\ [p.]\ \textbf{1.}~wear a diaper\ \ $\bullet$\ \ \setlength\topsep{0pt}\textbf{\foreignlanguage{arabic}{اِتْفَوَّط}}\ {\color{gray}\texttt{/\sffamily {{\sffamily ʔitfawwatˤ}}/}\color{black}}\ [c.]\ \ $\bullet$\ \ \setlength\topsep{0pt}\textbf{\foreignlanguage{arabic}{يِتْفَوَّط}}\ {\color{gray}\texttt{/\sffamily {{\sffamily jitfawwatˤ}}/}\color{black}}\ [i.]\ \color{gray}(msa. \foreignlanguage{arabic}{يرتدي حفّاظة}~\foreignlanguage{arabic}{\textbf{١.}})\color{black}\  \begin{flushright}\color{gray}\foreignlanguage{arabic}{\textbf{\underline{\foreignlanguage{arabic}{أمثلة}}}: عمرك 7 سنين ولساتك بتِتْفَوَّط؟}\end{flushright}\color{black}} \vspace{2mm}

{\setlength\topsep{0pt}\textbf{\foreignlanguage{arabic}{فَوَّط}}\ {\color{gray}\texttt{/\sffamily {{\sffamily fawwatˤ}}/}\color{black}}\ \textsc{verb}\ [p.]\ \textbf{1.}~make sb wear a diaper\ \ $\bullet$\ \ \setlength\topsep{0pt}\textbf{\foreignlanguage{arabic}{فَوِّط}}\ {\color{gray}\texttt{/\sffamily {{\sffamily fawwitˤ}}/}\color{black}}\ [c.]\ \ $\bullet$\ \ \setlength\topsep{0pt}\textbf{\foreignlanguage{arabic}{يفَوِّط}}\ {\color{gray}\texttt{/\sffamily {{\sffamily jfawwitˤ}}/}\color{black}}\ [i.]\ \color{gray}(msa. \foreignlanguage{arabic}{يجعل شخص يرتدي حفّاظة}~\foreignlanguage{arabic}{\textbf{١.}})\color{black}\  \begin{flushright}\color{gray}\foreignlanguage{arabic}{\textbf{\underline{\foreignlanguage{arabic}{أمثلة}}}: إِذا بدك ترتاحي من غلبة التنظيف تبع كل يوم نصيحة فَوطيه بس يي ينام}\end{flushright}\color{black}} \vspace{2mm}

{\setlength\topsep{0pt}\textbf{\foreignlanguage{arabic}{فُوطَة}}\ {\color{gray}\texttt{/\sffamily {{\sffamily fuːtˤa}}/}\color{black}}\ \textsc{noun}\ [f.]\ \color{gray}(msa. \foreignlanguage{arabic}{قطعة قماش تستخدم للمسح}~\foreignlanguage{arabic}{\textbf{٢.}}  \foreignlanguage{arabic}{حفّاظة}~\foreignlanguage{arabic}{\textbf{١.}})\color{black}\ \textbf{1.}~diaper  \textbf{2.}~wiper rag\ \ $\bullet$\ \ \setlength\topsep{0pt}\textbf{\foreignlanguage{arabic}{فوَط}}\ {\color{gray}\texttt{/\sffamily {{\sffamily fuwatˤ}}/}\color{black}}\ [pl.]\  \begin{flushright}\color{gray}\foreignlanguage{arabic}{\textbf{\underline{\foreignlanguage{arabic}{أمثلة}}}: أخمد مش ملحق عولاده مصاريف حليب وفوَط\ $\bullet$\ \  بتجيبي فُوطَة وبتبليها بمي سخنة وبتمسحي حوالين الوسخ شوي شوي}\end{flushright}\color{black}} \vspace{2mm}

\vspace{-3mm}
\markboth{\color{blue}\foreignlanguage{arabic}{ف.و.ق}\color{blue}{}}{\color{blue}\foreignlanguage{arabic}{ف.و.ق}\color{blue}{}}\subsection*{\color{blue}\foreignlanguage{arabic}{ف.و.ق}\color{blue}{}\index{\color{blue}\foreignlanguage{arabic}{ف.و.ق}\color{blue}{}}} 

{\setlength\topsep{0pt}\textbf{\foreignlanguage{arabic}{تَفَوُّق}}\ {\color{gray}\texttt{/\sffamily {{\sffamily tafawwuq}}/}\color{black}}\ \textsc{noun}\ [m.]\ \color{gray}(msa. \foreignlanguage{arabic}{تَفَوُّق}~\foreignlanguage{arabic}{\textbf{١.}})\color{black}\ \textbf{1.}~excellence\  \begin{flushright}\color{gray}\foreignlanguage{arabic}{\textbf{\underline{\foreignlanguage{arabic}{أمثلة}}}: بتمنالك كل التّفَوُّق والنجاح بحياتك}\end{flushright}\color{black}} \vspace{2mm}

{\setlength\topsep{0pt}\textbf{\foreignlanguage{arabic}{تْفَوَّق}}\ {\color{gray}\texttt{/\sffamily {{\sffamily tfawwaq}}/}\color{black}}\ \textsc{verb}\ [p.]\ \textbf{1.}~excel  \textbf{2.}~outperform\ \ $\bullet$\ \ \setlength\topsep{0pt}\textbf{\foreignlanguage{arabic}{اِتْفَوَّق}}\ {\color{gray}\texttt{/\sffamily {{\sffamily ʔitfawwaq}}/}\color{black}}\ [c.]\ \ $\bullet$\ \ \setlength\topsep{0pt}\textbf{\foreignlanguage{arabic}{يِتْفَوَّق}}\ {\color{gray}\texttt{/\sffamily {{\sffamily jitfawwaq}}/}\color{black}}\ [i.]\ \color{gray}(msa. \foreignlanguage{arabic}{يتميَّز}~\foreignlanguage{arabic}{\textbf{٢.}}  \foreignlanguage{arabic}{يَتَفوَّق}~\foreignlanguage{arabic}{\textbf{١.}})\color{black}\  \begin{flushright}\color{gray}\foreignlanguage{arabic}{\textbf{\underline{\foreignlanguage{arabic}{أمثلة}}}: والله ما شاء الله عليها تْفَوَّقت عليهم كلهم}\end{flushright}\color{black}} \vspace{2mm}

{\setlength\topsep{0pt}\textbf{\foreignlanguage{arabic}{فَائِق}}\ {\color{gray}\texttt{/\sffamily {{\sffamily faːʔiq}}/}\color{black}}\ \textsc{adj}\ [m.]\ \textbf{1.}~super  \textbf{2.}~excessive\  \begin{flushright}\color{gray}\foreignlanguage{arabic}{\textbf{\underline{\foreignlanguage{arabic}{أمثلة}}}: بدورش عبنت فائِقة الجمال بس كمان إِشي ومنه}\end{flushright}\color{black}} \vspace{2mm}

{\setlength\topsep{0pt}\textbf{\foreignlanguage{arabic}{فَاق}}\ {\color{gray}\texttt{/\sffamily {{\sffamily faːq}}/}\color{black}}\ \textsc{verb}\ [p.]\ \textbf{1.}~exceed  \textbf{2.}~surpass\ \ $\bullet$\ \ \setlength\topsep{0pt}\textbf{\foreignlanguage{arabic}{فُوق}}\ {\color{gray}\texttt{/\sffamily {{\sffamily fuːq}}/}\color{black}}\ [c.]\ \ $\bullet$\ \ \setlength\topsep{0pt}\textbf{\foreignlanguage{arabic}{يفُوق}}\ {\color{gray}\texttt{/\sffamily {{\sffamily jfuːq}}/}\color{black}}\ [i.]\ \color{gray}(msa. \foreignlanguage{arabic}{يَفُوق}~\foreignlanguage{arabic}{\textbf{١.}})\color{black}\  \begin{flushright}\color{gray}\foreignlanguage{arabic}{\textbf{\underline{\foreignlanguage{arabic}{أمثلة}}}: لعب الشباب بالمباراة اليوم فاق كل التوقعات}\end{flushright}\color{black}} \vspace{2mm}

{\setlength\topsep{0pt}\textbf{\foreignlanguage{arabic}{فَوق}}\ {\color{gray}\texttt{/\sffamily {{\sffamily foː(q)}}/}\color{black}}\ \textsc{adv}\ \color{gray}(msa. \foreignlanguage{arabic}{فوق}~\foreignlanguage{arabic}{\textbf{٢.}}  \foreignlanguage{arabic}{أعلى}~\foreignlanguage{arabic}{\textbf{١.}})\color{black}\ \textbf{1.}~above  \textbf{2.}~up\ \ $\bullet$\ \ \textsc{ph.} \color{gray} \foreignlanguage{arabic}{لَو يصِيرُوَا اِجْرَيك فَوق ورَاسَك تَحِت}\color{black}\ {\color{gray}\texttt{/{\sffamily law ʔijsˤiːruː ʔi(dʒ)reːk foː(q) wraːsak lataħat}/}\color{black}}\ \textbf{1.}~when pigs fly\ \ $\bullet$\ \ \textsc{ph.} \color{gray} \foreignlanguage{arabic}{مِن فَوق لَفَوق}\color{black}\ {\color{gray}\texttt{/{\sffamily min foː(q) lafoː(q)}/}\color{black}}\ \color{gray} (msa. \foreignlanguage{arabic}{القيام بعمل شيء بعجلة وبدون إِتقان}~\foreignlanguage{arabic}{\textbf{١.}})\color{black}\ \textbf{1.}~to do sth in a hurry and not duly\ \ $\bullet$\ \ \textsc{ph.} \color{gray} \foreignlanguage{arabic}{فَوق فَوق}\color{black}\ {\color{gray}\texttt{/{\sffamily foː(q) foː(q)}/}\color{black}}\ \textbf{1.}~very rich.  \textbf{2.}~high-profile people who have connections and who can make decisions\  \begin{flushright}\color{gray}\foreignlanguage{arabic}{\textbf{\underline{\foreignlanguage{arabic}{أمثلة}}}:  الموضوع صار فْوق فْوق ما بيدنا شي نعمله\ $\bullet$\ \  أم محمد بتحكي إِنه وضع العريس فَوق فَوق\ $\bullet$\ \  مش ضروري ترتبي وتشطفي الدار كل يوم بحق الله, عادري كنسي ومسحي مِن فوق لفوق ما حدا داري عنك\ $\bullet$\ \  لو يصيروا اجريك فوق وراسَك تحت مش رح أخطبلك هالكرنيبة بنت الكرنيبة\ $\bullet$\ \  اتطلع فوق}\end{flushright}\color{black}} \vspace{2mm}

{\setlength\topsep{0pt}\textbf{\foreignlanguage{arabic}{فَوق}}\ {\color{gray}\texttt{/\sffamily {{\sffamily foː(q)}}/}\color{black}}\ \textsc{noun}\ [m.]\ \color{gray}(msa. \foreignlanguage{arabic}{فوق}~\foreignlanguage{arabic}{\textbf{٢.}}  \foreignlanguage{arabic}{أعلى}~\foreignlanguage{arabic}{\textbf{١.}})\color{black}\ \textbf{1.}~above  \textbf{2.}~up\ \ $\bullet$\ \ \textsc{ph.} \color{gray} \foreignlanguage{arabic}{فَوق بَعَض}\color{black}\ {\color{gray}\texttt{/{\sffamily foː(q) baʕadˤ}/}\color{black}}\ \color{gray} (msa. \foreignlanguage{arabic}{مزدحم جداً}~\foreignlanguage{arabic}{\textbf{١.}})\color{black}\ \textbf{1.}~very crowded\ \ $\bullet$\ \ \textsc{ph.} \color{gray} \foreignlanguage{arabic}{فَوق حَقُّه دُقُّه}\color{black}\ {\color{gray}\texttt{/{\sffamily foːq ħaqqo duqqo}/}\color{black}}\ \textbf{1.}~brazenly unfair\  \begin{flushright}\color{gray}\foreignlanguage{arabic}{\textbf{\underline{\foreignlanguage{arabic}{أمثلة}}}: يعني شو؟ فُوق حَقُّه دُقُّه كمان. أنت ما بتخاف من الله.\ $\bullet$\ \  رحنا عالحسبة النّاس فوق بَعَض\ $\bullet$\ \  بتلاقيه فوق الطاولة اللي باوضة الضيوف}\end{flushright}\color{black}} \vspace{2mm}

{\setlength\topsep{0pt}\textbf{\foreignlanguage{arabic}{فَوقَانِي}}\ {\color{gray}\texttt{/\sffamily {{\sffamily foː(q)aːni}}/}\color{black}}\ \textsc{adj}\ [m.]\ \textbf{1.}~up  \textbf{2.}~relating to the floor up\  \begin{flushright}\color{gray}\foreignlanguage{arabic}{\textbf{\underline{\foreignlanguage{arabic}{أمثلة}}}: المحل موجود بالطابق الفوقانِي}\end{flushright}\color{black}} \vspace{2mm}

{\setlength\topsep{0pt}\textbf{\foreignlanguage{arabic}{فَوقِيِّة}}\ {\color{gray}\texttt{/\sffamily {{\sffamily fawqijje}}/}\color{black}}\ \textsc{noun}\ [f.]\ \color{gray}(msa. \foreignlanguage{arabic}{فوقِيَّة}~\foreignlanguage{arabic}{\textbf{١.}})\color{black}\ \textbf{1.}~superiority\  \begin{flushright}\color{gray}\foreignlanguage{arabic}{\textbf{\underline{\foreignlanguage{arabic}{أمثلة}}}: بحس إِنه عندها نظرة فوقِيِّة وبتتعامل بفوقِيِّة مع تبعون القرى والمخيمات}\end{flushright}\color{black}} \vspace{2mm}

{\setlength\topsep{0pt}\textbf{\foreignlanguage{arabic}{مُتَفَوِّق}}\ {\color{gray}\texttt{/\sffamily {{\sffamily mutafawwiq}}/}\color{black}}\ \textsc{adj}\ [m.]\ \color{gray}(msa. \foreignlanguage{arabic}{مُتَفَوِّق}~\foreignlanguage{arabic}{\textbf{١.}})\color{black}\ \textbf{1.}~excellent\  \begin{flushright}\color{gray}\foreignlanguage{arabic}{\textbf{\underline{\foreignlanguage{arabic}{أمثلة}}}: بنتها الصغيرة مُتَفَوِّقة ودايماً بتطلع من الأوائِل}\end{flushright}\color{black}} \vspace{2mm}

\vspace{-3mm}
\markboth{\color{blue}\foreignlanguage{arabic}{ف.و.ل}\color{blue}{}}{\color{blue}\foreignlanguage{arabic}{ف.و.ل}\color{blue}{}}\subsection*{\color{blue}\foreignlanguage{arabic}{ف.و.ل}\color{blue}{}\index{\color{blue}\foreignlanguage{arabic}{ف.و.ل}\color{blue}{}}} 

{\setlength\topsep{0pt}\textbf{\foreignlanguage{arabic}{فَوَالِة}}\ {\color{gray}\texttt{/\sffamily {{\sffamily fawaːle}}/}\color{black}}\ \textsc{noun}\ [f.]\ \color{gray}(msa. \foreignlanguage{arabic}{طعام مستعجل من نواشف ومقالي يقدمه أهل الحمولة لأهل المتوفى}~\foreignlanguage{arabic}{\textbf{١.}})\color{black}\ \textbf{1.}~It is the food that the people bring to the family of the deceased person. This food is known as n a w aa sh i f, i.e., the food that is eaten on breakfast or dinner without cooking.  \textbf{2.}~such as, thyme, olives, pickles, labneh and cheese.\ } \vspace{2mm}

{\setlength\topsep{0pt}\textbf{\foreignlanguage{arabic}{فُول}}\footnote{Collective noun}\ \ {\color{gray}\texttt{/\sffamily {{\sffamily fuːl}}/}\color{black}}\ \textsc{noun}\ [m.]\ \color{gray}(msa. \foreignlanguage{arabic}{فول}~\foreignlanguage{arabic}{\textbf{١.}})\color{black}\ \textbf{1.}~beans\  \begin{flushright}\color{gray}\foreignlanguage{arabic}{\textbf{\underline{\foreignlanguage{arabic}{أمثلة}}}: طابخين فول مع لبن تعال اتفضل كل معنا}\end{flushright}\color{black}} \vspace{2mm}

{\setlength\topsep{0pt}\textbf{\foreignlanguage{arabic}{فُولِة}}\footnote{Unit noun}\ \ {\color{gray}\texttt{/\sffamily {{\sffamily fuːle}}/}\color{black}}\ \textsc{noun}\ [f.]\ \textbf{1.}~one grain of beans\ \ $\bullet$\ \ \textsc{ph.} \color{gray} \foreignlanguage{arabic}{مَا يتنيل بثمه فولة}\color{black}\ {\color{gray}\texttt{/{\sffamily maː btinbal bθimmo fuːle}/}\color{black}}\ \textbf{1.}~to spill the beans\ \ $\bullet$\ \ \textsc{ph.} \color{gray} \foreignlanguage{arabic}{كل فولة وإِلهَا كيَالهَا}\color{black}\ {\color{gray}\texttt{/{\sffamily kul fuːlew ʔilha kajjaːlha}/}\color{black}}\ \color{gray} (msa. \foreignlanguage{arabic}{كل إِنسان له نصيب مكتوب}~\foreignlanguage{arabic}{\textbf{١.}})\color{black}\ \textbf{1.}~It is an idiomatic expression that means that every lady will find a suitor/match who will admire her the way she is.\  \begin{flushright}\color{gray}\foreignlanguage{arabic}{\textbf{\underline{\foreignlanguage{arabic}{أمثلة}}}: أخوها ما يْتِنْيَل بْثِمُّه فُولِة}\end{flushright}\color{black}} \vspace{2mm}

\vspace{-3mm}
\markboth{\color{blue}\foreignlanguage{arabic}{ف.ي}\color{blue}{}}{\color{blue}\foreignlanguage{arabic}{ف.ي}\color{blue}{}}\subsection*{\color{blue}\foreignlanguage{arabic}{ف.ي}\color{blue}{}\index{\color{blue}\foreignlanguage{arabic}{ف.ي}\color{blue}{}}} 

{\setlength\topsep{0pt}\textbf{\foreignlanguage{arabic}{فَيّ}}\ {\color{gray}\texttt{/\sffamily {{\sffamily fajj}}/}\color{black}}\ \textsc{noun}\ [m.]\ \textbf{1.}~shadow\ \ $\bullet$\ \ \textsc{ph.} \color{gray} \foreignlanguage{arabic}{فَيّ ومَيّ}\color{black}\ {\color{gray}\texttt{/{\sffamily fajj wum\#jj}/}\color{black}}\ \color{gray} (msa. \foreignlanguage{arabic}{كل شيء متوفِّر}~\foreignlanguage{arabic}{\textbf{١.}})\color{black}\ \textbf{1.}~life is just a bowl of cherries\  \begin{flushright}\color{gray}\foreignlanguage{arabic}{\textbf{\underline{\foreignlanguage{arabic}{أمثلة}}}: شو ناقْصِك فهميني؟ فَي ومَي واحنا بألف نعمة\ $\bullet$\ \  تعال اقعد بالفَي}\end{flushright}\color{black}} \vspace{2mm}

{\setlength\topsep{0pt}\textbf{\foreignlanguage{arabic}{فِي}}\ {\color{gray}\texttt{/\sffamily {{\sffamily fiː}}/}\color{black}}\ \textsc{prep}\ \color{gray}(msa. \foreignlanguage{arabic}{في}~\foreignlanguage{arabic}{\textbf{١.}})\color{black}\ \textbf{1.}~in  \textbf{2.}~by\ \ $\bullet$\ \ \textsc{ph.} \color{gray} \foreignlanguage{arabic}{فِيه}\color{black}\ {\color{gray}\texttt{/{\sffamily fiː}/}\color{black}}\ \color{gray} (msa. \foreignlanguage{arabic}{هناك}~\foreignlanguage{arabic}{\textbf{١.}})\color{black}\ \textbf{1.}~there is(expletive)\ \ $\bullet$\ \ \textsc{ph.} \color{gray} \foreignlanguage{arabic}{فِش}\color{black}\ {\color{gray}\texttt{/{\sffamily fiʃ}/}\color{black}}\ \textbf{1.}~there is no.  \textbf{2.}~nothing\ \ $\bullet$\ \ \textsc{ph.} \color{gray} \foreignlanguage{arabic}{فِش خَوَاص}\color{black}\ {\color{gray}\texttt{/{\sffamily fiʃ xawaːsˤ}/}\color{black}}\ \color{gray} (msa. \foreignlanguage{arabic}{لا مفَر أو لا يوجد أي خيار آخَر}~\foreignlanguage{arabic}{\textbf{١.}})\color{black}\ \textbf{1.}~there is no escape.  \textbf{2.}~no other option\  \begin{flushright}\color{gray}\foreignlanguage{arabic}{\textbf{\underline{\foreignlanguage{arabic}{أمثلة}}}: لازم نعزمها ولا بتبعبع لكل الجارات فِش خَواص\ $\bullet$\ \  مية مرة حكيتلك فِش معي ولا تعريفِة\ $\bullet$\ \   في عندكم مي تتحمّموا؟\ $\bullet$\ \  في زلمة إِجى عندي وحكالي انه بقى صاحبك أيام مدرسة الوكالة\ $\bullet$\ \  الكَوّاش عنا بنلم فيه القش والوسخ\ $\bullet$\ \  جوا العمارة فيها سخانات شمسية\ $\bullet$\ \  والله كان شرشحته في اللي عمله لهالمسكينة\ $\bullet$\ \  في وقت العصر تقريباً\ $\bullet$\ \  همي ساكنين في أبعد وأوسخ مكان بالدنيا}\end{flushright}\color{black}} \vspace{2mm}

\vspace{-3mm}
\markboth{\color{blue}\foreignlanguage{arabic}{ف.ي.د}\color{blue}{}}{\color{blue}\foreignlanguage{arabic}{ف.ي.د}\color{blue}{}}\subsection*{\color{blue}\foreignlanguage{arabic}{ف.ي.د}\color{blue}{}\index{\color{blue}\foreignlanguage{arabic}{ف.ي.د}\color{blue}{}}} 

{\setlength\topsep{0pt}\textbf{\foreignlanguage{arabic}{اِسْتَفَاد}}\ {\color{gray}\texttt{/\sffamily {{\sffamily ʔistafaːd}}/}\color{black}}\ \textsc{verb}\ [p.]\ \textbf{1.}~benefit\ \ $\bullet$\ \ \setlength\topsep{0pt}\textbf{\foreignlanguage{arabic}{اِسْتَفِيد}}\ {\color{gray}\texttt{/\sffamily {{\sffamily ʔistafiːd}}/}\color{black}}\ [c.]\ \ $\bullet$\ \ \setlength\topsep{0pt}\textbf{\foreignlanguage{arabic}{يِسْتَفِيد}}\ {\color{gray}\texttt{/\sffamily {{\sffamily jistafiːd}}/}\color{black}}\ [i.]\ \color{gray}(msa. \foreignlanguage{arabic}{يَستَفِيد}~\foreignlanguage{arabic}{\textbf{١.}})\color{black}\  \begin{flushright}\color{gray}\foreignlanguage{arabic}{\textbf{\underline{\foreignlanguage{arabic}{أمثلة}}}: أخذت حبة أكامول بس ما استفدتش كثير عليها}\end{flushright}\color{black}} \vspace{2mm}

{\setlength\topsep{0pt}\textbf{\foreignlanguage{arabic}{فَائِدِة}}\ {\color{gray}\texttt{/\sffamily {{\sffamily faːʔide}}/}\color{black}}\ \textsc{noun}\ [f.]\ \textbf{1.}~benefit  \textbf{2.}~use  \textbf{3.}~interest (econ.)\ } \vspace{2mm}

{\setlength\topsep{0pt}\textbf{\foreignlanguage{arabic}{فَاد}}\ {\color{gray}\texttt{/\sffamily {{\sffamily faːd}}/}\color{black}}\ \textsc{verb}\ [p.]\ \textbf{1.}~make sb benefit from sth\ \ $\bullet$\ \ \setlength\topsep{0pt}\textbf{\foreignlanguage{arabic}{فِيد}}\ {\color{gray}\texttt{/\sffamily {{\sffamily fiːd}}/}\color{black}}\ [c.]\ \ $\bullet$\ \ \setlength\topsep{0pt}\textbf{\foreignlanguage{arabic}{يفِيد}}\ {\color{gray}\texttt{/\sffamily {{\sffamily jfiːd}}/}\color{black}}\ [i.]\  \begin{flushright}\color{gray}\foreignlanguage{arabic}{\textbf{\underline{\foreignlanguage{arabic}{أمثلة}}}: أنا حابب أفِيدك عمي وأشغِّل ولادك كلهم معي}\end{flushright}\color{black}} \vspace{2mm}

{\setlength\topsep{0pt}\textbf{\foreignlanguage{arabic}{فَايِد}}\ {\color{gray}\texttt{/\sffamily {{\sffamily faːjid}}/}\color{black}}\ \textsc{noun\textunderscore act}\ [m.]\ \textbf{1.}~benefitting\  \begin{flushright}\color{gray}\foreignlanguage{arabic}{\textbf{\underline{\foreignlanguage{arabic}{أمثلة}}}: بشو فايِدني أنت ها؟ ما ضل بالخم الا ممعوط الذّنب}\end{flushright}\color{black}} \vspace{2mm}

{\setlength\topsep{0pt}\textbf{\foreignlanguage{arabic}{فَايْدِة}}\ {\color{gray}\texttt{/\sffamily {{\sffamily faːjde}}/}\color{black}}\ \textsc{noun}\ [f.]\ \color{gray}(msa. \foreignlanguage{arabic}{فائدة}~\foreignlanguage{arabic}{\textbf{١.}})\color{black}\ \textbf{1.}~benefit  \textbf{2.}~interest\ \ $\bullet$\ \ \setlength\topsep{0pt}\textbf{\foreignlanguage{arabic}{فوَائِد}}\ {\color{gray}\texttt{/\sffamily {{\sffamily fawaːʔid}}/}\color{black}}\ [pl.]\  \begin{flushright}\color{gray}\foreignlanguage{arabic}{\textbf{\underline{\foreignlanguage{arabic}{أمثلة}}}: فش أي فايْدِة لهيك مشاريع}\end{flushright}\color{black}} \vspace{2mm}

{\setlength\topsep{0pt}\textbf{\foreignlanguage{arabic}{فَود}}\ {\color{gray}\texttt{/\sffamily {{\sffamily foːd}}/}\color{black}}\ \textsc{noun}\ [m.]\ (src. \color{gray}\foreignlanguage{arabic}{نابلس > قرى}\color{black})\ \color{gray}(msa. \foreignlanguage{arabic}{مَهْر}~\foreignlanguage{arabic}{\textbf{١.}})\color{black}\ \textbf{1.}~dowry\ } \vspace{2mm}

{\setlength\topsep{0pt}\textbf{\foreignlanguage{arabic}{فَيد}}\ {\color{gray}\texttt{/\sffamily {{\sffamily feːd}}/}\color{black}}\ \textsc{noun}\ [m.]\ (src. \color{gray}\foreignlanguage{arabic}{طولكرم}\color{black})\ \color{gray}(msa. \foreignlanguage{arabic}{مَهْر}~\foreignlanguage{arabic}{\textbf{١.}})\color{black}\ \textbf{1.}~dowry\  \begin{flushright}\color{gray}\foreignlanguage{arabic}{\textbf{\underline{\foreignlanguage{arabic}{أمثلة}}}: ان شاء الله الليلة بنزور الجماعة نقطِّع الفيد}\end{flushright}\color{black}} \vspace{2mm}

{\setlength\topsep{0pt}\textbf{\foreignlanguage{arabic}{فِيد}}\ {\color{gray}\texttt{/\sffamily {{\sffamily fiːd}}/}\color{black}}\ \textsc{noun}\ [m.]\ \color{gray}(msa. \foreignlanguage{arabic}{مهر}~\foreignlanguage{arabic}{\textbf{١.}})\color{black}\ \textbf{1.}~dowry\  \begin{flushright}\color{gray}\foreignlanguage{arabic}{\textbf{\underline{\foreignlanguage{arabic}{أمثلة}}}: قديش فِيدْها بنت أبو السعيد؟}\end{flushright}\color{black}} \vspace{2mm}

{\setlength\topsep{0pt}\textbf{\foreignlanguage{arabic}{مَفَاد}}\ {\color{gray}\texttt{/\sffamily {{\sffamily mafaːd}}/}\color{black}}\ \textsc{noun}\ [m.]\ \textbf{1.}~meaning  \textbf{2.}~content\ } \vspace{2mm}

{\setlength\topsep{0pt}\textbf{\foreignlanguage{arabic}{مُفِيد}}\ {\color{gray}\texttt{/\sffamily {{\sffamily mufiːd}}/}\color{black}}\ \textsc{adj}\ [m.]\ \color{gray}(msa. \foreignlanguage{arabic}{مُفيد}~\foreignlanguage{arabic}{\textbf{١.}})\color{black}\ \textbf{1.}~beneficial\  \begin{flushright}\color{gray}\foreignlanguage{arabic}{\textbf{\underline{\foreignlanguage{arabic}{أمثلة}}}: كل معلقة قزحة مطحونة مع عسل عالريق كثير مُفيد هالاشي للامساك اللي عندك}\end{flushright}\color{black}} \vspace{2mm}

\vspace{-3mm}
\markboth{\color{blue}\foreignlanguage{arabic}{ف.ي.د.ي.و}\color{blue}{ (ntws)}}{\color{blue}\foreignlanguage{arabic}{ف.ي.د.ي.و}\color{blue}{ (ntws)}}\subsection*{\color{blue}\foreignlanguage{arabic}{ف.ي.د.ي.و}\color{blue}{ (ntws)}\index{\color{blue}\foreignlanguage{arabic}{ف.ي.د.ي.و}\color{blue}{ (ntws)}}} 

{\setlength\topsep{0pt}\textbf{\foreignlanguage{arabic}{فِيدْيَو}}\ {\color{gray}\texttt{/\sffamily {{\sffamily viːdjoː}}/}\color{black}}\ \textsc{noun}\ [m.]\ \textbf{1.}~video\  \begin{flushright}\color{gray}\foreignlanguage{arabic}{\textbf{\underline{\foreignlanguage{arabic}{أمثلة}}}: بلكته من كثر مابيبعثلي فِيدْيَوهات عالواتس. والله جنني!}\end{flushright}\color{black}} \vspace{2mm}

\vspace{-3mm}
\markboth{\color{blue}\foreignlanguage{arabic}{ف.ي.ش}\color{blue}{}}{\color{blue}\foreignlanguage{arabic}{ف.ي.ش}\color{blue}{}}\subsection*{\color{blue}\foreignlanguage{arabic}{ف.ي.ش}\color{blue}{}\index{\color{blue}\foreignlanguage{arabic}{ف.ي.ش}\color{blue}{}}} 

{\setlength\topsep{0pt}\textbf{\foreignlanguage{arabic}{فِيش}}\ {\color{gray}\texttt{/\sffamily {{\sffamily fiːʃ}}/}\color{black}}\ \textsc{noun}\ [m.]\ \textbf{1.}~plug  \textbf{2.}~plug adapter\ \ $\bullet$\ \ \setlength\topsep{0pt}\textbf{\foreignlanguage{arabic}{فْيَاش}}\ {\color{gray}\texttt{/\sffamily {{\sffamily fjaːʃ}}/}\color{black}}\ [pl.]\ \ $\bullet$\ \ \setlength\topsep{0pt}\textbf{\foreignlanguage{arabic}{فِيَش}}\ {\color{gray}\texttt{/\sffamily {{\sffamily fijaʃ}}/}\color{black}}\ [pl.]\  \begin{flushright}\color{gray}\foreignlanguage{arabic}{\textbf{\underline{\foreignlanguage{arabic}{أمثلة}}}: كل الفْياش اللي عندي فقعن}\end{flushright}\color{black}} \vspace{2mm}

{\setlength\topsep{0pt}\textbf{\foreignlanguage{arabic}{فِيشِة}}\ {\color{gray}\texttt{/\sffamily {{\sffamily fiːʃe}}/}\color{black}}\ \textsc{noun}\ [f.]\ \textbf{1.}~a shawl that is made from silk or wool that Palestinian women in cities used to wear on their heads and shoulder .\ \ $\bullet$\ \ \setlength\topsep{0pt}\textbf{\foreignlanguage{arabic}{فِيَش}}\ {\color{gray}\texttt{/\sffamily {{\sffamily fijaʃ}}/}\color{black}}\ [pl.]\ } \vspace{2mm}

\vspace{-3mm}
\markboth{\color{blue}\foreignlanguage{arabic}{ف.ي.ض}\color{blue}{}}{\color{blue}\foreignlanguage{arabic}{ف.ي.ض}\color{blue}{}}\subsection*{\color{blue}\foreignlanguage{arabic}{ف.ي.ض}\color{blue}{}\index{\color{blue}\foreignlanguage{arabic}{ف.ي.ض}\color{blue}{}}} 

{\setlength\topsep{0pt}\textbf{\foreignlanguage{arabic}{فَائِض}}\ {\color{gray}\texttt{/\sffamily {{\sffamily faːʔi(dˤ)}}/}\color{black}}\ \textsc{adj}\ [m.]\ \textbf{1.}~superflous\  \begin{flushright}\color{gray}\foreignlanguage{arabic}{\textbf{\underline{\foreignlanguage{arabic}{أمثلة}}}: لما كان في فائِض من الميزانية لهالسنة رجعلهم لرئاسة الوكالة}\end{flushright}\color{black}} \vspace{2mm}

{\setlength\topsep{0pt}\textbf{\foreignlanguage{arabic}{فَاض}}\ {\color{gray}\texttt{/\sffamily {{\sffamily faː(dˤ)}}/}\color{black}}\ \textsc{verb}\ [p.]\ \textbf{1.}~exceed  \textbf{2.}~overflow\ \ $\bullet$\ \ \setlength\topsep{0pt}\textbf{\foreignlanguage{arabic}{فِيض}}\ {\color{gray}\texttt{/\sffamily {{\sffamily fiː(dˤ)}}/}\color{black}}\ [c.]\ \ $\bullet$\ \ \setlength\topsep{0pt}\textbf{\foreignlanguage{arabic}{يفِيض}}\ {\color{gray}\texttt{/\sffamily {{\sffamily jfiː(dˤ)}}/}\color{black}}\ [i.]\  \begin{flushright}\color{gray}\foreignlanguage{arabic}{\textbf{\underline{\foreignlanguage{arabic}{أمثلة}}}: خفت تفِيض المي عشان هيك خليته يشيك عالمزاريب}\end{flushright}\color{black}} \vspace{2mm}

{\setlength\topsep{0pt}\textbf{\foreignlanguage{arabic}{فَايِض}}\ {\color{gray}\texttt{/\sffamily {{\sffamily faːji(dˤ)}}/}\color{black}}\ \textsc{noun\textunderscore act}\ [m.]\ \textbf{1.}~exceeding  \textbf{2.}~overflowing\  \begin{flushright}\color{gray}\foreignlanguage{arabic}{\textbf{\underline{\foreignlanguage{arabic}{أمثلة}}}: المي بقت فايضة عكل البناية}\end{flushright}\color{black}} \vspace{2mm}

{\setlength\topsep{0pt}\textbf{\foreignlanguage{arabic}{فَيض}}\ {\color{gray}\texttt{/\sffamily {{\sffamily feː(dˤ)}}/}\color{black}}\ \textsc{noun}\ [m.]\ \textbf{1.}~a superfluity of sth\ \ $\bullet$\ \ \textsc{ph.} \color{gray} \foreignlanguage{arabic}{اِتْرُكْهَا عَفَيض الله}\color{black}\ {\color{gray}\texttt{/{\sffamily ʔutrukha ʕafeː(dˤ) ʔalˤlˤa}/}\color{black}}\ \textbf{1.}~man proposes, God disposes\  \begin{flushright}\color{gray}\foreignlanguage{arabic}{\textbf{\underline{\foreignlanguage{arabic}{أمثلة}}}: خلاص اتركها عفِيض الله يازلمة!}\end{flushright}\color{black}} \vspace{2mm}

{\setlength\topsep{0pt}\textbf{\foreignlanguage{arabic}{فَيَضَان}}\ {\color{gray}\texttt{/\sffamily {{\sffamily faja(dˤ)aːn}}/}\color{black}}\ \textsc{noun}\ [m.]\ \textbf{1.}~flood\ } \vspace{2mm}

\vspace{-3mm}
\markboth{\color{blue}\foreignlanguage{arabic}{ف.ي.ع}\color{blue}{}}{\color{blue}\foreignlanguage{arabic}{ف.ي.ع}\color{blue}{}}\subsection*{\color{blue}\foreignlanguage{arabic}{ف.ي.ع}\color{blue}{}\index{\color{blue}\foreignlanguage{arabic}{ف.ي.ع}\color{blue}{}}} 

{\setlength\topsep{0pt}\textbf{\foreignlanguage{arabic}{اِسْتَفيَع}}\ {\color{gray}\texttt{/\sffamily {{\sffamily ʔistajaʕ}}/}\color{black}}\ \textsc{verb}\ [p.]\ \textbf{1.}~consider sb as cool and funky in a way that does not go along with the standards of the society\ \ $\bullet$\ \ \setlength\topsep{0pt}\textbf{\foreignlanguage{arabic}{اِسْتَفْيِع}}\ {\color{gray}\texttt{/\sffamily {{\sffamily ʔiʔistajiʕ}}/}\color{black}}\ [c.]\ \ $\bullet$\ \ \setlength\topsep{0pt}\textbf{\foreignlanguage{arabic}{يِسْتَفْيِع}}\ {\color{gray}\texttt{/\sffamily {{\sffamily jiʔistajiʕ}}/}\color{black}}\ [i.]\ } \vspace{2mm}

{\setlength\topsep{0pt}\textbf{\foreignlanguage{arabic}{فَاع}}\ {\color{gray}\texttt{/\sffamily {{\sffamily faːʕ}}/}\color{black}}\ \textsc{verb}\ [p.]\ \textbf{1.}~stroll around for pleasure.  \textbf{2.}~go out and visit many places (without any restrictions).  \textbf{3.}~spread rapidly.  \textbf{4.}~overflow (sewage).  \textbf{5.}~yell at sb and tell him/her off\ \ $\bullet$\ \ \setlength\topsep{0pt}\textbf{\foreignlanguage{arabic}{فِيع}}\ {\color{gray}\texttt{/\sffamily {{\sffamily fiːʕ}}/}\color{black}}\ [c.]\ \ $\bullet$\ \ \setlength\topsep{0pt}\textbf{\foreignlanguage{arabic}{يفِيع}}\ {\color{gray}\texttt{/\sffamily {{\sffamily jfiːʕ}}/}\color{black}}\ [i.]\  \begin{flushright}\color{gray}\foreignlanguage{arabic}{\textbf{\underline{\foreignlanguage{arabic}{أمثلة}}}: اسمعوا بدنا نْفِيع برام الله شو رأيكم؟\ $\bullet$\ \  فاع فيِّي كأنِّي أنا السبب بكل شي صارله\ $\bullet$\ \  الله لا يورجيك لما فاعَت المجاري \ $\bullet$\ \  فاع النمل بالدار}\end{flushright}\color{black}} \vspace{2mm}

{\setlength\topsep{0pt}\textbf{\foreignlanguage{arabic}{فَايِع}}\ {\color{gray}\texttt{/\sffamily {{\sffamily faːjiʕ}}/}\color{black}}\ \textsc{adj}\ [m.]\ \color{gray}(msa. \foreignlanguage{arabic}{غير تقليدي}~\foreignlanguage{arabic}{\textbf{١.}})\color{black}\ \textbf{1.}~funky  \textbf{2.}~cool\  \begin{flushright}\color{gray}\foreignlanguage{arabic}{\textbf{\underline{\foreignlanguage{arabic}{أمثلة}}}: الأب فايِع والأم الله يستر عليها مش مخلية حدا من شرها}\end{flushright}\color{black}} \vspace{2mm}

{\setlength\topsep{0pt}\textbf{\foreignlanguage{arabic}{فَايِع}}\ {\color{gray}\texttt{/\sffamily {{\sffamily faːjiʕ}}/}\color{black}}\ \textsc{noun\textunderscore act}\ [m.]\ \textbf{1.}~strolling around for pleasure.  \textbf{2.}~going out and visit many places.  \textbf{3.}~spread rapidly.  \textbf{4.}~overflowing (sewage).  \textbf{5.}~yelling at sb and tell him/her off\  \begin{flushright}\color{gray}\foreignlanguage{arabic}{\textbf{\underline{\foreignlanguage{arabic}{أمثلة}}}: المجاري فايعة يكفيك شرها\ $\bullet$\ \  ضليتني فايِع برة طول اليوم}\end{flushright}\color{black}} \vspace{2mm}

{\setlength\topsep{0pt}\textbf{\foreignlanguage{arabic}{فَيَاعَة}}\ {\color{gray}\texttt{/\sffamily {{\sffamily faːjaːʕa}}/}\color{black}}\ \textsc{noun}\ [f.]\ \textbf{1.}~the state of being cool and funky in a way that does not go along with the standards of the society\ } \vspace{2mm}

{\setlength\topsep{0pt}\textbf{\foreignlanguage{arabic}{فَيَّع}}\ {\color{gray}\texttt{/\sffamily {{\sffamily fajjaʕ}}/}\color{black}}\ \textsc{verb}\ [p.]\ \textbf{1.}~make sb cool and funky.  \textbf{2.}~make sb stroll around for pleasure.  \textbf{3.}~make sb go out and visit many places\ \ $\bullet$\ \ \setlength\topsep{0pt}\textbf{\foreignlanguage{arabic}{فيِّع}}\ {\color{gray}\texttt{/\sffamily {{\sffamily fajjiʕ}}/}\color{black}}\ [c.]\ \ $\bullet$\ \ \setlength\topsep{0pt}\textbf{\foreignlanguage{arabic}{يفيِّع}}\ {\color{gray}\texttt{/\sffamily {{\sffamily jfajjiʕ}}/}\color{black}}\ [i.]\  \begin{flushright}\color{gray}\foreignlanguage{arabic}{\textbf{\underline{\foreignlanguage{arabic}{أمثلة}}}: والله غير أفيعِك يا خالتو بس أنت تعالي عندي عرام الله}\end{flushright}\color{black}} \vspace{2mm}

{\setlength\topsep{0pt}\textbf{\foreignlanguage{arabic}{مِسْتَفْيِع}}\ {\color{gray}\texttt{/\sffamily {{\sffamily miʔistajiʕ}}/}\color{black}}\ \textsc{noun\textunderscore act}\ [m.]\ \textbf{1.}~considering sb as cool and funky in a way that does not go along with the standards of the society\  \begin{flushright}\color{gray}\foreignlanguage{arabic}{\textbf{\underline{\foreignlanguage{arabic}{أمثلة}}}: أنا كنت مِسْتَفْيِعة حالي بالأول بس شفت فاتِن عرفت إِني مؤدبة}\end{flushright}\color{black}} \vspace{2mm}

\vspace{-3mm}
\markboth{\color{blue}\foreignlanguage{arabic}{ف.ي.ق}\color{blue}{}}{\color{blue}\foreignlanguage{arabic}{ف.ي.ق}\color{blue}{}}\subsection*{\color{blue}\foreignlanguage{arabic}{ف.ي.ق}\color{blue}{}\index{\color{blue}\foreignlanguage{arabic}{ف.ي.ق}\color{blue}{}}} 

{\setlength\topsep{0pt}\textbf{\foreignlanguage{arabic}{فَاق}}\ {\color{gray}\texttt{/\sffamily {{\sffamily faː(q)}}/}\color{black}}\ \textsc{verb}\ [p.]\ \textbf{1.}~wake up.  \textbf{2.}~start to realize\ \ $\bullet$\ \ \setlength\topsep{0pt}\textbf{\foreignlanguage{arabic}{فِيق}}\ {\color{gray}\texttt{/\sffamily {{\sffamily fiː(q)}}/}\color{black}}\ [c.]\ \ $\bullet$\ \ \setlength\topsep{0pt}\textbf{\foreignlanguage{arabic}{يفِيق}}\ {\color{gray}\texttt{/\sffamily {{\sffamily jfiː(q)}}/}\color{black}}\ [i.]\ \color{gray}(msa. \foreignlanguage{arabic}{بدأ يدرك}~\foreignlanguage{arabic}{\textbf{٢.}}  \foreignlanguage{arabic}{استيقظ}~\foreignlanguage{arabic}{\textbf{١.}})\color{black}\  \begin{flushright}\color{gray}\foreignlanguage{arabic}{\textbf{\underline{\foreignlanguage{arabic}{أمثلة}}}: أنت كل يوم بِتفيق قبل الشحادة وبنتها\ $\bullet$\ \  أنا متى فقت عحالي؟ لما صفِّيت عالحديدة وبطل حيلتي اللضى}\end{flushright}\color{black}} \vspace{2mm}

{\setlength\topsep{0pt}\textbf{\foreignlanguage{arabic}{فَايِق}}\ {\color{gray}\texttt{/\sffamily {{\sffamily faːjiq}}/}\color{black}}\ \textsc{adj}\ [m.]\ \textbf{1.}~awake  \textbf{2.}~alert  \textbf{3.}~be in a good mood to do sth\ \ $\bullet$\ \ \textsc{ph.} \color{gray} \foreignlanguage{arabic}{فَايق و رَايق}\color{black}\ {\color{gray}\texttt{/{\sffamily faːji(q) wuraːji(q)}/}\color{black}}\ \textbf{1.}~to be at peace with the world\  \begin{flushright}\color{gray}\foreignlanguage{arabic}{\textbf{\underline{\foreignlanguage{arabic}{أمثلة}}}: شو؟ شايفك فايِق و رايِق عساعة هالصبح\ $\bullet$\ \  أنا مش فايِقلك أبداً}\end{flushright}\color{black}} \vspace{2mm}

{\setlength\topsep{0pt}\textbf{\foreignlanguage{arabic}{فَيَّق}}\ {\color{gray}\texttt{/\sffamily {{\sffamily fajja(q)}}/}\color{black}}\ \textsc{verb}\ [p.]\ \textbf{1.}~wake sb up\ \ $\bullet$\ \ \setlength\topsep{0pt}\textbf{\foreignlanguage{arabic}{فَيِّق}}\ {\color{gray}\texttt{/\sffamily {{\sffamily fajji(q)}}/}\color{black}}\ [c.]\ \ $\bullet$\ \ \setlength\topsep{0pt}\textbf{\foreignlanguage{arabic}{يفَيِّق}}\ {\color{gray}\texttt{/\sffamily {{\sffamily jfajji(q)}}/}\color{black}}\ [i.]\ \color{gray}(msa. \foreignlanguage{arabic}{يوقِظ}~\foreignlanguage{arabic}{\textbf{١.}})\color{black}\  \begin{flushright}\color{gray}\foreignlanguage{arabic}{\textbf{\underline{\foreignlanguage{arabic}{أمثلة}}}: فَيِّقني الصبح بكير ألحق المعبر}\end{flushright}\color{black}} \vspace{2mm}

\vspace{-3mm}
\markboth{\color{blue}\foreignlanguage{arabic}{ف.ي.ل}\color{blue}{}}{\color{blue}\foreignlanguage{arabic}{ف.ي.ل}\color{blue}{}}\subsection*{\color{blue}\foreignlanguage{arabic}{ف.ي.ل}\color{blue}{}\index{\color{blue}\foreignlanguage{arabic}{ف.ي.ل}\color{blue}{}}} 

{\setlength\topsep{0pt}\textbf{\foreignlanguage{arabic}{فِيل}}\ {\color{gray}\texttt{/\sffamily {{\sffamily fiːl}}/}\color{black}}\ \textsc{noun}\ [m.]\ \color{gray}(msa. \foreignlanguage{arabic}{فِيل}~\foreignlanguage{arabic}{\textbf{١.}})\color{black}\ \textbf{1.}~elephant\ \ $\bullet$\ \ \setlength\topsep{0pt}\textbf{\foreignlanguage{arabic}{فِيلَة}}\ {\color{gray}\texttt{/\sffamily {{\sffamily fijala}}/}\color{black}}\ [pl.]\ \ $\bullet$\ \ \textsc{ph.} \color{gray} \foreignlanguage{arabic}{صَايرة قد الفِيل}\color{black}\ {\color{gray}\texttt{/{\sffamily sˤaːjra (q)add ʔilfiːl}/}\color{black}}\ \textbf{1.}~It is an idiomatic expression that means that sb has gained a lot of weight\  \begin{flushright}\color{gray}\foreignlanguage{arabic}{\textbf{\underline{\foreignlanguage{arabic}{أمثلة}}}: صايرة قد الفِيل بعد الحمل}\end{flushright}\color{black}} \vspace{2mm}

\vspace{-3mm}
\markboth{\color{blue}\foreignlanguage{arabic}{ف.ي.ل}\color{blue}{ (ntws)}}{\color{blue}\foreignlanguage{arabic}{ف.ي.ل}\color{blue}{ (ntws)}}\subsection*{\color{blue}\foreignlanguage{arabic}{ف.ي.ل}\color{blue}{ (ntws)}\index{\color{blue}\foreignlanguage{arabic}{ف.ي.ل}\color{blue}{ (ntws)}}} 

{\setlength\topsep{0pt}\textbf{\foreignlanguage{arabic}{فَايْل}}\footnote{English loanword}\ \ {\color{gray}\texttt{/\sffamily {{\sffamily faːjl}}/}\color{black}}\ \textsc{noun}\ [m.]\ \textbf{1.}~file\ } \vspace{2mm}

{\setlength\topsep{0pt}\textbf{\foreignlanguage{arabic}{فَيَل}}\footnote{English loanword}\ \ {\color{gray}\texttt{/\sffamily {{\sffamily fajal}}/}\color{black}}\ \textsc{noun}\ [m.]\ \color{gray}(msa. \foreignlanguage{arabic}{ملف}~\foreignlanguage{arabic}{\textbf{١.}})\color{black}\ \textbf{1.}~file\  \begin{flushright}\color{gray}\foreignlanguage{arabic}{\textbf{\underline{\foreignlanguage{arabic}{أمثلة}}}: ضل علي آخر فَيَل من هالمجموعة وبكرة ان شاء الله بكمل المجموعة الثانية}\end{flushright}\color{black}} \vspace{2mm}

{\setlength\topsep{0pt}\textbf{\foreignlanguage{arabic}{فَيَّل}}\footnote{English loanword}\ \ {\color{gray}\texttt{/\sffamily {{\sffamily fajjal}}/}\color{black}}\ \textsc{verb}\ [p.]\ \textbf{1.}~put papers in a file\ \ $\bullet$\ \ \setlength\topsep{0pt}\textbf{\foreignlanguage{arabic}{فَيِّل}}\footnote{English loanword}\ \ {\color{gray}\texttt{/\sffamily {{\sffamily fajjil}}/}\color{black}}\ [c.]\ \ $\bullet$\ \ \setlength\topsep{0pt}\textbf{\foreignlanguage{arabic}{يفَيِّل}}\footnote{English loanword}\ \ {\color{gray}\texttt{/\sffamily {{\sffamily jfajjil}}/}\color{black}}\ [i.]\  \begin{flushright}\color{gray}\foreignlanguage{arabic}{\textbf{\underline{\foreignlanguage{arabic}{أمثلة}}}: خلي أحمد يفَيِّللي هالأوراق}\end{flushright}\color{black}} \vspace{2mm}

\end{multicols}

\end{document}


% \include{letter_sections/ق.tex}
% \include{letter_sections/ك.tex}
% \include{letter_sections/ل.tex}
% \include{letter_sections/م.tex}
% \include{letter_sections/ن.tex}
% \include{letter_sections/ه.tex}
% \include{letter_sections/و.tex}
% \include{letter_sections/ي.tex}

%-----------------------------------------------------------
% INDEX & BIBLIOGRAPHY
%-----------------------------------------------------------
\printindex
\printbibliography

\end{document}